\section{Proof of the controllability properties}

This section is devoted to study the control properties for the parabolic system \eqref{heat_frac}. We begin by proving the null controllability result.

\begin{proof}[Proof of Theorem \ref{null_control_thm}]
First of all, for all $\varphi^T\in L^2(-1,1)$, let $\varphi(x,t)$ be the unique solution to the adjoint system
\begin{align}\label{heat_frac_adj}
	\begin{cases}
		-\varphi_t + \fl{s}{\varphi} = 0, & (x,t)\in (-1,1)\times(0,T)
		\\
		\varphi = 0, & (x,t)\in [\,\RR\setminus(-1,1)\,]\times(0,T)
		\\
		\varphi(x,T) = \varphi^T(x), & x\in (-1,1).
	\end{cases}
\end{align}

Multiplying \eqref{heat_frac} by $\varphi$ and integrating over $(-1,1)\times (0,T)$, it is straightforward to check that $z(x,T)=0$ if and only if 
\begin{align}\label{control_id}
	\int_0^T\int_{-1}^1 \varphi(x,t)g(x,t)\mathbf{1}_{\omega}(x)\,dxdt = -\int_{-1}^1 u_0(x)\varphi(x,0)\,dx,
\end{align}

In turn, it is classical that \eqref{control_id} is equivalent to the existence of a constant $C>0$ such that the following observability inequality holds
\begin{align}\label{obs}
	\norm{\varphi(x,0)}{L^2(-1,1)}^2\leq C\int_0^T\left|\,\int_{-1}^1 \varphi(x,t)g(x,t)\mathbf{1}_{\omega}(x)\,dx\,\right|^2\,dt,
\end{align}

Notice that $\varphi$ can be expressed in the basis of the eigenfunctions of the fractional Laplacian on $(-1,1)$ with zero Dirichlet boundary conditions. Namely,
\begin{align}\label{adj_sol}
	\varphi(x,t) = \sum_{k\geq 1} \varphi_ke^{-\lambda_k(T-t)}\varrho_k(x), 
\end{align}
where $\varphi_k = \langle \varphi^T,\varrho_k\rangle$ and, for $k\geq 1$, $\varrho_k(x)$ are the solutions to the following eigenvalue problem 
\begin{align*}
	\begin{cases}
		\fl{s}{\varrho_k} = \lambda_k\varrho_k, & x\in (-1,1), \;\; k\in\NN
		\\
		\varrho_k = 0, & x\in \,\RR\setminus(-1,1).
	\end{cases}
\end{align*}
	
Now, plugging \eqref{adj_sol} into \eqref{obs}, using the orthonormality of the eigenfunctions $\varrho_k$ as a basis of $L^2(-1,1)$ and employing the change of variables $T-t\mapsto t$, the observability inequality becomes 
\begin{align}\label{obs_spectr}
	\sum_{k\geq 1} |\varphi_k|^2e^{-2\lambda_k T} \leq C\int_0^T\left|\,\sum_{k\geq 1} \varphi_kg_k(t)e^{-\lambda_k t}\right|^2\,dt, 
\end{align}
where $g_k = \langle g\mathbf{1}_{\omega},\varrho_k\rangle$. 
	
By means of the classical moment method (\cite{fattorini1971exact}), inequalities of the form \eqref{obs_spectr} are well known to be true if and only if \eqref{eigen_cond} holds and the eigenfunctions $\varrho_k$ satisfy the following lower bound:
\begin{align}\label{eigenf_bound}
	\norm{\varrho_k}{L^2(\omega)}\geq C>0, \,\,\forall \, k\geq 1,
\end{align}
where the constant $C$ is independent of $k$. The proof of \eqref{eigenf_bound} is an easy adaptation of the one of \cite[Lemma 2]{kwasnicki2012eigenvalues}. Moreover, according to \cite{kulczycki2010spectral,kwasnicki2012eigenvalues} we have 
\begin{align*}
	\lambda_k = \left(\frac{k\pi}{2}-\frac{(1-s)\pi}{4}\right)^{2s}+O\left(\frac{1}{k}\right).
\end{align*}
	
Therefore, we easily see that the condition \eqref{eigen_cond} is satisfied if and only if $s>1/2$. If $s\leq 1/2$, instead, the series diverges, since it behaves as an harmonic series. In conclusion, the observability inequality \eqref{obs} holds true when $s>1/2$, but it is false when $s\leq 1/2$. This concludes the proof. 
\end{proof}

Even if for $s\leq 1/2$ null controllability for \eqref{heat_frac} fails, Theorem \ref{approx_control_thm} ensures that, for all $s\in(0,1)$, we still have approximate controllability. This is consequence of a unique continuation property for the fractional Laplacian, which has been obtained in \cite{fall2014unique}.

\begin{proof}[Proof of Theorem \ref{approx_control_thm}]
It is classical that the result is true as soon as one has the following unique continuation property for the solution to the adjoint equation \eqref{heat_frac_adj} (see, e.g., \cite[Theorem 5.2]{micu2004introduction}).
	
\MyQuote{Given $s\in(0,1)$ and $\varphi^T_0\in L^2(-1,1)$, let $\varphi$ be the unique solution to the system \eqref{heat_frac_adj}. Let $\omega\subset (-1,1)$ be an arbitrary open set. If $\varphi = 0$ on $\omega\times(0,T)$, then $\varphi = 0$ on $(-1,1)\times(0,T)$.}
	
Therefore, we are reduced to the proof of the property ($\mathcal P$). To this end, let us recall that $\varphi$ can be expressed in the form \eqref{adj_sol} and let us assume that 
\begin{align}\label{uc}
	\varphi=0 \textrm{ in } \omega\times(0,T). 
\end{align}
Let $\{\psi_{k_j}\}_{1\leq k\leq m_k}$ be an orthonormal basis of $\ker(\lambda_k-\fl{s}{})$. Then, \eqref{adj_sol} can be rewritten as
\begin{align*}
	\varphi(x,t) = \sum_{k\geq 1} \left(\sum_{j=1}^{m_k} \varphi_{k_j}\psi_{k_j}(x)\right)e^{-\lambda_k(T-t)}, \;\;\; (x,t)\in (-1,1)\times(-\infty, T). 
\end{align*}

Let $z\in\CC$ with $\eta:=\Re(z)>0$ and let $N\in\NN$. Since the functions $\psi_{k_j}$, $1\leq j\leq m_k$, $1\leq k\leq N$ are orthonormal, we have that
\begin{align*}
	\norm{\sum_{k=1}^N \left(\sum_{j=1}^{m_k} \varphi_{k_j}\psi_{k_j}(x)\right)e^{z(t-T)}e^{-\lambda_k(T-t)}}{L^2(-1,1)}^2 & \leq \sum_{k=1}^N \left(\sum_{j=1}^{m_k} |\varphi_{k_j}|^2\right)e^{2\eta(t-T)}e^{-2\lambda_k(T-t)}
	\\
	& \leq \sum_{k\geq 1} \left(\sum_{j=1}^{m_k} |\varphi_{k_j}|^2\right)e^{2\eta(t-T)}e^{-2\lambda_k(T-t)} \leq Ce^{2\eta(t-T)}\norm{\varphi^T}{L^2(-1,1)}^2.
\end{align*}
Hence, letting 
\begin{align*}
	w_N(x,t):= \sum_{k=1}^N \left(\sum_{j=1}^{m_k} \varphi_{k_j}\psi_{k_j}(x)\right)e^{z(t-T)}e^{-\lambda_k(T-t)},
\end{align*}
we have shown that $\norm{w_T(x,t)}{L^2(-1,1)}\leq Ce^{\eta(t-T)}\norm{\varphi^T}{L^2(-1,1)}$. Moreover, we have
\begin{align*}
	\int_{-\infty}^T e^{\eta(t-T)}\norm{\varphi^T}{L^2(-1,1)}\,dt = \frac{1}{\eta}\norm{\varphi^T}{L^2(-1,1)}\int_0^{+\infty} e^{-\tau}\,d\tau = \frac{1}{\eta}\norm{\varphi^T}{L^2(-1,1)}.
\end{align*}
Therefore, we can apply the Dominated Convergence Theorem, obtaining
\begin{align}\label{dct_identity}
	\lim_{N\to+\infty}\int_{-\infty}^T w_N(x,t)\,dt &= \int_{-\infty}^T\lim_{N\to+\infty} w_N(x,t)\,dt = \int_{-\infty}^T e^{z(t-T)} \sum_{k=1}^{+\infty} \left(\sum_{j=1}^{m_k} \varphi_{k_j}\psi_{k_j}(x)\right)e^{-\lambda_k(T-t)}\,dt \notag
	\\
	&=\sum_{k=1}^{+\infty}\sum_{j=1}^{m_k} \varphi_{k_j}\psi_{k_j}(x) \int_{-\infty}^T e^{z(t-T)}e^{-\lambda_k(T-t)}\,dt =\sum_{k=1}^{+\infty}\sum_{j=1}^{m_k} \varphi_{k_j}\psi_{k_j}(x) \int_0^{+\infty} e^{-(z+\lambda_k)\tau}\,d\tau \notag
	\\
	&=\sum_{k=1}^{+\infty}\sum_{j=1}^{m_k} \frac{\varphi_{k_j}}{z+\lambda_k}\psi_{k_j}(x), \;\;\; x\in (-1,1)\;\Re(z)>0.
\end{align} 
It follows from \eqref{uc} and \eqref{dct_identity} that 
\begin{align*}
	\sum_{k=1}^{+\infty}\sum_{j=1}^{m_k} \frac{\varphi_{k_j}}{z+\lambda_k}\psi_{k_j}(x)=0, \;\;\; x\in\omega,\;\Re(z)>0.
\end{align*}

This holds for every $z\in\CC\setminus\{-\lambda_k\}_{k\in\NN}$, using the analytic continuation in $z$. Hence, taking a suitable small circle around $-\lambda_{\ell}$ not including $\{-\lambda_k\}_{k\neq\ell}$ and integrating on that circle we get that
\begin{align*}
	w_\ell:=\sum_{j=1}^{m_{\ell}} \varphi_{\ell_j}\psi_{\ell_j}(x)=0, \;\;\; x\in\omega.
\end{align*}
	
According to \cite[Theorem 1.4]{fall2014unique}, $\fl{s}{}$ has the unique continuation property in the sense that if $\lambda_k$ is an eigenvalue of $\fl{s}{}$ on (-1,1) with Dirichlet boundary conditions, and $(\fl{s}{}-\lambda_k)\varrho_k = 0$ in $(-1,1)$ with $\varrho_k = 0$ in $\omega$, then $\varrho_k = 0$ in $(-1,1)$. 
This can applied to $w_{\ell}$, in order to conclude $w_{\ell} = 0$ in $(-1,1)$ for every $\ell$. Since $\{\psi_{\ell_j}\}_{1\leq j\leq m_{\ell}}$ are linearly independent in $L^2(-1,1)$, we get $\varphi_{\ell_j} = 0$, $1\leq j\leq m_k$, $\ell\in\NN$. It follows that $\varphi^T=0$ and hence, $\varphi=0$ in $(-1,1)\times(0,T)$, meaning that $\varphi$ enjoys the property $(\mathcal{P})$. As an immediate consequence, we have that our original equation \eqref{heat_frac} is approximately controllable. Our proof is then concluded. 
\end{proof}

\begin{remark}
According to \cite{fall2014unique}, the elliptic unique continuation property for the fractional Laplacian holds in any space dimension. In view of that, Theorem \ref{approx_control_thm} may be extended also to the case $N>1$. On the other hand, the same does not applies to Theorem \eqref{null_control_thm}. Indeed, the proof of this result uses arguments that are designed specifically for one-dimensional problems (\cite{fattorini1971exact}). If one would analyze the null-controllability in a general multi-dimensional setting, other tools (such as a Carleman estimate) are needed. As far as we know, these techniques have not been fully developed yet for problems involving the fractional Laplacian on a domain. 
\end{remark}
%Therefore, given any initial datum $z_0\in L^2(-1,1)$ we are interested in computing numerically the control function $g$ that drives the solution $z$ to zero in time $T$. Before describing the methodology that we shall adopt, we recall the existing theoretical results on the controllability of the fractional heat equation \eqref{heat_frac}. This will give us a hint about what we should expect from our simulations, and will provide a validation of the accuracy of our numerical method.



