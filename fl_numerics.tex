\documentclass[preprint,10pt]{article}

\usepackage{amsfonts}
\usepackage{amsmath}
\usepackage{amsthm}
\usepackage[utf8]{inputenc}
\usepackage[english]{babel}
\usepackage{epstopdf}
\usepackage{bmpsize}
\usepackage{epsfig}
\usepackage{hyperref}
\usepackage[english]{varioref}
\usepackage{amssymb}
\usepackage{multicol}
\usepackage{dcolumn}
\usepackage{geometry}
\usepackage{fancyhdr}
\usepackage[mathcal]{eucal}
\usepackage{mathrsfs}
\usepackage{color, colortbl}
\usepackage{microtype}
\usepackage{longtable}
\usepackage[toc,page]{appendix}
\usepackage{bm}
\usepackage{pifont}
\usepackage{fleqn}
\usepackage{graphicx}
\usepackage{txfonts}
\usepackage{subfig}
\usepackage{fullpage}
%\usepackage{refcheck}
%\usepackage[pagewise]{lineno}\linenumbers
\usepackage[pagewise]{lineno}
\DeclareMathAlphabet{\mathbbm}{U}{bbm}{m}{n}% from bbm.sty
%
% fig4tex.tex  Version 1.9, November 14, 2011
% Copyright 2001-2011 Yvon Lafranche
%
% Author:  Yvon Lafranche
%          IRMAR, Universite de Rennes 1 - France
%          Yvon.Lafranche@univ-rennes1.fr
%
% Web Site: http://perso.univ-rennes1.fr/yvon.lafranche/fig4tex
%
%%%%%%%%%%%%%%%%%%%%%%%%%%%%%%%%%%%%%%%%%%%%%%%%%%%%%%%%%%%%%%%%%%%%%%%%%%%%%%%
\ifx\figforTeXisloaded\relax\endinput\else\global\let\figforTeXisloaded=\relax\fi
\message{version 1.9}
%
% This package has been tested with plain TeX and LaTeX on UNIX systems
% (including Linux and Mac OS X) and Windows.
%
% Uncomment the following line for TeXtures on MacOS 9:
%\let\TeXturesonMacOSltX=\relax
%%%%%%%%%%%%%%%%%%%%%%%%%%%%%%%%%%%%%%%%%%%%%%%%%%%%%%%%%%%%%%%%%%%%%%%%%%%%%%%
%
% OBJECT
%
% Fig4TeX is a set of TeX macros allowing to :
%  . create a figure as a graphical file at compilation time of the TeX document,
%  . write text on the figure under total control of the user.
% For details, see the documentation file fig4texDOC.pdf.
%
% PRINCIPLE
%
% With the help of adequate macros, the user defines some characteristic points.
% These points are used to build the figure which is stored in a so-called graphical
% file. Each of these points, and eventually other ones, can also be used as an
% attach point of a text printed on the figure. Thus, an interesting feature is to
% have the same geometric description both to draw the figure and to locate the text.
%
% USAGE
%
% The user macros are divided in two main groups:
% 1. Macros beginning with \figdraw are commands used to create the graphical file.
%    They have an effect only if a new graphical file is created.
% 2. Macros beginning with \fig can themselves be divided in two classes, so-called
%    mute and non-mute macros. Mute macros are mainly related to the definition of
%    the geometry. Non-mute macros are the only ones that write something on the
%    figure. Their names begin with \figwrite.
%
% EXAMPLE
%
% Put the following lines in a source file and compile it with TeX.
% According to the engine used, the output will be a DVI file or a PDF file.
%
%     \input fig4tex.tex
%% 1. Definition of characteristic points
%     \figinit{pt}
%     \figpt 2:(-10, -10)
%     \figpt 3:(130, 20)
%     \figpt 1:(80, 120)
%     \figptbary  5:c.g.[1,2,3 ; 1,1,1]
%% 2. Creation of the graphical file
%     \figdrawbegin{}
%     \figdrawtrimesh 2 [1,2,3]
%     \figdrawnormal 20,0.5 [3,1]
%     \figdrawend
%% 3. Writing text on the figure
%     \figvisu{\figBoxA}{Example}{%
%     \figwriten 1:(4)
%     \figwritew 2:(2)
%     \figwritee 3:(2)
%     \figptendnormal 4 :: 20,0.5 [3,1]\figwriten 4:$\vec{n}_2$(4)
%     \figset write(mark=$\bullet$)\figwrites 5:$G$(4)
%     }
%% Display the figure centered in the page
%     \centerline{\box\figBoxA}
%     \bye
%
%%%%%%%%%%%%%%%%%%%%%%%%%%%%%%%%%%%%%%%%%%%%%%%%%%%%%%%%%%%%%%%%%%%%%%%%%%%%%%%
%
%                      Geometrical macros - Mute macros
%                      --------------------------------
% --------- Control macros
% Initialization before the creation of a new figure. It is necessary to call this
% macro in case of successive figures. No check is performed on the arguments.
% First argument : Choice of the unit and the scale factor.
%  The unit can be one those defined in the TeX Book, namely: pt (TeX point),
%  pc (pica, 1pc = 12pt), in (inch, 1in = 72.27pt), bp (big point, 72bp = 1in),
%  cm (centimeter, 2.54cm = 1in), mm (millimeter, 10mm = 1cm), dd (point didot,
%  1157dd = 1238pt), cc (cicero, 1cc = 12dd), sp (scaled point, 65536sp = 1pt)
%  By default, pt is assumed and the scale factor is 1. For example, \figinit{in}
%  is equivalent to \figinit{2.54cm}.
% Second argument (optional) : Choice of the space dimension and, in 3D, the
%  projection type.
%  If this argument is equal to 2D or absent, then geometry in the plane is assumed,
%  otherwise geometry in three dimensions is performed.
%  Moreover, if this argument is equal to:
%   . orthogonal, then the orthogonal projection is used,
%   . realistic, then the realistic projection is used,
%   . 3D or cavalier (or anything else), then the cavalier projection is used.
%    \figinit{ScaleFactorUnit} or \figinit{ScaleFactorUnit, 2D}
%    \figinit{ScaleFactorUnit, X} with X in {3D, cavalier, orthogonal, realistic}
% Insertion of the file FileName in the page, scaled by ScaleFactor which by default
% equals to 1. This macro must be used to include a graphical file created by Fig4TeX.
% If the argument is empty, blank or numeric, the default file name is used, i.e. the
% included file is the last file created by the command \figdrawbegin{}. Otherwise,
% the first argument is assumed to be the valid file name of an existing file.
% If given, the file name must not begin with a digit.
% See also \figinsertE.
%    \figinsert{}         or \figinsert{ScaleFactor}
%    \figinsert{FileName} or \figinsert{FileName, ScaleFactor}
% Insertion of the external file FileName in the page, scaled by ScaleFactor
% which by default equals to 1. In PDF mode, this macro allows to include
% so-called "external" files such as JBIG2, JPEG, PDF or PNG files.
% In DVI mode, this macro allows to include PostScript files only.
%    \figinsertE{FileName} or \figinsertE{FileName, ScaleFactor}
% Draws a rectangular grid with horizontal and vertical steps HX and HY, with
% numerical values corresponding to the bounding box read in the file FileName.
%    \figscan FileName(HX,HY)
% Setting parameters acting on the appearance of the figure, which are to be given
% as pairs of the form attribute = value. For each keyword, the attributes are given
% below, along with the possibles values, their default value and their definition:
% - keyword = projection
%  . depth (or lambda) = real in [0,1] (default = 0.5): depth reduction coefficient
%    (for the cavalier projection).
%  . distance = real in user coordinates (default = 5 x largest dimension of
%    the 3D bounding box): observation distance (for the realistic projection).
%  . latitude (or theta) = angle in degrees (default = 25): latitude of the
%    observation direction.
%  . longitude (or psi) = angle in degrees (default = 40): longitude of the
%    observation direction.
%  . targetpt = point number (default = center of the 3D bounding box): target
%    point (for the realistic projection).
%  Nota: These attributes must be set before step 2 (\figdrawbegin) and 3 (\figvisu).
%        The cavalier projection is only defined by depth and psi. In this case,
%        values of psi lying in [0,180] correspond to a view from above and to
%        a view from beneath for values lying in [-180,0].
%        The orthogonal projection is only defined by theta and psi.
%        The realistic projection is defined by theta, psi, distance and targetpt.
%        For those two projections, the observation direction is defined by the
%        longitude psi and the latitude theta, with (psi, theta) = (0, 0)
%        corresponding to the Ox line.
%    \figset projection (attribute1=value1, attribute2=value2,...)
%    \figset keyword (attribute1=value1, attribute2=value2,...)
% Prints to the terminal the current settings.
%    \figshowsettings
% Creation of a Vbox containing the figure and the legends defined by the Commands,
% with a Caption centered below. Commands and Caption can be void.
% Three boxes are preallocated and are available to the user: \figBoxA, \figBoxB
% and \figBoxC (a single one is generally sufficient for a whole document since
% it can be reused several times).
% The numbers of the current figure and of the next one are also available via
% the expandable macros \figforTeXFigno and \figforTeXnextFigno.
%    \figvisu{Vbox}{Caption}{Commands}
%
% --------- Basic macros
% Definition of the point NewPt whose coordinates are (X,Y) or (X,Y,Z), with a joined Text.
% If Z is missing, then Z=0 is assumed. The default Text is $A_i$ where i is the number of
% the point (see \figset write(ptname={...})).
%    \figpt NewPt :Text(X,Y)
%    \figpt NewPt :Text(X,Y,Z)
% Barycenter or centroid (with a joined Text) of N points bearing integer coefficients
%    \figptbary NewPt :Text[Pt1,... ,PtN ; Coef1,... ,CoefN]
% Barycenter or centroid (with a joined Text) of N points bearing real coefficients
%    \figptbaryR NewPt :Text[Pt1,... ,PtN ; Coef1,... ,CoefN]
% Definition of NewPt, with an associated Text, with the same coordinates as Pt.
%    \figptcopy NewPt :Text/Pt/
% Computation of geometrical parameters specified by a keyword.
% The general calling sequence is
%         \figget keyword = \Result [Arg1,...ArgN]
% where \Result is a macro that holds the expected result and Argi are the arguments
% needed to compute this value. \Result is a macro whose name is chosen by the user and
% can then be used as a symbolic numerical constant.
% For each keyword, the definition of the associated parameter is given below along with
% the calling sequence:
% - keyword = angle
%   Calling sequence in 2D: \figget angle = \Result [Center,Pt1,Pt2]
%   Calling sequence in 3D: \figget angle = \Result [Center,Pt1,Pt2,Pt3]
%   Computation of the value \Result (in degrees) of the oriented angle (CP1, CP2), where
%   C=Center, P1=Pt1, P2=Pt2.
%   In 3D, the plane is oriented so that the angle is measured counterclockwise around the
%   vector CP1 x CP3, where P3=Pt3. Notice that C, P1, P2 and P3 must be coplanar, but P3
%   must not lie on the line (C,P1).
% - keyword = distance
%   Calling sequence: \figget distance = \Result [Pt1,Pt2]
%   Computation of the euclidian distance \Result between the points Pt1 and Pt2.
%    \figget keyword = \Result [Arg1,...ArgN]
% Definition of the vector NewVect with components (X,Y) or (X,Y,Z).
% If Z is missing, then Z=0 is assumed.
%    \figvectC NewVect (X,Y)
%    \figvectC NewVect (X,Y,Z)
% Definition of the vector NewVect which is:
% . in 2D, the vector with origin point Pt1 and end point Pt2 rotated by pi/2
%   (so normal to [Pt1,Pt2]),
% . in 3D, the vector product P1P2 / ||P1P2|| x P1P3 / ||P1P3||, ie a vector normal to
%   the plane defined by the 3 points and so that (P1P2, P1P3, NewVect) is a positively
%   oriented basis.
%    \figvectN NewVect [Pt1,Pt2]
%    \figvectN NewVect [Pt1,Pt2,Pt3]
% Definition of the vector NewVect which is:
% . in 2D, the vector Vector rotated by pi/2
% . in 3D, the vector product Vector1 / ||Vector1|| x Vector2 / ||Vector2|| so that
% (Vector1, Vector2, NewVect) is a positively oriented basis.
%    \figvectNV NewVect [Vector]
%    \figvectNV NewVect [Vector1, Vector2]
% Definition of the vector NewVect with origin point Pt1 and end point Pt2.
%    \figvectP NewVect [Pt1,Pt2]
% Definition of the unitary vector NewVect which is Vector normalized according to
% the unit and the scale factor chosen by the user (see \figinit).
%    \figvectU NewVect [Vector]
%
% --------- Macros for elementary geometry
% --- Transformation macros (one result) ---
% Image NewPt (with a joined Text) of point Pt by the mapping defined by:
%  . an invariant point InvPt
%  . the matrix A whose coefficients are Aij, i,j in [1,d], d=2 in 2D or d=3 in 3D.
% If I=InvPt, P=Pt, P'=NewPt, P' is computed as IP' = A (IP).
%    \figptmap NewPt :Text= Pt /InvPt/A11,A12 ; A21,A22/
%    \figptmap NewPt :Text= Pt /InvPt/A11,A12,A13 ; A21,A22,A23 ; A31,A32,A33/
% Image NewPt of point Pt by the homothety of center Center and of ratio Ratio
% with a joined Text
%    \figpthom NewPt :Text= Pt /Center, Ratio/
% Image NewPt of point Pt by the inversion of center Center and of ratio Ratio
% with a joined Text
%    \figptinv NewPt :Text= Pt /Center, Ratio/
% Image NewPt (with a joined Text) of point Pt by the rotation defined:
% . in 2D, by the center Center and the angle Angle (in degrees),
% . in 3D, by the axis (Center, Vector) and the angle Angle (in degrees).
%   Pt is rotated by Angle around the axis. The sense of rotation is such that
%   (CP, CP', Vector) is a positively oriented basis, where C=Center, P=Pt, P'=NewPt.
%    \figptrot NewPt :Text= Pt /Center, Angle/
%    \figptrot NewPt :Text= Pt /Center, Angle, Vector/
% Image NewPt (with a joined Text) of point Pt by the orthogonal symmetry
% with respect to:
% . in 2D, the line defined by the points LinePt1 and LinePt2,
% . in 3D, the plane defined by the point PlanePt and the vector NormalVector normal
% to the plane.
%    \figptsym NewPt :Text= Pt /LinePt1, LinePt2/
%    \figptsym NewPt :Text= Pt /PlanePt, NormalVector/
% Image NewPt of point Pt by the translation of vector Lambda * Vector
% with a joined Text.
%    \figpttra NewPt :Text= Pt /Lambda, Vector/
% Image NewPt of point Pt by the translation of vector (X,Y) in 2D, (X,Y,Z) in 3D
% with a joined Text.
%    \figpttraC NewPt :Text= Pt /X,Y/
%    \figpttraC NewPt :Text= Pt /X,Y,Z/
%
% Image NewPt of point Pt by the orthogonal projection onto the line (LinePt1,LinePt2),
% with a joined Text.
%    \figptorthoprojline NewPt :Text= Pt /LinePt1, LinePt2/
% Image NewPt of point Pt by the orthogonal projection onto the plane defined by
% the point PlanePt and the normal vector NormalVector, with a joined Text.
%    \figptorthoprojplane NewPt :Text= Pt /PlanePt, NormalVector/
%
% --- Transformation macros (multiple result) ---
% Images NewPt1, NewPt1+1,..., NewPt1+(N-1) of points Pt1, Pt2, ..., PtN
% by the mapping defined by:
%  . an invariant point InvPt
%  . the matrix A whose coefficients are Aij, i,j in [1,d], d=2 in 2D or d=3 in 3D.
% If I=InvPt, P=Pti, P'=NewPti, P' is computed as IP' = A (IP).
% See following information at \figptstra.
%    \figptsmap NewPt1 = Pt1, Pt2, ..., PtN /InvPt/A11,A12 ; A21,A22/
%    \figptsmap NewPt1 = Pt1, Pt2, ..., PtN /InvPt/A11,A12,A13; A21,A22,A23; A31,A32,A33/
% Images NewPt1, NewPt1+1,..., NewPt1+(N-1) of points Pt1, Pt2, ..., PtN
% by the homothety of center Center and of ratio Ratio.
% See following information at \figptstra.
%    \figptshom NewPt1 = Pt1, Pt2, ..., PtN /Center, Ratio/
% Images NewPt1, NewPt1+1,..., NewPt1+(N-1) of points Pt1, Pt2, ..., PtN
% by the rotation defined:
% . in 2D, by the center Center and the angle Angle (in degrees),
% . in 3D, by the axis (Center, Vector) and the angle Angle (in degrees). Each point
%   Pti is rotated by Angle around the axis. The sense of rotation is such that
%   (CP, CP', Vector) is a positively oriented basis, where C=Center, P=Pti, P'=NewPti.
% See following information at \figptstra.
%    \figptsrot NewPt1 = Pt1, Pt2, ..., PtN /Center, Angle/
%    \figptsrot NewPt1 = Pt1, Pt2, ..., PtN /Center, Angle, Vector/
% Images NewPt1, NewPt1+1,..., NewPt1+(N-1) of points Pt1, Pt2, ..., PtN
% by the orthogonal symmetry with respect to:
% . in 2D, the line defined by the points LinePt1 and LinePt2,
% . in 3D, the plane defined by the point PlanePt and the vector NormalVector normal
% to the plane. See following information at \figptstra.
%    \figptssym NewPt1 = Pt1, Pt2, ..., PtN /LinePt1, LinePt2/
%    \figptssym NewPt1 = Pt1, Pt2, ..., PtN /PlanePt, NormalVector/
% Images NewPt1, NewPt1+1,..., NewPt1+(N-1) of points Pt1, Pt2, ..., PtN
% by the translation of vector Lambda * Vector
% Text eventually previously associated with the result points is lost.
% The result points NewPti have successive numbers.
% The data points Pti can be given in any order.
% WARNING : Let Ir (resp. Id) the set of the result points numbers (resp. the set
% of the given points numbers). Let J the intersection of Ir and Id.
% 1) Ir = Id, or J is empty, or NewPt1 does not belong to J: no problem.
% 2) Otherwise, the result given by the macro MAY BE WRONG, at least partially:
%    this is because, in this case, NewPt1 belongs to J, and the given points are
%    taken into account sequentially, beginning from the first element in the list.
%    \figptstra NewPt1 = Pt1, Pt2, ..., PtN /Lambda, Vector/
%
% Images NewPt1, NewPt1+1,..., NewPt1+(N-1) of points Pt1, Pt2, ..., PtN
% by the orthogonal projection onto the line (LinePt1, LinePt2).
%    \figptsorthoprojline NewPt1 = Pt1, Pt2, ..., PtN /LinePt1, LinePt2/
% Images NewPt1, NewPt1+1,..., NewPt1+(N-1) of points Pt1, Pt2, ..., PtN
% by the orthogonal projection onto the plane defined by the point PlanePt and the
% normal vector NormalVector.
%    \figptsorthoprojplane NewPt1 = Pt1, Pt2, ..., PtN /PlanePt, NormalVector/
%
% --- Geometrical construction macros ---
% Intersection of the line defined by the point LinePt1 and the vector Vector1
%             and the line defined by the point LinePt2 and the vector Vector2
% with a joined Text
%    \figptinterlines NewPt :Text[LinePt1,Vector1; LinePt2,Vector2]
% Intersection of the line defined by the point LinePt and the vector Vector
%             and the plane defined by the point PlanePt and the normal vector NormalVector
% with a joined Text.
%    \figptinterlineplane NewPt :Text[LinePt,Vector; PlanePt,NormalVector]
% End points NewPt1, NewPt1+1 (and NewPt1+2 in 3D) of the arrows corresponding to
% the axes, drawn by the macro \figdrawaxes. The text $x$, $y$ (and $z$ in 3D) is
% automatically joined to the points, so that writing the legend with the
% \figwriteX macros is more straightforward. For example, in 2D, one can write
% something like:  \figwrites NewPt1:(3pt) \figwritew NewPt1+1:(3pt)
% Remember that these macros can override the default text setting.
% The short form \figptsaxes NewPt1 : Origin(L) is equivalent to
% \figptsaxes NewPt1 : Origin(0,L, 0,L) in 2D and
% \figptsaxes NewPt1 : Origin(0,L, 0,L, 0,L) in 3D.
% NOTA: For simplicity of use, the arguments of this macro are deduced from
%       those of \figdrawaxes.
%    \figptsaxes NewPt1 : Origin(X1,X2, Y1,Y2)
%    \figptsaxes NewPt1 : Origin(X1,X2, Y1,Y2, Z1,Z2)
%    \figptsaxes NewPt1 : Origin(L)
% End point (with a joined Text) of the "exterior normal" to the segment [Pt1, Pt2].
% The length of the normal vector is Length. Lambda is the barycentric coordinate
% of the origin of the normal with respect to the segment [Pt1, Pt2], which sets the
% position of the vector along [Pt1, Pt2]. In 3D-mode, it works only in the plane Z=0.
%    \figptendnormal NewPt :Text: Length,Lambda [Pt1,Pt2]
% Intersections NewPt1 and NewPt2 (=NewPt1+1) of the two circles defined by their
% center and radius, (Center1,Radius1) and (Center2,Radius2).
% NewPt1 and NewPt2 must be different from Center1 and Center2.
% NewPt1 and NewPt2 are ordered so that the angle (NewPt1, Center1, NewPt2) is positive.
% If the two circles do not intersect, then NewPt1=Center1 and NewPt2=Center2.
% In 3D, the plane containing the two circles is defined by the three points Center1,
% Center2 and PlanePt (which must not lie on the same line). The three points
% (NewPt1, Center1, NewPt2) defined the same orientation as (Center2, Center1, PlanePt).
%    \figptsintercirc NewPt1 [Center1,Radius1 ; Center2,Radius2]
%    \figptsintercirc NewPt1 [Center1,Radius1 ; Center2,Radius2 ; PlanePt]
% Intersections NewPt1 and NewPt2 (=NewPt1+1) of the line (LinePt1, LinePt2) and
% the ellipse defined by Center, XRad, YRad and Inclination (see \figptell).
% NewPt1 and NewPt2 must be different from LinePt1.
% NewPt1 and NewPt2 are ordered so that the vectors NewPt1NewPt2 and LinePt1LinePt2
% have the same direction.
% If the line does not intersect the ellipse, then NewPt1=LinePt1 and NewPt2=LinePt2.
% In 3D-mode, it works only in the plane Z=0.
%    \figptsinterlinell NewPt1 [Center,XRad,YRad,Inclination ; LinePt1,LinePt2]
% Intersections NewPt1 and NewPt2 (=NewPt1+1) of the line (LinePt1, LinePt2) and
% the ellipse defined by its Center, and the end points of its axes, A1=PtAxis1 and
% A2=PtAxis2. The local axes are then defined by the basis (CA1, CA2).
% NewPt1 and NewPt2 must be different from LinePt1.
% NewPt1 and NewPt2 are ordered so that the vectors NewPt1NewPt2 and LinePt1LinePt2
% have the same direction.
% If the line does not intersect the ellipse, then NewPt1=LinePt1 and NewPt2=LinePt2.
% In 3D, the 5 data points must lie in the same plane ; this is not checked.
%    \figptsinterlinellP NewPt1 [Center,PtAxis1,PtAxis2 ; LinePt1,LinePt2]
% Apparent intersection NewPt (with a joined Text) of the segment [SegPt1,SegPt2]
% and the line (LinePt1,LinePt2). NewPt is the visible limit of the segment hidden
% by an object an edge of which is lying on the line.
% If the projection of the line intersects the segment, the solution NewPt lies
% between SegPt1 and SegPt2, otherwise NewPt = SegPt1.
%    \figptvisilimSL NewPt:Text[SegPt1,SegPt2 ; LinePt1,LinePt2]
%
% --- Macros related to the triangle ---
% Center NewPt of the circumscribed circle to the triangle (Pt1,Pt2,Pt3) with a joined Text.
%    \figptcircumcenter NewPt :Text[Pt1,Pt2,Pt3]
% Center NewPt (with a joined Text) of the escribed circle to the triangle (Pt1,Pt2,Pt3)
% opposite to Pt1.
%    \figptexcenter NewPt :Text[Pt1,Pt2,Pt3]
% Center NewPt (with a joined Text) of the inscribed circle to the triangle (Pt1,Pt2,Pt3).
%    \figptincenter NewPt :Text[Pt1,Pt2,Pt3]
% Orthocenter NewPt of the triangle (Pt1,Pt2,Pt3) with a joined Text.
%    \figptorthocenter NewPt :Text[Pt1,Pt2,Pt3]
%
% --------- Macros related to arcs and curves
% Creation of the point NewPt, with an associated Text, on the circle defined by its Center,
% and its Radius. The position of the point is set by the Angle given in degrees.
% In 3D, the circle is lying in the plane (Center,Pt1,Pt2) = (C,P1,P2) and
% the Angle is measured counterclockwise around the vector CP1 x CP2, starting from
% the half-line (C,P1).
%    \figptcirc NewPt :Text: Center;Radius (Angle)
%    \figptcirc NewPt :Text: Center,Pt1,Pt2;Radius (Angle)
% Creation of the point NewPt, with an associated Text, on the ellipse defined by Center,
% XRad, YRad and Inclination. Inclination is the rotation angle of the local axes with
% respect to the paper sheet. The position of the point is set by the parametrization
% Angle given in local axes. The two angles are given in degrees.
% In 3D-mode, it works only in the plane Z=0.
%    \figptell NewPt :Text: Center;XRad,YRad (Angle,Inclination)
% Creation of the point NewPt, with an associated Text, on the ellipse defined by its Center,
% and the end points of its axes, A1=PtAxis1 and A2=PtAxis2. The local axes are then
% defined by the basis (CA1, CA2). The position of the point is set by the parametrization
% Angle given in degrees, starting from the half-line (C,A1) and measured counterclockwise
% around the vector CA1 x CA2.
%    \figptellP NewPt :Text: Center,PtAxis1,PtAxis2 (Angle)
% Computes the point NewPt (with a joined Text) lying on the cubic Bezier curve defined
% by the four control points Pt1, Pt2, Pt3 and Pt4, for the parameter value t.
% We recall that if t=0, NewPt=Pt1 and if t=1, NewPt=Pt4.
%    \figptBezier NewPt :Text: t [Pt1,Pt2,Pt3,Pt4]
% Computes the two control points NewPt1 and NewPt2 (=NewPt1+1) so that the cubic Bezier
% curve defined by the control points Pt1, NewPt1, NewPt2 and Pt4 interpolates the
% four points Pt1, Pt2, Pt3 and Pt4 with respective parameter values of 0, 1/3, 2/3 and 1.
% The result points numbers can be anyone including Pt1, Pt2, Pt3 and Pt4.
%    \figptscontrol NewPt1 [Pt1,Pt2,Pt3,Pt4]
% Computes the control points NewPt1, NewPt1+1,..., NewPt1+M used by the macro
% \figdrawcurve when it is invoqued by  \figdrawcurve [Pt0,Pt1,... ,PtN,PtN+1]
% which draws a curve that consists in N-1 cubic Bezier arcs.
% \NbArcs is a TeX macro whose name is chosen by the user. It can be considered
% as a variable and on output, its value is N-1 so that the calling sequence
%  \figdrawBezier \NbArcs [NewPt1, NewPt1+1,..., NewPt1+M] draws exactly the same curve.
% As a consequence, the data points Pti are duplicated on output : indeed, we have
% NewPt1+J = Pti with J=3*(i-1), i=1,...N.
% The result points numbers must be different from any of Pt0,Pt1,... ,PtN,PtN+1.
% On output, a total of M+1 points are created, with M = 3 * \NbArcs. 
% NOTA : this macro uses the current value of curve roudness.
%    \figptscontrolcurve NewPt1, \NbArcs [Pt0,Pt1,... ,PtN,PtN+1]
% Curvature center NewPt (with a joined Text) at the point lying on the cubic Bezier curve
% defined by the four control points Pt1, Pt2, Pt3 and Pt4, for the parameter value t.
%    \figptcurvcenter NewPt :Text: t [Pt1,Pt2,Pt3,Pt4]
% Computes the vector NewVect corresponding to the derivative of order n of the cubic Bezier
% curve defined by the four control points Pt1, Pt2, Pt3 and Pt4, for the parameter value t.
% The order n must be equal to 1 or 2.
%    \figvectDBezier NewVect : n, t [Pt1,Pt2,Pt3,Pt4]
%
%
%                      Writing macros - Non-mute macros
%                      --------------------------------
% --------- Control macros
% To write the coordinates of a point. To be used in the joined text argument
% of the macros that create points and the non-mute macros \figwrit* .
% Only NDec decimals are printed. See also \figsetroundcoord.
%    \figcoord{NDec}

% Setting the attributes the \figwriteXX macros depend on.
% The general calling sequence is
%         \figset keyword (attribute1=value1, attribute2=value2,...)
% Here, the required keyword is "write". See the macro \figset in the section
% "Macros for graphical generation" for the other keywords. The attributes are
% given below, along with the possible values and their definition.
% The current values can be obtained with the help of \figshowsettings.
% - keyword = write
%  . mark = symbol : controls how the point position is shown ; the point marker
%                    to be written can be a point (.) or $\bullet$ for example.
%                    By default, nothing is written.
%  . ptname = text : controls the definition of the point name ; the default name
%                    for the point i is $A_i$: \figset write(ptname={$X^{(#1)}$})
%                    changes it to $X^{(i)}$ for example.
%  . roundcoord = yes/no : switches on or off the rounding of decimals printed
%                          with \figcoord. The initial value is yes.
%    \figset write(attribute1=value1, attribute2=value2,...)
% Shows on the figure the location of every point defined since the beginning
% of the session, whose number lies in the interval [Nmin, Nmax].
% Writes a bullet at each location point along with the number of the point or
% the joined text if any.
% CAUTION :
% If Nmax-Nmin is too large, an error message "! TeX capacity exceeded" may occur.
%    \figshowpts[Nmin, Nmax]
% --------- Writing macros
% Writing a Text after Pt1, Pt2, ..., PtN according to TeX's alignment.
%    \figwrite[Pt1, Pt2, ..., PtN]{Text}
% Writing a Text vertically and horizontally centered at Pt1, Pt2, ..., PtN.
%    \figwritec[Pt1, Pt2, ..., PtN]{Text}
% Writing the point marker at the locations defined by Pt1, Pt2, ..., PtN.
%    \figwritep[Pt1, Pt2, ..., PtN]
% Writing the point marker at the locations defined by Pt1, Pt2, ..., PtN, with a Text
% placed, with respect to each point, at a given Distance from each point towards
% the west, the east, the north or the south. Distance measures the shortest path
% between each Pti and the bounding box of the Text.
%    \figwritew Pt1, Pt2, ..., PtN :Text(Distance)
%    \figwritee Pt1, Pt2, ..., PtN :Text(Distance)
%    \figwriten Pt1, Pt2, ..., PtN :Text(Distance)
%    \figwrites Pt1, Pt2, ..., PtN :Text(Distance)
% Writing the point marker at the locations defined by Pt1, Pt2, ..., PtN, with a Text
% placed, with respect to each point, at a given Distance from each point towards
% the north-west, the south-west, the north-east or the south-east. Distance measures
% the shortest path between each Pti and the bounding box of the Text.
%    \figwritenw Pt1, Pt2, ..., PtN :Text(Distance)
%    \figwritesw Pt1, Pt2, ..., PtN :Text(Distance)
%    \figwritene Pt1, Pt2, ..., PtN :Text(Distance)
%    \figwritese Pt1, Pt2, ..., PtN :Text(Distance)
% Writing a point marker at the locations defined by Pt1, Pt2, ..., PtN, with a Text
% placed, with respect to each point, at a given Distance from each point towards
% the west, the east, the north or the south (BASELINE version).
% If X=w or e, Distance measures the length between Pti and the end (resp. the beginning)
% of the Text, while the baseline of the Text is set to Pti's ordinate.
% If X=n or s, the Text is horizontally centered and Distance measures the length between
% Pti and the baseline of the Text.
%    \figwritebw Pt1, Pt2, ..., PtN :Text(Distance)
%    \figwritebe Pt1, Pt2, ..., PtN :Text(Distance)
%    \figwritebn Pt1, Pt2, ..., PtN :Text(Distance)
%    \figwritebs Pt1, Pt2, ..., PtN :Text(Distance)
% Writing the point marker at the locations defined by Pt1, Pt2, ..., PtN, with a Text
% placed, with respect to each point, at a horizontal distance DistanceX (west or east)
% and a vertical distance DistanceY between the point and the mid-height of the text.
%    \figwritegcw Pt1, Pt2, ..., PtN :Text(DistanceX,DistanceY)
%    \figwritegce Pt1, Pt2, ..., PtN :Text(DistanceX,DistanceY)
% Writing the point marker at the locations defined by Pt1, Pt2, ..., PtN, with a Text
% placed, with respect to each point, at a horizontal distance DistanceX (west or east)
% and a vertical distance DistanceY from the bottom of the bounding box of the Text if
% DistanceY > 0, from its top if DistanceY < 0. If DistanceY = 0, the Text is vertically
% centered.
%    \figwritegw Pt1, Pt2, ..., PtN :Text(DistanceX,DistanceY)
%    \figwritege Pt1, Pt2, ..., PtN :Text(DistanceX,DistanceY)
%
%                      Macros for graphical generation
%                      -------------------------------
% --------- Control macros
% Starting the creation of a graphical file whose name is FileName.
% If the argument FileName is empty or blank, an internal default file name is used,
% which is always updated, regardless the update mode.
% See \figinsert for details.
%    \figdrawbegin{} or \figdrawbegin{FileName}
% End of the current graphical file.
%    \figdrawend
% To reset one or several groups of graphical attributes to their default values.
% The groups of attributes are denoted with keywords which are:
%  - NO keyword or keyword = general to reset the general (or top-level) attributes,
%  - altitude to reset the altitude attributes,
%  - arrowhead to reset the arrowhead attributes,
%  - curve to reset the (\figdraw)curve attributes,
%  - general to reset the general (or top-level) attributes,
%  - flowchart to reset the flow chart attributes,
%  - mesh to reset the (\figdraw)mesh attributes,
%  - trimesh to reset the (\figdraw)trimesh attributes.
%    \figreset{keyword1, keyword2,...}
% Setting graphical attributes or parameters acting on the appearance of the figure,
% which are to be given as pairs of the form  attribute = value.
% The general calling sequence is
%         \figset keyword (attribute1=value1, attribute2=value2,...)
% For each keyword, the attributes are given below, along with the possible values
% and their definition.
% The current values can be obtained with the help of \figshowsettings.
% - NO keyword or keyword = general
%  . color = color definition : line or area color in cmyk or rgb color code,
%    or in gray tone (real number between 0 (black) and 1 (white).
%  . dash = index or dash pattern : line style given by Index (Index = 1 to 10,
%    1 meaning "solid line") or by Pattern.
%  . fillmode = yes/no : setting the filling mode to "no" tells the drawing macros
%    to draw lines ; setting the filling mode to "yes" tells the macros to fill
%    the appropriate area using the current color or gray shade.
%  . join = miter (V), round (U) or bevel (\_/) : line join parameter that establishes
%    the shape to be put at the corners of a polygonal line. Best rendering is
%    obtained with join=round when more than 2 segments meet at a corner.
%  . update = yes/no : setting the update mode to "yes" (resp. to "no") before
%    \figdrawbegin enforces (resp. avoids) the graphical file to be recreated at
%    each compilation.
%  . width = dimension in PostScript units : line width.
%    Note : has no effect on a straight line after \figset (fillmode=yes).
% - keyword = altitude
%  All attributes are DDV (they have a dynamic default value).
%  The following attributes affect the aspect of the extension of the base line drawn
%  when the foot of the altitude lies outside the triangle:
%  . blcolor = color definition : controls the color of the base line,
%  . bldash = index or dash pattern : controls the style of the base line,
%  . blwidth = dimension in PostScript units : controls the width of the base line.
%  The following attributes affect the aspect of the square of the altitude:
%  . sqcolor = color definition : controls the color of the square,
%  . sqdash = index or dash pattern : controls the style of the square,
%  . sqwidth = dimension in PostScript units : controls the width of the square.
% - keyword = arrowhead
%  . angle = real in degrees : arrowhead opening half-angle.
%  . fillmode = yes/no : arrowhead filling switch, which tells whether the interior
%    of the arrowhead must be filled or not.
%  . length = real in user coordinates : length of each edge of the arrowhead.
%  . out = yes/no : arrowhead "outside" switch, which tells whether the arrowhead
%    must be drawn outside the body of the arrow (as if it was rotated by 180 degrees
%    around the end point) or not.
%  . ratio = real in [0,1] : arrowhead ratio defining the arrowhead length
%    to be ratio x (body length).
%   Nota : The two attributes length and ratio are mutually exclusive ; the default is
%          to use the length.
% - keyword = curve
%  . roundness = real usually in [0,0.5] : roundness of a C1 curve.
% - keyword = flowchart
%  . arrowposition = real in [0,1] : arrowhead position along the path,
%    0 corresponding to the beginning and 1 to the end of the path.
%  . arrowrefpt = start/end : reference point of the arrowhead to its start point or
%    to its end point ; the arrowhead position is computed with respect to this point.
%  . bgcolor = color definition : controls the color of the frame background.
%  . line = polygon/curve : tells \figdrawfcconnect to draw the path between two nodes
%    using polygonal lines or smooth curved lines.
%  . padding = real in user coordinates (see xpadding and ypadding): space surrounding
%    the text inside a frame. Both horizontal and vertical padding are set to the
%    same value. This attribute allows to reduce or enlarge the frames drawn at the
%    nodes of the flow chart, starting from the internal default dimensions.
%  . radius = positive real in user coordinates : radius of the circular arcs to be
%    drawn at the corners of the path when the line is polygonal.
%  . shape = rectangle, ellipse or lozenge : shape of the frames.
%  . thickcolor = color definition : controls the color of the thickness of the frames.
%  . thickness = real in user coordinates : thickness of the frames.
%  . xpadding = real in user coordinates : horizontal space around the text inside a frame.
%  . ypadding = real in user coordinates : vertical space around the text inside a frame.
%  Notice that the frame border color is the general color.
% - keyword = mesh
%  . diag = integer in {-1,0,1} : controls the drawing of a grid mesh.
%    If diag=0, each cell is empty ; if diag=1 (resp. diag=-1), the SW-NE diagonal (resp.
%    the NW-SE diagonal) is drawn inside each cell.
%  The following DDV attributes affect the aspect of the lines used to draw the mesh
%  inside the quadrangle:
%  . color = color definition : controls the color of the lines,
%  . dash = index or dash pattern : controls the style of the lines,
%  . width = dimension in PostScript units : controls the width of the lines.
% - keyword = projection : see "Control macros" in section "Geometrical macros - Mute macros".
% - keyword = trimesh
%  The following DDV attributes affect the aspect of the lines used to draw the mesh
%  inside the triangle:
%  . color = color definition : controls the color of the lines,
%  . dash = index or dash pattern : controls the style of the lines,
%  . width = dimension in PostScript units : controls the width of the lines.
% - keyword = write : see "Control macros" in section "Writing macros - Non-mute macros".
%    \figset keyword (attribute1=value1, attribute2=value2,...)
% Global setting of the default values of graphical attributes, which are to be given
% as pairs of the form attribute=NewDefaultValue.
% The attributes are set up to the end of the document or up to next call to this macro.
% For each keyword, the attributes are listed below ; see \figset for their definition
% (particular case : \figsetdefault arrowhead(length=DimWithUnit)).
% - NO keyword or keyword = general: color, dash, fillmode, join, update, width,
% - keyword = arrowhead: angle, fillmode, length, out, ratio,
% - keyword = curve: roundness,
% - keyword = flowchart: arrowposition, arrowrefpt, bgcolor, line, padding, shape,
%             thickness, xpadding, ypadding,
% - keyword = mesh: diag.
%    \figsetdefault keyword (attribute1=value1, attribute2=value2,...)
%
% --------- Basic drawing macros
% Circle of center Center and of radius Radius.
% In 3D, the circle is in the plane defined the 3 points Center, Pt1 and Pt2 ;
% Pt1 and Pt2 do not need to lie on the circle.
%    \figdrawcirc Center (Radius)
%    \figdrawcirc Center,Pt1,Pt2 (Radius)
% Line defined by N points, closed if the last point number equals the first one.
%    \figdrawline[Pt1,Pt2,... ,PtN]
% Line defined by N points, closed if the last point number equals the first one.
% The points are given by their coordinates, each of them separated by a white space.
%    \figdrawlineC(X1 Y1, X2 Y2,..., XN YN)
%    \figdrawlineC(X1 Y1 Z1, X2 Y2 Z2,..., XN YN ZN)
% Line defined by points read from the text file Filename, which contains the
% coordinates of the points, one point per line, given as  X Y  in 2D and as
% X Y Z  in 3D (the values must be separated by a white space and nothing else).
% The line is closed if the last point equals the first one.
%    \figdrawlineF{Filename}
%
% --------- Other drawing macros
% --- Arc ---
% Circular arc of center Center and radius Radius limited by the angles
% Ang1 and Ang2 given in degrees.
% In 2D, the angles are measured counterclockwise.
% In 3D, the arc lies in the plane defined by the 3 points Center, Pt1 and Pt2.
% The angles are measured counterclockwise around the vector CP1 x CP2, starting
% from the half-line (C, P1), where C=Center, P1=Pt1, P2=Pt2.
%    \figdrawarccirc Center ; Radius (Ang1,Ang2)
%    \figdrawarccirc Center,Pt1,Pt2 ; Radius (Ang1,Ang2)
% Point version of \figdrawarccirc
% Circular arc of center Center and of radius Radius limited by the two half-lines
% (Center, Pt1) and (Center, Pt2).
% Let N be the normal vector that orients the plane in which lies the arc.
% In 2D, N is defined as usual by N = Z = X x Y.
% In 3D, N = CP1 x CP3, where C=Center, P1=Pt1, P3=Pt3.
% The arc is drawn from Pt1 towards Pt2 turning counterclockwise around the axis (C; N).
% Notice that C, P1, P2 and P3 must be coplanar, but P3 must not lie on the line (C,P1).
%    \figdrawarccircP Center ; Radius [Pt1,Pt2]
%    \figdrawarccircP Center ; Radius [Pt1,Pt2,Pt3]
% Arc of ellipse of center Center, of axes XRad and YRad, limited by the parametrization
% angles Ang1 and Ang2. The major axis is turned by an angle Inclination from the
% horizontal line. The angles are given in degrees and measured counterclockwise.
% In 3D-mode, it works only in the plane Z=0.
%    \figdrawarcell Center ; XRad,YRad (Ang1,Ang2, Inclination)
% Arc of ellipse of center Center limited by the parametrization angles Ang1 and Ang2.
% The end point of the major axis is PtAxis1 and the end point of the minor axis is PtAxis2.
% The angles are given in degrees and measured counterclockwise around the vector CA1 x CA2.
%    \figdrawarcellPA Center,PtAxis1,PtAxis2 (Ang1, Ang2)
% Arc of ellipse of center Center limited by the two half-lines (Center,Pt1)
% and (Center,Pt2). The end point of the major axis is PtAxis1 and the end point
% of the minor axis is PtAxis2. The arc is drawn counterclockwise around the vector
% CA1 x CA2, where C=Center, A1=PtAxis1, A2=PtAxis2.
%    \figdrawarcellPP Center,PtAxis1,PtAxis2 [Pt1,Pt2]
% --- Arrow ---
% Arrow from Pt1 to Pt2.
% The arrowhead is drawn according to the \figdrawarrowhead macro settings.
%    \figdrawarrow [Pt1,Pt2]
% Arrow that consists of the cubic Bezier curve defined by the four control points
% Pt1, Pt2, Pt3 and Pt4, and the arrowhead at point Pt4.
%    \figdrawarrowBezier [Pt1,Pt2,Pt3,Pt4]
% Circular arrow such that the circular arc is centered at Center, has the radius Radius
% and is limited by the angles Ang1 and Ang2 given in degrees.
% If Ang2 > Ang1, the arrow is drawn counterclockwise, else it is drawn clockwise.
% The arrowhead is drawn according to the \figdrawarrowhead macro settings.
% In 3D, the arc lies in the plane defined the 3 points Center, Pt1 and Pt2.
% The angles are measured counterclockwise around the vector CP1 x CP2, starting
% from the half-line (C, P1), where C=Center, P1=Pt1, P2=Pt2.
%    \figdrawarrowcirc Center ; Radius (Ang1,Ang2)
%    \figdrawarrowcirc Center,Pt1,Pt2 ; Radius (Ang1,Ang2)
% Point version of \figdrawarrowcirc
% Circular arrow such that the circular arc is centered at Center, has the radius
% |Radius| and is limited by the two half-lines (Center, Pt1) and (Center, Pt2).
% Let N be the normal vector that orients the plane in which lies the arc.
% In 2D, N is defined as usual by N = Z = X x Y.
% In 3D, N = CP1 x CP3, where C=Center, P1=Pt1, P3=Pt3.
% The arrow is drawn from Pt1 towards Pt2 turning around the axis (Center; N):
%  . counterclockwise if Radius > 0,
%  . clockwise if Radius < 0.
% Notice that C, P1, P2 and P3 must be coplanar, but P3 must not lie on the line (C,P1).
%    \figdrawarrowcircP Center ; Radius [Pt1,Pt2]
%    \figdrawarrowcircP Center ; Radius [Pt1,Pt2,Pt3]
% Arrowhead of the arrow from Pt1 to Pt2. The segment [Pt1, Pt2] (body) is not drawn.
% The appearance of the arrowhead can be modified with the help of attributes set by
% the macro \figset arrowhead(...).
%    \figdrawarrowhead [Pt1,Pt2]
% Axes defined by their Origin and coordinate ranges. Each axis is parallel to the
% corresponding absolute axis and is drawn as an oriented arrow from start value
% to stop value.
% The short form \figdrawaxes Origin(L) is equivalent to \figdrawaxes Origin(0,L, 0,L)
% in 2D and \figdrawaxes Origin(0,L, 0,L, 0,L) in 3D.
% The end points of the axes are given by \figptsaxes.
%    \figdrawaxes Origin(X1,X2, Y1,Y2)
%    \figdrawaxes Origin(X1,X2, Y1,Y2, Z1,Z2)
%    \figdrawaxes Origin(L)
% --- Curve ---
% Bezier curve defined by N cubic arcs. The arc number i is defined by the four
% control points P_{j}, P_{j+1}, P_{j+2}, P_{j+3} with j=3i-2.
% The curve interpolates each 3 points beginning with the first, i.e. at P_{3i-2},
% i=1,...N+1. At these points, the curve is only C0. To obtain G1 continuity at P_j,
% the points P_{j-1}, P_j and P_{j+1} must be aligned.
% The total number of points must be 3N+1 and is not checked.
%    \figdrawBezier N [Pt_1, ..., Pt_{3N+1}]
% C1 curve that interpolates the points P1, P2,... ,Pn. The curve consists of
% n-1 Bezier cubic arcs. The direction of the tangent at Pi is given by Pi-1Pi+1,
% i=1,... ,n. To get a C1 closed curve, the last three points must be the same
% as the first three ones.
% The shape of the curve can be modified by a roundness coefficient to be
% set by the macro \figset curve(...). The best values for this coefficient are
% in the interval [0.15, 0.3] (0 gives a polygonal line).
%    \figdrawcurve [Pt0,Pt1,... ,PtN,PtN+1]
% --- Flow chart ---
% Connections between nodes in a flow chart. Each connection path can be a polygonal
% line or a curved line. An arrowhead is drawn by default at the middle point of the
% path joining the points Pt1,Pt2,... ,PtN.
% The attributes of the path are set by the macro \figset flowchart(...).
% The frames at the nodes are drawn with the help of the macro \figdrawfcnode.
%    \figdrawfcconnect[Pt1,Pt2,... ,PtN]
% Frames containing Text at each node centered on points Pt1,Pt2,... ,PtN in a flow chart.
% If the Text argument is empty, the text joined to the points when they are defined
% is used ; otherwise, this Text is used for every point Pt1,Pt2,... ,PtN. The frame
% surrounding the text is reduced or enlarged according to the padding dimensions.
% This attribute along with the other ones is set by the macro \figset flowchart(...).
% The connections between the nodes are usually drawn by the macro \figdrawfcconnect.
% If no arrows are wanted, the best is to say \figset arrowhead(length=0) ; the macros
% \figdrawline or \figdrawcurve can also be used.
%    \figdrawfcnode[Pt1,Pt2,... ,PtN]{Text}
% --- Triangle related macros ---
% Altitude from Pt1 in the triangle (Pt1, Pt2, Pt3).
% Dim is the dimension of the square at the foot H of the altitude which is drawn on
% either side of the altitude according to the order of the 2 points Pt2 and Pt3.
% If H is outside the the segment [Pt2, Pt3], a line is drawn to support the square.
% The graphical attributes of this line and the square are set by \figset altitude(...).
%    \figdrawaltitude Dim [Pt1,Pt2,Pt3]
% Exterior normal of length Length to the segment [Pt1, Pt2]. Lambda is the barycentric
% coordinate of the origin of the normal with respect to the segment [Pt1, Pt2], which
% sets the position of the vector along [Pt1, Pt2].
% In 3D-mode, it works only in the plane Z=0.
%    \figdrawnormal Length,Lambda [Pt1,Pt2]
% --- Grid ---
% Mesh of N1 x N2 intervals on the quadrangle (Pt1, Pt2, Pt3, Pt4), N1 along
% [Pt1, Pt2] and [Pt3, Pt4], N2 along the 2 other segments. A flag set by
% \figset mesh(diag=...) allows to draw a diagonal line inside each cell. Other
% graphical settings govern the appearance of the internal mesh (see \figset mesh).
% The border line attributes are the general ones.
%    \figdrawmesh N1,N2 [Pt1,Pt2,Pt3,Pt4]
% Mesh of the type Type triangle in the triangle (Pt1, Pt2, Pt3). The internal
% mesh is drawn according to specific graphical settings (see \figset trimesh).
% The border line attributes are the general ones.
%    \figdrawtrimesh Type [Pt1,Pt2,Pt3]
%
%%%%%%%%%%%%%%%%%%%%%%%%%%%%%%%%%%%%%%%%%%%%%%%%%%%%%%%%%%%%%%%%%%%%%%%%%%%%%%%
% Points with numbers >= 0 are devoted to the user.
% Points with numbers <  0 are reserved to internal use.
%%%%%%%%%%%%%%%%%%%%%%%%%%%%%%%%%%%%%%%%%%%%%%%%%%%%%%%%%%%%%%%%%%%%%%%%%%%%%%%
\catcode`\@=11
\ifx\ctr@ln@m\undefined\else%
    \immediate\write16{*** Fig4TeX WARNING : \string\ctr@ln@m\space already defined.}\fi
\def\ctr@ln@m#1{\ifx#1\undefined\else%
    \immediate\write16{*** Fig4TeX WARNING : \string#1 already defined.}\fi}
\ctr@ln@m\ctr@ld@f
\def\ctr@ld@f#1#2{\ctr@ln@m#2#1#2}
\ctr@ld@f\def\ctr@ln@w#1#2{\ctr@ln@m#2\csname#1\endcsname#2}
{\catcode`\/=0 \catcode`/\=12 /ctr@ld@f/gdef/BS@{\}}
\ctr@ld@f\def\ctr@lcsn@m#1{\expandafter\ifx\csname#1\endcsname\relax\else%
    \immediate\write16{*** Fig4TeX WARNING : \BS@\expandafter\string#1\space already defined.}\fi}
\ctr@ld@f\edef\colonc@tcode{\the\catcode`\:}
\ctr@ld@f\edef\semicolonc@tcode{\the\catcode`\;}
\ctr@ld@f\def\t@stc@tcodech@nge{{\let\c@tcodech@nged=\z@%
    \ifnum\colonc@tcode=\the\catcode`\:\else\let\c@tcodech@nged=\@ne\fi%
    \ifnum\semicolonc@tcode=\the\catcode`\;\else\let\c@tcodech@nged=\@ne\fi%
    \ifx\c@tcodech@nged\@ne%
    \immediate\write16{}
    \immediate\write16{!!!=============================================================!!!}
    \immediate\write16{ Fig4TeX WARNING:}
    \immediate\write16{ The category code of some characters has been changed, which will}
    \immediate\write16{ result in an error (message "Runaway argument?").}
    \immediate\write16{ This probably comes from another package that changed the category}
    \immediate\write16{ code after Fig4TeX was loaded. If that proves to be exact, the}
    \immediate\write16{ solution is to exchange the loading commands on top of your file}
    \immediate\write16{ so that Fig4TeX is loaded last. For example, in LaTeX, we should}
    \immediate\write16{ say :}
    \immediate\write16{\BS@ usepackage[french]{babel}}
    \immediate\write16{\BS@ usepackage{fig4tex}}
    \immediate\write16{!!!=============================================================!!!}
    \immediate\write16{}
    \fi}}
\ctr@ld@f\def\FigforTeX{F\kern-.05em i\kern-.05em g\kern-.1em\raise-.14em\hbox{4}\kern-.19em\TeX}
\ctr@ld@f\def\W@rnmesoldA#1{\W@rnmesold}
\ctr@ld@f\def\W@rnmesoldAB#1(#2){\W@rnmesold}
\ctr@ld@f\def\W@rnmesold{%
    \immediate\write16{}
    \immediate\write16{!!!=============================================================!!!}
    \immediate\write16{ Fig4TeX WARNING:}
    \immediate\write16{ The file to be compiled is not compatible with the current version}
    \immediate\write16{ of Fig4TeX. To fix that, upgrade the source file (mainly change \BS@ ps*}
    \immediate\write16{ macros by \BS@ fig* macros), or use fig4tex184.tex instead (\BS@ input fig4tex184}
    \immediate\write16{ or \BS@ usepackage{fig4tex184}).}
    \immediate\write16{!!!=============================================================!!!}
    \immediate\write16{}}
\ctr@ln@m\psbeginfig\let\psbeginfig\W@rnmesoldA
\ctr@ln@m\psset\let\psset\W@rnmesoldAB
\ctr@ln@m\pssetdefault\let\pssetdefault\W@rnmesoldAB
\ctr@ln@m\pssetupdate\let\pssetupdate\W@rnmesoldA
\ctr@ln@w{newdimen}\epsil@n\epsil@n=0.00005pt
\ctr@ln@w{newdimen}\Cepsil@n\Cepsil@n=0.005pt
\ctr@ln@w{newdimen}\dcq@\dcq@=254pt
\ctr@ln@w{newdimen}\PI@\PI@=3.141592pt
\ctr@ln@w{newdimen}\DemiPI@deg\DemiPI@deg=90pt
\ctr@ln@w{newdimen}\PI@deg\PI@deg=180pt
\ctr@ln@w{newdimen}\DePI@deg\DePI@deg=360pt
\ctr@ld@f\chardef\t@n=10
\ctr@ld@f\chardef\c@nt=100
\ctr@ld@f\chardef\@lxxiv=74
\ctr@ld@f\chardef\@xci=91
\ctr@ld@f\mathchardef\@nMnCQn=9949
\ctr@ld@f\chardef\@vi=6
\ctr@ld@f\chardef\@xxx=30
\ctr@ld@f\chardef\@lvi=56
\ctr@ld@f\chardef\@@lxxi=71
\ctr@ld@f\chardef\@lxxxv=85
\ctr@ld@f\mathchardef\@@mmmmlxviii=4068
\ctr@ld@f\mathchardef\@ccclx=360
\ctr@ld@f\mathchardef\@dccxx=720
\ctr@ln@w{newcount}\p@rtent \ctr@ln@w{newcount}\f@ctech \ctr@ln@w{newcount}\result@tent
\ctr@ln@w{newdimen}\v@lmin \ctr@ln@w{newdimen}\v@lmax \ctr@ln@w{newdimen}\v@leur
\ctr@ln@w{newdimen}\result@t\ctr@ln@w{newdimen}\result@@t
\ctr@ln@w{newdimen}\mili@u \ctr@ln@w{newdimen}\c@rre \ctr@ln@w{newdimen}\delt@
\ctr@ld@f\def\degT@rd{0.017453 }  % pi/180
\ctr@ld@f\def\rdT@deg{57.295779 } % 180/pi
\ctr@ln@m\v@leurseule
{\catcode`p=12 \catcode`t=12 \gdef\v@leurseule#1pt{#1}}
\ctr@ld@f\def\repdecn@mb#1{\expandafter\v@leurseule\the#1\space}
\ctr@ld@f\def\arct@n#1(#2,#3){{\v@lmin=#2\v@lmax=#3%
    \maxim@m{\mili@u}{-\v@lmin}{\v@lmin}\maxim@m{\c@rre}{-\v@lmax}{\v@lmax}%
    \delt@=\mili@u\m@ech\mili@u%
    \ifdim\c@rre>\@nMnCQn\mili@u\divide\v@lmax\tw@\c@lATAN\v@leur(\z@,\v@lmax)% DY > 9949 DX
    \else%
    \maxim@m{\mili@u}{-\v@lmin}{\v@lmin}\maxim@m{\c@rre}{-\v@lmax}{\v@lmax}%
    \m@ech\c@rre%
    \ifdim\mili@u>\@nMnCQn\c@rre\divide\v@lmin\tw@% DX > 9949 DY
    \maxim@m{\mili@u}{-\v@lmin}{\v@lmin}\c@lATAN\v@leur(\mili@u,\z@)%
    \else\c@lATAN\v@leur(\delt@,\v@lmax)\fi\fi%
    \ifdim\v@lmin<\z@\v@leur=-\v@leur\ifdim\v@lmax<\z@\advance\v@leur-\PI@%
    \else\advance\v@leur\PI@\fi\fi%
    \global\result@t=\v@leur}#1=\result@t}
\ctr@ld@f\def\m@ech#1{\ifdim#1>1.646pt\divide\mili@u\t@n\divide\c@rre\t@n\m@ech#1\fi}
\ctr@ld@f\def\c@lATAN#1(#2,#3){{\v@lmin=#2\v@lmax=#3\v@leur=\z@\delt@=\tw@ pt%
    \un@iter{0.785398}{\v@lmax<}%
    \un@iter{0.463648}{\v@lmax<}%
    \un@iter{0.244979}{\v@lmax<}%
    \un@iter{0.124355}{\v@lmax<}%
    \un@iter{0.062419}{\v@lmax<}%
    \un@iter{0.031240}{\v@lmax<}%
    \un@iter{0.015624}{\v@lmax<}%
    \un@iter{0.007812}{\v@lmax<}%
    \un@iter{0.003906}{\v@lmax<}%
    \un@iter{0.001953}{\v@lmax<}%
    \un@iter{0.000976}{\v@lmax<}%
    \un@iter{0.000488}{\v@lmax<}%
    \un@iter{0.000244}{\v@lmax<}%
    \un@iter{0.000122}{\v@lmax<}%
    \un@iter{0.000061}{\v@lmax<}%
    \un@iter{0.000030}{\v@lmax<}%
    \un@iter{0.000015}{\v@lmax<}%
    \global\result@t=\v@leur}#1=\result@t}
\ctr@ld@f\def\un@iter#1#2{%
    \divide\delt@\tw@\edef\dpmn@{\repdecn@mb{\delt@}}%
    \mili@u=\v@lmin%
    \ifdim#2\z@%
      \advance\v@lmin-\dpmn@\v@lmax\advance\v@lmax\dpmn@\mili@u%
      \advance\v@leur-#1pt%
    \else%
      \advance\v@lmin\dpmn@\v@lmax\advance\v@lmax-\dpmn@\mili@u%
      \advance\v@leur#1pt%
    \fi}
\ctr@ld@f\def\c@ssin#1#2#3{\expandafter\ifx\csname COS@\number#3\endcsname\relax\c@lCS{#3pt}%
    \expandafter\xdef\csname COS@\number#3\endcsname{\repdecn@mb\result@t}%
    \expandafter\xdef\csname SIN@\number#3\endcsname{\repdecn@mb\result@@t}\fi%
    \edef#1{\csname COS@\number#3\endcsname}\edef#2{\csname SIN@\number#3\endcsname}}
\ctr@ld@f\def\c@lCS#1{{\mili@u=#1\p@rtent=\@ne%
    \relax\ifdim\mili@u<\z@\red@ng<-\else\red@ng>+\fi\f@ctech=\p@rtent%
    \relax\ifdim\mili@u<\z@\mili@u=-\mili@u\f@ctech=-\f@ctech\fi\c@@lCS}}
\ctr@ld@f\def\c@@lCS{\v@lmin=\mili@u\c@rre=-\mili@u\advance\c@rre\DemiPI@deg\v@lmax=\c@rre%
    \mili@u\@@lxxi\mili@u\divide\mili@u\@@mmmmlxviii%
    \edef\v@larg{\repdecn@mb{\mili@u}}\mili@u=-\v@larg\mili@u%
    \edef\v@lmxde{\repdecn@mb{\mili@u}}%
    \c@rre\@@lxxi\c@rre\divide\c@rre\@@mmmmlxviii%
    \edef\v@largC{\repdecn@mb{\c@rre}}\c@rre=-\v@largC\c@rre%
    \edef\v@lmxdeC{\repdecn@mb{\c@rre}}%
    \fctc@s\mili@u\v@lmin\global\result@t\p@rtent\v@leur%
    \let\t@mp=\v@larg\let\v@larg=\v@largC\let\v@largC=\t@mp%
    \let\t@mp=\v@lmxde\let\v@lmxde=\v@lmxdeC\let\v@lmxdeC=\t@mp%
    \fctc@s\c@rre\v@lmax\global\result@@t\f@ctech\v@leur}
\ctr@ld@f\def\fctc@s#1#2{\v@leur=#1\relax\ifdim#2<\@lxxxv\p@\cosser@h\else\sinser@t\fi}
\ctr@ld@f\def\cosser@h{\advance\v@leur\@lvi\p@\divide\v@leur\@lvi%
    \v@leur=\v@lmxde\v@leur\advance\v@leur\@xxx\p@%
    \v@leur=\v@lmxde\v@leur\advance\v@leur\@ccclx\p@%
    \v@leur=\v@lmxde\v@leur\advance\v@leur\@dccxx\p@\divide\v@leur\@dccxx}
\ctr@ld@f\def\sinser@t{\v@leur=\v@lmxdeC\p@\advance\v@leur\@vi\p@%
    \v@leur=\v@largC\v@leur\divide\v@leur\@vi}
\ctr@ld@f\def\red@ng#1#2{\relax\ifdim\mili@u#1#2\DemiPI@deg\advance\mili@u#2-\PI@deg%
    \p@rtent=-\p@rtent\red@ng#1#2\fi}
\ctr@ld@f\def\pr@c@lCS#1#2#3{\ctr@lcsn@m{COS@\number#3 }%
    \expandafter\xdef\csname COS@\number#3\endcsname{#1}%
    \expandafter\xdef\csname SIN@\number#3\endcsname{#2}}
\pr@c@lCS{1}{0}{0}
\pr@c@lCS{0.7071}{0.7071}{45}\pr@c@lCS{0.7071}{-0.7071}{-45}
\pr@c@lCS{0}{1}{90}          \pr@c@lCS{0}{-1}{-90}
\pr@c@lCS{-1}{0}{180}        \pr@c@lCS{-1}{0}{-180}
\pr@c@lCS{0}{-1}{270}        \pr@c@lCS{0}{1}{-270}
\ctr@ld@f\def\invers@#1#2{{\v@leur=#2\maxim@m{\v@lmax}{-\v@leur}{\v@leur}%
    \f@ctech=\@ne\m@inv@rs%
    \multiply\v@leur\f@ctech\edef\v@lv@leur{\repdecn@mb{\v@leur}}%
    \p@rtentiere{\p@rtent}{\v@leur}\v@lmin=\p@\divide\v@lmin\p@rtent%
    \inv@rs@\multiply\v@lmax\f@ctech\global\result@t=\v@lmax}#1=\result@t}
\ctr@ld@f\def\m@inv@rs{\ifdim\v@lmax<\p@\multiply\v@lmax\t@n\multiply\f@ctech\t@n\m@inv@rs\fi}
\ctr@ld@f\def\inv@rs@{\v@lmax=-\v@lmin\v@lmax=\v@lv@leur\v@lmax%
    \advance\v@lmax\tw@ pt\v@lmax=\repdecn@mb{\v@lmin}\v@lmax%
    \delt@=\v@lmax\advance\delt@-\v@lmin\ifdim\delt@<\z@\delt@=-\delt@\fi%
    \ifdim\delt@>\epsil@n\v@lmin=\v@lmax\inv@rs@\fi}
\ctr@ld@f\def\minim@m#1#2#3{\relax\ifdim#2<#3#1=#2\else#1=#3\fi}
\ctr@ld@f\def\maxim@m#1#2#3{\relax\ifdim#2>#3#1=#2\else#1=#3\fi}
\ctr@ld@f\def\p@rtentiere#1#2{#1=#2\divide#1by65536 }
\ctr@ld@f\def\r@undint#1#2{{\v@leur=#2\divide\v@leur\t@n\p@rtentiere{\p@rtent}{\v@leur}%
    \v@leur=\p@rtent pt\global\result@t=\t@n\v@leur}#1=\result@t}
\ctr@ld@f\def\sqrt@#1#2{{\v@leur=#2%
    \minim@m{\v@lmin}{\p@}{\v@leur}\maxim@m{\v@lmax}{\p@}{\v@leur}%
    \f@ctech=\@ne\m@sqrt@\sqrt@@%
    \mili@u=\v@lmin\advance\mili@u\v@lmax\divide\mili@u\tw@\multiply\mili@u\f@ctech%
    \global\result@t=\mili@u}#1=\result@t}
\ctr@ld@f\def\m@sqrt@{\ifdim\v@leur>\dcq@\divide\v@leur\c@nt\v@lmax=\v@leur%
    \multiply\f@ctech\t@n\m@sqrt@\fi}
\ctr@ld@f\def\sqrt@@{\mili@u=\v@lmin\advance\mili@u\v@lmax\divide\mili@u\tw@%
    \c@rre=\repdecn@mb{\mili@u}\mili@u%
    \ifdim\c@rre<\v@leur\v@lmin=\mili@u\else\v@lmax=\mili@u\fi%
    \delt@=\v@lmax\advance\delt@-\v@lmin\ifdim\delt@>\epsil@n\sqrt@@\fi}
\ctr@ld@f\def\extrairelepremi@r#1\de#2{\expandafter\lepremi@r#2@#1#2}
\ctr@ld@f\def\lepremi@r#1,#2@#3#4{\def#3{#1}\def#4{#2}\ignorespaces}
\ctr@ld@f\def\@cfor#1:=#2\do#3{%
  \edef\@fortemp{#2}%
  \ifx\@fortemp\empty\else\@cforloop#2,\@nil,\@nil\@@#1{#3}\fi}
\ctr@ln@m\@nextwhile
\ctr@ld@f\def\@cforloop#1,#2\@@#3#4{%
  \def#3{#1}%
  \ifx#3\Fig@nnil\let\@nextwhile=\Fig@fornoop\else#4\relax\let\@nextwhile=\@cforloop\fi%
  \@nextwhile#2\@@#3{#4}}

\ctr@ld@f\def\@ecfor#1:=#2\do#3{%
  \def\@@cfor{\@cfor#1:=}%
  \edef\@@@cfor{#2}%
  \expandafter\@@cfor\@@@cfor\do{#3}}
\ctr@ld@f\def\Fig@nnil{\@nil}
\ctr@ld@f\def\Fig@fornoop#1\@@#2#3{}
\ctr@ln@m\list@@rg
\ctr@ld@f\def\trtlis@rg#1#2{\def\list@@rg{#1}%
    \@ecfor\p@rv@l:=\list@@rg\do{\expandafter#2\p@rv@l|}}
\ctr@ld@f\def\trtlis@rgtok#1{\let@xte={}\let\n@xt\addt@t@xt\addt@t@xt #1}
\ctr@ln@m\M@cro
\ctr@ln@m\n@xt
\ctr@ld@f\def\addt@t@xt#1{\if#1|\let\n@xt\relax\else%
    \if#1,\expandafter\M@cro\the\let@xte|\let@xte={}%
    \else\let@xte=\expandafter{\the\let@xte #1}\fi\fi\n@xt}
\ctr@ln@w{newbox}\b@xvisu
\ctr@ln@w{newtoks}\let@xte
\ctr@ln@w{newif}\ifitis@K
\ctr@ln@w{newcount}\s@mme
\ctr@ln@w{newcount}\l@mbd@un \ctr@ln@w{newcount}\l@mbd@de
\ctr@ln@w{newcount}\superc@ntr@l\superc@ntr@l=\@ne        % Controle impose
\ctr@ln@w{newcount}\typec@ntr@l\typec@ntr@l=\superc@ntr@l % Controle souhaite
\ctr@ln@w{newdimen}\v@lX  \ctr@ln@w{newdimen}\v@lY  \ctr@ln@w{newdimen}\v@lZ
\ctr@ln@w{newdimen}\v@lXa \ctr@ln@w{newdimen}\v@lYa \ctr@ln@w{newdimen}\v@lZa
\ctr@ln@w{newdimen}\unit@\unit@=\p@ % Initialisation a la valeur par defaut.
\ctr@ld@f\def\unit@util{pt}
\ctr@ld@f\def\ptT@ptps{0.996264 }
\ctr@ld@f\def\ptpsT@pt{1.00375 }
\ctr@ld@f\def\ptT@unit@{1} % Initialisation correspondant a la valeur par defaut de \unit@
\ctr@ld@f\def\setunit@#1{\def\unit@util{#1}\setunit@@#1:\invers@{\result@t}{\unit@}%
    \edef\ptT@unit@{\repdecn@mb\result@t}}
\ctr@ld@f\def\setunit@@#1#2:{\ifcat#1a\unit@=\@ne#1#2\else\unit@=#1#2\fi}
\ctr@ld@f\def\d@fm@cdim#1#2{{\v@leur=#2\v@leur=\ptT@unit@\v@leur\xdef#1{\repdecn@mb\v@leur}}}
\ctr@ln@w{newif}\ifBdingB@x\BdingB@xtrue
\ctr@ln@w{newdimen}\c@@rdXmin \ctr@ln@w{newdimen}\c@@rdYmin  % Dimensions de la BoundingBox
\ctr@ln@w{newdimen}\c@@rdXmax \ctr@ln@w{newdimen}\c@@rdYmax
\ctr@ld@f\def\b@undb@x#1#2{\ifBdingB@x%
    \relax\ifdim#1<\c@@rdXmin\global\c@@rdXmin=#1\fi%
    \relax\ifdim#2<\c@@rdYmin\global\c@@rdYmin=#2\fi%
    \relax\ifdim#1>\c@@rdXmax\global\c@@rdXmax=#1\fi%
    \relax\ifdim#2>\c@@rdYmax\global\c@@rdYmax=#2\fi\fi}
\ctr@ld@f\def\b@undb@xP#1{{\Figg@tXY{#1}\b@undb@x{\v@lX}{\v@lY}}}
\ctr@ld@f\def\ellBB@x#1;#2,#3(#4,#5,#6){{\s@uvc@ntr@l\et@tellBB@x%
    \setc@ntr@l{2}\figptell-2::#1;#2,#3(#4,#6)\b@undb@xP{-2}%
    \figptell-2::#1;#2,#3(#5,#6)\b@undb@xP{-2}%
    \c@ssin{\C@}{\S@}{#6}\v@lmin=\C@ pt\v@lmax=\S@ pt%
    \mili@u=#3\v@lmin\delt@=#2\v@lmax\arct@n\v@leur(\delt@,\mili@u)%
    \mili@u=-#3\v@lmax\delt@=#2\v@lmin\arct@n\c@rre(\delt@,\mili@u)%
    \v@leur=\rdT@deg\v@leur\advance\v@leur-\DePI@deg%
    \c@rre=\rdT@deg\c@rre\advance\c@rre-\DePI@deg%
    \v@lmin=#4pt\v@lmax=#5pt%
    \loop\ifdim\v@leur<\v@lmax\ifdim\v@leur>\v@lmin%
    \edef\@ngle{\repdecn@mb\v@leur}\figptell-2::#1;#2,#3(\@ngle,#6)%
    \b@undb@xP{-2}\fi\advance\v@leur\PI@deg\repeat%
    \loop\ifdim\c@rre<\v@lmax\ifdim\c@rre>\v@lmin%
    \edef\@ngle{\repdecn@mb\c@rre}\figptell-2::#1;#2,#3(\@ngle,#6)%
    \b@undb@xP{-2}\fi\advance\c@rre\PI@deg\repeat%
    \resetc@ntr@l\et@tellBB@x}\ignorespaces}
\ctr@ld@f\def\initb@undb@x{\c@@rdXmin=\maxdimen\c@@rdYmin=\maxdimen%
    \c@@rdXmax=-\maxdimen\c@@rdYmax=-\maxdimen}
\ctr@ld@f\def\c@ntr@lnum#1{%
    \relax\ifnum\typec@ntr@l=\@ne%
    \ifnum#1<\z@%
    \immediate\write16{*** Forbidden point number (#1). Abort.}\end\fi\fi%
    \set@bjc@de{#1}}
\ctr@ln@m\objc@de
\ctr@ld@f\def\set@bjc@de#1{\edef\objc@de{@BJ\ifnum#1<\z@ M\romannumeral-#1\else\romannumeral#1\fi}}
\s@mme=\m@ne\loop\ifnum\s@mme>-19
  \set@bjc@de{\s@mme}\ctr@lcsn@m\objc@de\ctr@lcsn@m{\objc@de T}
\advance\s@mme\m@ne\repeat
\s@mme=\@ne\loop\ifnum\s@mme<6
  \set@bjc@de{\s@mme}\ctr@lcsn@m\objc@de\ctr@lcsn@m{\objc@de T}
\advance\s@mme\@ne\repeat
\ctr@ld@f\def\setc@ntr@l#1{\ifnum\superc@ntr@l>#1\typec@ntr@l=\superc@ntr@l%
    \else\typec@ntr@l=#1\fi}
\ctr@ld@f\def\resetc@ntr@l#1{\global\superc@ntr@l=#1\setc@ntr@l{#1}}
\ctr@ld@f\def\s@uvc@ntr@l#1{\edef#1{\the\superc@ntr@l}}
\ctr@ln@m\c@lproscal
\ctr@ld@f\def\c@lproscalDD#1[#2,#3]{{\Figg@tXY{#2}%
    \edef\Xu@{\repdecn@mb{\v@lX}}\edef\Yu@{\repdecn@mb{\v@lY}}\Figg@tXY{#3}%
    \global\result@t=\Xu@\v@lX\global\advance\result@t\Yu@\v@lY}#1=\result@t}
\ctr@ld@f\def\c@lproscalTD#1[#2,#3]{{\Figg@tXY{#2}\edef\Xu@{\repdecn@mb{\v@lX}}%
    \edef\Yu@{\repdecn@mb{\v@lY}}\edef\Zu@{\repdecn@mb{\v@lZ}}%
    \Figg@tXY{#3}\global\result@t=\Xu@\v@lX\global\advance\result@t\Yu@\v@lY%
    \global\advance\result@t\Zu@\v@lZ}#1=\result@t}
\ctr@ld@f\def\c@lprovec#1{%
    \det@rmC\v@lZa(\v@lX,\v@lY,\v@lmin,\v@lmax)%
    \det@rmC\v@lXa(\v@lY,\v@lZ,\v@lmax,\v@leur)%
    \det@rmC\v@lYa(\v@lZ,\v@lX,\v@leur,\v@lmin)%
    \Figv@ctCreg#1(\v@lXa,\v@lYa,\v@lZa)}
\ctr@ld@f\def\det@rm#1[#2,#3]{{\Figg@tXY{#2}\Figg@tXYa{#3}%
    \delt@=\repdecn@mb{\v@lX}\v@lYa\advance\delt@-\repdecn@mb{\v@lY}\v@lXa%
    \global\result@t=\delt@}#1=\result@t}
\ctr@ld@f\def\det@rmC#1(#2,#3,#4,#5){{\global\result@t=\repdecn@mb{#2}#5%
    \global\advance\result@t-\repdecn@mb{#3}#4}#1=\result@t}
\ctr@ld@f\def\getredf@ctDD#1(#2,#3){{\maxim@m{\v@lXa}{-#2}{#2}\maxim@m{\v@lYa}{-#3}{#3}%
    \maxim@m{\v@lXa}{\v@lXa}{\v@lYa}% \v@lXa = ||X||inf
    \ifdim\v@lXa>\@xci pt\divide\v@lXa\@xci%
    \p@rtentiere{\p@rtent}{\v@lXa}\advance\p@rtent\@ne\else\p@rtent=\@ne\fi%
    \global\result@tent=\p@rtent}#1=\result@tent\ignorespaces}
\ctr@ld@f\def\getredf@ctTD#1(#2,#3,#4){{\maxim@m{\v@lXa}{-#2}{#2}\maxim@m{\v@lYa}{-#3}{#3}%
    \maxim@m{\v@lZa}{-#4}{#4}\maxim@m{\v@lXa}{\v@lXa}{\v@lYa}%
    \maxim@m{\v@lXa}{\v@lXa}{\v@lZa}% \v@lXa = ||X||inf
    \ifdim\v@lXa>\@lxxiv pt\divide\v@lXa\@lxxiv%
    \p@rtentiere{\p@rtent}{\v@lXa}\advance\p@rtent\@ne\else\p@rtent=\@ne\fi%
    \global\result@tent=\p@rtent}#1=\result@tent\ignorespaces}
\ctr@ln@m\getredf@ctB
\ctr@ld@f\def\getredf@ctBDD#1{\getredf@ctDD#1(\v@lX,\v@lY)}
\ctr@ld@f\def\getredf@ctBTD#1{\getredf@ctTD#1(\v@lX,\v@lY,\v@lZ)}
\ctr@ld@f\def\FigptintercircB@zDD#1:#2:#3,#4[#5,#6,#7,#8]{{\s@uvc@ntr@l\et@tfigptintercircB@zDD%
    \setc@ntr@l{2}\figvectPDD-1[#5,#8]\Figg@tXY{-1}\getredf@ctDD\f@ctech(\v@lX,\v@lY)%
    \mili@u=#4\unit@\divide\mili@u\f@ctech\c@rre=\repdecn@mb{\mili@u}\mili@u%
    \figptBezierDD-5::#3[#5,#6,#7,#8]%
    \v@lmin=#3\p@\v@lmax=\v@lmin\advance\v@lmax0.1\p@%
    \loop\edef\T@{\repdecn@mb{\v@lmax}}\figptBezierDD-2::\T@[#5,#6,#7,#8]%
    \figvectPDD-1[-5,-2]\n@rmeucCDD{\delt@}{-1}\ifdim\delt@<\c@rre\v@lmin=\v@lmax%
    \advance\v@lmax0.1\p@\repeat%
    \loop\mili@u=\v@lmin\advance\mili@u\v@lmax%
    \divide\mili@u\tw@\edef\T@{\repdecn@mb{\mili@u}}\figptBezierDD-2::\T@[#5,#6,#7,#8]%
    \figvectPDD-1[-5,-2]\n@rmeucCDD{\delt@}{-1}\ifdim\delt@>\c@rre\v@lmax=\mili@u%
    \else\v@lmin=\mili@u\fi\v@leur=\v@lmax\advance\v@leur-\v@lmin%
    \ifdim\v@leur>\epsil@n\repeat\figptcopyDD#1:#2/-2/%
    \resetc@ntr@l\et@tfigptintercircB@zDD}\ignorespaces}
\ctr@ln@m\figptinterlines
\ctr@ld@f\def\inters@cDD#1:#2[#3,#4;#5,#6]{{\s@uvc@ntr@l\et@tinters@cDD%
    \setc@ntr@l{2}\vecunit@{-1}{#4}\vecunit@{-2}{#6}%
    \Figg@tXY{-1}\setc@ntr@l{1}\Figg@tXYa{#3}%
    \edef\A@{\repdecn@mb{\v@lX}}\edef\B@{\repdecn@mb{\v@lY}}%
    \v@lmin=\B@\v@lXa\advance\v@lmin-\A@\v@lYa%
    \Figg@tXYa{#5}\setc@ntr@l{2}\Figg@tXY{-2}%
    \edef\C@{\repdecn@mb{\v@lX}}\edef\D@{\repdecn@mb{\v@lY}}%
    \v@lmax=\D@\v@lXa\advance\v@lmax-\C@\v@lYa%
    \delt@=\A@\v@lY\advance\delt@-\B@\v@lX%
    \invers@{\v@leur}{\delt@}\edef\v@ldelta{\repdecn@mb{\v@leur}}%
    \v@lXa=\A@\v@lmax\advance\v@lXa-\C@\v@lmin%
    \v@lYa=\B@\v@lmax\advance\v@lYa-\D@\v@lmin%
    \v@lXa=\v@ldelta\v@lXa\v@lYa=\v@ldelta\v@lYa%
    \setc@ntr@l{1}\Figp@intregDD#1:{#2}(\v@lXa,\v@lYa)%
    \resetc@ntr@l\et@tinters@cDD}\ignorespaces}
\ctr@ld@f\def\inters@cTD#1:#2[#3,#4;#5,#6]{{\s@uvc@ntr@l\et@tinters@cTD%
    \setc@ntr@l{2}\figvectNVTD-1[#4,#6]\figvectNVTD-2[#6,-1]\figvectPTD-1[#3,#5]%
    \r@pPSTD\v@leur[-2,-1,#4]\edef\v@lcoef{\repdecn@mb{\v@leur}}%
    \figpttraTD#1:{#2}=#3/\v@lcoef,#4/\resetc@ntr@l\et@tinters@cTD}\ignorespaces}
\ctr@ld@f\def\r@pPSTD#1[#2,#3,#4]{{\Figg@tXY{#2}\edef\Xu@{\repdecn@mb{\v@lX}}%
    \edef\Yu@{\repdecn@mb{\v@lY}}\edef\Zu@{\repdecn@mb{\v@lZ}}%
    \Figg@tXY{#3}\v@lmin=\Xu@\v@lX\advance\v@lmin\Yu@\v@lY\advance\v@lmin\Zu@\v@lZ%
    \Figg@tXY{#4}\v@lmax=\Xu@\v@lX\advance\v@lmax\Yu@\v@lY\advance\v@lmax\Zu@\v@lZ%
    \invers@{\v@leur}{\v@lmax}\global\result@t=\repdecn@mb{\v@leur}\v@lmin}%
    #1=\result@t}
\ctr@ln@m\n@rminf
\ctr@ld@f\def\n@rminfDD#1#2{{\Figg@tXY{#2}\maxim@m{\v@lX}{\v@lX}{-\v@lX}%
    \maxim@m{\v@lY}{\v@lY}{-\v@lY}\maxim@m{\global\result@t}{\v@lX}{\v@lY}}%
    #1=\result@t}
\ctr@ld@f\def\n@rminfTD#1#2{{\Figg@tXY{#2}\maxim@m{\v@lX}{\v@lX}{-\v@lX}%
    \maxim@m{\v@lY}{\v@lY}{-\v@lY}\maxim@m{\v@lZ}{\v@lZ}{-\v@lZ}%
    \maxim@m{\v@lX}{\v@lX}{\v@lY}\maxim@m{\global\result@t}{\v@lX}{\v@lZ}}%
    #1=\result@t}
\ctr@ln@m\n@rmeucC
\ctr@ld@f\def\n@rmeucCDD#1#2{\Figg@tXY{#2}\divide\v@lX\f@ctech\divide\v@lY\f@ctech%
    #1=\repdecn@mb{\v@lX}\v@lX\v@lX=\repdecn@mb{\v@lY}\v@lY\advance#1\v@lX}
\ctr@ld@f\def\n@rmeucCTD#1#2{\Figg@tXY{#2}%
    \divide\v@lX\f@ctech\divide\v@lY\f@ctech\divide\v@lZ\f@ctech%
    #1=\repdecn@mb{\v@lX}\v@lX\v@lX=\repdecn@mb{\v@lY}\v@lY\advance#1\v@lX%
    \v@lX=\repdecn@mb{\v@lZ}\v@lZ\advance#1\v@lX}
\ctr@ln@m\n@rmeucSV
\ctr@ld@f\def\n@rmeucSVDD#1#2{{\Figg@tXY{#2}%
    \v@lXa=\repdecn@mb{\v@lX}\v@lX\v@lYa=\repdecn@mb{\v@lY}\v@lY%
    \advance\v@lXa\v@lYa\sqrt@{\global\result@t}{\v@lXa}}#1=\result@t}
\ctr@ld@f\def\n@rmeucSVTD#1#2{{\Figg@tXY{#2}\v@lXa=\repdecn@mb{\v@lX}\v@lX%
    \v@lYa=\repdecn@mb{\v@lY}\v@lY\v@lZa=\repdecn@mb{\v@lZ}\v@lZ%
    \advance\v@lXa\v@lYa\advance\v@lXa\v@lZa\sqrt@{\global\result@t}{\v@lXa}}#1=\result@t}
\ctr@ln@m\n@rmeuc
\ctr@ld@f\def\n@rmeucDD#1#2{{\Figg@tXY{#2}\getredf@ctDD\f@ctech(\v@lX,\v@lY)%
    \divide\v@lX\f@ctech\divide\v@lY\f@ctech%
    \v@lXa=\repdecn@mb{\v@lX}\v@lX\v@lYa=\repdecn@mb{\v@lY}\v@lY%
    \advance\v@lXa\v@lYa\sqrt@{\global\result@t}{\v@lXa}%
    \global\multiply\result@t\f@ctech}#1=\result@t}
\ctr@ld@f\def\n@rmeucTD#1#2{{\Figg@tXY{#2}\getredf@ctTD\f@ctech(\v@lX,\v@lY,\v@lZ)%
    \divide\v@lX\f@ctech\divide\v@lY\f@ctech\divide\v@lZ\f@ctech%
    \v@lXa=\repdecn@mb{\v@lX}\v@lX%
    \v@lYa=\repdecn@mb{\v@lY}\v@lY\v@lZa=\repdecn@mb{\v@lZ}\v@lZ%
    \advance\v@lXa\v@lYa\advance\v@lXa\v@lZa\sqrt@{\global\result@t}{\v@lXa}%
    \global\multiply\result@t\f@ctech}#1=\result@t}
\ctr@ln@m\vecunit@
\ctr@ld@f\def\vecunit@DD#1#2{{\Figg@tXY{#2}\getredf@ctDD\f@ctech(\v@lX,\v@lY)%
    \divide\v@lX\f@ctech\divide\v@lY\f@ctech%
    \Figv@ctCreg#1(\v@lX,\v@lY)\n@rmeucSV{\v@lYa}{#1}%
    \invers@{\v@lXa}{\v@lYa}\edef\v@lv@lXa{\repdecn@mb{\v@lXa}}%
    \v@lX=\v@lv@lXa\v@lX\v@lY=\v@lv@lXa\v@lY%
    \Figv@ctCreg#1(\v@lX,\v@lY)\multiply\v@lYa\f@ctech\global\result@t=\v@lYa}}
\ctr@ld@f\def\vecunit@TD#1#2{{\Figg@tXY{#2}\getredf@ctTD\f@ctech(\v@lX,\v@lY,\v@lZ)%
    \divide\v@lX\f@ctech\divide\v@lY\f@ctech\divide\v@lZ\f@ctech%
    \Figv@ctCreg#1(\v@lX,\v@lY,\v@lZ)\n@rmeucSV{\v@lYa}{#1}%
    \invers@{\v@lXa}{\v@lYa}\edef\v@lv@lXa{\repdecn@mb{\v@lXa}}%
    \v@lX=\v@lv@lXa\v@lX\v@lY=\v@lv@lXa\v@lY\v@lZ=\v@lv@lXa\v@lZ%
    \Figv@ctCreg#1(\v@lX,\v@lY,\v@lZ)\multiply\v@lYa\f@ctech\global\result@t=\v@lYa}}
\ctr@ld@f\def\vecunitC@TD[#1,#2]{\Figg@tXYa{#1}\Figg@tXY{#2}%
    \advance\v@lX-\v@lXa\advance\v@lY-\v@lYa\advance\v@lZ-\v@lZa\c@lvecunitTD}
\ctr@ld@f\def\vecunitCV@TD#1{\Figg@tXY{#1}\c@lvecunitTD}
\ctr@ld@f\def\c@lvecunitTD{\getredf@ctTD\f@ctech(\v@lX,\v@lY,\v@lZ)%
    \divide\v@lX\f@ctech\divide\v@lY\f@ctech\divide\v@lZ\f@ctech%
    \v@lXa=\repdecn@mb{\v@lX}\v@lX%
    \v@lYa=\repdecn@mb{\v@lY}\v@lY\v@lZa=\repdecn@mb{\v@lZ}\v@lZ%
    \advance\v@lXa\v@lYa\advance\v@lXa\v@lZa\sqrt@{\v@lYa}{\v@lXa}%
    \invers@{\v@lXa}{\v@lYa}\edef\v@lv@lXa{\repdecn@mb{\v@lXa}}%
    \v@lX=\v@lv@lXa\v@lX\v@lY=\v@lv@lXa\v@lY\v@lZ=\v@lv@lXa\v@lZ}
\ctr@ln@m\figgetangle
\ctr@ld@f\def\figgetangleDD#1[#2,#3,#4]{\ifGR@cri{\s@uvc@ntr@l\et@tfiggetangleDD\setc@ntr@l{2}%
    \figvectPDD-1[#2,#3]\figvectPDD-2[#2,#4]\vecunit@{-1}{-1}%
    \c@lproscalDD\delt@[-2,-1]\figvectNVDD-1[-1]\c@lproscalDD\v@leur[-2,-1]%
    \arct@n\v@lmax(\delt@,\v@leur)\v@lmax=\rdT@deg\v@lmax%
    \ifdim\v@lmax<\z@\advance\v@lmax\DePI@deg\fi\xdef#1{\repdecn@mb{\v@lmax}}%
    \resetc@ntr@l\et@tfiggetangleDD}\ignorespaces\fi}
\ctr@ld@f\def\figgetangleTD#1[#2,#3,#4,#5]{\ifGR@cri{\s@uvc@ntr@l\et@tfiggetangleTD\setc@ntr@l{2}%
    \figvectPTD-1[#2,#3]\figvectPTD-2[#2,#5]\figvectNVTD-3[-1,-2]%
    \figvectPTD-2[#2,#4]\figvectNVTD-4[-3,-1]%
    \vecunit@{-1}{-1}\c@lproscalTD\delt@[-2,-1]\c@lproscalTD\v@leur[-2,-4]%
    \arct@n\v@lmax(\delt@,\v@leur)\v@lmax=\rdT@deg\v@lmax%
    \ifdim\v@lmax<\z@\advance\v@lmax\DePI@deg\fi\xdef#1{\repdecn@mb{\v@lmax}}%
    \resetc@ntr@l\et@tfiggetangleTD}\ignorespaces\fi}    
\ctr@ld@f\def\figgetdist#1[#2,#3]{\ifGR@cri{\s@uvc@ntr@l\et@tfiggetdist\setc@ntr@l{2}%
    \figvectP-1[#2,#3]\n@rmeuc{\v@lX}{-1}\v@lX=\ptT@unit@\v@lX\xdef#1{\repdecn@mb{\v@lX}}%
    \resetc@ntr@l\et@tfiggetdist}\ignorespaces\fi}
\ctr@ld@f\def\figget#1=#2[#3]{\keln@mun#1|%
    \def\n@mref{a}\ifx\l@debut\n@mref\figgetangle#2[#3]\else% angle
    \def\n@mref{d}\ifx\l@debut\n@mref\figgetdist#2[#3]\else% distance
    \W@rnmeskwd{figget}{#1}\fi\fi\ignorespaces}
\ctr@ld@f\def\Figg@tT#1{\c@ntr@lnum{#1}%
    {\expandafter\expandafter\expandafter\extr@ctT\csname\objc@de\endcsname:%
     \ifnum\B@@ltxt=\z@\ptn@me{#1}\else\csname\objc@de T\endcsname\fi}}
\ctr@ld@f\def\extr@ctT#1,#2,#3/#4:{\def\B@@ltxt{#3}}
\ctr@ld@f\def\Figg@tXY#1{\c@ntr@lnum{#1}%
    \expandafter\expandafter\expandafter\extr@ctC\csname\objc@de\endcsname:}
\ctr@ln@m\extr@ctC
\ctr@ld@f\def\extr@ctCDD#1/#2,#3,#4:{\v@lX=#2\v@lY=#3}
\ctr@ld@f\def\extr@ctCTD#1/#2,#3,#4:{\v@lX=#2\v@lY=#3\v@lZ=#4}
\ctr@ld@f\def\Figg@tXYa#1{\c@ntr@lnum{#1}%
    \expandafter\expandafter\expandafter\extr@ctCa\csname\objc@de\endcsname:}
\ctr@ln@m\extr@ctCa
\ctr@ld@f\def\extr@ctCaDD#1/#2,#3,#4:{\v@lXa=#2\v@lYa=#3}
\ctr@ld@f\def\extr@ctCaTD#1/#2,#3,#4:{\v@lXa=#2\v@lYa=#3\v@lZa=#4}
\ctr@ln@m\t@xt@
\ctr@ld@f\def\figinit#1{\t@stc@tcodech@nge\initpr@lim\Figinit@#1,:\initpss@ttings\ignorespaces}
\ctr@ld@f\def\Figinit@#1,#2:{\setunit@{#1}\def\t@xt@{#2}\ifx\t@xt@\empty\else\Figinit@@#2:\fi}
\ctr@ld@f\def\Figinit@@#1#2:{\if#12 \else\Figs@tproj{#1}\initTD@\fi}
\ctr@ln@w{newif}\ifTr@isDim
\ctr@ld@f\def\UnD@fined{UNDEFINED}
\ctr@ln@m\@utoFN
\ctr@ln@m\@utoFInDone
\ctr@ln@m\disob@unit
\ctr@ld@f\def\initpr@lim{\initb@undb@x\figsetmark{}\figsetptname{$A_{##1}$}\def\Sc@leFact{1}%
    \initDD@\figsetroundcoord{yes}\GR@critrue\expandafter\setupd@te\D@FTupdate:%
    \edef\disob@unit{\UnD@fined}\edef\t@rgetpt{\UnD@fined}\gdef\@utoFInDone{1}\gdef\@utoFN{0}}
\ctr@ld@f\def\initDD@{\Tr@isDimfalse%
    \ifPDFm@ke%
     \let\Ps@rcerc=\Ps@rcercBz%
     \let\Ps@rell=\Ps@rellBz%
    \fi
    \let\c@lDCUn=\c@lDCUnDD%
    \let\c@lDCDeux=\c@lDCDeuxDD%
    \let\c@ldefproj=\relax%
    \let\c@lproscal=\c@lproscalDD%
    \let\c@lprojSP=\relax%
    \let\extr@ctC=\extr@ctCDD%
    \let\extr@ctCa=\extr@ctCaDD%
    \let\extr@ctCF=\extr@ctCFDD%
    \let\Figp@intreg=\Figp@intregDD%
    \let\Figpts@xes=\Figpts@xesDD%
    \let\getredf@ctB=\getredf@ctBDD%
    \let\n@rmeucSV=\n@rmeucSVDD\let\n@rmeuc=\n@rmeucDD\let\n@rmeucC\n@rmeucCDD\let\n@rminf=\n@rminfDD%
    \let\pr@dMatV=\pr@dMatVDD%
    \let\Q@@xes=\Q@@xesDD%
    \let\vecunit@=\vecunit@DD%
    \let\figcoord=\figcoordDD%
    \let\figgetangle=\figgetangleDD%
    \let\figpt=\figptDD%
    \let\figptBezier=\figptBezierDD%
    \let\figptbary=\figptbaryDD%
    \let\figptcirc=\figptcircDD%
    \let\figptcircumcenter=\figptcircumcenterDD%
    \let\figptcopy=\figptcopyDD%
    \let\figptcurvcenter=\figptcurvcenterDD%
    \let\figptell=\figptellDD%
    \let\figptendnormal=\figptendnormalDD%
    \let\figptinterlineplane=\figptinterlineplaneDD%
    \let\figptinterlines=\inters@cDD%
    \let\figptorthocenter=\figptorthocenterDD%
    \let\figptorthoprojline=\figptorthoprojlineDD%
    \let\figptorthoprojplane=\figptorthoprojplaneDD%
    \let\figptrot=\figptrotDD%
    \let\figptscontrol=\figptscontrolDD%
    \let\figptsintercirc=\figptsintercircDD%
    \let\figptsinterlinell=\figptsinterlinellDD%
    \let\figptsorthoprojline=\figptsorthoprojlineDD%
    \let\figptorthoprojplane=\figptorthoprojplaneDD%
    \let\figptsrot=\figptsrotDD%
    \let\figptssym=\figptssymDD%
    \let\figptstra=\figptstraDD%
    \let\figptsym=\figptsymDD%
    \let\figpttraC=\figpttraCDD%
    \let\figpttra=\figpttraDD%
    \let\figptvisilimSL=\figptvisilimSLDD%
    \let\figsetobdist=\figsetobdistDD%
    \let\figsettarget=\figsettargetDD%
    \let\figsetview=\figsetviewDD%
    \let\figvectDBezier=\figvectDBezierDD%
    \let\figvectN=\figvectNDD%
    \let\figvectNV=\figvectNVDD%
    \let\figvectP=\figvectPDD%
    \let\figvectU=\figvectUDD%
    \let\figdrawarccircP=\Q@arccircPDD%
    \let\figdrawarccirc=\Q@arccircDD%
    \let\figdrawarcell=\Q@arcellDD%
    \let\figdrawarcellPA=\Q@arcellPADD%
    \let\figdrawarrowBezier=\Q@arrowBezierDD%
    \let\figdrawarrowcircP=\Q@arrowcircPDD%
    \let\figdrawarrowcirc=\Q@arrowcircDD%
    \let\figdrawarrowhead=\Q@arrowheadDD%
    \let\figdrawarrow=\Q@arrowDD%
    \let\figdrawBezier=\Q@BezierDD%
    \let\figdrawcirc=\Q@circDD%
    \let\figdrawcurve=\Q@curveDD%
    \let\figdrawnormal=\Q@normalDD%
    }
\ctr@ld@f\def\initTD@{\Tr@isDimtrue\initb@undb@xTD\newt@rgetptfalse\newdis@bfalse%
    \let\c@lDCUn=\c@lDCUnTD%
    \let\c@lDCDeux=\c@lDCDeuxTD%
    \let\c@ldefproj=\c@ldefprojTD%
    \let\c@lproscal=\c@lproscalTD%
    \let\extr@ctC=\extr@ctCTD%
    \let\extr@ctCa=\extr@ctCaTD%
    \let\extr@ctCF=\extr@ctCFTD%
    \let\Figp@intreg=\Figp@intregTD%
    \let\Figpts@xes=\Figpts@xesTD%
    \let\getredf@ctB=\getredf@ctBTD%
    \let\n@rmeucSV=\n@rmeucSVTD\let\n@rmeuc=\n@rmeucTD\let\n@rmeucC\n@rmeucCTD\let\n@rminf=\n@rminfTD%
    \let\pr@dMatV=\pr@dMatVTD%
    \let\Q@@xes=\Q@@xesTD%
    \let\vecunit@=\vecunit@TD%
    \let\figcoord=\figcoordTD%
    \let\figgetangle=\figgetangleTD%
    \let\figpt=\figptTD%
    \let\figptBezier=\figptBezierTD%
    \let\figptbary=\figptbaryTD%
    \let\figptcirc=\figptcircTD%
    \let\figptcircumcenter=\figptcircumcenterTD%
    \let\figptcopy=\figptcopyTD%
    \let\figptcurvcenter=\figptcurvcenterTD%
    \let\figptinterlineplane=\figptinterlineplaneTD%
    \let\figptinterlines=\inters@cTD%
    \let\figptorthocenter=\figptorthocenterTD%
    \let\figptorthoprojline=\figptorthoprojlineTD%
    \let\figptorthoprojplane=\figptorthoprojplaneTD%
    \let\figptrot=\figptrotTD%
    \let\figptscontrol=\figptscontrolTD%
    \let\figptsintercirc=\figptsintercircTD%
    \let\figptsorthoprojline=\figptsorthoprojlineTD%
    \let\figptsorthoprojplane=\figptsorthoprojplaneTD%
    \let\figptsrot=\figptsrotTD%
    \let\figptssym=\figptssymTD%
    \let\figptstra=\figptstraTD%
    \let\figptsym=\figptsymTD%
    \let\figpttraC=\figpttraCTD%
    \let\figpttra=\figpttraTD%
    \let\figptvisilimSL=\figptvisilimSLTD%
    \let\figsetobdist=\figsetobdistTD%
    \let\figsettarget=\figsettargetTD%
    \let\figsetview=\figsetviewTD%
    \let\figvectDBezier=\figvectDBezierTD%
    \let\figvectN=\figvectNTD%
    \let\figvectNV=\figvectNVTD%
    \let\figvectP=\figvectPTD%
    \let\figvectU=\figvectUTD%
    \let\figdrawarccircP=\Q@arccircPTD%
    \let\figdrawarccirc=\Q@arccircTD%
    \let\figdrawarcell=\Q@arcellTD%
    \let\figdrawarcellPA=\Q@arcellPATD%
    \let\figdrawarrowBezier=\Q@arrowBezierTD%
    \let\figdrawarrowcircP=\Q@arrowcircPTD%
    \let\figdrawarrowcirc=\Q@arrowcircTD%
    \let\figdrawarrowhead=\Q@arrowheadTD%
    \let\figdrawarrow=\Q@arrowTD%
    \let\figdrawBezier=\Q@BezierTD%
    \let\figdrawcirc=\Q@circTD%
    \let\figdrawcurve=\Q@curveTD%
    }
\ctr@ld@f\def\un@v@ilable#1{\immediate\write16{*** The macro #1 is not available in the current context.}}
\ctr@ld@f\def\figinsert#1{{\def\t@xt@{#1}\relax%
    \ifx\t@xt@\empty\ifnum\@utoFInDone>\z@\Figinsert@\DefGIfilen@me,:\fi%
    \else\expandafter\FiginsertNu@#1 :\fi}\ignorespaces}
\ctr@ld@f\def\FiginsertNu@#1 #2:{\def\t@xt@{#1}\relax\ifx\t@xt@\empty\def\t@xt@{#2}%
    \ifx\t@xt@\empty\ifnum\@utoFInDone>\z@\Figinsert@\DefGIfilen@me,:\fi%
    \else\FiginsertNu@#2:\fi\else\expandafter\FiginsertNd@#1 #2:\fi}
\ctr@ld@f\def\FiginsertNd@#1#2:{\ifcat#1a\Figinsert@#1#2,:\else%
    \ifnum\@utoFInDone>\z@\Figinsert@\DefGIfilen@me,#1#2,:\fi\fi}
\ctr@ln@m\Sc@leFact
\ctr@ld@f\def\Figinsert@#1,#2:{\def\t@xt@{#2}\ifx\t@xt@\empty\xdef\Sc@leFact{1}\else%
    \X@rgdeux@#2\xdef\Sc@leFact{\@rgdeux}\fi%
    \Figdisc@rdLTS{#1}{\t@xt@}\@psfgetbb{\t@xt@}%
    \v@lX=\@psfllx\p@\v@lX=\ptpsT@pt\v@lX\v@lX=\Sc@leFact\v@lX%
    \v@lY=\@psflly\p@\v@lY=\ptpsT@pt\v@lY\v@lY=\Sc@leFact\v@lY%
    \b@undb@x{\v@lX}{\v@lY}%
    \v@lX=\@psfurx\p@\v@lX=\ptpsT@pt\v@lX\v@lX=\Sc@leFact\v@lX%
    \v@lY=\@psfury\p@\v@lY=\ptpsT@pt\v@lY\v@lY=\Sc@leFact\v@lY%
    \b@undb@x{\v@lX}{\v@lY}%
    \ifPDFm@ke\Figinclud@PDF{\t@xt@}{\Sc@leFact}\else%
    \v@lX=\c@nt pt\v@lX=\Sc@leFact\v@lX\edef\F@ct{\repdecn@mb{\v@lX}}%
    \ifx\TeXturesonMacOSltX\special{postscriptfile #1 vscale=\F@ct\space hscale=\F@ct}%
    \else\special{psfile=#1 vscale=\F@ct\space hscale=\F@ct}\fi\fi%
    \message{[\t@xt@]}\ignorespaces}
\ctr@ld@f\def\Figdisc@rdLTS#1#2{\expandafter\Figdisc@rdLTS@#1 :#2}
\ctr@ld@f\def\Figdisc@rdLTS@#1 #2:#3{\def#3{#1}\relax\ifx#3\empty\expandafter\Figdisc@rdLTS@#2:#3\fi}
\ctr@ld@f\def\figinsertE#1{\FiginsertE@#1,:\ignorespaces}
\ctr@ld@f\def\FiginsertE@#1,#2:{{\def\t@xt@{#2}\ifx\t@xt@\empty\xdef\Sc@leFact{1}\else%
    \X@rgdeux@#2\xdef\Sc@leFact{\@rgdeux}\fi%
    \Figdisc@rdLTS{#1}{\t@xt@}\pdfximage{\t@xt@}%
    \setbox\Gb@x=\hbox{\pdfrefximage\pdflastximage}%
    \v@lX=\z@\v@lY=-\Sc@leFact\dp\Gb@x\b@undb@x{\v@lX}{\v@lY}%
    \advance\v@lX\Sc@leFact\wd\Gb@x\advance\v@lY\Sc@leFact\dp\Gb@x%
    \advance\v@lY\Sc@leFact\ht\Gb@x\b@undb@x{\v@lX}{\v@lY}%
    \v@lX=\Sc@leFact\wd\Gb@x\pdfximage width \v@lX {\t@xt@}%
    \rlap{\pdfrefximage\pdflastximage}\message{[\t@xt@]}}\ignorespaces}
\ctr@ld@f\def\X@rgdeux@#1,{\edef\@rgdeux{#1}}
\ctr@ln@m\figpt
\ctr@ld@f\def\figptDD#1:#2(#3,#4){\ifGR@cri\c@ntr@lnum{#1}%
    {\v@lX=#3\unit@\v@lY=#4\unit@\Fig@dmpt{#2}{\z@}}\ignorespaces\fi}
\ctr@ld@f\def\Fig@dmpt#1#2{\def\t@xt@{#1}\ifx\t@xt@\empty\def\B@@ltxt{\z@}%
    \else\expandafter\gdef\csname\objc@de T\endcsname{#1}\def\B@@ltxt{\@ne}\fi%
    \expandafter\xdef\csname\objc@de\endcsname{\ifitis@vect@r\C@dCl@svect%
    \else\C@dCl@spt\fi,\z@,\B@@ltxt/\the\v@lX,\the\v@lY,#2}}
\ctr@ld@f\def\C@dCl@spt{P}
\ctr@ld@f\def\C@dCl@svect{V}
\ctr@ln@m\c@@rdYZ
\ctr@ln@m\c@@rdY
\ctr@ld@f\def\figptTD#1:#2(#3,#4){\ifGR@cri\c@ntr@lnum{#1}%
    \def\c@@rdYZ{#4,0,0}\extrairelepremi@r\c@@rdY\de\c@@rdYZ%
    \extrairelepremi@r\c@@rdZ\de\c@@rdYZ%
    {\v@lX=#3\unit@\v@lY=\c@@rdY\unit@\v@lZ=\c@@rdZ\unit@\Fig@dmpt{#2}{\the\v@lZ}%
    \b@undb@xTD{\v@lX}{\v@lY}{\v@lZ}}\ignorespaces\fi}
\ctr@ln@m\Figp@intreg
\ctr@ld@f\def\Figp@intregDD#1:#2(#3,#4){\c@ntr@lnum{#1}%
    {\result@t=#4\v@lX=#3\v@lY=\result@t\Fig@dmpt{#2}{\z@}}\ignorespaces}
\ctr@ld@f\def\Figp@intregTD#1:#2(#3,#4){\c@ntr@lnum{#1}%
    \def\c@@rdYZ{#4,\z@,\z@}\extrairelepremi@r\c@@rdY\de\c@@rdYZ%
    \extrairelepremi@r\c@@rdZ\de\c@@rdYZ%
    {\v@lX=#3\v@lY=\c@@rdY\v@lZ=\c@@rdZ\Fig@dmpt{#2}{\the\v@lZ}%
    \b@undb@xTD{\v@lX}{\v@lY}{\v@lZ}}\ignorespaces}
\ctr@ln@m\figptBezier
\ctr@ld@f\def\figptBezierDD#1:#2:#3[#4,#5,#6,#7]{\ifGR@cri{\s@uvc@ntr@l\et@tfigptBezierDD%
    \FigptBezier@#3[#4,#5,#6,#7]\Figp@intregDD#1:{#2}(\v@lX,\v@lY)%
    \resetc@ntr@l\et@tfigptBezierDD}\ignorespaces\fi}
\ctr@ld@f\def\figptBezierTD#1:#2:#3[#4,#5,#6,#7]{\ifGR@cri{\s@uvc@ntr@l\et@tfigptBezierTD%
    \FigptBezier@#3[#4,#5,#6,#7]\Figp@intregTD#1:{#2}(\v@lX,\v@lY,\v@lZ)%
    \resetc@ntr@l\et@tfigptBezierTD}\ignorespaces\fi}
\ctr@ld@f\def\FigptBezier@#1[#2,#3,#4,#5]{\setc@ntr@l{2}%
    \edef\T@{#1}\v@leur=\p@\advance\v@leur-#1pt\edef\UNmT@{\repdecn@mb{\v@leur}}%
    \figptcopy-4:/#2/\figptcopy-3:/#3/\figptcopy-2:/#4/\figptcopy-1:/#5/%
    \l@mbd@un=-4 \l@mbd@de=-\thr@@\p@rtent=\m@ne\c@lDecast%
    \l@mbd@un=-4 \l@mbd@de=-\thr@@\p@rtent=-\tw@\c@lDecast%
    \l@mbd@un=-4 \l@mbd@de=-\thr@@\p@rtent=-\thr@@\c@lDecast\Figg@tXY{-4}}
\ctr@ln@m\c@lDCUn
\ctr@ld@f\def\c@lDCUnDD#1#2{\Figg@tXY{#1}\v@lX=\UNmT@\v@lX\v@lY=\UNmT@\v@lY%
    \Figg@tXYa{#2}\advance\v@lX\T@\v@lXa\advance\v@lY\T@\v@lYa%
    \Figp@intregDD#1:(\v@lX,\v@lY)}
\ctr@ld@f\def\c@lDCUnTD#1#2{\Figg@tXY{#1}\v@lX=\UNmT@\v@lX\v@lY=\UNmT@\v@lY\v@lZ=\UNmT@\v@lZ%
    \Figg@tXYa{#2}\advance\v@lX\T@\v@lXa\advance\v@lY\T@\v@lYa\advance\v@lZ\T@\v@lZa%
    \Figp@intregTD#1:(\v@lX,\v@lY,\v@lZ)}
\ctr@ld@f\def\c@lDecast{\relax\ifnum\l@mbd@un<\p@rtent\c@lDCUn{\l@mbd@un}{\l@mbd@de}%
    \advance\l@mbd@un\@ne\advance\l@mbd@de\@ne\c@lDecast\fi}
\ctr@ld@f\def\figptmap#1:#2=#3/#4/#5/{\ifGR@cri{\s@uvc@ntr@l\et@tfigptmap%
    \setc@ntr@l{2}\figvectP-1[#4,#3]\Figg@tXY{-1}%
    \pr@dMatV/#5/\figpttra#1:{#2}=#4/1,-1/%
    \resetc@ntr@l\et@tfigptmap}\ignorespaces\fi}
\ctr@ln@m\pr@dMatV
\ctr@ld@f\def\pr@dMatVDD/#1,#2;#3,#4/{\v@lXa=#1\v@lX\advance\v@lXa#2\v@lY%
    \v@lYa=#3\v@lX\advance\v@lYa#4\v@lY\Figv@ctCreg-1(\v@lXa,\v@lYa)}
\ctr@ld@f\def\pr@dMatVTD/#1,#2,#3;#4,#5,#6;#7,#8,#9/{%
    \v@lXa=#1\v@lX\advance\v@lXa#2\v@lY\advance\v@lXa#3\v@lZ%
    \v@lYa=#4\v@lX\advance\v@lYa#5\v@lY\advance\v@lYa#6\v@lZ%
    \v@lZa=#7\v@lX\advance\v@lZa#8\v@lY\advance\v@lZa#9\v@lZ%
    \Figv@ctCreg-1(\v@lXa,\v@lYa,\v@lZa)}
\ctr@ln@m\figptbary
\ctr@ld@f\def\figptbaryDD#1:#2[#3;#4]{\ifGR@cri{\edef\list@num{#3}\extrairelepremi@r\p@int\de\list@num%
    \s@mme=\z@\@ecfor\c@ef:=#4\do{\advance\s@mme\c@ef}%
    \edef\listec@ef{#4,0}\extrairelepremi@r\c@ef\de\listec@ef%
    \Figg@tXY{\p@int}\divide\v@lX\s@mme\divide\v@lY\s@mme%
    \multiply\v@lX\c@ef\multiply\v@lY\c@ef%
    \@ecfor\p@int:=\list@num\do{\extrairelepremi@r\c@ef\de\listec@ef%
           \Figg@tXYa{\p@int}\divide\v@lXa\s@mme\divide\v@lYa\s@mme%
           \multiply\v@lXa\c@ef\multiply\v@lYa\c@ef%
           \advance\v@lX\v@lXa\advance\v@lY\v@lYa}%
    \Figp@intregDD#1:{#2}(\v@lX,\v@lY)}\ignorespaces\fi}
\ctr@ld@f\def\figptbaryTD#1:#2[#3;#4]{\ifGR@cri{\edef\list@num{#3}\extrairelepremi@r\p@int\de\list@num%
    \s@mme=\z@\@ecfor\c@ef:=#4\do{\advance\s@mme\c@ef}%
    \edef\listec@ef{#4,0}\extrairelepremi@r\c@ef\de\listec@ef%
    \Figg@tXY{\p@int}\divide\v@lX\s@mme\divide\v@lY\s@mme\divide\v@lZ\s@mme%
    \multiply\v@lX\c@ef\multiply\v@lY\c@ef\multiply\v@lZ\c@ef%
    \@ecfor\p@int:=\list@num\do{\extrairelepremi@r\c@ef\de\listec@ef%
           \Figg@tXYa{\p@int}\divide\v@lXa\s@mme\divide\v@lYa\s@mme\divide\v@lZa\s@mme%
           \multiply\v@lXa\c@ef\multiply\v@lYa\c@ef\multiply\v@lZa\c@ef%
           \advance\v@lX\v@lXa\advance\v@lY\v@lYa\advance\v@lZ\v@lZa}%
    \Figp@intregTD#1:{#2}(\v@lX,\v@lY,\v@lZ)}\ignorespaces\fi}
\ctr@ld@f\def\figptbaryR#1:#2[#3;#4]{\ifGR@cri{%
    \v@leur=\z@\@ecfor\c@ef:=#4\do{\maxim@m{\v@lmax}{\c@ef pt}{-\c@ef pt}%
    \ifdim\v@lmax>\v@leur\v@leur=\v@lmax\fi}%
    \ifdim\v@leur<\p@\f@ctech=\@M\else\ifdim\v@leur<\t@n\p@\f@ctech=\@m\else%
    \ifdim\v@leur<\c@nt\p@\f@ctech=\c@nt\else\ifdim\v@leur<\@m\p@\f@ctech=\t@n\else%
    \f@ctech=\@ne\fi\fi\fi\fi%
    \def\listec@ef{0}%
    \@ecfor\c@ef:=#4\do{\sc@lec@nvRI{\c@ef pt}\edef\listec@ef{\listec@ef,\the\s@mme}}%
    \extrairelepremi@r\c@ef\de\listec@ef\figptbary#1:#2[#3;\listec@ef]}\ignorespaces\fi}
\ctr@ld@f\def\sc@lec@nvRI#1{\v@leur=#1\p@rtentiere{\s@mme}{\v@leur}\advance\v@leur-\s@mme\p@%
    \multiply\v@leur\f@ctech\p@rtentiere{\p@rtent}{\v@leur}%
    \multiply\s@mme\f@ctech\advance\s@mme\p@rtent}
\ctr@ln@m\figptcirc
\ctr@ld@f\def\figptcircDD#1:#2:#3;#4(#5){\ifGR@cri{\s@uvc@ntr@l\et@tfigptcircDD%
    \c@lptellDD#1:{#2}:#3;#4,#4(#5)\resetc@ntr@l\et@tfigptcircDD}\ignorespaces\fi}
\ctr@ld@f\def\figptcircTD#1:#2:#3,#4,#5;#6(#7){\ifGR@cri{\s@uvc@ntr@l\et@tfigptcircTD%
    \setc@ntr@l{2}\c@lExtAxes#3,#4,#5(#6)\figptellP#1:{#2}:#3,-4,-5(#7)%
    \resetc@ntr@l\et@tfigptcircTD}\ignorespaces\fi}
\ctr@ln@m\figptcircumcenter
\ctr@ld@f\def\figptcircumcenterDD#1:#2[#3,#4,#5]{\ifGR@cri{\s@uvc@ntr@l\et@tfigptcircumcenterDD%
    \setc@ntr@l{2}\figvectNDD-5[#3,#4]\figptbaryDD-3:[#3,#4;1,1]%
                  \figvectNDD-6[#4,#5]\figptbaryDD-4:[#4,#5;1,1]%
    \resetc@ntr@l{2}\inters@cDD#1:{#2}[-3,-5;-4,-6]%
    \resetc@ntr@l\et@tfigptcircumcenterDD}\ignorespaces\fi}
\ctr@ld@f\def\figptcircumcenterTD#1:#2[#3,#4,#5]{\ifGR@cri{\s@uvc@ntr@l\et@tfigptcircumcenterTD%
    \setc@ntr@l{2}\figvectNTD-1[#3,#4,#5]%
    \figvectPTD-3[#3,#4]\figvectNVTD-5[-1,-3]\figptbaryTD-3:[#3,#4;1,1]%
    \figvectPTD-4[#4,#5]\figvectNVTD-6[-1,-4]\figptbaryTD-4:[#4,#5;1,1]%
    \resetc@ntr@l{2}\inters@cTD#1:{#2}[-3,-5;-4,-6]%
    \resetc@ntr@l\et@tfigptcircumcenterTD}\ignorespaces\fi}
\ctr@ln@m\figptcopy
\ctr@ld@f\def\figptcopyDD#1:#2/#3/{\ifGR@cri{\Figg@tXY{#3}%
    \Figp@intregDD#1:{#2}(\v@lX,\v@lY)}\ignorespaces\fi}
\ctr@ld@f\def\figptcopyTD#1:#2/#3/{\ifGR@cri{\Figg@tXY{#3}%
    \Figp@intregTD#1:{#2}(\v@lX,\v@lY,\v@lZ)}\ignorespaces\fi}
\ctr@ln@m\figptcurvcenter
\ctr@ld@f\def\figptcurvcenterDD#1:#2:#3[#4,#5,#6,#7]{\ifGR@cri{\s@uvc@ntr@l\et@tfigptcurvcenterDD%
    \setc@ntr@l{2}\c@lcurvradDD#3[#4,#5,#6,#7]\edef\Sprim@{\repdecn@mb{\result@t}}%
    \figptBezierDD-1::#3[#4,#5,#6,#7]\figpttraDD#1:{#2}=-1/\Sprim@,-5/%
    \resetc@ntr@l\et@tfigptcurvcenterDD}\ignorespaces\fi}
\ctr@ld@f\def\figptcurvcenterTD#1:#2:#3[#4,#5,#6,#7]{\ifGR@cri{\s@uvc@ntr@l\et@tfigptcurvcenterTD%
    \setc@ntr@l{2}\figvectDBezierTD -5:1,#3[#4,#5,#6,#7]%
    \figvectDBezierTD -6:2,#3[#4,#5,#6,#7]\vecunit@TD{-5}{-5}%
    \edef\Sprim@{\repdecn@mb{\result@t}}\figvectNVTD-1[-6,-5]%
    \figvectNVTD-5[-5,-1]\c@lproscalTD\v@leur[-6,-5]%
    \invers@{\v@leur}{\v@leur}\v@leur=\Sprim@\v@leur\v@leur=\Sprim@\v@leur%
    \figptBezierTD-1::#3[#4,#5,#6,#7]\edef\Sprim@{\repdecn@mb{\v@leur}}%
    \figpttraTD#1:{#2}=-1/\Sprim@,-5/\resetc@ntr@l\et@tfigptcurvcenterTD}\ignorespaces\fi}
\ctr@ld@f\def\c@lcurvradDD#1[#2,#3,#4,#5]{{\figvectDBezierDD -5:1,#1[#2,#3,#4,#5]%
    \figvectDBezierDD -6:2,#1[#2,#3,#4,#5]\vecunit@DD{-5}{-5}%
    \edef\Sprim@{\repdecn@mb{\result@t}}\figvectNVDD-5[-5]\c@lproscalDD\v@leur[-6,-5]%
    \invers@{\v@leur}{\v@leur}\v@leur=\Sprim@\v@leur\v@leur=\Sprim@\v@leur%
    \global\result@t=\v@leur}}
\ctr@ln@m\figptell
\ctr@ld@f\def\figptellDD#1:#2:#3;#4,#5(#6,#7){\ifGR@cri{\s@uvc@ntr@l\et@tfigptell%
    \c@lptellDD#1::#3;#4,#5(#6)\figptrotDD#1:{#2}=#1/#3,#7/%
    \resetc@ntr@l\et@tfigptell}\ignorespaces\fi}
\ctr@ld@f\def\c@lptellDD#1:#2:#3;#4,#5(#6){\c@ssin{\C@}{\S@}{#6}\v@lmin=\C@ pt\v@lmax=\S@ pt%
    \v@lmin=#4\v@lmin\v@lmax=#5\v@lmax%
    \edef\Xc@mp{\repdecn@mb{\v@lmin}}\edef\Yc@mp{\repdecn@mb{\v@lmax}}%
    \setc@ntr@l{2}\figvectC-1(\Xc@mp,\Yc@mp)\figpttraDD#1:{#2}=#3/1,-1/}
\ctr@ld@f\def\figptellP#1:#2:#3,#4,#5(#6){\ifGR@cri{\s@uvc@ntr@l\et@tfigptellP%
    \setc@ntr@l{2}\figvectP-1[#3,#4]\figvectP-2[#3,#5]%
    \v@leur=#6pt\c@lptellP{#3}{-1}{-2}\figptcopy#1:{#2}/-3/%
    \resetc@ntr@l\et@tfigptellP}\ignorespaces\fi}
\ctr@ln@m\@ngle
\ctr@ld@f\def\c@lptellP#1#2#3{\edef\@ngle{\repdecn@mb\v@leur}\c@ssin{\C@}{\S@}{\@ngle}%
    \figpttra-3:=#1/\C@,#2/\figpttra-3:=-3/\S@,#3/}
\ctr@ln@m\figptendnormal
\ctr@ld@f\def\figptendnormalDD#1:#2:#3,#4[#5,#6]{\ifGR@cri{\s@uvc@ntr@l\et@tfigptendnormal%
    \Figg@tXYa{#5}\Figg@tXY{#6}%
    \advance\v@lX-\v@lXa\advance\v@lY-\v@lYa%
    \setc@ntr@l{2}\Figv@ctCreg-1(\v@lX,\v@lY)\vecunit@{-1}{-1}\Figg@tXY{-1}%
    \delt@=#3\unit@\maxim@m{\delt@}{\delt@}{-\delt@}\edef\l@ngueur{\repdecn@mb{\delt@}}%
    \v@lX=\l@ngueur\v@lX\v@lY=\l@ngueur\v@lY%
    \delt@=\p@\advance\delt@-#4pt\edef\l@ngueur{\repdecn@mb{\delt@}}%
    \figptbaryR-1:[#5,#6;#4,\l@ngueur]\Figg@tXYa{-1}%
    \advance\v@lXa\v@lY\advance\v@lYa-\v@lX%
    \setc@ntr@l{1}\Figp@intregDD#1:{#2}(\v@lXa,\v@lYa)\resetc@ntr@l\et@tfigptendnormal}%
    \ignorespaces\fi}
\ctr@ld@f\def\figptexcenter#1:#2[#3,#4,#5]{\ifGR@cri{\let@xte={-}% for computational macro
    \Figptexinsc@nter#1:#2[#3,#4,#5]}\ignorespaces\fi}
\ctr@ld@f\def\figptincenter#1:#2[#3,#4,#5]{\ifGR@cri{\let@xte={}% for computational macro
    \Figptexinsc@nter#1:#2[#3,#4,#5]}\ignorespaces\fi}
\ctr@ld@f\let\figptinscribedcenter=\figptincenter% for compatibility with older versions
\ctr@ld@f\def\Figptexinsc@nter#1:#2[#3,#4,#5]{%
    \figgetdist\LA@[#4,#5]\figgetdist\LB@[#3,#5]\figgetdist\LC@[#3,#4]%
    \figptbaryR#1:{#2}[#3,#4,#5;\the\let@xte\LA@,\LB@,\LC@]}
\ctr@ln@m\figptinterlineplane
\ctr@ld@f\def\figptinterlineplaneDD{\un@v@ilable{figptinterlineplane}}
\ctr@ld@f\def\figptinterlineplaneTD#1:#2[#3,#4;#5,#6]{\ifGR@cri{\s@uvc@ntr@l\et@tfigptinterlineplane%
    \setc@ntr@l{2}\figvectPTD-1[#3,#5]\vecunit@TD{-2}{#6}%
    \r@pPSTD\v@leur[-2,-1,#4]\edef\v@lcoef{\repdecn@mb{\v@leur}}%
    \figpttraTD#1:{#2}=#3/\v@lcoef,#4/\resetc@ntr@l\et@tfigptinterlineplane}\ignorespaces\fi}
\ctr@ln@m\figptorthocenter
\ctr@ld@f\def\figptorthocenterDD#1:#2[#3,#4,#5]{\ifGR@cri{\s@uvc@ntr@l\et@tfigptorthocenterDD%
    \setc@ntr@l{2}\figvectNDD-3[#3,#4]\figvectNDD-4[#4,#5]%
    \resetc@ntr@l{2}\inters@cDD#1:{#2}[#5,-3;#3,-4]%
    \resetc@ntr@l\et@tfigptorthocenterDD}\ignorespaces\fi}
\ctr@ld@f\def\figptorthocenterTD#1:#2[#3,#4,#5]{\ifGR@cri{\s@uvc@ntr@l\et@tfigptorthocenterTD%
    \setc@ntr@l{2}\figvectNTD-1[#3,#4,#5]%
    \figvectPTD-2[#3,#4]\figvectNVTD-3[-1,-2]%
    \figvectPTD-2[#4,#5]\figvectNVTD-4[-1,-2]%
    \resetc@ntr@l{2}\inters@cTD#1:{#2}[#5,-3;#3,-4]%
    \resetc@ntr@l\et@tfigptorthocenterTD}\ignorespaces\fi}
\ctr@ln@m\figptorthoprojline
\ctr@ld@f\def\figptorthoprojlineDD#1:#2=#3/#4,#5/{\ifGR@cri{\s@uvc@ntr@l\et@tfigptorthoprojlineDD%
    \setc@ntr@l{2}\figvectPDD-3[#4,#5]\figvectNVDD-4[-3]\resetc@ntr@l{2}%
    \inters@cDD#1:{#2}[#3,-4;#4,-3]\resetc@ntr@l\et@tfigptorthoprojlineDD}\ignorespaces\fi}
\ctr@ld@f\def\figptorthoprojlineTD#1:#2=#3/#4,#5/{\ifGR@cri{\s@uvc@ntr@l\et@tfigptorthoprojlineTD%
    \setc@ntr@l{2}\figvectPTD-1[#4,#3]\figvectPTD-2[#4,#5]\vecunit@TD{-2}{-2}%
    \c@lproscalTD\v@leur[-1,-2]\edef\v@lcoef{\repdecn@mb{\v@leur}}%
    \figpttraTD#1:{#2}=#4/\v@lcoef,-2/\resetc@ntr@l\et@tfigptorthoprojlineTD}\ignorespaces\fi}
\ctr@ln@m\figptorthoprojplane
\ctr@ld@f\def\figptorthoprojplaneDD{\un@v@ilable{figptorthoprojplane}}
\ctr@ld@f\def\figptorthoprojplaneTD#1:#2=#3/#4,#5/{\ifGR@cri{\s@uvc@ntr@l\et@tfigptorthoprojplane%
    \setc@ntr@l{2}\figvectPTD-1[#3,#4]\vecunit@TD{-2}{#5}%
    \c@lproscalTD\v@leur[-1,-2]\edef\v@lcoef{\repdecn@mb{\v@leur}}%
    \figpttraTD#1:{#2}=#3/\v@lcoef,-2/\resetc@ntr@l\et@tfigptorthoprojplane}\ignorespaces\fi}
\ctr@ld@f\def\figpthom#1:#2=#3/#4,#5/{\ifGR@cri{\s@uvc@ntr@l\et@tfigpthom%
    \setc@ntr@l{2}\figvectP-1[#4,#3]\figpttra#1:{#2}=#4/#5,-1/%
    \resetc@ntr@l\et@tfigpthom}\ignorespaces\fi}
\ctr@ld@f\def\figptinv#1:#2=#3/#4,#5/{\ifGR@cri{\s@uvc@ntr@l\et@tfigptinv%
    \setc@ntr@l{2}\figvectP-1[#4,#3]\Figg@tXY{-1}%
    \getredf@ctB\f@ctech\n@rmeucC{\delt@}{-1}%
    \delt@=\ptT@unit@\delt@\delt@=\ptT@unit@\delt@%
    \invers@{\delt@}{\delt@}\multiply\f@ctech\f@ctech\divide\delt@\f@ctech%
    \delt@=#5\delt@\edef\v@lcoef{\repdecn@mb{\delt@}}\figpttra#1:{#2}=#4/\v@lcoef,-1/%
    \resetc@ntr@l\et@tfigptinv}\ignorespaces\fi}
\ctr@ln@m\figptrot
\ctr@ld@f\def\figptrotDD#1:#2=#3/#4,#5/{\ifGR@cri{\s@uvc@ntr@l\et@tfigptrotDD%
    \c@ssin{\C@}{\S@}{#5}\setc@ntr@l{2}\figvectPDD-1[#4,#3]\Figg@tXY{-1}%
    \v@lXa=\C@\v@lX\advance\v@lXa-\S@\v@lY%
    \v@lYa=\S@\v@lX\advance\v@lYa\C@\v@lY%
    \Figv@ctCreg-1(\v@lXa,\v@lYa)\figpttraDD#1:{#2}=#4/1,-1/%
    \resetc@ntr@l\et@tfigptrotDD}\ignorespaces\fi}
\ctr@ld@f\def\figptrotTD#1:#2=#3/#4,#5,#6/{\ifGR@cri{\s@uvc@ntr@l\et@tfigptrotTD%
    \c@ssin{\C@}{\S@}{#5}%
    \setc@ntr@l{2}\figptorthoprojplaneTD-3:=#4/#3,#6/\figvectPTD-2[-3,#3]%
    \n@rmeucTD\v@leur{-2}\ifdim\v@leur<\Cepsil@n\Figg@tXYa{#3}\else%
    \edef\v@lcoef{\repdecn@mb{\v@leur}}\figvectNVTD-1[#6,-2]%
    \Figg@tXYa{-1}\v@lXa=\v@lcoef\v@lXa\v@lYa=\v@lcoef\v@lYa\v@lZa=\v@lcoef\v@lZa%
    \v@lXa=\S@\v@lXa\v@lYa=\S@\v@lYa\v@lZa=\S@\v@lZa\Figg@tXY{-2}%
    \advance\v@lXa\C@\v@lX\advance\v@lYa\C@\v@lY\advance\v@lZa\C@\v@lZ%
    \Figg@tXY{-3}\advance\v@lXa\v@lX\advance\v@lYa\v@lY\advance\v@lZa\v@lZ\fi%
    \Figp@intregTD#1:{#2}(\v@lXa,\v@lYa,\v@lZa)\resetc@ntr@l\et@tfigptrotTD}\ignorespaces\fi}
\ctr@ln@m\figptsym
\ctr@ld@f\def\figptsymDD#1:#2=#3/#4,#5/{\ifGR@cri{\s@uvc@ntr@l\et@tfigptsymDD%
    \resetc@ntr@l{2}\figptorthoprojlineDD-5:=#3/#4,#5/\figvectPDD-2[#3,-5]%
    \figpttraDD#1:{#2}=#3/2,-2/\resetc@ntr@l\et@tfigptsymDD}\ignorespaces\fi}
\ctr@ld@f\def\figptsymTD#1:#2=#3/#4,#5/{\ifGR@cri{\s@uvc@ntr@l\et@tfigptsymTD%
    \resetc@ntr@l{2}\figptorthoprojplaneTD-3:=#3/#4,#5/\figvectPTD-2[#3,-3]%
    \figpttraTD#1:{#2}=#3/2,-2/\resetc@ntr@l\et@tfigptsymTD}\ignorespaces\fi}
\ctr@ln@m\figpttra
\ctr@ld@f\def\figpttraDD#1:#2=#3/#4,#5/{\ifGR@cri{\Figg@tXYa{#5}\v@lXa=#4\v@lXa\v@lYa=#4\v@lYa%
    \Figg@tXY{#3}\advance\v@lX\v@lXa\advance\v@lY\v@lYa%
    \Figp@intregDD#1:{#2}(\v@lX,\v@lY)}\ignorespaces\fi}
\ctr@ld@f\def\figpttraTD#1:#2=#3/#4,#5/{\ifGR@cri{\Figg@tXYa{#5}\v@lXa=#4\v@lXa\v@lYa=#4\v@lYa%
    \v@lZa=#4\v@lZa\Figg@tXY{#3}\advance\v@lX\v@lXa\advance\v@lY\v@lYa%
    \advance\v@lZ\v@lZa\Figp@intregTD#1:{#2}(\v@lX,\v@lY,\v@lZ)}\ignorespaces\fi}
\ctr@ln@m\figpttraC
\ctr@ld@f\def\figpttraCDD#1:#2=#3/#4,#5/{\ifGR@cri{\v@lXa=#4\unit@\v@lYa=#5\unit@%
    \Figg@tXY{#3}\advance\v@lX\v@lXa\advance\v@lY\v@lYa%
    \Figp@intregDD#1:{#2}(\v@lX,\v@lY)}\ignorespaces\fi}
\ctr@ld@f\def\figpttraCTD#1:#2=#3/#4,#5,#6/{\ifGR@cri{\v@lXa=#4\unit@\v@lYa=#5\unit@\v@lZa=#6\unit@%
    \Figg@tXY{#3}\advance\v@lX\v@lXa\advance\v@lY\v@lYa\advance\v@lZ\v@lZa%
    \Figp@intregTD#1:{#2}(\v@lX,\v@lY,\v@lZ)}\ignorespaces\fi}
\ctr@ld@f\def\figptsaxes#1:#2(#3){\ifGR@cri{\an@lys@xes#3,:\ifx\t@xt@\empty%
    \ifTr@isDim\Figpts@xes#1:#2(0,#3,0,#3,0,#3)\else\Figpts@xes#1:#2(0,#3,0,#3)\fi%
    \else\Figpts@xes#1:#2(#3)\fi}\ignorespaces\fi}
\ctr@ln@m\Figpts@xes
\ctr@ld@f\def\Figpts@xesDD#1:#2(#3,#4,#5,#6){%
    \s@mme=#1\figpttraC\the\s@mme:$x$=#2/#4,0/%
    \advance\s@mme\@ne\figpttraC\the\s@mme:$y$=#2/0,#6/}
\ctr@ld@f\def\Figpts@xesTD#1:#2(#3,#4,#5,#6,#7,#8){%
    \s@mme=#1\figpttraC\the\s@mme:$x$=#2/#4,0,0/%
    \advance\s@mme\@ne\figpttraC\the\s@mme:$y$=#2/0,#6,0/%
    \advance\s@mme\@ne\figpttraC\the\s@mme:$z$=#2/0,0,#8/}
\ctr@ld@f\def\figptsmap#1=#2/#3/#4/{\ifGR@cri{\s@uvc@ntr@l\et@tfigptsmap%
    \setc@ntr@l{2}\def\list@num{#2}\s@mme=#1%
    \@ecfor\p@int:=\list@num\do{\figvectP-1[#3,\p@int]\Figg@tXY{-1}%
    \pr@dMatV/#4/\figpttra\the\s@mme:=#3/1,-1/\advance\s@mme\@ne}%
    \resetc@ntr@l\et@tfigptsmap}\ignorespaces\fi}
\ctr@ln@m\figptscontrol
\ctr@ld@f\def\figptscontrolDD#1[#2,#3,#4,#5]{\ifGR@cri{\s@uvc@ntr@l\et@tfigptscontrolDD\setc@ntr@l{2}%
    \v@lX=\z@\v@lY=\z@\Figtr@nptDD{-5}{#2}\Figtr@nptDD{2}{#5}%
    \divide\v@lX\@vi\divide\v@lY\@vi%
    \Figtr@nptDD{3}{#3}\Figtr@nptDD{-1.5}{#4}\Figp@intregDD-1:(\v@lX,\v@lY)%
    \v@lX=\z@\v@lY=\z@\Figtr@nptDD{2}{#2}\Figtr@nptDD{-5}{#5}%
    \divide\v@lX\@vi\divide\v@lY\@vi\Figtr@nptDD{-1.5}{#3}\Figtr@nptDD{3}{#4}%
    \s@mme=#1\advance\s@mme\@ne\Figp@intregDD\the\s@mme:(\v@lX,\v@lY)%
    \figptcopyDD#1:/-1/\resetc@ntr@l\et@tfigptscontrolDD}\ignorespaces\fi}
\ctr@ld@f\def\figptscontrolTD#1[#2,#3,#4,#5]{\ifGR@cri{\s@uvc@ntr@l\et@tfigptscontrolTD\setc@ntr@l{2}%
    \v@lX=\z@\v@lY=\z@\v@lZ=\z@\Figtr@nptTD{-5}{#2}\Figtr@nptTD{2}{#5}%
    \divide\v@lX\@vi\divide\v@lY\@vi\divide\v@lZ\@vi%
    \Figtr@nptTD{3}{#3}\Figtr@nptTD{-1.5}{#4}\Figp@intregTD-1:(\v@lX,\v@lY,\v@lZ)%
    \v@lX=\z@\v@lY=\z@\v@lZ=\z@\Figtr@nptTD{2}{#2}\Figtr@nptTD{-5}{#5}%
    \divide\v@lX\@vi\divide\v@lY\@vi\divide\v@lZ\@vi\Figtr@nptTD{-1.5}{#3}\Figtr@nptTD{3}{#4}%
    \s@mme=#1\advance\s@mme\@ne\Figp@intregTD\the\s@mme:(\v@lX,\v@lY,\v@lZ)%
    \figptcopyTD#1:/-1/\resetc@ntr@l\et@tfigptscontrolTD}\ignorespaces\fi}
\ctr@ld@f\def\Figtr@nptDD#1#2{\Figg@tXYa{#2}\v@lXa=#1\v@lXa\v@lYa=#1\v@lYa%
    \advance\v@lX\v@lXa\advance\v@lY\v@lYa}
\ctr@ld@f\def\Figtr@nptTD#1#2{\Figg@tXYa{#2}\v@lXa=#1\v@lXa\v@lYa=#1\v@lYa\v@lZa=#1\v@lZa%
    \advance\v@lX\v@lXa\advance\v@lY\v@lYa\advance\v@lZ\v@lZa}
\ctr@ld@f\def\figptscontrolcurve#1,#2[#3]{\ifGR@cri{\s@uvc@ntr@l\et@tfigptscontrolcurve%
    \def\list@num{#3}\extrairelepremi@r\Ak@\de\list@num%
    \extrairelepremi@r\Ai@\de\list@num\extrairelepremi@r\Aj@\de\list@num%
    \s@mme=#1\figptcopy\the\s@mme:/\Ai@/%
    \setc@ntr@l{2}\figvectP -1[\Ak@,\Aj@]%
    \@ecfor\Ak@:=\list@num\do{\advance\s@mme\@ne\figpttra\the\s@mme:=\Ai@/\curv@roundness,-1/%
       \figvectP -1[\Ai@,\Ak@]\advance\s@mme\@ne\figpttra\the\s@mme:=\Aj@/-\curv@roundness,-1/%
       \advance\s@mme\@ne\figptcopy\the\s@mme:/\Aj@/%
       \edef\Ai@{\Aj@}\edef\Aj@{\Ak@}}\advance\s@mme-#1\divide\s@mme\thr@@%
       \xdef#2{\the\s@mme}%
    \resetc@ntr@l\et@tfigptscontrolcurve}\ignorespaces\fi}
\ctr@ln@m\figptsintercirc
\ctr@ld@f\def\figptsintercircDD#1[#2,#3;#4,#5]{\ifGR@cri{\s@uvc@ntr@l\et@tfigptsintercircDD%
    \setc@ntr@l{2}\let\c@lNVintc=\c@lNVintcDD\Figptsintercirc@#1[#2,#3;#4,#5]%    
    \resetc@ntr@l\et@tfigptsintercircDD}\ignorespaces\fi}
\ctr@ld@f\def\figptsintercircTD#1[#2,#3;#4,#5;#6]{\ifGR@cri{\s@uvc@ntr@l\et@tfigptsintercircTD%
    \setc@ntr@l{2}\let\c@lNVintc=\c@lNVintcTD\vecunitC@TD[#2,#6]%
    \Figv@ctCreg-3(\v@lX,\v@lY,\v@lZ)\Figptsintercirc@#1[#2,#3;#4,#5]%
    \resetc@ntr@l\et@tfigptsintercircTD}\ignorespaces\fi}
\ctr@ld@f\def\Figptsintercirc@#1[#2,#3;#4,#5]{\figvectP-1[#2,#4]%
    \vecunit@{-1}{-1}\delt@=\result@t\f@ctech=\result@tent%
    \s@mme=#1\advance\s@mme\@ne\figptcopy#1:/#2/\figptcopy\the\s@mme:/#4/%
    \ifdim\delt@=\z@\else%
    \v@lmin=#3\unit@\v@lmax=#5\unit@\v@leur=\v@lmin\advance\v@leur\v@lmax%
    \ifdim\v@leur>\delt@%
    \v@leur=\v@lmin\advance\v@leur-\v@lmax\maxim@m{\v@leur}{\v@leur}{-\v@leur}%
    \ifdim\v@leur<\delt@%
    \divide\v@lmin\f@ctech\divide\v@lmax\f@ctech\divide\delt@\f@ctech%
    \v@lmin=\repdecn@mb{\v@lmin}\v@lmin\v@lmax=\repdecn@mb{\v@lmax}\v@lmax%
    \invers@{\v@leur}{\delt@}\advance\v@lmax-\v@lmin%
    \v@lmax=-\repdecn@mb{\v@leur}\v@lmax\advance\delt@\v@lmax\delt@=.5\delt@%
    \v@lmax=\delt@\multiply\v@lmax\f@ctech%
    \edef\t@ille{\repdecn@mb{\v@lmax}}\figpttra-2:=#2/\t@ille,-1/%
    \delt@=\repdecn@mb{\delt@}\delt@\advance\v@lmin-\delt@%
    \sqrt@{\v@leur}{\v@lmin}\multiply\v@leur\f@ctech\edef\t@ille{\repdecn@mb{\v@leur}}%
    \c@lNVintc\figpttra#1:=-2/-\t@ille,-1/\figpttra\the\s@mme:=-2/\t@ille,-1/\fi\fi\fi}
\ctr@ld@f\def\c@lNVintcDD{\Figg@tXY{-1}\Figv@ctCreg-1(-\v@lY,\v@lX)} % <=> \figvectNVDD-1[-1]
\ctr@ld@f\def\c@lNVintcTD{{\Figg@tXY{-3}\v@lmin=\v@lX\v@lmax=\v@lY\v@leur=\v@lZ%
    \Figg@tXY{-1}\c@lprovec{-3}\vecunit@{-3}{-3}% <=> \figvectNVTD-3[-1,-3]\vecunit@{-3}{-3}
    \Figg@tXY{-1}\v@lmin=\v@lX\v@lmax=\v@lY%
    \v@leur=\v@lZ\Figg@tXY{-3}\c@lprovec{-1}}} % <=> \figvectNVTD-1[-3,-1]
\ctr@ln@m\figptsinterlinell
\ctr@ld@f\def\figptsinterlinellDD#1[#2,#3,#4,#5;#6,#7]{\ifGR@cri{\s@uvc@ntr@l\et@tfigptsinterlinellDD%
    \figptcopy#1:/#6/\s@mme=#1\advance\s@mme\@ne\figptcopy\the\s@mme:/#7/%
    \v@lmin=#3\unit@\v@lmax=#4\unit@% a, b
    \setc@ntr@l{2}\figptbaryDD-4:[#6,#7;1,1]\figptsrotDD-3=-4,#7/#2,-#5/% D et rotation
    \Figg@tXY{-3}\Figg@tXYa{#2}\advance\v@lX-\v@lXa\advance\v@lY-\v@lYa% alpha, beta
    \figvectP-1[-3,-2]\Figg@tXYa{-1}\figvectP-3[-4,#7]\Figptsint@rLE{#1}% u1, u2
    \resetc@ntr@l\et@tfigptsinterlinellDD}\ignorespaces\fi}
\ctr@ld@f\def\figptsinterlinellP#1[#2,#3,#4;#5,#6]{\ifGR@cri{\s@uvc@ntr@l\et@tfigptsinterlinellP%
    \figptcopy#1:/#5/\s@mme=#1\advance\s@mme\@ne\figptcopy\the\s@mme:/#6/\setc@ntr@l{2}%
    \figvectP-1[#2,#3]\vecunit@{-1}{-1}\v@lmin=\result@t% a
    \figvectP-2[#2,#4]\vecunit@{-2}{-2}\v@lmax=\result@t% b
    \figptbary-4:[#5,#6;1,1]% D
    \figvectP-3[#2,-4]\c@lproscal\v@lX[-3,-1]\c@lproscal\v@lY[-3,-2]% alpha, beta
    \figvectP-3[-4,#6]\c@lproscal\v@lXa[-3,-1]\c@lproscal\v@lYa[-3,-2]% u1, u2
    \Figptsint@rLE{#1}\resetc@ntr@l\et@tfigptsinterlinellP}\ignorespaces\fi}
\ctr@ld@f\def\Figptsint@rLE#1{%
    \getredf@ctDD\f@ctech(\v@lmin,\v@lmax)%
    \getredf@ctDD\p@rtent(\v@lX,\v@lY)\ifnum\p@rtent>\f@ctech\f@ctech=\p@rtent\fi%
    \getredf@ctDD\p@rtent(\v@lXa,\v@lYa)\ifnum\p@rtent>\f@ctech\f@ctech=\p@rtent\fi%
    \divide\v@lmin\f@ctech\divide\v@lmax\f@ctech\divide\v@lX\f@ctech\divide\v@lY\f@ctech%
    \divide\v@lXa\f@ctech\divide\v@lYa\f@ctech%
    \c@rre=\repdecn@mb\v@lXa\v@lmax\mili@u=\repdecn@mb\v@lYa\v@lmin%
    \getredf@ctDD\f@ctech(\c@rre,\mili@u)%
    \c@rre=\repdecn@mb\v@lX\v@lmax\mili@u=\repdecn@mb\v@lY\v@lmin%
    \getredf@ctDD\p@rtent(\c@rre,\mili@u)\ifnum\p@rtent>\f@ctech\f@ctech=\p@rtent\fi%
    \divide\v@lmin\f@ctech\divide\v@lmax\f@ctech\divide\v@lX\f@ctech\divide\v@lY\f@ctech%
    \divide\v@lXa\f@ctech\divide\v@lYa\f@ctech%
    \v@lmin=\repdecn@mb{\v@lmin}\v@lmin\v@lmax=\repdecn@mb{\v@lmax}\v@lmax%
    \edef\G@xde{\repdecn@mb\v@lmin}\edef\P@xde{\repdecn@mb\v@lmax}%
    \c@rre=-\v@lmax\v@leur=\repdecn@mb\v@lY\v@lY\advance\c@rre\v@leur\c@rre=\G@xde\c@rre%
    \v@leur=\repdecn@mb\v@lX\v@lX\v@leur=\P@xde\v@leur\advance\c@rre\v@leur% C
    \v@lmin=\repdecn@mb\v@lYa\v@lmin\v@lmax=\repdecn@mb\v@lXa\v@lmax%
    \mili@u=\repdecn@mb\v@lX\v@lmax\advance\mili@u\repdecn@mb\v@lY\v@lmin% B
    \v@lmax=\repdecn@mb\v@lXa\v@lmax\advance\v@lmax\repdecn@mb\v@lYa\v@lmin% A
    \ifdim\v@lmax>\epsil@n%
    \maxim@m{\v@leur}{\c@rre}{-\c@rre}\maxim@m{\v@lmin}{\mili@u}{-\mili@u}%
    \maxim@m{\v@leur}{\v@leur}{\v@lmin}\maxim@m{\v@lmin}{\v@lmax}{-\v@lmax}%
    \maxim@m{\v@leur}{\v@leur}{\v@lmin}\p@rtentiere{\p@rtent}{\v@leur}\advance\p@rtent\@ne%
    \divide\c@rre\p@rtent\divide\mili@u\p@rtent\divide\v@lmax\p@rtent%
    \delt@=\repdecn@mb{\mili@u}\mili@u\v@leur=\repdecn@mb{\v@lmax}\c@rre%
    \advance\delt@-\v@leur\ifdim\delt@<\z@\else\sqrt@\delt@\delt@%
    \invers@\v@lmax\v@lmax\edef\Uns@rAp{\repdecn@mb\v@lmax}%
    \v@leur=-\mili@u\advance\v@leur-\delt@\v@leur=\Uns@rAp\v@leur%
    \edef\t@ille{\repdecn@mb\v@leur}\figpttra#1:=-4/\t@ille,-3/\s@mme=#1\advance\s@mme\@ne%
    \v@leur=-\mili@u\advance\v@leur\delt@\v@leur=\Uns@rAp\v@leur%
    \edef\t@ille{\repdecn@mb\v@leur}\figpttra\the\s@mme:=-4/\t@ille,-3/\fi\fi}
\ctr@ln@m\figptsorthoprojline
\ctr@ld@f\def\figptsorthoprojlineDD#1=#2/#3,#4/{\ifGR@cri{\s@uvc@ntr@l\et@tfigptsorthoprojlineDD%
    \setc@ntr@l{2}\figvectPDD-3[#3,#4]\figvectNVDD-4[-3]\resetc@ntr@l{2}%
    \def\list@num{#2}\s@mme=#1\@ecfor\p@int:=\list@num\do{%
    \inters@cDD\the\s@mme:[\p@int,-4;#3,-3]\advance\s@mme\@ne}%
    \resetc@ntr@l\et@tfigptsorthoprojlineDD}\ignorespaces\fi}
\ctr@ld@f\def\figptsorthoprojlineTD#1=#2/#3,#4/{\ifGR@cri{\s@uvc@ntr@l\et@tfigptsorthoprojlineTD%
    \setc@ntr@l{2}\figvectPTD-2[#3,#4]\vecunit@TD{-2}{-2}%
    \def\list@num{#2}\s@mme=#1\@ecfor\p@int:=\list@num\do{%
    \figvectPTD-1[#3,\p@int]\c@lproscalTD\v@leur[-1,-2]%
    \edef\v@lcoef{\repdecn@mb{\v@leur}}\figpttraTD\the\s@mme:=#3/\v@lcoef,-2/%
    \advance\s@mme\@ne}\resetc@ntr@l\et@tfigptsorthoprojlineTD}\ignorespaces\fi}
\ctr@ln@m\figptsorthoprojplane
\ctr@ld@f\def\figptsorthoprojplaneDD{\un@v@ilable{figptsorthoprojplane}}
\ctr@ld@f\def\figptsorthoprojplaneTD#1=#2/#3,#4/{\ifGR@cri{\s@uvc@ntr@l\et@tfigptsorthoprojplane%
    \setc@ntr@l{2}\vecunit@TD{-2}{#4}%
    \def\list@num{#2}\s@mme=#1\@ecfor\p@int:=\list@num\do{\figvectPTD-1[\p@int,#3]%
    \c@lproscalTD\v@leur[-1,-2]\edef\v@lcoef{\repdecn@mb{\v@leur}}%
    \figpttraTD\the\s@mme:=\p@int/\v@lcoef,-2/\advance\s@mme\@ne}%
    \resetc@ntr@l\et@tfigptsorthoprojplane}\ignorespaces\fi}
\ctr@ld@f\def\figptshom#1=#2/#3,#4/{\ifGR@cri{\s@uvc@ntr@l\et@tfigptshom%
    \setc@ntr@l{2}\def\list@num{#2}\s@mme=#1%
    \@ecfor\p@int:=\list@num\do{\figvectP-1[#3,\p@int]%
    \figpttra\the\s@mme:=#3/#4,-1/\advance\s@mme\@ne}%
    \resetc@ntr@l\et@tfigptshom}\ignorespaces\fi}
\ctr@ld@f\def\figptsinv#1=#2/#3,#4/{\ifGR@cri{\s@uvc@ntr@l\et@tfigptsinv%
    \setc@ntr@l{2}\def\list@num{#2}\s@mme=#1%
    \@ecfor\p@int:=\list@num\do{\figvectP-1[#3,\p@int]\Figg@tXY{-1}%
    \getredf@ctB\f@ctech\n@rmeucC{\delt@}{-1}%
    \delt@=\ptT@unit@\delt@\delt@=\ptT@unit@\delt@%
    \invers@{\delt@}{\delt@}\multiply\f@ctech\f@ctech\divide\delt@\f@ctech%
    \delt@=#4\delt@\edef\v@lcoef{\repdecn@mb{\delt@}}\figpttra\the\s@mme:=#3/\v@lcoef,-1/%
    \advance\s@mme\@ne}\resetc@ntr@l\et@tfigptsinv}\ignorespaces\fi}
\ctr@ln@m\figptsrot
\ctr@ld@f\def\figptsrotDD#1=#2/#3,#4/{\ifGR@cri{\s@uvc@ntr@l\et@tfigptsrotDD%
    \c@ssin{\C@}{\S@}{#4}\setc@ntr@l{2}\def\list@num{#2}\s@mme=#1%
    \@ecfor\p@int:=\list@num\do{\figvectPDD-1[#3,\p@int]\Figg@tXY{-1}%
    \v@lXa=\C@\v@lX\advance\v@lXa-\S@\v@lY%
    \v@lYa=\S@\v@lX\advance\v@lYa\C@\v@lY%
    \Figv@ctCreg-1(\v@lXa,\v@lYa)\figpttraDD\the\s@mme:=#3/1,-1/\advance\s@mme\@ne}%
    \resetc@ntr@l\et@tfigptsrotDD}\ignorespaces\fi}
\ctr@ld@f\def\figptsrotTD#1=#2/#3,#4,#5/{\ifGR@cri{\s@uvc@ntr@l\et@tfigptsrotTD%
    \c@ssin{\C@}{\S@}{#4}%
    \setc@ntr@l{2}\def\list@num{#2}\s@mme=#1%
    \@ecfor\p@int:=\list@num\do{\figptorthoprojplaneTD-3:=#3/\p@int,#5/%
    \figvectPTD-2[-3,\p@int]%
    \figvectNVTD-1[#5,-2]\n@rmeucTD\v@leur{-2}\edef\v@lcoef{\repdecn@mb{\v@leur}}%
    \Figg@tXYa{-1}\v@lXa=\v@lcoef\v@lXa\v@lYa=\v@lcoef\v@lYa\v@lZa=\v@lcoef\v@lZa%
    \v@lXa=\S@\v@lXa\v@lYa=\S@\v@lYa\v@lZa=\S@\v@lZa\Figg@tXY{-2}%
    \advance\v@lXa\C@\v@lX\advance\v@lYa\C@\v@lY\advance\v@lZa\C@\v@lZ%
    \Figg@tXY{-3}\advance\v@lXa\v@lX\advance\v@lYa\v@lY\advance\v@lZa\v@lZ%
    \Figp@intregTD\the\s@mme:(\v@lXa,\v@lYa,\v@lZa)\advance\s@mme\@ne}%
    \resetc@ntr@l\et@tfigptsrotTD}\ignorespaces\fi}
\ctr@ln@m\figptssym
\ctr@ld@f\def\figptssymDD#1=#2/#3,#4/{\ifGR@cri{\s@uvc@ntr@l\et@tfigptssymDD%
    \setc@ntr@l{2}\figvectPDD-3[#3,#4]\Figg@tXY{-3}\Figv@ctCreg-4(-\v@lY,\v@lX)%
    \resetc@ntr@l{2}\def\list@num{#2}\s@mme=#1%
    \@ecfor\p@int:=\list@num\do{\inters@cDD-5:[#3,-3;\p@int,-4]\figvectPDD-2[\p@int,-5]%
    \figpttraDD\the\s@mme:=\p@int/2,-2/\advance\s@mme\@ne}%
    \resetc@ntr@l\et@tfigptssymDD}\ignorespaces\fi}
\ctr@ld@f\def\figptssymTD#1=#2/#3,#4/{\ifGR@cri{\s@uvc@ntr@l\et@tfigptssymTD%
    \setc@ntr@l{2}\vecunit@TD{-2}{#4}\def\list@num{#2}\s@mme=#1%
    \@ecfor\p@int:=\list@num\do{\figvectPTD-1[\p@int,#3]%
    \c@lproscalTD\v@leur[-1,-2]\v@leur=2\v@leur\edef\v@lcoef{\repdecn@mb{\v@leur}}%
    \figpttraTD\the\s@mme:=\p@int/\v@lcoef,-2/\advance\s@mme\@ne}%
    \resetc@ntr@l\et@tfigptssymTD}\ignorespaces\fi}
\ctr@ln@m\figptstra
\ctr@ld@f\def\figptstraDD#1=#2/#3,#4/{\ifGR@cri{\Figg@tXYa{#4}\v@lXa=#3\v@lXa\v@lYa=#3\v@lYa%
    \def\list@num{#2}\s@mme=#1\@ecfor\p@int:=\list@num\do{\Figg@tXY{\p@int}%
    \advance\v@lX\v@lXa\advance\v@lY\v@lYa%
    \Figp@intregDD\the\s@mme:(\v@lX,\v@lY)\advance\s@mme\@ne}}\ignorespaces\fi}
\ctr@ld@f\def\figptstraTD#1=#2/#3,#4/{\ifGR@cri{\Figg@tXYa{#4}\v@lXa=#3\v@lXa\v@lYa=#3\v@lYa%
    \v@lZa=#3\v@lZa\def\list@num{#2}\s@mme=#1\@ecfor\p@int:=\list@num\do{\Figg@tXY{\p@int}%
    \advance\v@lX\v@lXa\advance\v@lY\v@lYa\advance\v@lZ\v@lZa%
    \Figp@intregTD\the\s@mme:(\v@lX,\v@lY,\v@lZ)\advance\s@mme\@ne}}\ignorespaces\fi}
\ctr@ln@m\figptvisilimSL
\ctr@ld@f\def\figptvisilimSLDD{\un@v@ilable{figptvisilimSL}}
\ctr@ld@f\def\figptvisilimSLTD#1:#2[#3,#4;#5,#6]{\ifGR@cri{\s@uvc@ntr@l\et@tfigptvisilimSLTD%
    \setc@ntr@l{2}\figvectP-1[#3,#4]\n@rminf{\delt@}{-1}%
    \ifcase\CUR@proj\v@lX=\cxa@\p@\v@lY=-\p@\v@lZ=\cxb@\p@% Proj cav
    \Figv@ctCreg-2(\v@lX,\v@lY,\v@lZ)\figvectP-3[#5,#6]\figvectNV-1[-2,-3]%
    \or\figvectP-1[#5,#6]\vecunitCV@TD{-1}\v@lmin=\v@lX\v@lmax=\v@lY% Proj ortho
    \v@leur=\v@lZ\v@lX=\cza@\p@\v@lY=\czb@\p@\v@lZ=\czc@\p@\c@lprovec{-1}%
    \or\c@ley@pt{-2}\figvectN-1[#5,#6,-2]\fi% Proj rea
    \edef\Ai@{#3}\edef\Aj@{#4}\figvectP-2[#5,\Ai@]\c@lproscal\v@leur[-1,-2]%
    \ifdim\v@leur>\z@\p@rtent=\@ne\else\p@rtent=\m@ne\fi%
    \figvectP-2[#5,\Aj@]\c@lproscal\v@leur[-1,-2]%
    \ifdim\p@rtent\v@leur>\z@\figptcopy#1:#2/#3/%
    \message{*** \BS@ figptvisilimSL: points are on the same side.}\else%
    \figptcopy-3:/#3/\figptcopy-4:/#4/%
    \loop\figptbary-5:[-3,-4;1,1]\figvectP-2[#5,-5]\c@lproscal\v@leur[-1,-2]%
    \ifdim\p@rtent\v@leur>\z@\figptcopy-3:/-5/\else\figptcopy-4:/-5/\fi%
    \divide\delt@\tw@\ifdim\delt@>\epsil@n\repeat%
    \figptbary#1:#2[-3,-4;1,1]\fi\resetc@ntr@l\et@tfigptvisilimSLTD}\ignorespaces\fi}
\ctr@ld@f\def\c@ley@pt#1{\t@stp@r\ifitis@K\v@lX=\cza@\p@\v@lY=\czb@\p@\v@lZ=\czc@\p@%
    \Figv@ctCreg-1(\v@lX,\v@lY,\v@lZ)\Figp@intreg-2:(\wd\Bt@rget,\ht\Bt@rget,\dp\Bt@rget)%
    \figpttra#1:=-2/-\disob@intern,-1/\else\end\fi}
\ctr@ld@f\def\t@stp@r{\itis@Ktrue\ifnewt@rgetpt\else\itis@Kfalse%
    \message{*** \BS@ figptvisilimXX: target point undefined.}\fi\ifnewdis@b\else%
    \itis@Kfalse\message{*** \BS@ figptvisilimXX: observation distance undefined.}\fi%
    \ifitis@K\else\message{*** This macro must be called after \BS@ figdrawbegin or after
    having set the missing parameter(s) with \BS@ figset proj()}\fi}
\ctr@ld@f\def\figscan#1(#2,#3){{\s@uvc@ntr@l\et@tfigscan\@psfgetbb{#1}\if@psfbbfound\else%
    \def\@psfllx{0}\def\@psflly{20}\def\@psfurx{540}\def\@psfury{640}\fi\figscan@{#2}{#3}%
    \resetc@ntr@l\et@tfigscan}\ignorespaces}
\ctr@ld@f\def\figscan@#1#2{%
    \unit@=\@ne bp\setc@ntr@l{2}\figsetmark{}%
    \def\minst@p{20pt}%
    \v@lX=\@psfllx\p@\v@lX=\Sc@leFact\v@lX\r@undint\v@lX\v@lX%
    \v@lY=\@psflly\p@\v@lY=\Sc@leFact\v@lY\ifdim\v@lY>\z@\r@undint\v@lY\v@lY\fi%
    \delt@=\@psfury\p@\delt@=\Sc@leFact\delt@%
    \advance\delt@-\v@lY\v@lXa=\@psfurx\p@\v@lXa=\Sc@leFact\v@lXa\v@leur=\minst@p%
    \edef\valv@lY{\repdecn@mb{\v@lY}}\edef\LgTr@it{\the\delt@}%
    \loop\ifdim\v@lX<\v@lXa\edef\valv@lX{\repdecn@mb{\v@lX}}%
    \figptDD -1:(\valv@lX,\valv@lY)\figwriten -1:\hbox{\vrule height\LgTr@it}(0)%
    \ifdim\v@leur<\minst@p\else\figsetmark{\raise-8bp\hbox{$\scriptscriptstyle\triangle$}}%
    \figwrites -1:\@ffichnb{0}{\valv@lX}(6)\v@leur=\z@\figsetmark{}\fi%
    \advance\v@leur#1pt\advance\v@lX#1pt\repeat%
    \def\minst@p{10pt}%
    \v@lX=\@psfllx\p@\v@lX=\Sc@leFact\v@lX\ifdim\v@lX>\z@\r@undint\v@lX\v@lX\fi%
    \v@lY=\@psflly\p@\v@lY=\Sc@leFact\v@lY\r@undint\v@lY\v@lY%
    \delt@=\@psfurx\p@\delt@=\Sc@leFact\delt@%
    \advance\delt@-\v@lX\v@lYa=\@psfury\p@\v@lYa=\Sc@leFact\v@lYa\v@leur=\minst@p%
    \edef\valv@lX{\repdecn@mb{\v@lX}}\edef\LgTr@it{\the\delt@}%
    \loop\ifdim\v@lY<\v@lYa\edef\valv@lY{\repdecn@mb{\v@lY}}%
    \figptDD -1:(\valv@lX,\valv@lY)\figwritee -1:\vbox{\hrule width\LgTr@it}(0)%
    \ifdim\v@leur<\minst@p\else\figsetmark{$\triangleright$\kern4bp}%
    \figwritew -1:\@ffichnb{0}{\valv@lY}(6)\v@leur=\z@\figsetmark{}\fi%
    \advance\v@leur#2pt\advance\v@lY#2pt\repeat}
\ctr@ld@f\let\figscanI=\figscan
\ctr@ld@f\def\figscan@E#1(#2,#3){{\s@uvc@ntr@l\et@tfigscan@E%
    \Figdisc@rdLTS{#1}{\t@xt@}\pdfximage{\t@xt@}%
    \setbox\Gb@x=\hbox{\pdfrefximage\pdflastximage}%
    \edef\@psfllx{0}\v@lY=-\dp\Gb@x\edef\@psflly{\repdecn@mb{\v@lY}}%
    \edef\@psfurx{\repdecn@mb{\wd\Gb@x}}%
    \v@lY=\dp\Gb@x\advance\v@lY\ht\Gb@x\edef\@psfury{\repdecn@mb{\v@lY}}%
    \figscan@{#2}{#3}\resetc@ntr@l\et@tfigscan@E}\ignorespaces}
\ctr@ld@f\def\figshowpts[#1,#2]{{\figsetmark{$\bullet$}\figsetptname{\bf ##1}%
    \p@rtent=#2\relax\ifnum\p@rtent<\z@\p@rtent=\z@\fi%
    \s@mme=#1\relax\ifnum\s@mme<\z@\s@mme=\z@\fi%
    \loop\ifnum\s@mme<\p@rtent\pt@rvect{\s@mme}%
    \ifitis@K\figwriten{\the\s@mme}:(4pt)\fi\advance\s@mme\@ne\repeat%
    \pt@rvect{\s@mme}\ifitis@K\figwriten{\the\s@mme}:(4pt)\fi}\ignorespaces}
\ctr@ld@f\def\pt@rvect#1{\set@bjc@de{#1}%
    \expandafter\expandafter\expandafter\inqpt@rvec\csname\objc@de\endcsname:}
\ctr@ld@f\def\inqpt@rvec#1#2:{\if#1\C@dCl@spt\itis@Ktrue\else\itis@Kfalse\fi}
\ctr@ld@f\def\figshowsettings{{%
    \immediate\write16{====================================================================}%
    \immediate\write16{ Current settings are (DDV means "with dynamic default value"):}%
    \immediate\write16{ --- GENERAL ---}%
    \immediate\write16{Scale factor and Unit = \unit@util\space (\the\unit@)
     \space -> \BS@ figinit{ScaleFactorUnit}}%
    \immediate\write16{Update mode = \ifGRupdatem@de yes\else no\fi
     \space-> \BS@ figset(update=yes/no) or \BS@ figsetdefault(update=yes/no)}%
    \immediate\write16{ --- WRITING ---}%
    \immediate\write16{Implicit point name = \ptn@me{i} \space-> \BS@ figset write(ptname={Name})}%
    \immediate\write16{Point marker = \the\c@nsymb \space -> \BS@ figset write(mark=Mark)}%
    \immediate\write16{Print rounded coordinates = \ifr@undcoord yes\else no\fi
     \space-> \BS@ figset write(roundcoord=yes/no)}%
    \immediate\write16{ --- GRAPHICAL (general) ---}%
    \immediate\write16{Color = \CUR@color \space-> \BS@ figset(color=ColorDefinition)}%
    \immediate\write16{Filling mode = \iffillm@de yes\else no\fi
     \space-> \BS@ figset(fillmode=yes/no)}%
    \immediate\write16{Line join = \CUR@join \space-> \BS@ figset(join=miter/round/bevel)}%
    \immediate\write16{Line style = \CUR@dash \space-> \BS@ figset(dash=Index/Pattern)}%
    \immediate\write16{Line width = \CUR@width
     \space-> \BS@ figset(width=real in PostScript units)}%
    \immediate\write16{ --- GRAPHICAL (specific) ---}%
    \immediate\write16{Altitude (all the following attributes are DDV):}%
    \immediate\write16{ Base line color =
     \ifx\DDV@blcolor\D@FTref general color\else\DDV@blcolor\fi
     \space-> \BS@ figset altitude(blcolor=ColorDefinition)}%
    \immediate\write16{ Base line style =
     \ifx\DDV@bldash\D@FTref general style\else\DDV@bldash\fi
     \space-> \BS@ figset altitude(bldash=Index/Pattern)}%
    \immediate\write16{ Base line width =
     \ifx\DDV@blwidth\D@FTref general width\else\DDV@blwidth\fi
     \space-> \BS@ figset altitude(blwidth=real in PostScript units)}%
    \immediate\write16{ Square line color =
     \ifx\DDV@sqcolor\D@FTref general color\else\DDV@sqcolor\fi
     \space-> \BS@ figset altitude(sqcolor=ColorDefinition)}%
    \immediate\write16{ Square line style =
     \ifx\DDV@sqdash\D@FTref general style\else\DDV@sqdash\fi
     \space-> \BS@ figset altitude(sqdash=Index/Pattern)}%
    \immediate\write16{ Square line width =
     \ifx\DDV@sqwidth\D@FTref general width\else\DDV@sqwidth\fi
     \space-> \BS@ figset altitude(sqwidth=real in PostScript units)}%
    \immediate\write16{Arrowhead:}%
    \immediate\write16{ (half-)Angle = \@rrowheadangle
     \space-> \BS@ figset arrowhead(angle=real in degrees)}%
    \immediate\write16{ Filling mode = \if@rrowhfill yes\else no\fi
     \space-> \BS@ figset arrowhead(fillmode=yes/no)}%
    \immediate\write16{ "Outside" = \if@rrowhout yes\else no\fi
     \space-> \BS@ figset arrowhead(out=yes/no)}%
    \immediate\write16{ Length = \@rrowheadlength
     \if@rrowratio\space(not active)\else\space(active)\fi
     \space-> \BS@ figset arrowhead(length=real in user coord.)}%
    \immediate\write16{ Ratio = \@rrowheadratio
     \if@rrowratio\space(active)\else\space(not active)\fi
     \space-> \BS@ figset arrowhead(ratio=real in [0,1])}%
    \immediate\write16{Curve:}%
    \immediate\write16{ Roundness = \curv@roundness
     \space-> \BS@ figset curve(roundness=real in [0,0.5])}%
    \immediate\write16{Flow chart:}%
    \immediate\write16{ Arrow position = \@rrowp@s
     \space-> \BS@ figset flowchart(arrowposition=real in [0,1])}%
    \immediate\write16{ Arrow reference point = \ifcase\@rrowr@fpt start\else end\fi
     \space-> \BS@ figset flowchart(arrowrefpt = start/end)}%     
    \immediate\write16{ Background color = \fcbgc@lor
     \space-> \BS@ figset flowchart(bgcolor=ColorDefinition)}%
    \immediate\write16{ Line type = \ifcase\fclin@typ@ curve\else polygon\fi
     \space-> \BS@ figset flowchart(line=polygon/curve)}%
    \immediate\write16{ Padding = (\Xp@dd, \Yp@dd)
     \space-> \BS@ figset flowchart(padding = real in user coord.)}%
    \immediate\write16{\space\space\space\space(or
     \BS@ figset flowchart(xpadding=real, ypadding=real) )}%
    \immediate\write16{ Radius = \fclin@r@d
     \space-> \BS@ figset flowchart(radius=positive real in user coord.)}%
    \immediate\write16{ Shape = \fcsh@pe
     \space-> \BS@ figset flowchart(shape = rectangle, ellipse or lozenge)}%
    \immediate\write16{ Thickness color (DDV) = 
     \ifx\DDV@thickcolor\D@FTref general color\else\DDV@thickcolor\fi
     \space-> \BS@ figset flowchart(thickcolor=ColorDefinition)}%
    \immediate\write16{ Thickness = \thickn@ss
     \space-> \BS@ figset flowchart(thickness = real in user coord.)}%
    \immediate\write16{Mesh:}%
    \immediate\write16{ Diagonal = \c@ntrolmesh
     \space-> \BS@ figset mesh(diag=integer in {-1,0,1})}%
    \immediate\write16{ Lines color (DDV) =
     \ifx\DDV@meshcolor\D@FTref general color\else\DDV@meshcolor\fi
     \space-> \BS@ figset mesh(color=ColorDefinition)}%
    \immediate\write16{ Lines style (DDV) =
     \ifx\DDV@meshdash\D@FTref general style\else\DDV@meshdash\fi
     \space-> \BS@ figset mesh(dash=Index/Pattern)}%
    \immediate\write16{ Lines width (DDV) =
     \ifx\DDV@meshwidth\D@FTref general width\else\DDV@meshwidth\fi
     \space-> \BS@ figset mesh(width=real in PostScript units)}%
    \immediate\write16{Trimesh:}%
    \immediate\write16{ Lines color (DDV) =
     \ifx\DDV@tmeshcolor\D@FTref general color\else\DDV@tmeshcolor\fi
     \space-> \BS@ figset trimesh(color=ColorDefinition)}%
    \immediate\write16{ Lines style (DDV) =
     \ifx\DDV@tmeshdash\D@FTref general style\else\DDV@tmeshdash\fi
     \space-> \BS@ figset trimesh(dash=Index/Pattern)}%
    \immediate\write16{ Lines width (DDV) =
     \ifx\DDV@tmeshwidth\D@FTref general width\else\DDV@tmeshwidth\fi
     \space-> \BS@ figset trimesh(width=real in PostScript units)}%
    \ifTr@isDim%
    \immediate\write16{ --- 3D to 2D PROJECTION ---}%
    \immediate\write16{Projection : \typ@proj \space-> \BS@ figinit{ScaleFactorUnit, ProjType}}%
    \immediate\write16{Longitude (psi) = \v@lPsi \space-> \BS@ figset proj(psi=real in degrees)}%
    \ifcase\CUR@proj\immediate\write16{Depth coeff. (Lambda)
     \space = \v@lTheta \space-> \BS@ figset proj(lambda=real in [0,1])}%
    \else\immediate\write16{Latitude (theta)
     \space = \v@lTheta \space-> \BS@ figset proj(theta=real in degrees)}%
    \fi%
    \ifnum\CUR@proj=\tw@%
    \immediate\write16{Observation distance = \disob@unit
     \space-> \BS@ figset proj(dist=real in user coord.)}%
    \immediate\write16{Target point = \t@rgetpt \space-> \BS@ figset proj(targetpt=pt number)}%
     \v@lX=\ptT@unit@\wd\Bt@rget\v@lY=\ptT@unit@\ht\Bt@rget\v@lZ=\ptT@unit@\dp\Bt@rget%
    \immediate\write16{ Its coordinates are
     (\repdecn@mb{\v@lX}, \repdecn@mb{\v@lY}, \repdecn@mb{\v@lZ})}%
    \fi%
    \fi%
    \immediate\write16{====================================================================}%
    \ignorespaces}}
\ctr@ln@w{newif}\ifitis@vect@r
\ctr@ld@f\def\figvectC#1(#2,#3){{\itis@vect@rtrue\figpt#1:(#2,#3)}\ignorespaces}
\ctr@ld@f\def\Figv@ctCreg#1(#2,#3){{\itis@vect@rtrue\Figp@intreg#1:(#2,#3)}\ignorespaces}
\ctr@ln@m\figvectDBezier
\ctr@ld@f\def\figvectDBezierDD#1:#2,#3[#4,#5,#6,#7]{\ifGR@cri{\s@uvc@ntr@l\et@tfigvectDBezierDD%
    \FigvectDBezier@#2,#3[#4,#5,#6,#7]\v@lX=\c@ef\v@lX\v@lY=\c@ef\v@lY%
    \Figv@ctCreg#1(\v@lX,\v@lY)\resetc@ntr@l\et@tfigvectDBezierDD}\ignorespaces\fi}
\ctr@ld@f\def\figvectDBezierTD#1:#2,#3[#4,#5,#6,#7]{\ifGR@cri{\s@uvc@ntr@l\et@tfigvectDBezierTD%
    \FigvectDBezier@#2,#3[#4,#5,#6,#7]\v@lX=\c@ef\v@lX\v@lY=\c@ef\v@lY\v@lZ=\c@ef\v@lZ%
    \Figv@ctCreg#1(\v@lX,\v@lY,\v@lZ)\resetc@ntr@l\et@tfigvectDBezierTD}\ignorespaces\fi}
\ctr@ld@f\def\FigvectDBezier@#1,#2[#3,#4,#5,#6]{\setc@ntr@l{2}%
    \edef\T@{#2}\v@leur=\p@\advance\v@leur-#2pt\edef\UNmT@{\repdecn@mb{\v@leur}}%
    \ifnum#1=\tw@\def\c@ef{6}\else\def\c@ef{3}\fi%
    \figptcopy-4:/#3/\figptcopy-3:/#4/\figptcopy-2:/#5/\figptcopy-1:/#6/%
    \l@mbd@un=-4 \l@mbd@de=-\thr@@\p@rtent=\m@ne\c@lDecast%
    \ifnum#1=\tw@\c@lDCDeux{-4}{-3}\c@lDCDeux{-3}{-2}\c@lDCDeux{-4}{-3}\else%
    \l@mbd@un=-4 \l@mbd@de=-\thr@@\p@rtent=-\tw@\c@lDecast%
    \c@lDCDeux{-4}{-3}\fi\Figg@tXY{-4}}
\ctr@ln@m\c@lDCDeux
\ctr@ld@f\def\c@lDCDeuxDD#1#2{\Figg@tXY{#2}\Figg@tXYa{#1}%
    \advance\v@lX-\v@lXa\advance\v@lY-\v@lYa\Figp@intregDD#1:(\v@lX,\v@lY)}
\ctr@ld@f\def\c@lDCDeuxTD#1#2{\Figg@tXY{#2}\Figg@tXYa{#1}\advance\v@lX-\v@lXa%
    \advance\v@lY-\v@lYa\advance\v@lZ-\v@lZa\Figp@intregTD#1:(\v@lX,\v@lY,\v@lZ)}
\ctr@ln@m\figvectN
\ctr@ld@f\def\figvectNDD#1[#2,#3]{\ifGR@cri{\Figg@tXYa{#2}\Figg@tXY{#3}%
    \advance\v@lX-\v@lXa\advance\v@lY-\v@lYa%
    \Figv@ctCreg#1(-\v@lY,\v@lX)}\ignorespaces\fi}
\ctr@ld@f\def\figvectNTD#1[#2,#3,#4]{\ifGR@cri{\vecunitC@TD[#2,#4]\v@lmin=\v@lX\v@lmax=\v@lY%
    \v@leur=\v@lZ\vecunitC@TD[#2,#3]\c@lprovec{#1}}\ignorespaces\fi}
\ctr@ln@m\figvectNV
\ctr@ld@f\def\figvectNVDD#1[#2]{\ifGR@cri{\Figg@tXY{#2}\Figv@ctCreg#1(-\v@lY,\v@lX)}\ignorespaces\fi}
\ctr@ld@f\def\figvectNVTD#1[#2,#3]{\ifGR@cri{\vecunitCV@TD{#3}\v@lmin=\v@lX\v@lmax=\v@lY%
    \v@leur=\v@lZ\vecunitCV@TD{#2}\c@lprovec{#1}}\ignorespaces\fi}
\ctr@ln@m\figvectP
\ctr@ld@f\def\figvectPDD#1[#2,#3]{\ifGR@cri{\Figg@tXYa{#2}\Figg@tXY{#3}%
    \advance\v@lX-\v@lXa\advance\v@lY-\v@lYa%
    \Figv@ctCreg#1(\v@lX,\v@lY)}\ignorespaces\fi}
\ctr@ld@f\def\figvectPTD#1[#2,#3]{\ifGR@cri{\Figg@tXYa{#2}\Figg@tXY{#3}%
    \advance\v@lX-\v@lXa\advance\v@lY-\v@lYa\advance\v@lZ-\v@lZa%
    \Figv@ctCreg#1(\v@lX,\v@lY,\v@lZ)}\ignorespaces\fi}
\ctr@ln@m\figvectU
\ctr@ld@f\def\figvectUDD#1[#2]{\ifGR@cri{\n@rmeuc\v@leur{#2}\invers@\v@leur\v@leur%
    \delt@=\repdecn@mb{\v@leur}\unit@\edef\v@ldelt@{\repdecn@mb{\delt@}}%
    \Figg@tXY{#2}\v@lX=\v@ldelt@\v@lX\v@lY=\v@ldelt@\v@lY%
    \Figv@ctCreg#1(\v@lX,\v@lY)}\ignorespaces\fi}
\ctr@ld@f\def\figvectUTD#1[#2]{\ifGR@cri{\n@rmeuc\v@leur{#2}\invers@\v@leur\v@leur%
    \delt@=\repdecn@mb{\v@leur}\unit@\edef\v@ldelt@{\repdecn@mb{\delt@}}%
    \Figg@tXY{#2}\v@lX=\v@ldelt@\v@lX\v@lY=\v@ldelt@\v@lY\v@lZ=\v@ldelt@\v@lZ%
    \Figv@ctCreg#1(\v@lX,\v@lY,\v@lZ)}\ignorespaces\fi}
\ctr@ld@f\def\figvisu#1#2#3{\c@ldefproj\initb@undb@x\xdef\figforTeXFigno{\figforTeXnextFigno}%
    \s@mme=\figforTeXnextFigno\advance\s@mme\@ne\xdef\figforTeXnextFigno{\number\s@mme}%
    \setbox\b@xvisu=\hbox{\ifnum\@utoFN>\z@\figinsert{}\gdef\@utoFInDone{0}\fi\ignorespaces#3}%
    \gdef\@utoFInDone{1}\gdef\@utoFN{0}%
    \v@lXa=-\c@@rdYmin\v@lYa=\c@@rdYmax\advance\v@lYa-\c@@rdYmin%
    \v@lX=\c@@rdXmax\advance\v@lX-\c@@rdXmin%
    \setbox#1=\hbox{#2}\v@lY=-\v@lX\maxim@m{\v@lX}{\v@lX}{\wd#1}%
    \advance\v@lY\v@lX\divide\v@lY\tw@\advance\v@lY-\c@@rdXmin%
    \setbox#1=\vbox{\parindent\z@\hsize=\v@lX\vskip\v@lYa%
    \rlap{\hskip\v@lY\smash{\raise\v@lXa\box\b@xvisu}}%
    \def\t@xt@{#2}\ifx\t@xt@\empty\else\medskip\centerline{#2}\fi}\wd#1=\v@lX}
\ctr@ld@f\def\figDecrementFigno{{\xdef\figforTeXnextFigno{\figforTeXFigno}%
    \s@mme=\figforTeXFigno\advance\s@mme\m@ne\xdef\figforTeXFigno{\number\s@mme}}}
\ctr@ln@w{newbox}\Bt@rget\setbox\Bt@rget=\null
\ctr@ln@w{newbox}\BminTD@\setbox\BminTD@=\null
\ctr@ln@w{newbox}\BmaxTD@\setbox\BmaxTD@=\null
\ctr@ln@w{newif}\ifnewt@rgetpt\ctr@ln@w{newif}\ifnewdis@b
\ctr@ld@f\def\b@undb@xTD#1#2#3{%
    \relax\ifdim#1<\wd\BminTD@\global\wd\BminTD@=#1\fi%
    \relax\ifdim#2<\ht\BminTD@\global\ht\BminTD@=#2\fi%
    \relax\ifdim#3<\dp\BminTD@\global\dp\BminTD@=#3\fi%
    \relax\ifdim#1>\wd\BmaxTD@\global\wd\BmaxTD@=#1\fi%
    \relax\ifdim#2>\ht\BmaxTD@\global\ht\BmaxTD@=#2\fi%
    \relax\ifdim#3>\dp\BmaxTD@\global\dp\BmaxTD@=#3\fi}
\ctr@ld@f\def\c@ldefdisob{{\ifdim\wd\BminTD@<\maxdimen\v@leur=\wd\BmaxTD@\advance\v@leur-\wd\BminTD@%
    \delt@=\ht\BmaxTD@\advance\delt@-\ht\BminTD@\maxim@m{\v@leur}{\v@leur}{\delt@}%
    \delt@=\dp\BmaxTD@\advance\delt@-\dp\BminTD@\maxim@m{\v@leur}{\v@leur}{\delt@}%
    \v@leur=5\v@leur\else\v@leur=800pt\fi\c@ldefdisob@{\v@leur}}}
\ctr@ln@m\disob@intern
\ctr@ln@m\disob@
\ctr@ln@m\divf@ctproj
\ctr@ld@f\def\c@ldefdisob@#1{{\v@leur=#1\ifdim\v@leur<\p@\v@leur=800pt\fi%
    \xdef\disob@intern{\repdecn@mb{\v@leur}}%
    \delt@=\ptT@unit@\v@leur\xdef\disob@unit{\repdecn@mb{\delt@}}%
    \f@ctech=\@ne\loop\ifdim\v@leur>\t@n pt\divide\v@leur\t@n\multiply\f@ctech\t@n\repeat%
    \xdef\disob@{\repdecn@mb{\v@leur}}\xdef\divf@ctproj{\the\f@ctech}}%
    \global\newdis@btrue}
\ctr@ln@m\t@rgetpt
\ctr@ld@f\def\c@ldeft@rgetpt{\newt@rgetpttrue\def\t@rgetpt{CenterBoundBox}{%
    \delt@=\wd\BmaxTD@\advance\delt@-\wd\BminTD@\divide\delt@\tw@%
    \v@leur=\wd\BminTD@\advance\v@leur\delt@\global\wd\Bt@rget=\v@leur%
    \delt@=\ht\BmaxTD@\advance\delt@-\ht\BminTD@\divide\delt@\tw@%
    \v@leur=\ht\BminTD@\advance\v@leur\delt@\global\ht\Bt@rget=\v@leur%
    \delt@=\dp\BmaxTD@\advance\delt@-\dp\BminTD@\divide\delt@\tw@%
    \v@leur=\dp\BminTD@\advance\v@leur\delt@\global\dp\Bt@rget=\v@leur}}
\ctr@ln@m\c@ldefproj
\ctr@ld@f\def\c@ldefprojTD{\ifnewt@rgetpt\else\c@ldeft@rgetpt\fi\ifnewdis@b\else\c@ldefdisob\fi}
\ctr@ld@f\def\c@lprojcav{% Projection cavaliere : X = x + y L cos t, Y = z + y L sin t
    \v@lZa=\cxa@\v@lY\advance\v@lX\v@lZa%
    \v@lZa=\cxb@\v@lY\v@lY=\v@lZ\advance\v@lY\v@lZa\ignorespaces}
\ctr@ln@m\v@lcoef
\ctr@ld@f\def\c@lprojrea{% Projection realiste
    \advance\v@lX-\wd\Bt@rget\advance\v@lY-\ht\Bt@rget\advance\v@lZ-\dp\Bt@rget%
    \v@lZa=\cza@\v@lX\advance\v@lZa\czb@\v@lY\advance\v@lZa\czc@\v@lZ%
    \divide\v@lZa\divf@ctproj\advance\v@lZa\disob@ pt\invers@{\v@lZa}{\v@lZa}%
    \v@lZa=\disob@\v@lZa\edef\v@lcoef{\repdecn@mb{\v@lZa}}%
    \v@lXa=\cxa@\v@lX\advance\v@lXa\cxb@\v@lY\v@lXa=\v@lcoef\v@lXa%
    \v@lY=\cyb@\v@lY\advance\v@lY\cya@\v@lX\advance\v@lY\cyc@\v@lZ%
    \v@lY=\v@lcoef\v@lY\v@lX=\v@lXa\ignorespaces}
\ctr@ld@f\def\c@lprojort{% Projection orthogonale
    \v@lXa=\cxa@\v@lX\advance\v@lXa\cxb@\v@lY%
    \v@lY=\cyb@\v@lY\advance\v@lY\cya@\v@lX\advance\v@lY\cyc@\v@lZ%
    \v@lX=\v@lXa\ignorespaces}
\ctr@ld@f\def\Figptpr@j#1:#2/#3/{{\Figg@tXY{#3}\superc@lprojSP%
    \Figp@intregDD#1:{#2}(\v@lX,\v@lY)}\ignorespaces}
\ctr@ln@m\figsetobdist
\ctr@ld@f\def\figsetobdistDD{\un@v@ilable{figsetobdist}}
\ctr@ld@f\def\figsetobdistTD(#1){{\ifCUR@PS\W@rnmesIgn{figset proj(dist=...)}%
    \else\v@leur=#1\unit@\c@ldefdisob@{\v@leur}\fi}\ignorespaces}
\ctr@ln@m\c@lprojSP
\ctr@ln@m\CUR@proj
\ctr@ln@m\typ@proj
\ctr@ln@m\superc@lprojSP
\ctr@ld@f\def\Figs@tproj#1{%
    \if#13 \def@ultproj\else\if#1c\def@ultproj%
    \else\if#1o\xdef\CUR@proj{1}\xdef\typ@proj{orthogonal}%
         \figsetviewTD(\def@ultpsi,\def@ulttheta)%
         \global\let\c@lprojSP=\c@lprojort\global\let\superc@lprojSP=\c@lprojort%
    \else\if#1r\xdef\CUR@proj{2}\xdef\typ@proj{realistic}%
         \figsetviewTD(\def@ultpsi,\def@ulttheta)%
         \global\let\c@lprojSP=\c@lprojrea\global\let\superc@lprojSP=\c@lprojrea%
    \else\def@ultproj\message{*** Unknown projection. Cavalier projection assumed.}%
    \fi\fi\fi\fi}
\ctr@ld@f\def\def@ultproj{\xdef\CUR@proj{0}\xdef\typ@proj{cavalier}\figsetviewTD(\def@ultpsi,0.5)%
         \global\let\c@lprojSP=\c@lprojcav\global\let\superc@lprojSP=\c@lprojcav}
\ctr@ln@m\figsettarget
\ctr@ld@f\def\figsettargetDD{\un@v@ilable{figsettarget}}
\ctr@ld@f\def\figsettargetTD[#1]{{\ifCUR@PS\W@rnmesIgn{figset proj(targetpt=...)}%
    \else\global\newt@rgetpttrue\xdef\t@rgetpt{#1}\Figg@tXY{#1}\global\wd\Bt@rget=\v@lX%
    \global\ht\Bt@rget=\v@lY\global\dp\Bt@rget=\v@lZ\fi}\ignorespaces}
\ctr@ln@m\figsetview
\ctr@ld@f\def\figsetviewDD{\un@v@ilable{figsetview}}
\ctr@ld@f\def\figsetviewTD(#1){\ifCUR@PS\W@rnmesIgn{figset proj(Psi|Theta|Lambda=...)}%
     \else\Figsetview@#1,:\fi\ignorespaces}
\ctr@ld@f\def\Figsetview@#1,#2:{{\xdef\v@lPsi{#1}\def\t@xt@{#2}%
    \ifx\t@xt@\empty\def\@rgdeux{\v@lTheta}\else\X@rgdeux@#2\fi%
    \c@ssin{\costhet@}{\sinthet@}{#1}\v@lmin=\costhet@ pt\v@lmax=\sinthet@ pt%
    \ifcase\CUR@proj%
    \v@leur=\@rgdeux\v@lmin\xdef\cxa@{\repdecn@mb{\v@leur}}%
    \v@leur=\@rgdeux\v@lmax\xdef\cxb@{\repdecn@mb{\v@leur}}\v@leur=\@rgdeux pt%
    \relax\ifdim\v@leur>\p@\message{*** Lambda too large ! See \BS@ figset proj() !}\fi%
    \else%
    \v@lmax=-\v@lmax\xdef\cxa@{\repdecn@mb{\v@lmax}}\xdef\cxb@{\costhet@}%
    \ifx\t@xt@\empty\edef\@rgdeux{\def@ulttheta}\fi\c@ssin{\C@}{\S@}{\@rgdeux}%
    \v@lmax=-\S@ pt%
    \v@leur=\v@lmax\v@leur=\costhet@\v@leur\xdef\cya@{\repdecn@mb{\v@leur}}%
    \v@leur=\v@lmax\v@leur=\sinthet@\v@leur\xdef\cyb@{\repdecn@mb{\v@leur}}%
    \xdef\cyc@{\C@}\v@lmin=-\C@ pt%
    \v@leur=\v@lmin\v@leur=\costhet@\v@leur\xdef\cza@{\repdecn@mb{\v@leur}}%
    \v@leur=\v@lmin\v@leur=\sinthet@\v@leur\xdef\czb@{\repdecn@mb{\v@leur}}%
    \xdef\czc@{\repdecn@mb{\v@lmax}}\fi%
    \xdef\v@lTheta{\@rgdeux}}}
\ctr@ld@f\def\def@ultpsi{40}
\ctr@ld@f\def\def@ulttheta{25}
\ctr@ln@m\l@debut
\ctr@ln@m\n@mref
\ctr@ld@f\def\Figsetpr@j#1=#2|{\keln@mtr#1|%
    \def\n@mref{dep}\ifx\l@debut\n@mref\Figsetd@p{#2}\else% depth (lambda)
    \def\n@mref{dis}\ifx\l@debut\n@mref%
     \ifnum\CUR@proj=\tw@\figsetobdist(#2)\else\Figset@rr\fi\else% dist
    \def\n@mref{lam}\ifx\l@debut\n@mref\Figsetd@p{#2}\else% depth (lambda)
    \def\n@mref{lat}\ifx\l@debut\n@mref\Figsetth@{#2}\else% latitude (theta)
    \def\n@mref{lon}\ifx\l@debut\n@mref\figsetview(#2)\else% longitude (psi)
    \def\n@mref{psi}\ifx\l@debut\n@mref\figsetview(#2)\else% longitude (psi)
    \def\n@mref{tar}\ifx\l@debut\n@mref%
     \ifnum\CUR@proj=\tw@\figsettarget[#2]\else\Figset@rr\fi\else% target point
    \def\n@mref{the}\ifx\l@debut\n@mref\Figsetth@{#2}\else% latitude (theta)
    \W@rnmesAttr{figset proj}{#1}\fi\fi\fi\fi\fi\fi\fi\fi}
\ctr@ld@f\def\Figsetd@p#1{\ifnum\CUR@proj=\z@\figsetview(\v@lPsi,#1)\else\Figset@rr\fi}
\ctr@ld@f\def\Figsetth@#1{\ifnum\CUR@proj=\z@\Figset@rr\else\figsetview(\v@lPsi,#1)\fi}
\ctr@ld@f\def\Figset@rr{\message{*** \BS@ figset proj(): Attribute "\n@mref" ignored, incompatible
    with current projection}}
\ctr@ld@f\def\initb@undb@xTD{\wd\BminTD@=\maxdimen\ht\BminTD@=\maxdimen\dp\BminTD@=\maxdimen%
    \wd\BmaxTD@=-\maxdimen\ht\BmaxTD@=-\maxdimen\dp\BmaxTD@=-\maxdimen}
\ctr@ln@w{newbox}\Gb@x      % boite a tout faire
\ctr@ln@w{newbox}\Gb@xSC    % boite qui contient le point marker
\ctr@ln@w{newtoks}\c@nsymb  % the point marker
\ctr@ln@w{newif}\ifr@undcoord\ctr@ln@w{newif}\ifunitpr@sent
\ctr@ld@f\def\unssqrttw@{0.707106 }
\ctr@ld@f\def\figAst{\raise-1.15ex\hbox{$\ast$}}
\ctr@ld@f\def\figBullet{\raise-1.15ex\hbox{$\bullet$}}
\ctr@ld@f\def\figCirc{\raise-1.15ex\hbox{$\circ$}}
\ctr@ld@f\def\figDiamond{\raise-1.15ex\hbox{$\diamond$}}%
\ctr@ld@f\def\boxit#1#2{\leavevmode\hbox{\vrule\vbox{\hrule\vglue#1%
    \vtop{\hbox{\kern#1{#2}\kern#1}\vglue#1\hrule}}\vrule}}
\ctr@ld@f\def\centertext#1#2{\vbox{\hsize#1\parindent0cm%
    \leftskip=0pt plus 1fil\rightskip=0pt plus 1fil\parfillskip=0pt{#2}}}
\ctr@ld@f\def\lefttext#1#2{\vbox{\hsize#1\parindent0cm\rightskip=0pt plus 1fil#2}}
\ctr@ld@f\def\c@nterpt{\ignorespaces%
    \kern-.5\wd\Gb@xSC%
    \raise-.5\ht\Gb@xSC\rlap{\hbox{\raise.5\dp\Gb@xSC\hbox{\copy\Gb@xSC}}}%
    \kern .5\wd\Gb@xSC\ignorespaces}
\ctr@ld@f\def\b@undb@xSC#1#2{{\v@lXa=#1\v@lYa=#2%
    \v@leur=\ht\Gb@xSC\advance\v@leur\dp\Gb@xSC%
    \advance\v@lXa-.5\wd\Gb@xSC\advance\v@lYa-.5\v@leur\b@undb@x{\v@lXa}{\v@lYa}%
    \advance\v@lXa\wd\Gb@xSC\advance\v@lYa\v@leur\b@undb@x{\v@lXa}{\v@lYa}}}
\ctr@ln@m\Dist@n
\ctr@ln@m\l@suite
\ctr@ld@f\def\@keldist#1#2{\edef\Dist@n{#2}\y@tiunit{\Dist@n}%
    \ifunitpr@sent#1=\Dist@n\else#1=\Dist@n\unit@\fi}
\ctr@ld@f\def\y@tiunit#1{\unitpr@sentfalse\expandafter\y@tiunit@#1:}
\ctr@ld@f\def\y@tiunit@#1#2:{\ifcat#1a\unitpr@senttrue\else\def\l@suite{#2}%
    \ifx\l@suite\empty\else\y@tiunit@#2:\fi\fi}
\ctr@ln@m\figcoord
\ctr@ld@f\def\figcoordDD#1{{\v@lX=\ptT@unit@\v@lX\v@lY=\ptT@unit@\v@lY%
    \ifr@undcoord\ifcase#1\v@leur=0.5pt\or\v@leur=0.05pt\or\v@leur=0.005pt%
    \or\v@leur=0.0005pt\else\v@leur=\z@\fi%
    \ifdim\v@lX<\z@\advance\v@lX-\v@leur\else\advance\v@lX\v@leur\fi%
    \ifdim\v@lY<\z@\advance\v@lY-\v@leur\else\advance\v@lY\v@leur\fi\fi%
    (\@ffichnb{#1}{\repdecn@mb{\v@lX}},\ifmmode\else\thinspace\fi%
    \@ffichnb{#1}{\repdecn@mb{\v@lY}})}}
\ctr@ld@f\def\@ffichnb#1#2{{\def\@@ffich{\@ffich#1(}\edef\n@mbre{#2}%
    \expandafter\@@ffich\n@mbre)}}
\ctr@ld@f\def\@ffich#1(#2.#3){{#2\ifnum#1>\z@.\fi\def\dig@ts{#3}\s@mme=\z@%
    \loop\ifnum\s@mme<#1\expandafter\@ffichdec\dig@ts:\advance\s@mme\@ne\repeat}}
\ctr@ld@f\def\@ffichdec#1#2:{\relax#1\def\dig@ts{#20}}
\ctr@ld@f\def\figcoordTD#1{{\v@lX=\ptT@unit@\v@lX\v@lY=\ptT@unit@\v@lY\v@lZ=\ptT@unit@\v@lZ%
    \ifr@undcoord\ifcase#1\v@leur=0.5pt\or\v@leur=0.05pt\or\v@leur=0.005pt%
    \or\v@leur=0.0005pt\else\v@leur=\z@\fi%
    \ifdim\v@lX<\z@\advance\v@lX-\v@leur\else\advance\v@lX\v@leur\fi%
    \ifdim\v@lY<\z@\advance\v@lY-\v@leur\else\advance\v@lY\v@leur\fi%
    \ifdim\v@lZ<\z@\advance\v@lZ-\v@leur\else\advance\v@lZ\v@leur\fi\fi%
    (\@ffichnb{#1}{\repdecn@mb{\v@lX}},\ifmmode\else\thinspace\fi%
     \@ffichnb{#1}{\repdecn@mb{\v@lY}},\ifmmode\else\thinspace\fi%
     \@ffichnb{#1}{\repdecn@mb{\v@lZ}})}}
\ctr@ld@f\def\figsetroundcoord#1{\expandafter\Figsetr@undcoord#1:\ignorespaces}
\ctr@ld@f\def\Figsetr@undcoord#1#2:{\if#1n\r@undcoordfalse\else\r@undcoordtrue\fi}
\ctr@ld@f\def\Figsetwr@te#1=#2|{\keln@mun#1|%
    \def\n@mref{m}\ifx\l@debut\n@mref\figsetmark{#2}\else% mark
    \def\n@mref{p}\ifx\l@debut\n@mref\figsetptname{#2}\else% ptname
    \def\n@mref{r}\ifx\l@debut\n@mref\figsetroundcoord{#2}\else% roundcoord
    \W@rnmesAttr{figset write}{#1}\fi\fi\fi}
\ctr@ld@f\def\figsetmark#1{\c@nsymb={#1}\setbox\Gb@xSC=\hbox{\the\c@nsymb}\ignorespaces}
\ctr@ln@m\ptn@me
\ctr@ld@f\def\figsetptname#1{\def\ptn@me##1{#1}\ignorespaces}
\ctr@ld@f\def\FigWrit@L#1:#2(#3,#4){\ignorespaces\@keldist\v@leur{#3}\@keldist\delt@{#4}%
    \C@rp@r@m\def\list@num{#1}\@ecfor\p@int:=\list@num\do{\FigWrit@pt{\p@int}{#2}}}
\ctr@ld@f\def\FigWrit@pt#1#2{\FigWp@r@m{#1}{#2}\Vc@rrect\figWp@si%
    \ifdim\wd\Gb@xSC>\z@\b@undb@xSC{\v@lX}{\v@lY}\fi\figWBB@x}
\ctr@ld@f\def\FigWp@r@m#1#2{\Figg@tXY{#1}%
    \setbox\Gb@x=\hbox{\def\t@xt@{#2}\ifx\t@xt@\empty\Figg@tT{#1}\else#2\fi}\c@lprojSP}
\ctr@ld@f\let\Vc@rrect=\relax
\ctr@ld@f\let\C@rp@r@m=\relax
\ctr@ld@f\def\figwrite[#1]#2{{\ignorespaces\def\list@num{#1}\@ecfor\p@int:=\list@num\do{%
    \setbox\Gb@x=\hbox{\def\t@xt@{#2}\ifx\t@xt@\empty\Figg@tT{\p@int}\else#2\fi}%
    \Figwrit@{\p@int}}}\ignorespaces}
\ctr@ld@f\def\Figwrit@#1{\Figg@tXY{#1}\c@lprojSP%
    \rlap{\kern\v@lX\raise\v@lY\hbox{\unhcopy\Gb@x}}\v@leur=\v@lY%
    \advance\v@lY\ht\Gb@x\b@undb@x{\v@lX}{\v@lY}\advance\v@lX\wd\Gb@x%
    \v@lY=\v@leur\advance\v@lY-\dp\Gb@x\b@undb@x{\v@lX}{\v@lY}}
\ctr@ld@f\def\figwritec[#1]#2{{\ignorespaces\def\list@num{#1}%
    \@ecfor\p@int:=\list@num\do{\Figwrit@c{\p@int}{#2}}}\ignorespaces}
\ctr@ld@f\def\Figwrit@c#1#2{\FigWp@r@m{#1}{#2}%
    \rlap{\kern\v@lX\raise\v@lY\hbox{\rlap{\kern-.5\wd\Gb@x%
    \raise-.5\ht\Gb@x\hbox{\raise.5\dp\Gb@x\hbox{\unhcopy\Gb@x}}}}}%
    \v@leur=\ht\Gb@x\advance\v@leur\dp\Gb@x%
    \advance\v@lX-.5\wd\Gb@x\advance\v@lY-.5\v@leur\b@undb@x{\v@lX}{\v@lY}%
    \advance\v@lX\wd\Gb@x\advance\v@lY\v@leur\b@undb@x{\v@lX}{\v@lY}}
\ctr@ld@f\def\figwritep[#1]{{\ignorespaces\def\list@num{#1}\setbox\Gb@x=\hbox{\c@nterpt}%
    \@ecfor\p@int:=\list@num\do{\Figwrit@{\p@int}}}\ignorespaces}
\ctr@ld@f\def\figwritew#1:#2(#3){\figwritegcw#1:{#2}(#3,0pt)}
\ctr@ld@f\def\figwritee#1:#2(#3){\figwritegce#1:{#2}(#3,0pt)}
\ctr@ld@f\def\figwriten#1:#2(#3){{\def\Vc@rrect{\v@lZ=\v@leur\advance\v@lZ\dp\Gb@x}%
    \Figwrit@NS#1:{#2}(#3)}\ignorespaces}
\ctr@ld@f\def\figwrites#1:#2(#3){{\def\Vc@rrect{\v@lZ=-\v@leur\advance\v@lZ-\ht\Gb@x}%
    \Figwrit@NS#1:{#2}(#3)}\ignorespaces}
\ctr@ld@f\def\Figwrit@NS#1:#2(#3){\let\figWp@si=\FigWp@siNS\let\figWBB@x=\FigWBB@xNS%
    \FigWrit@L#1:{#2}(#3,0pt)}
\ctr@ld@f\def\FigWp@siNS{\rlap{\kern\v@lX\raise\v@lY\hbox{\rlap{\kern-.5\wd\Gb@x%
    \raise\v@lZ\hbox{\unhcopy\Gb@x}}\c@nterpt}}}
\ctr@ld@f\def\FigWBB@xNS{\advance\v@lY\v@lZ%
    \advance\v@lY-\dp\Gb@x\advance\v@lX-.5\wd\Gb@x\b@undb@x{\v@lX}{\v@lY}%
    \advance\v@lY\ht\Gb@x\advance\v@lY\dp\Gb@x%
    \advance\v@lX\wd\Gb@x\b@undb@x{\v@lX}{\v@lY}}
\ctr@ld@f\def\figwritenw#1:#2(#3){{\let\figWp@si=\FigWp@sigW\let\figWBB@x=\FigWBB@xgWE%
    \def\C@rp@r@m{\v@leur=\unssqrttw@\v@leur\delt@=\v@leur%
    \ifdim\delt@=\z@\delt@=\epsil@n\fi}\let@xte={-}\FigWrit@L#1:{#2}(#3,0pt)}\ignorespaces}
\ctr@ld@f\def\figwritesw#1:#2(#3){{\let\figWp@si=\FigWp@sigW\let\figWBB@x=\FigWBB@xgWE%
    \def\C@rp@r@m{\v@leur=\unssqrttw@\v@leur\delt@=-\v@leur%
    \ifdim\delt@=\z@\delt@=-\epsil@n\fi}\let@xte={-}\FigWrit@L#1:{#2}(#3,0pt)}\ignorespaces}
\ctr@ld@f\def\figwritene#1:#2(#3){{\let\figWp@si=\FigWp@sigE\let\figWBB@x=\FigWBB@xgWE%
    \def\C@rp@r@m{\v@leur=\unssqrttw@\v@leur\delt@=\v@leur%
    \ifdim\delt@=\z@\delt@=\epsil@n\fi}\let@xte={}\FigWrit@L#1:{#2}(#3,0pt)}\ignorespaces}
\ctr@ld@f\def\figwritese#1:#2(#3){{\let\figWp@si=\FigWp@sigE\let\figWBB@x=\FigWBB@xgWE%
    \def\C@rp@r@m{\v@leur=\unssqrttw@\v@leur\delt@=-\v@leur%
    \ifdim\delt@=\z@\delt@=-\epsil@n\fi}\let@xte={}\FigWrit@L#1:{#2}(#3,0pt)}\ignorespaces}
\ctr@ld@f\def\figwritegw#1:#2(#3,#4){{\let\figWp@si=\FigWp@sigW\let\figWBB@x=\FigWBB@xgWE%
    \let@xte={-}\FigWrit@L#1:{#2}(#3,#4)}\ignorespaces}
\ctr@ld@f\def\figwritege#1:#2(#3,#4){{\let\figWp@si=\FigWp@sigE\let\figWBB@x=\FigWBB@xgWE%
    \let@xte={}\FigWrit@L#1:{#2}(#3,#4)}\ignorespaces}
\ctr@ld@f\def\FigWp@sigW{\v@lXa=\z@\v@lYa=\ht\Gb@x\advance\v@lYa\dp\Gb@x%
    \ifdim\delt@>\z@\relax%
    \rlap{\kern\v@lX\raise\v@lY\hbox{\rlap{\kern-\wd\Gb@x\kern-\v@leur%
          \raise\delt@\hbox{\raise\dp\Gb@x\hbox{\unhcopy\Gb@x}}}\c@nterpt}}%
    \else\ifdim\delt@<\z@\relax\v@lYa=-\v@lYa%
    \rlap{\kern\v@lX\raise\v@lY\hbox{\rlap{\kern-\wd\Gb@x\kern-\v@leur%
          \raise\delt@\hbox{\raise-\ht\Gb@x\hbox{\unhcopy\Gb@x}}}\c@nterpt}}%
    \else\v@lXa=-.5\v@lYa%
    \rlap{\kern\v@lX\raise\v@lY\hbox{\rlap{\kern-\wd\Gb@x\kern-\v@leur%
          \raise-.5\ht\Gb@x\hbox{\raise.5\dp\Gb@x\hbox{\unhcopy\Gb@x}}}\c@nterpt}}%
    \fi\fi}
\ctr@ld@f\def\FigWp@sigE{\v@lXa=\z@\v@lYa=\ht\Gb@x\advance\v@lYa\dp\Gb@x%
    \ifdim\delt@>\z@\relax%
    \rlap{\kern\v@lX\raise\v@lY\hbox{\c@nterpt\kern\v@leur%
          \raise\delt@\hbox{\raise\dp\Gb@x\hbox{\unhcopy\Gb@x}}}}%
    \else\ifdim\delt@<\z@\relax\v@lYa=-\v@lYa%
    \rlap{\kern\v@lX\raise\v@lY\hbox{\c@nterpt\kern\v@leur%
          \raise\delt@\hbox{\raise-\ht\Gb@x\hbox{\unhcopy\Gb@x}}}}%
    \else\v@lXa=-.5\v@lYa%
    \rlap{\kern\v@lX\raise\v@lY\hbox{\c@nterpt\kern\v@leur%
          \raise-.5\ht\Gb@x\hbox{\raise.5\dp\Gb@x\hbox{\unhcopy\Gb@x}}}}%
    \fi\fi}
\ctr@ld@f\def\FigWBB@xgWE{\advance\v@lY\delt@%
    \advance\v@lX\the\let@xte\v@leur\advance\v@lY\v@lXa\b@undb@x{\v@lX}{\v@lY}%
    \advance\v@lX\the\let@xte\wd\Gb@x\advance\v@lY\v@lYa\b@undb@x{\v@lX}{\v@lY}}
\ctr@ld@f\def\figwritegcw#1:#2(#3,#4){{\let\figWp@si=\FigWp@sigcW\let\figWBB@x=\FigWBB@xgcWE%
    \let@xte={-}\FigWrit@L#1:{#2}(#3,#4)}\ignorespaces}
\ctr@ld@f\def\figwritegce#1:#2(#3,#4){{\let\figWp@si=\FigWp@sigcE\let\figWBB@x=\FigWBB@xgcWE%
    \let@xte={}\FigWrit@L#1:{#2}(#3,#4)}\ignorespaces}
\ctr@ld@f\def\FigWp@sigcW{\rlap{\kern\v@lX\raise\v@lY\hbox{\rlap{\kern-\wd\Gb@x\kern-\v@leur%
     \raise-.5\ht\Gb@x\hbox{\raise\delt@\hbox{\raise.5\dp\Gb@x\hbox{\unhcopy\Gb@x}}}}%
     \c@nterpt}}}
\ctr@ld@f\def\FigWp@sigcE{\rlap{\kern\v@lX\raise\v@lY\hbox{\c@nterpt\kern\v@leur%
    \raise-.5\ht\Gb@x\hbox{\raise\delt@\hbox{\raise.5\dp\Gb@x\hbox{\unhcopy\Gb@x}}}}}}
\ctr@ld@f\def\FigWBB@xgcWE{\v@lZ=\ht\Gb@x\advance\v@lZ\dp\Gb@x%
    \advance\v@lX\the\let@xte\v@leur\advance\v@lY\delt@\advance\v@lY.5\v@lZ%
    \b@undb@x{\v@lX}{\v@lY}%
    \advance\v@lX\the\let@xte\wd\Gb@x\advance\v@lY-\v@lZ\b@undb@x{\v@lX}{\v@lY}}
\ctr@ld@f\def\figwritebn#1:#2(#3){{\def\Vc@rrect{\v@lZ=\v@leur}\Figwrit@NS#1:{#2}(#3)}\ignorespaces}
\ctr@ld@f\def\figwritebs#1:#2(#3){{\def\Vc@rrect{\v@lZ=-\v@leur}\Figwrit@NS#1:{#2}(#3)}\ignorespaces}
\ctr@ld@f\def\figwritebw#1:#2(#3){{\let\figWp@si=\FigWp@sibW\let\figWBB@x=\FigWBB@xbWE%
    \let@xte={-}\FigWrit@L#1:{#2}(#3,0pt)}\ignorespaces}
\ctr@ld@f\def\figwritebe#1:#2(#3){{\let\figWp@si=\FigWp@sibE\let\figWBB@x=\FigWBB@xbWE%
    \let@xte={}\FigWrit@L#1:{#2}(#3,0pt)}\ignorespaces}
\ctr@ld@f\def\FigWp@sibW{\rlap{\kern\v@lX\raise\v@lY\hbox{\rlap{\kern-\wd\Gb@x\kern-\v@leur%
          \hbox{\unhcopy\Gb@x}}\c@nterpt}}}
\ctr@ld@f\def\FigWp@sibE{\rlap{\kern\v@lX\raise\v@lY\hbox{\c@nterpt\kern\v@leur%
          \hbox{\unhcopy\Gb@x}}}}
\ctr@ld@f\def\FigWBB@xbWE{\v@lZ=\ht\Gb@x\advance\v@lZ\dp\Gb@x%
    \advance\v@lX\the\let@xte\v@leur\advance\v@lY\ht\Gb@x\b@undb@x{\v@lX}{\v@lY}%
    \advance\v@lX\the\let@xte\wd\Gb@x\advance\v@lY-\v@lZ\b@undb@x{\v@lX}{\v@lY}}
\ctr@ln@w{newread}\frf@g  \ctr@ln@w{newwrite}\fwf@g
\ctr@ln@w{newif}\ifCUR@PS
\ctr@ln@w{newif}\ifGR@cri
\ctr@ln@w{newif}\ifUse@llipse
\ctr@ln@w{newif}\ifGRdebugm@de \GRdebugm@defalse 
\ctr@ln@w{newif}\ifPDFm@ke
\ifx\pdfliteral\undefined\else\ifnum\pdfoutput>\z@\PDFm@ketrue\fi\fi
\ctr@ld@f\def\initPDF@rDVI{%
\ifPDFm@ke
 \let\figscan=\figscan@E
 \let\newGr@FN=\newGr@FNPDF
 \ctr@ld@f\def\c@mcurveto{c}
 \ctr@ld@f\def\c@mfill{f}
 \ctr@ld@f\def\c@mgsave{q}
 \ctr@ld@f\def\c@mgrestore{Q}
 \ctr@ld@f\def\c@mlineto{l}
 \ctr@ld@f\def\c@mmoveto{m}
 \ctr@ld@f\def\c@msetgray{g}     \ctr@ld@f\def\c@msetgrayStroke{G}
 \ctr@ld@f\def\c@msetcmykcolor{k}\ctr@ld@f\def\c@msetcmykcolorStroke{K}
 \ctr@ld@f\def\c@msetrgbcolor{rg}\ctr@ld@f\def\c@msetrgbcolorStroke{RG}
 \ctr@ld@f\def\d@fprimarC@lor{\CUR@color\space\CUR@colorc@md%
               \space\CUR@color\space\CUR@colorc@mdStroke}
 \ctr@ld@f\def\c@msetdash{d}
 \ctr@ld@f\def\c@msetlinejoin{j}
 \ctr@ld@f\def\c@msetlinewidth{w}
 \ctr@ld@f\def\f@gclosestroke{\immediate\write\fwf@g{s}}
 \ctr@ld@f\def\f@gfill{\immediate\write\fwf@g{\fillc@md}}% Voir la def de \fillc@md ****
 \ctr@ld@f\def\f@gnewpath{}
 \ctr@ld@f\def\f@gstroke{\immediate\write\fwf@g{S}}
\else
 \let\figinsertE=\figinsert
 \let\newGr@FN=\newGr@FNDVI
 \ctr@ld@f\def\c@mcurveto{curveto}
 \ctr@ld@f\def\c@mfill{fill}
 \ctr@ld@f\def\c@mgsave{gsave}
 \ctr@ld@f\def\c@mgrestore{grestore}
 \ctr@ld@f\def\c@mlineto{lineto}
 \ctr@ld@f\def\c@mmoveto{moveto}
 \ctr@ld@f\def\c@msetgray{setgray}          \ctr@ld@f\def\c@msetgrayStroke{}
 \ctr@ld@f\def\c@msetcmykcolor{setcmykcolor}\ctr@ld@f\def\c@msetcmykcolorStroke{}
 \ctr@ld@f\def\c@msetrgbcolor{setrgbcolor}  \ctr@ld@f\def\c@msetrgbcolorStroke{}
 \ctr@ld@f\def\d@fprimarC@lor{\CUR@color\space\CUR@colorc@md}
 \ctr@ld@f\def\c@msetdash{setdash}
 \ctr@ld@f\def\c@msetlinejoin{setlinejoin}
 \ctr@ld@f\def\c@msetlinewidth{setlinewidth}
 \ctr@ld@f\def\f@gclosestroke{\immediate\write\fwf@g{closepath\space stroke}}
 \ctr@ld@f\def\f@gfill{\immediate\write\fwf@g{\fillc@md}}
 \ctr@ld@f\def\f@gnewpath{\immediate\write\fwf@g{newpath}}
 \ctr@ld@f\def\f@gstroke{\immediate\write\fwf@g{stroke}}
\fi}
\ctr@ld@f\def\c@pypsfile#1#2{\c@pyfil@{\immediate\write#1}{#2}}
\ctr@ld@f\def\Figinclud@PDF#1#2{\openin\frf@g=#1\pdfliteral{q #2 0 0 #2 0 0 cm}%
    \c@pyfil@{\pdfliteral}{\frf@g}\pdfliteral{Q}\closein\frf@g}
\ctr@ln@w{newif}\ifmored@ta
\ctr@ln@m\bl@nkline
\ctr@ld@f\def\c@pyfil@#1#2{\def\bl@nkline{\par}{\catcode`\%=12
    \loop\ifeof#2\mored@tafalse\else\mored@tatrue\immediate\read#2 to\tr@c
    \ifx\tr@c\bl@nkline\else#1{\tr@c}\fi\fi\ifmored@ta\repeat}}
\ctr@ld@f\def\keln@mun#1#2|{\def\l@debut{#1}\def\l@suite{#2}}
\ctr@ld@f\def\keln@mde#1#2#3|{\def\l@debut{#1#2}\def\l@suite{#3}}
\ctr@ld@f\def\keln@mtr#1#2#3#4|{\def\l@debut{#1#2#3}\def\l@suite{#4}}
\ctr@ld@f\def\keln@mqu#1#2#3#4#5|{\def\l@debut{#1#2#3#4}\def\l@suite{#5}}
\ctr@ld@f\let\@psffilein=\frf@g % file to \read
\ctr@ln@w{newif}\if@psffileok    % continue looking for the bounding box?
\ctr@ln@w{newif}\if@psfbbfound   % success?
\ctr@ln@w{newif}\if@psfverbose   % report what you're making?
\@psfverbosetrue
\ctr@ln@m\@psfllx \ctr@ln@m\@psflly
\ctr@ln@m\@psfurx \ctr@ln@m\@psfury
\ctr@ln@m\resetcolonc@tcode
\ctr@ld@f\def\@psfgetbb#1{\global\@psfbbfoundfalse%
\global\def\@psfllx{0}\global\def\@psflly{0}%
\global\def\@psfurx{30}\global\def\@psfury{30}%
\openin\@psffilein=#1\relax
\ifeof\@psffilein\errmessage{I couldn't open #1, will ignore it}\else
   \edef\resetcolonc@tcode{\catcode`\noexpand\:\the\catcode`\:\relax}%
   {\@psffileoktrue \chardef\other=12
    \def\do##1{\catcode`##1=\other}\dospecials \catcode`\ =10 \resetcolonc@tcode
    \loop
       \read\@psffilein to \@psffileline
       \ifeof\@psffilein\@psffileokfalse\else
          \expandafter\@psfaux\@psffileline:. \\%
       \fi
   \if@psffileok\repeat
   \if@psfbbfound\else
    \if@psfverbose\message{No bounding box comment in #1; using defaults}\fi\fi
   }\closein\@psffilein\fi}%
\ctr@ln@m\@psfbblit
\ctr@ln@m\@psfpercent
{\catcode`\%=12 \global\let\@psfpercent=%\global\def\@psfbblit{%BoundingBox}}%
\ctr@ln@m\@psfaux
\long\def\@psfaux#1#2:#3\\{\ifx#1\@psfpercent
   \def\testit{#2}\ifx\testit\@psfbblit
      \@psfgrab #3 . . . \\%
      \@psffileokfalse
      \global\@psfbbfoundtrue
   \fi\else\ifx#1\par\else\@psffileokfalse\fi\fi}%
\ctr@ld@f\def\@psfempty{}%
\ctr@ld@f\def\@psfgrab #1 #2 #3 #4 #5\\{%
\global\def\@psfllx{#1}\ifx\@psfllx\@psfempty
      \@psfgrab #2 #3 #4 #5 .\\\else
   \global\def\@psflly{#2}%
   \global\def\@psfurx{#3}\global\def\@psfury{#4}\fi}%
\ctr@ld@f\def\PSwrit@cmd#1#2#3{{\Figg@tXY{#1}\c@lprojSP\b@undb@x{\v@lX}{\v@lY}%
    \v@lX=\ptT@ptps\v@lX\v@lY=\ptT@ptps\v@lY%
    \immediate\write#3{\repdecn@mb{\v@lX}\space\repdecn@mb{\v@lY}\space#2}}}
\ctr@ld@f\def\PSwrit@cmdS#1#2#3#4#5{{\Figg@tXY{#1}\c@lprojSP\b@undb@x{\v@lX}{\v@lY}%
    \global\result@t=\v@lX\global\result@@t=\v@lY%
    \v@lX=\ptT@ptps\v@lX\v@lY=\ptT@ptps\v@lY%
    \immediate\write#3{\repdecn@mb{\v@lX}\space\repdecn@mb{\v@lY}\space#2}}%
    \edef#4{\the\result@t}\edef#5{\the\result@@t}}
\ctr@ld@f\def\update@ttr#1#2#3{\Figdisc@rdLTS{#3}{\n@mref}%
    \ifx\n@mref\D@FTref#2{#1}\else#2{#3}\fi}
\ctr@ld@f\def\D@FTref{default}
\ctr@ld@f\def\W@rnmesAttr#1#2{%
    \immediate\write16{*** Unknown attribute: \BS@ #1(..., #2=...)}}
\ctr@ld@f\def\W@rnmeskwd#1#2{%
    \immediate\write16{*** Unknown keyword #2 in \BS@ #1}}
\ctr@ld@f\def\W@rnmesIgn#1{\immediate\write16{*** \BS@ #1 is ignored inside a
     \BS@ figdrawbegin-\BS@ figdrawend block.}}
\ctr@ld@f\def\Psset@lti#1=#2|{\keln@mtr#1|%
    \def\n@mref{blc}\ifx\l@debut\n@mref\update@ttr\D@FTref\P@setblcolor{#2}\else% base line color
    \def\n@mref{bld}\ifx\l@debut\n@mref\update@ttr\D@FTref\P@setbldash{#2}\else% base line dash
    \def\n@mref{blw}\ifx\l@debut\n@mref\update@ttr\D@FTref\P@setblwidth{#2}\else% base line width
    \def\n@mref{sqc}\ifx\l@debut\n@mref\update@ttr\D@FTref\P@setsqcolor{#2}\else% square color
    \def\n@mref{sqd}\ifx\l@debut\n@mref\update@ttr\D@FTref\P@setsqdash{#2}\else% square dash
    \def\n@mref{sqw}\ifx\l@debut\n@mref\update@ttr\D@FTref\P@setsqwidth{#2}\else% square width
    \W@rnmesAttr{figset altitude}{#1}\fi\fi\fi\fi\fi\fi}
\ctr@ln@m\DDV@blcolor
\ctr@ld@f\def\P@setblcolor#1{\edef\DDV@blcolor{#1}}
\ctr@ln@m\DDV@bldash
\ctr@ld@f\def\P@setbldash#1{\edef\DDV@bldash{#1}}
\ctr@ln@m\DDV@blwidth
\ctr@ld@f\def\P@setblwidth#1{\edef\DDV@blwidth{#1}}
\ctr@ln@m\DDV@sqcolor
\ctr@ld@f\def\P@setsqcolor#1{\edef\DDV@sqcolor{#1}}
\ctr@ln@m\DDV@sqdash
\ctr@ld@f\def\P@setsqdash#1{\edef\DDV@sqdash{#1}}
\ctr@ln@m\DDV@sqwidth
\ctr@ld@f\def\P@setsqwidth#1{\edef\DDV@sqwidth{#1}}
\ctr@ld@f\def\figdrawaltitude#1[#2,#3,#4]{{\ifCUR@PS\ifGR@cri%
    \PSc@mment{altitude Square Dim=#1, Triangle=[#2 / #3,#4]}%
    \s@uvc@ntr@l\et@tpsaltitude\resetc@ntr@l{2}\figptorthoprojline-5:=#2/#3,#4/%
    \figvectP -1[#3,#4]\n@rminf{\v@leur}{-1}\vecunit@{-3}{-1}%
    \figvectP -1[-5,#3]\n@rminf{\v@lmin}{-1}\figvectP -2[-5,#4]\n@rminf{\v@lmax}{-2}%
    \ifdim\v@lmin<\v@lmax\s@mme=#3\else\v@lmax=\v@lmin\s@mme=#4\fi%
    \figvectP -4[-5,#2]\vecunit@{-4}{-4}\delt@=#1\unit@%
    \edef\t@ille{\repdecn@mb{\delt@}}\figpttra-1:=-5/\t@ille,-3/%
    \figptstra-3=-5,-1/\t@ille,-4/\figdrawline[#2,-5]%
    \Pss@tspecifSt{color=\DDV@sqcolor,dash=\DDV@sqdash,width=\DDV@sqwidth}%
    \figdrawline[-1,-2,-3]%
    \Psrest@reSt{color=\DDV@sqcolor,dash=\DDV@sqdash,width=\DDV@sqwidth}%
    \ifdim\v@leur<\v@lmax%
    \Pss@tspecifSt{color=\DDV@blcolor,dash=\DDV@bldash,width=\DDV@blwidth}%
    \figdrawline[-5,\the\s@mme]%
    \Psrest@reSt{color=\DDV@blcolor,dash=\DDV@bldash,width=\DDV@blwidth}%
    \fi\PSc@mment{End altitude}\resetc@ntr@l\et@tpsaltitude\fi\fi}}
\ctr@ld@f\def\Ps@rcerc#1;#2(#3,#4){\ellBB@x#1;#2,#2(#3,#4,0)%
    \f@gnewpath{\delt@=#2\unit@\delt@=\ptT@ptps\delt@%
    \BdingB@xfalse%
    \PSwrit@cmd{#1}{\repdecn@mb{\delt@}\space #3\space #4\space arc}{\fwf@g}}}
\ctr@ln@m\figdrawarccirc
\ctr@ld@f\def\Q@arccircDD#1;#2(#3,#4){\ifCUR@PS\ifGR@cri%
    \PSc@mment{arccircDD Center=#1 ; Radius=#2 (Ang1=#3, Ang2=#4)}%
    \iffillm@de\Ps@rcerc#1;#2(#3,#4)%
    \f@gfill%
    \else\Ps@rcerc#1;#2(#3,#4)\f@gstroke\fi%
    \PSc@mment{End arccircDD}\fi\fi}
\ctr@ld@f\def\Q@arccircTD#1,#2,#3;#4(#5,#6){{\ifCUR@PS\ifGR@cri\s@uvc@ntr@l\et@tpsarccircTD%
    \PSc@mment{arccircTD Center=#1,P1=#2,P2=#3 ; Radius=#4 (Ang1=#5, Ang2=#6)}%
    \setc@ntr@l{2}\c@lExtAxes#1,#2,#3(#4)\Q@arcellPATD#1,-4,-5(#5,#6)%
    \PSc@mment{End arccircTD}\resetc@ntr@l\et@tpsarccircTD\fi\fi}}
\ctr@ld@f\def\c@lExtAxes#1,#2,#3(#4){%
    \figvectPTD-5[#1,#2]\vecunit@{-5}{-5}\figvectNTD-4[#1,#2,#3]\vecunit@{-4}{-4}%
    \figvectNVTD-3[-4,-5]\delt@=#4\unit@\edef\r@yon{\repdecn@mb{\delt@}}%
    \figpttra-4:=#1/\r@yon,-5/\figpttra-5:=#1/\r@yon,-3/}
\ctr@ln@m\figdrawarccircP
\ctr@ld@f\def\Q@arccircPDD#1;#2[#3,#4]{{\ifCUR@PS\ifGR@cri\s@uvc@ntr@l\et@tpsarccircPDD%
    \PSc@mment{arccircPDD Center=#1; Radius=#2, [P1=#3, P2=#4]}%
    \Ps@ngleparam#1;#2[#3,#4]\ifdim\v@lmin>\v@lmax\advance\v@lmax\DePI@deg\fi%
    \edef\@ngdeb{\repdecn@mb{\v@lmin}}\edef\@ngfin{\repdecn@mb{\v@lmax}}%
    \figdrawarccirc#1;\r@dius(\@ngdeb,\@ngfin)%
    \PSc@mment{End arccircPDD}\resetc@ntr@l\et@tpsarccircPDD\fi\fi}}
\ctr@ld@f\def\Q@arccircPTD#1;#2[#3,#4,#5]{{\ifCUR@PS\ifGR@cri\s@uvc@ntr@l\et@tpsarccircPTD%
    \PSc@mment{arccircPTD Center=#1; Radius=#2, [P1=#3, P2=#4, P3=#5]}%
    \setc@ntr@l{2}\c@lExtAxes#1,#3,#5(#2)\figdrawarcellPP#1,-4,-5[#3,#4]%
    \PSc@mment{End arccircPTD}\resetc@ntr@l\et@tpsarccircPTD\fi\fi}}
\ctr@ld@f\def\Ps@ngleparam#1;#2[#3,#4]{\setc@ntr@l{2}%
    \figvectPDD-1[#1,#3]\vecunit@{-1}{-1}\Figg@tXY{-1}\arct@n\v@lmin(\v@lX,\v@lY)%
    \figvectPDD-2[#1,#4]\vecunit@{-2}{-2}\Figg@tXY{-2}\arct@n\v@lmax(\v@lX,\v@lY)%
    \v@lmin=\rdT@deg\v@lmin\v@lmax=\rdT@deg\v@lmax%
    \v@leur=#2pt\maxim@m{\mili@u}{-\v@leur}{\v@leur}%
    \edef\r@dius{\repdecn@mb{\mili@u}}}
\ctr@ld@f\def\Ps@rcercBz#1;#2(#3,#4){\Ps@rellBz#1;#2,#2(#3,#4,0)}
\ctr@ld@f\def\Ps@rellBz#1;#2,#3(#4,#5,#6){%
    \ellBB@x#1;#2,#3(#4,#5,#6)\BdingB@xfalse%
    \c@lNbarcs{#4}{#5}\v@leur=#4pt\setc@ntr@l{2}\figptell-13::#1;#2,#3(#4,#6)%
    \f@gnewpath\PSwrit@cmd{-13}{\c@mmoveto}{\fwf@g}%
    \s@mme=\z@\bcl@rellBz#1;#2,#3(#6)\BdingB@xtrue}
\ctr@ld@f\def\bcl@rellBz#1;#2,#3(#4){\relax%
    \ifnum\s@mme<\p@rtent\advance\s@mme\@ne%
    \advance\v@leur\delt@\edef\@ngle{\repdecn@mb\v@leur}\figptell-14::#1;#2,#3(\@ngle,#4)%
    \advance\v@leur\delt@\edef\@ngle{\repdecn@mb\v@leur}\figptell-15::#1;#2,#3(\@ngle,#4)%
    \advance\v@leur\delt@\edef\@ngle{\repdecn@mb\v@leur}\figptell-16::#1;#2,#3(\@ngle,#4)%
    \figptscontrolDD-18[-13,-14,-15,-16]%
    \PSwrit@cmd{-18}{}{\fwf@g}\PSwrit@cmd{-17}{}{\fwf@g}%
    \PSwrit@cmd{-16}{\c@mcurveto}{\fwf@g}%
    \figptcopyDD-13:/-16/\bcl@rellBz#1;#2,#3(#4)\fi}
\ctr@ld@f\def\Ps@rell#1;#2,#3(#4,#5,#6){\ellBB@x#1;#2,#3(#4,#5,#6)%
    \f@gnewpath{\v@lmin=#2\unit@\v@lmin=\ptT@ptps\v@lmin%
    \v@lmax=#3\unit@\v@lmax=\ptT@ptps\v@lmax\BdingB@xfalse%
    \PSwrit@cmd{#1}%
    {#6\space\repdecn@mb{\v@lmin}\space\repdecn@mb{\v@lmax}\space #4\space #5\space ellipse}{\fwf@g}}%
    \global\Use@llipsetrue}
\ctr@ln@m\figdrawarcell
\ctr@ld@f\def\Q@arcellDD#1;#2,#3(#4,#5,#6){{\ifCUR@PS\ifGR@cri%
    \PSc@mment{arcellDD Center=#1 ; XRad=#2, YRad=#3 (Ang1=#4, Ang2=#5, Inclination=#6)}%
    \iffillm@de\Ps@rell#1;#2,#3(#4,#5,#6)%
    \f@gfill%
    \else\Ps@rell#1;#2,#3(#4,#5,#6)\f@gstroke\fi%
    \PSc@mment{End arcellDD}\fi\fi}}
\ctr@ld@f\def\Q@arcellTD#1;#2,#3(#4,#5,#6){{\ifCUR@PS\ifGR@cri\s@uvc@ntr@l\et@tpsarcellTD%
    \PSc@mment{arcellTD Center=#1 ; XRad=#2, YRad=#3 (Ang1=#4, Ang2=#5, Inclination=#6)}%
    \setc@ntr@l{2}\figpttraC -8:=#1/#2,0,0/\figpttraC -7:=#1/0,#3,0/%
    \figvectC -4(0,0,1)\figptsrot -8=-8,-7/#1,#6,-4/\Q@arcellPATD#1,-8,-7(#4,#5)%
    \PSc@mment{End arcellTD}\resetc@ntr@l\et@tpsarcellTD\fi\fi}}
\ctr@ln@m\figdrawarcellPA
\ctr@ld@f\def\Q@arcellPADD#1,#2,#3(#4,#5){{\ifCUR@PS\ifGR@cri\s@uvc@ntr@l\et@tpsarcellPADD%
    \PSc@mment{arcellPADD Center=#1,PtAxis1=#2,PtAxis2=#3 (Ang1=#4, Ang2=#5)}%
    \setc@ntr@l{2}\figvectPDD-1[#1,#2]\vecunit@DD{-1}{-1}\v@lX=\ptT@unit@\result@t%
    \edef\XR@d{\repdecn@mb{\v@lX}}\Figg@tXY{-1}\arct@n\v@lmin(\v@lX,\v@lY)%
    \v@lmin=\rdT@deg\v@lmin\edef\Inclin@{\repdecn@mb{\v@lmin}}%
    \figgetdist\YR@d[#1,#3]\Q@arcellDD#1;\XR@d,\YR@d(#4,#5,\Inclin@)%
    \PSc@mment{End arcellPADD}\resetc@ntr@l\et@tpsarcellPADD\fi\fi}}
\ctr@ld@f\def\Q@arcellPATD#1,#2,#3(#4,#5){{\ifCUR@PS\ifGR@cri\s@uvc@ntr@l\et@tpsarcellPATD%
    \PSc@mment{arcellPATD Center=#1,PtAxis1=#2,PtAxis2=#3 (Ang1=#4, Ang2=#5)}%
    \iffillm@de\Ps@rellPATD#1,#2,#3(#4,#5)%
    \f@gfill%
    \else\Ps@rellPATD#1,#2,#3(#4,#5)\f@gstroke\fi%
    \PSc@mment{End arcellPATD}\resetc@ntr@l\et@tpsarcellPATD\fi\fi}}
\ctr@ld@f\def\Ps@rellPATD#1,#2,#3(#4,#5){\let\c@lprojSP=\relax%
    \setc@ntr@l{2}\figvectPTD-1[#1,#2]\figvectPTD-2[#1,#3]\c@lNbarcs{#4}{#5}%
    \v@leur=#4pt\c@lptellP{#1}{-1}{-2}\Figptpr@j-5:/-3/%
    \f@gnewpath\PSwrit@cmdS{-5}{\c@mmoveto}{\fwf@g}{\X@un}{\Y@un}%
    \edef\C@nt@r{#1}\s@mme=\z@\bcl@rellPATD}
\ctr@ld@f\def\bcl@rellPATD{\relax%
    \ifnum\s@mme<\p@rtent\advance\s@mme\@ne%
    \advance\v@leur\delt@\c@lptellP{\C@nt@r}{-1}{-2}\Figptpr@j-4:/-3/%
    \advance\v@leur\delt@\c@lptellP{\C@nt@r}{-1}{-2}\Figptpr@j-6:/-3/%
    \advance\v@leur\delt@\c@lptellP{\C@nt@r}{-1}{-2}\Figptpr@j-3:/-3/%
    \v@lX=\z@\v@lY=\z@\Figtr@nptDD{-5}{-5}\Figtr@nptDD{2}{-3}%
    \divide\v@lX\@vi\divide\v@lY\@vi%
    \Figtr@nptDD{3}{-4}\Figtr@nptDD{-1.5}{-6}\v@lmin=\v@lX\v@lmax=\v@lY%
    \v@lX=\z@\v@lY=\z@\Figtr@nptDD{2}{-5}\Figtr@nptDD{-5}{-3}%
    \divide\v@lX\@vi\divide\v@lY\@vi\Figtr@nptDD{-1.5}{-4}\Figtr@nptDD{3}{-6}%
    \BdingB@xfalse%
    \Figp@intregDD-4:(\v@lmin,\v@lmax)\PSwrit@cmdS{-4}{}{\fwf@g}{\X@de}{\Y@de}%
    \Figp@intregDD-4:(\v@lX,\v@lY)\PSwrit@cmdS{-4}{}{\fwf@g}{\X@tr}{\Y@tr}%
    \BdingB@xtrue\PSwrit@cmdS{-3}{\c@mcurveto}{\fwf@g}{\X@qu}{\Y@qu}%
    \B@zierBB@x{1}{\Y@un}(\X@un,\X@de,\X@tr,\X@qu)%
    \B@zierBB@x{2}{\X@un}(\Y@un,\Y@de,\Y@tr,\Y@qu)%
    \edef\X@un{\X@qu}\edef\Y@un{\Y@qu}\figptcopyDD-5:/-3/\bcl@rellPATD\fi}
\ctr@ld@f\def\c@lNbarcs#1#2{%
    \delt@=#2pt\advance\delt@-#1pt\maxim@m{\v@lmax}{\delt@}{-\delt@}%
    \v@leur=\v@lmax\divide\v@leur45 \p@rtentiere{\p@rtent}{\v@leur}\advance\p@rtent\@ne%
    \s@mme=\p@rtent\multiply\s@mme\thr@@\divide\delt@\s@mme}
\ctr@ld@f\def\figdrawarcellPP#1,#2,#3[#4,#5]{{\ifCUR@PS\ifGR@cri\s@uvc@ntr@l\et@tpsarcellPP%
    \PSc@mment{arcellPP Center=#1,PtAxis1=#2,PtAxis2=#3 [Point1=#4, Point2=#5]}%
    \setc@ntr@l{2}\figvectP-2[#1,#3]\vecunit@{-2}{-2}\v@lmin=\result@t%
    \invers@{\v@lmax}{\v@lmin}%
    \figvectP-1[#1,#2]\vecunit@{-1}{-1}\v@leur=\result@t%
    \v@leur=\repdecn@mb{\v@lmax}\v@leur\edef\AsB@{\repdecn@mb{\v@leur}}% a/b
    \c@lAngle{#1}{#4}{\v@lmin}\edef\@ngdeb{\repdecn@mb{\v@lmin}}%
    \c@lAngle{#1}{#5}{\v@lmax}\ifdim\v@lmin>\v@lmax\advance\v@lmax\DePI@deg\fi%
    \edef\@ngfin{\repdecn@mb{\v@lmax}}\figdrawarcellPA#1,#2,#3(\@ngdeb,\@ngfin)%
    \PSc@mment{End arcellPP}\resetc@ntr@l\et@tpsarcellPP\fi\fi}}
\ctr@ld@f\def\c@lAngle#1#2#3{\figvectP-3[#1,#2]%
    \c@lproscal\delt@[-3,-1]\c@lproscal\v@leur[-3,-2]%
    \v@leur=\AsB@\v@leur\arct@n#3(\delt@,\v@leur)#3=\rdT@deg#3}
\ctr@ln@w{newif}\if@rrowratio\@rrowratiotrue
\ctr@ln@w{newif}\if@rrowhfill
\ctr@ln@w{newif}\if@rrowhout
\ctr@ld@f\def\Psset@rrowhe@d#1=#2|{\keln@mun#1|%
    \def\n@mref{a}\ifx\l@debut\n@mref\update@ttr\D@FTarrowheadangle\Q@s@tarrowheadangle{#2}\else% angle
    \def\n@mref{f}\ifx\l@debut\n@mref\update@ttr\D@FTarrowheadfill\Q@s@tarrowheadfill{#2}\else% fillmode
    \def\n@mref{l}\ifx\l@debut\n@mref\update@ttr\D@FTarrowheadlength\Q@s@tarrowheadlength{#2}\else% length
    \def\n@mref{o}\ifx\l@debut\n@mref\update@ttr\D@FTarrowheadout\Q@s@tarrowheadout{#2}\else% out
    \def\n@mref{r}\ifx\l@debut\n@mref\update@ttr\D@FTarrowheadratio\Q@s@tarrowheadratio{#2}\else% ratio
    \W@rnmesAttr{figset arrowhead}{#1}\fi\fi\fi\fi\fi}
\ctr@ln@m\@rrowheadangle
\ctr@ln@m\C@AHANG \ctr@ln@m\S@AHANG \ctr@ln@m\UNSS@N
\ctr@ld@f\def\Q@s@tarrowheadangle#1{\edef\@rrowheadangle{#1}{\c@ssin{\C@}{\S@}{#1}%
    \xdef\C@AHANG{\C@}\xdef\S@AHANG{\S@}\v@lmax=\S@ pt%
    \invers@{\v@leur}{\v@lmax}\maxim@m{\v@leur}{\v@leur}{-\v@leur}%
    \xdef\UNSS@N{\the\v@leur}}}
\ctr@ld@f\def\Q@s@tarrowheadfill#1{\expandafter\set@rrowhfill#1:}
\ctr@ld@f\def\set@rrowhfill#1#2:{\if#1n\@rrowhfillfalse\else\@rrowhfilltrue\fi}
\ctr@ld@f\def\Q@s@tarrowheadout#1{\expandafter\set@rrowhout#1:}
\ctr@ld@f\def\set@rrowhout#1#2:{\if#1n\@rrowhoutfalse\else\@rrowhouttrue\fi}
\ctr@ln@m\@rrowheadlength
\ctr@ld@f\def\Q@s@tarrowheadlength#1{\edef\@rrowheadlength{#1}\@rrowratiofalse}
\ctr@ln@m\@rrowheadratio
\ctr@ld@f\def\Q@s@tarrowheadratio#1{\edef\@rrowheadratio{#1}\@rrowratiotrue}
\ctr@ln@m\D@FTarrowheadlength
\ctr@ld@f\def\figresetarrowhead{%
    \Q@s@tarrowheadangle{\D@FTarrowheadangle}%
    \Q@s@tarrowheadfill{\D@FTarrowheadfill}%
    \Q@s@tarrowheadout{\D@FTarrowheadout}%
    \Q@s@tarrowheadratio{\D@FTarrowheadratio}%
    \d@fm@cdim\D@FTarrowheadlength{\D@FTh@rdahlength}% Valeur par defaut...
    \Q@s@tarrowheadlength{\D@FTarrowheadlength}}
\ctr@ld@f\def\D@FTarrowheadratio{0.1}
\ctr@ld@f\def\D@FTarrowheadangle{20}
\ctr@ld@f\def\D@FTarrowheadfill{no}
\ctr@ld@f\def\D@FTarrowheadout{no}
\ctr@ld@f\def\D@FTh@rdahlength{8pt}
\ctr@ln@m\figdrawarrow
\ctr@ld@f\def\Q@arrowDD[#1,#2]{{\ifCUR@PS\ifGR@cri\s@uvc@ntr@l\et@tpsarrow%
    \PSc@mment{arrowDD [Pt1,Pt2]=[#1,#2]}\Q@s@tfillmode{no}%
    \Q@arrowheadDD[#1,#2]\setc@ntr@l{2}\figdrawline[#1,-3]%
    \PSc@mment{End arrowDD}\resetc@ntr@l\et@tpsarrow\fi\fi}}
\ctr@ld@f\def\Q@arrowTD[#1,#2]{{\ifCUR@PS\ifGR@cri\s@uvc@ntr@l\et@tpsarrowTD%
    \PSc@mment{arrowTD [Pt1,Pt2]=[#1,#2]}\resetc@ntr@l{2}%
    \Figptpr@j-5:/#1/\Figptpr@j-6:/#2/\let\c@lprojSP=\relax\Q@arrowDD[-5,-6]%
    \PSc@mment{End arrowTD}\resetc@ntr@l\et@tpsarrowTD\fi\fi}}
\ctr@ln@m\figdrawarrowhead
\ctr@ld@f\def\Q@arrowheadDD[#1,#2]{{\ifCUR@PS\ifGR@cri\s@uvc@ntr@l\et@tpsarrowheadDD%
    \if@rrowhfill\def\@hangle{-\@rrowheadangle}\else\def\@hangle{\@rrowheadangle}\fi%
    \if@rrowratio%
    \if@rrowhout\def\@hratio{-\@rrowheadratio}\else\def\@hratio{\@rrowheadratio}\fi%
    \PSc@mment{arrowheadDD Ratio=\@hratio, Angle=\@hangle, [Pt1,Pt2]=[#1,#2]}%
    \Ps@rrowhead\@hratio,\@hangle[#1,#2]%
    \else%
    \if@rrowhout\def\@hlength{-\@rrowheadlength}\else\def\@hlength{\@rrowheadlength}\fi%
    \PSc@mment{arrowheadDD Length=\@hlength, Angle=\@hangle, [Pt1,Pt2]=[#1,#2]}%
    \Ps@rrowheadfd\@hlength,\@hangle[#1,#2]%
    \fi%
    \PSc@mment{End arrowheadDD}\resetc@ntr@l\et@tpsarrowheadDD\fi\fi}}
\ctr@ld@f\def\Q@arrowheadTD[#1,#2]{{\ifCUR@PS\ifGR@cri\s@uvc@ntr@l\et@tpsarrowheadTD%
    \PSc@mment{arrowheadTD [Pt1,Pt2]=[#1,#2]}\resetc@ntr@l{2}%
    \Figptpr@j-5:/#1/\Figptpr@j-6:/#2/\let\c@lprojSP=\relax\Q@arrowheadDD[-5,-6]%
    \PSc@mment{End arrowheadTD}\resetc@ntr@l\et@tpsarrowheadTD\fi\fi}}
\ctr@ld@f\def\Ps@rrowhead#1,#2[#3,#4]{\v@leur=#1\p@\maxim@m{\v@leur}{\v@leur}{-\v@leur}%
    \ifdim\v@leur>\Cepsil@n{% Arrow is not degenerated
    \PSc@mment{@rrowhead Ratio=#1, Angle=#2, [Pt1,Pt2]=[#3,#4]}\v@leur=\UNSS@N%
    \v@leur=\CUR@width\v@leur\v@leur=\ptpsT@pt\v@leur\delt@=.5\v@leur% = width / (2 sin(Angle))
    \setc@ntr@l{2}\figvectPDD-3[#4,#3]%
    \Figg@tXY{-3}\v@lX=#1\v@lX\v@lY=#1\v@lY\Figv@ctCreg-3(\v@lX,\v@lY)%
    \vecunit@{-4}{-3}\mili@u=\result@t%
    \ifdim#2pt>\z@\v@lXa=-\C@AHANG\delt@%
     \edef\c@ef{\repdecn@mb{\v@lXa}}\figpttraDD-3:=-3/\c@ef,-4/\fi%
    \edef\c@ef{\repdecn@mb{\delt@}}%
    \v@lXa=\mili@u\v@lXa=\C@AHANG\v@lXa%
    \v@lYa=\ptpsT@pt\p@\v@lYa=\CUR@width\v@lYa\v@lYa=\sDcc@ngle\v@lYa%
    \advance\v@lXa-\v@lYa\gdef\sDcc@ngle{0}%
    \ifdim\v@lXa>\v@leur\edef\c@efendpt{\repdecn@mb{\v@leur}}%
    \else\edef\c@efendpt{\repdecn@mb{\v@lXa}}\fi%
    \Figg@tXY{-3}\v@lmin=\v@lX\v@lmax=\v@lY%
    \v@lXa=\C@AHANG\v@lmin\v@lYa=\S@AHANG\v@lmax\advance\v@lXa\v@lYa%
    \v@lYa=-\S@AHANG\v@lmin\v@lX=\C@AHANG\v@lmax\advance\v@lYa\v@lX%
    \setc@ntr@l{1}\Figg@tXY{#4}\advance\v@lX\v@lXa\advance\v@lY\v@lYa%
    \setc@ntr@l{2}\Figp@intregDD-2:(\v@lX,\v@lY)%
    \v@lXa=\C@AHANG\v@lmin\v@lYa=-\S@AHANG\v@lmax\advance\v@lXa\v@lYa%
    \v@lYa=\S@AHANG\v@lmin\v@lX=\C@AHANG\v@lmax\advance\v@lYa\v@lX%
    \setc@ntr@l{1}\Figg@tXY{#4}\advance\v@lX\v@lXa\advance\v@lY\v@lYa%
    \setc@ntr@l{2}\Figp@intregDD-1:(\v@lX,\v@lY)%
    \ifdim#2pt<\z@\fillm@detrue\figdrawline[-2,#4,-1]% fill
    \else\figptstraDD-3=#4,-2,-1/\c@ef,-4/\s@uvdash{\typ@dash}\Q@s@tdash{\D@FTdash}%
    \figdrawline[-2,-3,-1]\Q@s@tdash{\typ@dash}\fi% no fill
    \ifdim#1pt>\z@\figpttraDD-3:=#4/\c@efendpt,-4/\else\figptcopyDD-3:/#4/\fi%
    \PSc@mment{End @rrowhead}}\fi}
\ctr@ld@f\def\sDcc@ngle{0}% Initialisation
\ctr@ld@f\def\Ps@rrowheadfd#1,#2[#3,#4]{{%
    \PSc@mment{@rrowheadfd Length=#1, Angle=#2, [Pt1,Pt2]=[#3,#4]}%
    \setc@ntr@l{2}\figvectPDD-1[#3,#4]\n@rmeucDD{\v@leur}{-1}\v@leur=\ptT@unit@\v@leur%
    \invers@{\v@leur}{\v@leur}\v@leur=#1\v@leur\edef\R@tio{\repdecn@mb{\v@leur}}%
    \Ps@rrowhead\R@tio,#2[#3,#4]\PSc@mment{End @rrowheadfd}}}
\ctr@ln@m\figdrawarrowBezier
\ctr@ld@f\def\Q@arrowBezierDD[#1,#2,#3,#4]{{\ifCUR@PS\ifGR@cri\s@uvc@ntr@l\et@tpsarrowBezierDD%
    \PSc@mment{arrowBezierDD Control points=#1,#2,#3,#4}\setc@ntr@l{2}%
    \if@rrowratio\c@larclengthDD\v@leur,10[#1,#2,#3,#4]\else\v@leur=\z@\fi%
    \Ps@rrowB@zDD\v@leur[#1,#2,#3,#4]%
    \PSc@mment{End arrowBezierDD}\resetc@ntr@l\et@tpsarrowBezierDD\fi\fi}}
\ctr@ld@f\def\Q@arrowBezierTD[#1,#2,#3,#4]{{\ifCUR@PS\ifGR@cri\s@uvc@ntr@l\et@tpsarrowBezierTD%
    \PSc@mment{arrowBezierTD Control points=#1,#2,#3,#4}\resetc@ntr@l{2}%
    \Figptpr@j-7:/#1/\Figptpr@j-8:/#2/\Figptpr@j-9:/#3/\Figptpr@j-10:/#4/%
    \let\c@lprojSP=\relax\ifnum\CUR@proj<\tw@\Q@arrowBezierDD[-7,-8,-9,-10]%
    \else\f@gnewpath\PSwrit@cmd{-7}{\c@mmoveto}{\fwf@g}%
    \if@rrowratio\c@larclengthDD\mili@u,10[-7,-8,-9,-10]\else\mili@u=\z@\fi%
    \p@rtent=\NBz@rcs\advance\p@rtent\m@ne\subB@zierTD\p@rtent[#1,#2,#3,#4]%
    \f@gstroke%
    \advance\v@lmin\p@rtent\delt@% Initialized in \subB@zierTD
    \v@leur=\v@lmin\advance\v@leur0.33333 \delt@\edef\unti@rs{\repdecn@mb{\v@leur}}%
    \v@leur=\v@lmin\advance\v@leur0.66666 \delt@\edef\deti@rs{\repdecn@mb{\v@leur}}%
    \figptcopyDD-8:/-10/\c@lsubBzarc\unti@rs,\deti@rs[#1,#2,#3,#4]%
    \figptcopyDD-8:/-4/\figptcopyDD-9:/-3/\Ps@rrowB@zDD\mili@u[-7,-8,-9,-10]\fi%
    \PSc@mment{End arrowBezierTD}\resetc@ntr@l\et@tpsarrowBezierTD\fi\fi}}
\ctr@ld@f\def\c@larclengthDD#1,#2[#3,#4,#5,#6]{{\p@rtent=#2\figptcopyDD-5:/#3/%
    \delt@=\p@\divide\delt@\p@rtent\c@rre=\z@\v@leur=\z@\s@mme=\z@%
    \loop\ifnum\s@mme<\p@rtent\advance\s@mme\@ne\advance\v@leur\delt@%
    \edef\T@{\repdecn@mb{\v@leur}}\figptBezierDD-6::\T@[#3,#4,#5,#6]%
    \figvectPDD-1[-5,-6]\n@rmeucDD{\mili@u}{-1}\advance\c@rre\mili@u%
    \figptcopyDD-5:/-6/\repeat\global\result@t=\ptT@unit@\c@rre}#1=\result@t}
\ctr@ld@f\def\Ps@rrowB@zDD#1[#2,#3,#4,#5]{{\Q@s@tfillmode{no}%
    \if@rrowratio\delt@=\@rrowheadratio#1\else\delt@=\@rrowheadlength pt\fi%
    \v@leur=\C@AHANG\delt@\edef\R@dius{\repdecn@mb{\v@leur}}%
    \FigptintercircB@zDD-5::0,\R@dius[#5,#4,#3,#2]%
    \Q@s@tarrowheadlength{\repdecn@mb{\delt@}}\Q@arrowheadDD[-5,#5]%
    \let\n@rmeuc=\n@rmeucDD\figgetdist\R@dius[#5,-3]%
    \FigptintercircB@zDD-6::0,\R@dius[#5,#4,#3,#2]%
    \figptBezierDD-5::0.33333[#5,#4,#3,#2]\figptBezierDD-3::0.66666[#5,#4,#3,#2]%
    \figptscontrolDD-5[-6,-5,-3,#2]\Q@BezierDD1[-6,-5,-4,#2]}}
\ctr@ln@m\figdrawarrowcirc
\ctr@ld@f\def\Q@arrowcircDD#1;#2(#3,#4){{\ifCUR@PS\ifGR@cri\s@uvc@ntr@l\et@tpsarrowcircDD%
    \PSc@mment{arrowcircDD Center=#1 ; Radius=#2 (Ang1=#3,Ang2=#4)}%
    \Q@s@tfillmode{no}\Pscirc@rrowhead#1;#2(#3,#4)%
    \setc@ntr@l{2}\figvectPDD -4[#1,-3]\vecunit@{-4}{-4}%
    \Figg@tXY{-4}\arct@n\v@lmin(\v@lX,\v@lY)%
    \v@lmin=\rdT@deg\v@lmin\v@leur=#4pt\advance\v@leur-\v@lmin%
    \maxim@m{\v@leur}{\v@leur}{-\v@leur}%
    \ifdim\v@leur>\DemiPI@deg\relax\ifdim\v@lmin<#4pt\advance\v@lmin\DePI@deg%
    \else\advance\v@lmin-\DePI@deg\fi\fi\edef\ar@ngle{\repdecn@mb{\v@lmin}}%
    \ifdim#3pt<#4pt\figdrawarccirc#1;#2(#3,\ar@ngle)\else\figdrawarccirc#1;#2(\ar@ngle,#3)\fi%
    \PSc@mment{End arrowcircDD}\resetc@ntr@l\et@tpsarrowcircDD\fi\fi}}
\ctr@ld@f\def\Q@arrowcircTD#1,#2,#3;#4(#5,#6){{\ifCUR@PS\ifGR@cri\s@uvc@ntr@l\et@tpsarrowcircTD%
    \PSc@mment{arrowcircTD Center=#1,P1=#2,P2=#3 ; Radius=#4 (Ang1=#5, Ang2=#6)}%
    \resetc@ntr@l{2}\c@lExtAxes#1,#2,#3(#4)\let\c@lprojSP=\relax%
    \figvectPTD-11[#1,-4]\figvectPTD-12[#1,-5]\c@lNbarcs{#5}{#6}%
    \if@rrowratio\v@lmax=\degT@rd\v@lmax\edef\D@lpha{\repdecn@mb{\v@lmax}}\fi%
    \advance\p@rtent\m@ne\mili@u=\z@%
    \v@leur=#5pt\c@lptellP{#1}{-11}{-12}\Figptpr@j-9:/-3/%
    \f@gnewpath\PSwrit@cmdS{-9}{\c@mmoveto}{\fwf@g}{\X@un}{\Y@un}%
    \edef\C@nt@r{#1}\s@mme=\z@\bcl@rcircTD\f@gstroke%
    \advance\v@leur\delt@\c@lptellP{#1}{-11}{-12}\Figptpr@j-5:/-3/%
    \advance\v@leur\delt@\c@lptellP{#1}{-11}{-12}\Figptpr@j-6:/-3/%
    \advance\v@leur\delt@\c@lptellP{#1}{-11}{-12}\Figptpr@j-10:/-3/%
    \figptscontrolDD-8[-9,-5,-6,-10]%
    \if@rrowratio\c@lcurvradDD0.5[-9,-8,-7,-10]\advance\mili@u\result@t%
    \maxim@m{\mili@u}{\mili@u}{-\mili@u}\mili@u=\ptT@unit@\mili@u%
    \mili@u=\D@lpha\mili@u\advance\p@rtent\@ne\divide\mili@u\p@rtent\fi%
    \Ps@rrowB@zDD\mili@u[-9,-8,-7,-10]%
    \PSc@mment{End arrowcircTD}\resetc@ntr@l\et@tpsarrowcircTD\fi\fi}}
\ctr@ld@f\def\bcl@rcircTD{\relax%
    \ifnum\s@mme<\p@rtent\advance\s@mme\@ne%
    \advance\v@leur\delt@\c@lptellP{\C@nt@r}{-11}{-12}\Figptpr@j-5:/-3/%
    \advance\v@leur\delt@\c@lptellP{\C@nt@r}{-11}{-12}\Figptpr@j-6:/-3/%
    \advance\v@leur\delt@\c@lptellP{\C@nt@r}{-11}{-12}\Figptpr@j-10:/-3/%
    \figptscontrolDD-8[-9,-5,-6,-10]\BdingB@xfalse%
    \PSwrit@cmdS{-8}{}{\fwf@g}{\X@de}{\Y@de}\PSwrit@cmdS{-7}{}{\fwf@g}{\X@tr}{\Y@tr}%
    \BdingB@xtrue\PSwrit@cmdS{-10}{\c@mcurveto}{\fwf@g}{\X@qu}{\Y@qu}%
    \if@rrowratio\c@lcurvradDD0.5[-9,-8,-7,-10]\advance\mili@u\result@t\fi%
    \B@zierBB@x{1}{\Y@un}(\X@un,\X@de,\X@tr,\X@qu)%
    \B@zierBB@x{2}{\X@un}(\Y@un,\Y@de,\Y@tr,\Y@qu)%
    \edef\X@un{\X@qu}\edef\Y@un{\Y@qu}\figptcopyDD-9:/-10/\bcl@rcircTD\fi}
\ctr@ld@f\def\Pscirc@rrowhead#1;#2(#3,#4){{%
    \PSc@mment{circ@rrowhead Center=#1 ; Radius=#2 (Ang1=#3,Ang2=#4)}%
    \v@leur=#2\unit@\edef\s@glen{\repdecn@mb{\v@leur}}\v@lY=\z@\v@lX=\v@leur%
    \resetc@ntr@l{2}\Figv@ctCreg-3(\v@lX,\v@lY)\figpttraDD-5:=#1/1,-3/%
    \figptrotDD-5:=-5/#1,#4/%
    \figvectPDD-3[#1,-5]\Figg@tXY{-3}\v@leur=\v@lX%
    \ifdim#3pt<#4pt\v@lX=\v@lY\v@lY=-\v@leur\else\v@lX=-\v@lY\v@lY=\v@leur\fi%
    \Figv@ctCreg-3(\v@lX,\v@lY)\vecunit@{-3}{-3}%
    \if@rrowratio\v@leur=#4pt\advance\v@leur-#3pt\maxim@m{\mili@u}{-\v@leur}{\v@leur}%
    \mili@u=\degT@rd\mili@u\v@leur=\s@glen\mili@u\edef\s@glen{\repdecn@mb{\v@leur}}%
    \mili@u=#2\mili@u\mili@u=\@rrowheadratio\mili@u\else\mili@u=\@rrowheadlength pt\fi%
    \figpttraDD-6:=-5/\s@glen,-3/\v@leur=#2pt\v@leur=2\v@leur%
    \invers@{\v@leur}{\v@leur}\c@rre=\repdecn@mb{\v@leur}\mili@u% = sin = L/(2R)
    \mili@u=\c@rre\mili@u=\repdecn@mb{\c@rre}\mili@u%
    \v@leur=\p@\advance\v@leur-\mili@u% \v@leur = cos*cos
    \invers@{\mili@u}{2\v@leur}\delt@=\c@rre\delt@=\repdecn@mb{\mili@u}\delt@%
    \xdef\sDcc@ngle{\repdecn@mb{\delt@}}% sin/(2*cos*cos) used in \Ps@rrowhead
    \sqrt@{\mili@u}{\v@leur}\arct@n\v@leur(\mili@u,\c@rre)%
    \v@leur=\rdT@deg\v@leur% \cor@ngle = atan(L/sqrt(4R*R-L*L))
    \ifdim#3pt<#4pt\v@leur=-\v@leur\fi%
    \if@rrowhout\v@leur=-\v@leur\fi\edef\cor@ngle{\repdecn@mb{\v@leur}}%
    \figptrotDD-6:=-6/-5,\cor@ngle/\Q@arrowheadDD[-6,-5]%
    \PSc@mment{End circ@rrowhead}}}
\ctr@ln@m\figdrawarrowcircP
\ctr@ld@f\def\Q@arrowcircPDD#1;#2[#3,#4]{{\ifCUR@PS\ifGR@cri%
    \PSc@mment{arrowcircPDD Center=#1; Radius=#2, [P1=#3,P2=#4]}%
    \s@uvc@ntr@l\et@tpsarrowcircPDD\Ps@ngleparam#1;#2[#3,#4]%
    \ifdim\v@leur>\z@\ifdim\v@lmin>\v@lmax\advance\v@lmax\DePI@deg\fi%
    \else\ifdim\v@lmin<\v@lmax\advance\v@lmin\DePI@deg\fi\fi%
    \edef\@ngdeb{\repdecn@mb{\v@lmin}}\edef\@ngfin{\repdecn@mb{\v@lmax}}%
    \figdrawarrowcirc#1;\r@dius(\@ngdeb,\@ngfin)%
    \PSc@mment{End arrowcircPDD}\resetc@ntr@l\et@tpsarrowcircPDD\fi\fi}}
\ctr@ld@f\def\Q@arrowcircPTD#1;#2[#3,#4,#5]{{\ifCUR@PS\ifGR@cri\s@uvc@ntr@l\et@tpsarrowcircPTD%
    \PSc@mment{arrowcircPTD Center=#1; Radius=#2, [P1=#3,P2=#4,P3=#5]}%
    \figgetangleTD\@ngfin[#1,#3,#4,#5]\v@leur=#2pt%
    \maxim@m{\mili@u}{-\v@leur}{\v@leur}\edef\r@dius{\repdecn@mb{\mili@u}}%
    \ifdim\v@leur<\z@\v@lmax=\@ngfin pt\advance\v@lmax-\DePI@deg%
    \edef\@ngfin{\repdecn@mb{\v@lmax}}\fi\Q@arrowcircTD#1,#3,#5;\r@dius(0,\@ngfin)%
    \PSc@mment{End arrowcircPTD}\resetc@ntr@l\et@tpsarrowcircPTD\fi\fi}}
\ctr@ld@f\def\figdrawaxes#1(#2){{\ifCUR@PS\ifGR@cri\s@uvc@ntr@l\et@tpsaxes%
    \PSc@mment{axes Origin=#1 Range=(#2)}\an@lys@xes#2,:\resetc@ntr@l{2}%
    \ifx\t@xt@\empty\ifTr@isDim\Q@@xes#1(0,#2,0,#2,0,#2)\else\Q@@xes#1(0,#2,0,#2)\fi%
    \else\Q@@xes#1(#2)\fi\PSc@mment{End axes}\resetc@ntr@l\et@tpsaxes\fi\fi}}
\ctr@ld@f\def\an@lys@xes#1,#2:{\def\t@xt@{#2}}
\ctr@ln@m\Q@@xes
\ctr@ld@f\def\Q@@xesDD#1(#2,#3,#4,#5){%
    \figpttraC-5:=#1/#2,0/\figpttraC-6:=#1/#3,0/\Q@arrowDD[-5,-6]%
    \figpttraC-5:=#1/0,#4/\figpttraC-6:=#1/0,#5/\Q@arrowDD[-5,-6]}
\ctr@ld@f\def\Q@@xesTD#1(#2,#3,#4,#5,#6,#7){%
    \figpttraC-7:=#1/#2,0,0/\figpttraC-8:=#1/#3,0,0/\Q@arrowTD[-7,-8]%
    \figpttraC-7:=#1/0,#4,0/\figpttraC-8:=#1/0,#5,0/\Q@arrowTD[-7,-8]%
    \figpttraC-7:=#1/0,0,#6/\figpttraC-8:=#1/0,0,#7/\Q@arrowTD[-7,-8]}
\ctr@ln@m\newGr@FN
\ctr@ld@f\def\newGr@FNPDF#1{\s@mme=\Gr@FNb\advance\s@mme\@ne\xdef\Gr@FNb{\number\s@mme}}
\ctr@ld@f\def\newGr@FNDVI#1{\newGr@FNPDF{}\xdef#1{\jobname GI\Gr@FNb.anx}}
\ctr@ld@f\def\figdrawbegin#1{\newGr@FN\DefGIfilen@me\gdef\@utoFN{0}%
    \def\t@xt@{#1}\relax\ifx\t@xt@\empty\GRupdatem@detrue%
    \gdef\@utoFN{1}\Psb@ginfig\DefGIfilen@me\else\expandafter\Psb@ginfigNu@#1 :\fi}
\ctr@ld@f\def\Psb@ginfigNu@#1 #2:{\def\t@xt@{#1}\relax\ifx\t@xt@\empty\def\t@xt@{#2}%
    \ifx\t@xt@\empty\GRupdatem@detrue\gdef\@utoFN{1}\Psb@ginfig\DefGIfilen@me%
    \else\Psb@ginfigNu@#2:\fi\else\Psb@ginfig{#1}\fi}
\ctr@ln@m\PSfilen@me \ctr@ln@m\auxfilen@me
\ctr@ld@f\def\Psb@ginfig#1{\ifCUR@PS\else%
    \edef\PSfilen@me{#1}\edef\auxfilen@me{\jobname.anx}%
    \ifGRupdatem@de\GR@critrue\else\openin\frf@g=\PSfilen@me\relax%
    \ifeof\frf@g\GR@critrue\else\GR@crifalse\fi\closein\frf@g\fi%
    \CUR@PStrue\c@ldefproj\expandafter\setupd@te\D@FTupdate:%
    \ifGR@cri\initb@undb@x%
    \immediate\openout\fwf@g=\auxfilen@me\initpss@ttings\fi%
    \fi}
\ctr@ld@f\def\Gr@FNb{0}
\ctr@ld@f\def\figforTeXFileno{\Gr@FNb}
\ctr@ld@f\def\figforTeXFigno{0 }
\ctr@ld@f\def\figforTeXnextFigno{1 }
\ctr@ld@f\edef\DefGIfilen@me{\jobname GI.anx}
\ctr@ld@f\def\initpss@ttings{\figreset{altitude,arrowhead,curve,general,flowchart,mesh,trimesh}%
    \Use@llipsefalse}
\ctr@ld@f\def\B@zierBB@x#1#2(#3,#4,#5,#6){{\c@rre=\t@n\epsil@n% Do not reduce this value
    \v@lmax=#4\advance\v@lmax-#5\v@lmax=\thr@@\v@lmax\advance\v@lmax#6\advance\v@lmax-#3%
    \mili@u=#4\mili@u=-\tw@\mili@u\advance\mili@u#3\advance\mili@u#5%
    \v@lmin=#4\advance\v@lmin-#3\maxim@m{\v@leur}{-\v@lmax}{\v@lmax}%
    \maxim@m{\delt@}{-\mili@u}{\mili@u}\maxim@m{\v@leur}{\v@leur}{\delt@}%
    \maxim@m{\delt@}{-\v@lmin}{\v@lmin}\maxim@m{\v@leur}{\v@leur}{\delt@}%
    \ifdim\v@leur>\c@rre\invers@{\v@leur}{\v@leur}\edef\Uns@rM@x{\repdecn@mb{\v@leur}}%
    \v@lmax=\Uns@rM@x\v@lmax\mili@u=\Uns@rM@x\mili@u\v@lmin=\Uns@rM@x\v@lmin%
    \maxim@m{\v@leur}{-\v@lmax}{\v@lmax}\ifdim\v@leur<\c@rre%
    \maxim@m{\v@leur}{-\mili@u}{\mili@u}\ifdim\v@leur<\c@rre\else%
    \invers@{\mili@u}{\mili@u}\v@leur=-0.5\v@lmin%
    \v@leur=\repdecn@mb{\mili@u}\v@leur\m@jBBB@x{\v@leur}{#1}{#2}(#3,#4,#5,#6)\fi%
    \else\delt@=\repdecn@mb{\mili@u}\mili@u\v@leur=\repdecn@mb{\v@lmax}\v@lmin%
    \advance\delt@-\v@leur\ifdim\delt@<\z@\else\invers@{\v@lmax}{\v@lmax}%
    \edef\Uns@rAp{\repdecn@mb{\v@lmax}}\sqrt@{\delt@}{\delt@}%
    \v@leur=-\mili@u\advance\v@leur\delt@\v@leur=\Uns@rAp\v@leur%
    \m@jBBB@x{\v@leur}{#1}{#2}(#3,#4,#5,#6)%
    \v@leur=-\mili@u\advance\v@leur-\delt@\v@leur=\Uns@rAp\v@leur%
    \m@jBBB@x{\v@leur}{#1}{#2}(#3,#4,#5,#6)\fi\fi\fi}}
\ctr@ld@f\def\m@jBBB@x#1#2#3(#4,#5,#6,#7){{\relax\ifdim#1>\z@\ifdim#1<\p@%
    \edef\T@{\repdecn@mb{#1}}\v@lX=\p@\advance\v@lX-#1\edef\UNmT@{\repdecn@mb{\v@lX}}%
    \v@lX=#4\v@lY=#5\v@lZ=#6\v@lXa=#7\v@lX=\UNmT@\v@lX\advance\v@lX\T@\v@lY%
    \v@lY=\UNmT@\v@lY\advance\v@lY\T@\v@lZ\v@lZ=\UNmT@\v@lZ\advance\v@lZ\T@\v@lXa%
    \v@lX=\UNmT@\v@lX\advance\v@lX\T@\v@lY\v@lY=\UNmT@\v@lY\advance\v@lY\T@\v@lZ%
    \v@lX=\UNmT@\v@lX\advance\v@lX\T@\v@lY%
    \ifcase#2\or\v@lY=#3\or\v@lY=\v@lX\v@lX=#3\fi\b@undb@x{\v@lX}{\v@lY}\fi\fi}}
\ctr@ld@f\def\PsB@zier#1[#2]{{\f@gnewpath%
    \s@mme=\z@\def\list@num{#2,0}\extrairelepremi@r\p@int\de\list@num%
    \PSwrit@cmdS{\p@int}{\c@mmoveto}{\fwf@g}{\X@un}{\Y@un}\p@rtent=#1\bclB@zier}}
\ctr@ld@f\def\bclB@zier{\relax%
    \ifnum\s@mme<\p@rtent\advance\s@mme\@ne\BdingB@xfalse%
    \extrairelepremi@r\p@int\de\list@num\PSwrit@cmdS{\p@int}{}{\fwf@g}{\X@de}{\Y@de}%
    \extrairelepremi@r\p@int\de\list@num\PSwrit@cmdS{\p@int}{}{\fwf@g}{\X@tr}{\Y@tr}%
    \BdingB@xtrue%
    \extrairelepremi@r\p@int\de\list@num\PSwrit@cmdS{\p@int}{\c@mcurveto}{\fwf@g}{\X@qu}{\Y@qu}%
    \B@zierBB@x{1}{\Y@un}(\X@un,\X@de,\X@tr,\X@qu)%
    \B@zierBB@x{2}{\X@un}(\Y@un,\Y@de,\Y@tr,\Y@qu)%
    \edef\X@un{\X@qu}\edef\Y@un{\Y@qu}\bclB@zier\fi}
\ctr@ln@m\figdrawBezier
\ctr@ld@f\def\Q@BezierDD#1[#2]{\ifCUR@PS\ifGR@cri%
    \PSc@mment{BezierDD N arcs=#1, Control points=#2}%
    \iffillm@de\PsB@zier#1[#2]%
    \f@gfill%
    \else\PsB@zier#1[#2]\f@gstroke\fi%
    \PSc@mment{End BezierDD}\fi\fi}
\ctr@ln@m\et@tpsBezierTD% ou doubler les {}
\ctr@ld@f\def\Q@BezierTD#1[#2]{\ifCUR@PS\ifGR@cri\s@uvc@ntr@l\et@tpsBezierTD%
    \PSc@mment{BezierTD N arcs=#1, Control points=#2}%
    \iffillm@de\PsB@zierTD#1[#2]%
    \f@gfill%
    \else\PsB@zierTD#1[#2]\f@gstroke\fi%
    \PSc@mment{End BezierTD}\resetc@ntr@l\et@tpsBezierTD\fi\fi}
\ctr@ld@f\def\PsB@zierTD#1[#2]{\ifnum\CUR@proj<\tw@\PsB@zier#1[#2]\else\PsB@zier@TD#1[#2]\fi}
\ctr@ld@f\def\PsB@zier@TD#1[#2]{{\f@gnewpath%
    \s@mme=\z@\def\list@num{#2,0}\extrairelepremi@r\p@int\de\list@num%
    \let\c@lprojSP=\relax\setc@ntr@l{2}\Figptpr@j-7:/\p@int/%
    \PSwrit@cmd{-7}{\c@mmoveto}{\fwf@g}%
    \loop\ifnum\s@mme<#1\advance\s@mme\@ne\extrairelepremi@r\p@intun\de\list@num%
    \extrairelepremi@r\p@intde\de\list@num\extrairelepremi@r\p@inttr\de\list@num%
    \subB@zierTD\NBz@rcs[\p@int,\p@intun,\p@intde,\p@inttr]\edef\p@int{\p@inttr}\repeat}}
\ctr@ld@f\def\subB@zierTD#1[#2,#3,#4,#5]{\delt@=\p@\divide\delt@\NBz@rcs\v@lmin=\z@%
    {\Figg@tXY{-7}\edef\X@un{\the\v@lX}\edef\Y@un{\the\v@lY}%
    \s@mme=\z@\loop\ifnum\s@mme<#1\advance\s@mme\@ne%
    \v@leur=\v@lmin\advance\v@leur0.33333 \delt@\edef\unti@rs{\repdecn@mb{\v@leur}}%
    \v@leur=\v@lmin\advance\v@leur0.66666 \delt@\edef\deti@rs{\repdecn@mb{\v@leur}}%
    \advance\v@lmin\delt@\edef\trti@rs{\repdecn@mb{\v@lmin}}%
    \figptBezierTD-8::\trti@rs[#2,#3,#4,#5]\Figptpr@j-8:/-8/%
    \c@lsubBzarc\unti@rs,\deti@rs[#2,#3,#4,#5]\BdingB@xfalse%
    \PSwrit@cmdS{-4}{}{\fwf@g}{\X@de}{\Y@de}\PSwrit@cmdS{-3}{}{\fwf@g}{\X@tr}{\Y@tr}%
    \BdingB@xtrue\PSwrit@cmdS{-8}{\c@mcurveto}{\fwf@g}{\X@qu}{\Y@qu}%
    \B@zierBB@x{1}{\Y@un}(\X@un,\X@de,\X@tr,\X@qu)%
    \B@zierBB@x{2}{\X@un}(\Y@un,\Y@de,\Y@tr,\Y@qu)%
    \edef\X@un{\X@qu}\edef\Y@un{\Y@qu}\figptcopyDD-7:/-8/\repeat}}
\ctr@ld@f\def\NBz@rcs{2}
\ctr@ld@f\def\c@lsubBzarc#1,#2[#3,#4,#5,#6]{\figptBezierTD-5::#1[#3,#4,#5,#6]%
    \figptBezierTD-6::#2[#3,#4,#5,#6]\Figptpr@j-4:/-5/\Figptpr@j-5:/-6/%
    \figptscontrolDD-4[-7,-4,-5,-8]}
\ctr@ln@m\figdrawcirc
\ctr@ld@f\def\Q@circDD#1(#2){\ifCUR@PS\ifGR@cri\PSc@mment{circDD Center=#1 (Radius=#2)}%
    \Q@arccircDD#1;#2(0,360)\PSc@mment{End circDD}\fi\fi}
\ctr@ld@f\def\Q@circTD#1,#2,#3(#4){\ifCUR@PS\ifGR@cri%
    \PSc@mment{circTD Center=#1,P1=#2,P2=#3 (Radius=#4)}%
    \Q@arccircTD#1,#2,#3;#4(0,360)\PSc@mment{End circTD}\fi\fi}
\ctr@ln@m\p@urcent
{\catcode`\%=12\gdef\p@urcent{%}}
\ctr@ld@f\def\PSc@mment#1{\ifGRdebugm@de\immediate\write\fwf@g{\p@urcent\space#1}\fi}
\ctr@ln@m\acc@louv \ctr@ln@m\acc@lfer
{\catcode`\[=1\catcode`\{=12\gdef\acc@louv[{}}
{\catcode`\]=2\catcode`\}=12\gdef\acc@lfer{}]]
\ctr@ld@f\def\PSdict@{\ifUse@llipse%
    \immediate\write\fwf@g{/ellipsedict 9 dict def ellipsedict /mtrx matrix put}%
    \immediate\write\fwf@g{/ellipse \acc@louv ellipsedict begin}%
    \immediate\write\fwf@g{ /endangle exch def /startangle exch def}%
    \immediate\write\fwf@g{ /yrad exch def /xrad exch def}%
    \immediate\write\fwf@g{ /rotangle exch def /y exch def /x exch def}%
    \immediate\write\fwf@g{ /savematrix mtrx currentmatrix def}%
    \immediate\write\fwf@g{ x y translate rotangle rotate xrad yrad scale}%
    \immediate\write\fwf@g{ 0 0 1 startangle endangle arc}%
    \immediate\write\fwf@g{ savematrix setmatrix end\acc@lfer def}%
    \fi\PShe@der{EndProlog}}
\ctr@ld@f\def\Pssetc@rve#1=#2|{\keln@mun#1|%
    \def\n@mref{r}\ifx\l@debut\n@mref\update@ttr\D@FTroundness\Q@s@troundness{#2}\else% roundness
    \W@rnmesAttr{figset curve}{#1}\fi}
\ctr@ln@m\curv@roundness
\ctr@ld@f\def\Q@s@troundness#1{\edef\curv@roundness{#1}}
\ctr@ld@f\def\D@FTroundness{0.2} % Valeur par defaut
\ctr@ln@m\figdrawcurve
\ctr@ld@f\def\Q@curveDD[#1]{{\ifCUR@PS\ifGR@cri\PSc@mment{curveDD Points=#1}%
    \s@uvc@ntr@l\et@tpscurveDD%
    \iffillm@de\Psc@rveDD\curv@roundness[#1]%
    \f@gfill%
    \else\Psc@rveDD\curv@roundness[#1]\f@gstroke\fi%
    \PSc@mment{End curveDD}\resetc@ntr@l\et@tpscurveDD\fi\fi}}
\ctr@ld@f\def\Q@curveTD[#1]{{\ifCUR@PS\ifGR@cri%
    \PSc@mment{curveTD Points=#1}\s@uvc@ntr@l\et@tpscurveTD\let\c@lprojSP=\relax%
    \iffillm@de\Psc@rveTD\curv@roundness[#1]%
    \f@gfill%
    \else\Psc@rveTD\curv@roundness[#1]\f@gstroke\fi%
    \PSc@mment{End curveTD}\resetc@ntr@l\et@tpscurveTD\fi\fi}}
\ctr@ld@f\def\Psc@rveDD#1[#2]{%
    \def\list@num{#2}\extrairelepremi@r\Ak@\de\list@num%
    \extrairelepremi@r\Ai@\de\list@num\extrairelepremi@r\Aj@\de\list@num%
    \f@gnewpath\PSwrit@cmdS{\Ai@}{\c@mmoveto}{\fwf@g}{\X@un}{\Y@un}%
    \setc@ntr@l{2}\figvectPDD -1[\Ak@,\Aj@]%
    \@ecfor\Ak@:=\list@num\do{\figpttraDD-2:=\Ai@/#1,-1/\BdingB@xfalse%
       \PSwrit@cmdS{-2}{}{\fwf@g}{\X@de}{\Y@de}%
       \figvectPDD -1[\Ai@,\Ak@]\figpttraDD-2:=\Aj@/-#1,-1/%
       \PSwrit@cmdS{-2}{}{\fwf@g}{\X@tr}{\Y@tr}\BdingB@xtrue%
       \PSwrit@cmdS{\Aj@}{\c@mcurveto}{\fwf@g}{\X@qu}{\Y@qu}%
       \B@zierBB@x{1}{\Y@un}(\X@un,\X@de,\X@tr,\X@qu)%
       \B@zierBB@x{2}{\X@un}(\Y@un,\Y@de,\Y@tr,\Y@qu)%
       \edef\X@un{\X@qu}\edef\Y@un{\Y@qu}\edef\Ai@{\Aj@}\edef\Aj@{\Ak@}}}
\ctr@ld@f\def\Psc@rveTD#1[#2]{\ifnum\CUR@proj<\tw@\Psc@rvePPTD#1[#2]\else\Psc@rveCPTD#1[#2]\fi}
\ctr@ld@f\def\Psc@rvePPTD#1[#2]{\setc@ntr@l{2}%
    \def\list@num{#2}\extrairelepremi@r\Ak@\de\list@num\Figptpr@j-5:/\Ak@/%
    \extrairelepremi@r\Ai@\de\list@num\Figptpr@j-3:/\Ai@/%
    \extrairelepremi@r\Aj@\de\list@num\Figptpr@j-4:/\Aj@/%
    \f@gnewpath\PSwrit@cmdS{-3}{\c@mmoveto}{\fwf@g}{\X@un}{\Y@un}%
    \figvectPDD -1[-5,-4]%
    \@ecfor\Ak@:=\list@num\do{\Figptpr@j-5:/\Ak@/\figpttraDD-2:=-3/#1,-1/%
       \BdingB@xfalse\PSwrit@cmdS{-2}{}{\fwf@g}{\X@de}{\Y@de}%
       \figvectPDD -1[-3,-5]\figpttraDD-2:=-4/-#1,-1/%
       \PSwrit@cmdS{-2}{}{\fwf@g}{\X@tr}{\Y@tr}\BdingB@xtrue%
       \PSwrit@cmdS{-4}{\c@mcurveto}{\fwf@g}{\X@qu}{\Y@qu}%
       \B@zierBB@x{1}{\Y@un}(\X@un,\X@de,\X@tr,\X@qu)%
       \B@zierBB@x{2}{\X@un}(\Y@un,\Y@de,\Y@tr,\Y@qu)%
       \edef\X@un{\X@qu}\edef\Y@un{\Y@qu}\figptcopyDD-3:/-4/\figptcopyDD-4:/-5/}}
\ctr@ld@f\def\Psc@rveCPTD#1[#2]{\setc@ntr@l{2}%
    \def\list@num{#2}\extrairelepremi@r\Ak@\de\list@num%
    \extrairelepremi@r\Ai@\de\list@num\extrairelepremi@r\Aj@\de\list@num%
    \Figptpr@j-7:/\Ai@/%
    \f@gnewpath\PSwrit@cmd{-7}{\c@mmoveto}{\fwf@g}%
    \figvectPTD -9[\Ak@,\Aj@]%
    \@ecfor\Ak@:=\list@num\do{\figpttraTD-10:=\Ai@/#1,-9/%
       \figvectPTD -9[\Ai@,\Ak@]\figpttraTD-11:=\Aj@/-#1,-9/%
       \subB@zierTD\NBz@rcs[\Ai@,-10,-11,\Aj@]\edef\Ai@{\Aj@}\edef\Aj@{\Ak@}}}
\ctr@ld@f\def\figdrawend{\ifCUR@PS\ifGR@cri\immediate\closeout\fwf@g%
    \immediate\openout\fwf@g=\PSfilen@me\relax%
    \ifPDFm@ke\PSBdingB@x\else%
    \immediate\write\fwf@g{\p@urcent\string!PS-Adobe-2.0 EPSF-2.0}%
    \PShe@der{Creator\string: TeX (fig4tex.tex)}%
    \PShe@der{Title\string: \PSfilen@me}%
    \PShe@der{CreationDate\string: \the\day/\the\month/\the\year}%
    \PSBdingB@x%
    \PShe@der{EndComments}\PSdict@\fi%
    \immediate\write\fwf@g{\c@mgsave}%
    \openin\frf@g=\auxfilen@me\c@pypsfile\fwf@g\frf@g\closein\frf@g%
    \immediate\write\fwf@g{\c@mgrestore}%
    \PSc@mment{End of file.}\immediate\closeout\fwf@g%
    \immediate\openout\fwf@g=\auxfilen@me\immediate\closeout\fwf@g%
    \immediate\write16{File \PSfilen@me\space created.}\fi\fi\CUR@PSfalse\GR@critrue}
\ctr@ld@f\def\PShe@der#1{\immediate\write\fwf@g{\p@urcent\p@urcent#1}}
\ctr@ld@f\def\PSBdingB@x{{\v@lX=\ptT@ptps\c@@rdXmin\v@lY=\ptT@ptps\c@@rdYmin%
     \v@lXa=\ptT@ptps\c@@rdXmax\v@lYa=\ptT@ptps\c@@rdYmax%
     \PShe@der{BoundingBox\string: \repdecn@mb{\v@lX}\space\repdecn@mb{\v@lY}%
     \space\repdecn@mb{\v@lXa}\space\repdecn@mb{\v@lYa}}}}
\ctr@ld@f\def\figdrawfcconnect[#1]{{\ifCUR@PS\ifGR@cri\PSc@mment{fcconnect Points=#1}%
    \Q@s@tfillmode{no}\s@uvc@ntr@l\et@tpsfcconnect\resetc@ntr@l{2}%
    \fcc@nnect@[#1]\resetc@ntr@l\et@tpsfcconnect\PSc@mment{End fcconnect}\fi\fi}}
\ctr@ld@f\def\fcc@nnect@[#1]{\let\N@rm=\n@rmeucDD\def\list@num{#1}%
    \extrairelepremi@r\Ai@\de\list@num\edef\pr@m{\Ai@}\v@leur=\z@\p@rtent=\@ne\c@llgtot%
    \ifcase\fclin@typ@\edef\list@num{[\pr@m,#1,\Ai@}\expandafter\figdrawcurve\list@num]%
    \else\ifdim\fclin@r@d\p@>\z@\Pslin@conge[#1]\else\figdrawline[#1]\fi\fi%
    \v@leur=\@rrowp@s\v@leur\edef\list@num{#1,\Ai@,0}%
    \extrairelepremi@r\Ai@\de\list@num\mili@u=\epsil@n\c@llgpart%
    \advance\mili@u-\epsil@n\advance\mili@u-\delt@\advance\v@leur-\mili@u%
    \ifcase\fclin@typ@\invers@\mili@u\delt@%
    \ifnum\@rrowr@fpt>\z@\advance\delt@-\v@leur\v@leur=\delt@\fi%
    \v@leur=\repdecn@mb\v@leur\mili@u\edef\v@lt{\repdecn@mb\v@leur}%
    \extrairelepremi@r\Ak@\de\list@num%
    \figvectPDD-1[\pr@m,\Aj@]\figpttraDD-6:=\Ai@/\curv@roundness,-1/%
    \figvectPDD-1[\Ak@,\Ai@]\figpttraDD-7:=\Aj@/\curv@roundness,-1/%
    \delt@=\@rrowheadlength\p@\delt@=\C@AHANG\delt@\edef\R@dius{\repdecn@mb{\delt@}}%
    \ifcase\@rrowr@fpt%
    \FigptintercircB@zDD-8::\v@lt,\R@dius[\Ai@,-6,-7,\Aj@]\Q@arrowheadDD[-5,-8]\else%
    \FigptintercircB@zDD-8::\v@lt,\R@dius[\Aj@,-7,-6,\Ai@]\Q@arrowheadDD[-8,-5]\fi%
    \else\advance\delt@-\v@leur%
    \p@rtentiere{\p@rtent}{\delt@}\edef\C@efun{\the\p@rtent}%
    \p@rtentiere{\p@rtent}{\v@leur}\edef\C@efde{\the\p@rtent}%
    \figptbaryDD-5:[\Ai@,\Aj@;\C@efun,\C@efde]\ifcase\@rrowr@fpt%
    \delt@=\@rrowheadlength\unit@\delt@=\C@AHANG\delt@\edef\t@ille{\repdecn@mb{\delt@}}%
    \figvectPDD-2[\Ai@,\Aj@]\vecunit@{-2}{-2}\figpttraDD-5:=-5/\t@ille,-2/\fi%
    \Q@arrowheadDD[\Ai@,-5]\fi}
\ctr@ld@f\def\c@llgtot{\@ecfor\Aj@:=\list@num\do{\figvectP-1[\Ai@,\Aj@]\N@rm\delt@{-1}%
    \advance\v@leur\delt@\advance\p@rtent\@ne\edef\Ai@{\Aj@}}}
\ctr@ld@f\def\c@llgpart{\extrairelepremi@r\Aj@\de\list@num\figvectP-1[\Ai@,\Aj@]\N@rm\delt@{-1}%
    \advance\mili@u\delt@\ifdim\mili@u<\v@leur\edef\pr@m{\Ai@}\edef\Ai@{\Aj@}\c@llgpart\fi}
\ctr@ld@f\def\Pslin@conge[#1]{\ifnum\p@rtent>\tw@{\def\list@num{#1}%
    \extrairelepremi@r\Ai@\de\list@num\extrairelepremi@r\Aj@\de\list@num%
    \figptcopy-6:/\Ai@/\figvectP-3[\Ai@,\Aj@]\vecunit@{-3}{-3}\v@lmax=\result@t%
    \@ecfor\Ak@:=\list@num\do{\figvectP-4[\Aj@,\Ak@]\vecunit@{-4}{-4}%
    \minim@m\v@lmin\v@lmax\result@t\v@lmax=\result@t%
    \det@rm\delt@[-3,-4]\maxim@m\mili@u{\delt@}{-\delt@}\ifdim\mili@u>\Cepsil@n%
    \ifdim\delt@>\z@\figgetangleDD\Angl@[\Aj@,\Ak@,\Ai@]\else%
    \figgetangleDD\Angl@[\Aj@,\Ai@,\Ak@]\fi%
    \v@leur=\PI@deg\advance\v@leur-\Angl@\p@\divide\v@leur\tw@%
    \edef\Angl@{\repdecn@mb\v@leur}\c@ssin{\C@}{\S@}{\Angl@}\v@leur=\fclin@r@d\unit@%
    \v@leur=\S@\v@leur\mili@u=\C@\p@\invers@\mili@u\mili@u%
    \v@leur=\repdecn@mb{\mili@u}\v@leur%
    \minim@m\v@leur\v@leur\v@lmin\edef\t@ille{\repdecn@mb{\v@leur}}%
    \figpttra-5:=\Aj@/-\t@ille,-3/\figdrawline[-6,-5]\figpttra-6:=\Aj@/\t@ille,-4/%
    \figvectNVDD-3[-3]\figvectNVDD-8[-4]\inters@cDD-7:[-5,-3;-6,-8]%
    \ifdim\delt@>\z@\figdrawarccircP-7;\fclin@r@d[-5,-6]\else\figdrawarccircP-7;\fclin@r@d[-6,-5]\fi%
    \else\figdrawline[-6,\Aj@]\figptcopy-6:/\Aj@/\fi% Points alignes
    \edef\Ai@{\Aj@}\edef\Aj@{\Ak@}\figptcopy-3:/-4/}\figdrawline[-6,\Aj@]}\else\figdrawline[#1]\fi}
\ctr@ld@f\def\figdrawfcnode[#1]#2{{\ifCUR@PS\ifGR@cri\PSc@mment{fcnode Points=#1}%
    \s@uvc@ntr@l\et@tpsfcnode\resetc@ntr@l{2}%
    \def\t@xt@{#2}\ifx\t@xt@\empty\def\g@tt@xt{\setbox\Gb@x=\hbox{\Figg@tT{\p@int}}}%
    \else\def\g@tt@xt{\setbox\Gb@x=\hbox{#2}}\fi%
    \v@lmin=\h@rdfcXp@dd\advance\v@lmin\Xp@dd\unit@\multiply\v@lmin\tw@%
    \v@lmax=\h@rdfcYp@dd\advance\v@lmax\Yp@dd\unit@\multiply\v@lmax\tw@%
    \Figv@ctCreg-8(\unit@,-\unit@)\def\list@num{#1}%
    \delt@=\CUR@width bp\divide\delt@\tw@%
    \fcn@de\PSc@mment{End fcnode}\resetc@ntr@l\et@tpsfcnode\fi\fi}}
\ctr@ld@f\def\d@butn@de{\g@tt@xt\v@lX=\wd\Gb@x%
    \v@lY=\ht\Gb@x\advance\v@lY\dp\Gb@x\advance\v@lX\v@lmin\advance\v@lY\v@lmax}
\ctr@ld@f\def\fcn@deE{%
    \@ecfor\p@int:=\list@num\do{\d@butn@de\v@lX=\unssqrttw@\v@lX\v@lY=\unssqrttw@\v@lY%
    \ifdim\thickn@ss\p@>\z@% Begin thickness
    \v@lXa=\v@lX\advance\v@lXa\delt@\v@lXa=\ptT@unit@\v@lXa\edef\XR@d{\repdecn@mb\v@lXa}%
    \v@lYa=\v@lY\advance\v@lYa\delt@\v@lYa=\ptT@unit@\v@lYa\edef\YR@d{\repdecn@mb\v@lYa}%
    \arct@n\v@leur(\v@lXa,\v@lYa)\v@leur=\rdT@deg\v@leur\edef\@nglde{\repdecn@mb\v@leur}%
    {\c@lptellDD-2::\p@int;\XR@d,\YR@d(\@nglde)}% \v@lmin & \v@lmax modified in \c@lptellDD
    \advance\v@leur-\PI@deg\edef\@nglun{\repdecn@mb\v@leur}%
    {\c@lptellDD-3::\p@int;\XR@d,\YR@d(\@nglun)}%
    \figptstra-6=-3,-2,\p@int/\thickn@ss,-8/\Q@s@tfillmode{yes}%
    \Pss@tspecifSt{color=\DDV@thickcolor}%
    \figdrawline[-2,-3,-6,-5]\figdrawarcell-4;\XR@d,\YR@d(\@nglun,\@nglde,0)%
    \Psrest@reSt{color=\DDV@thickcolor}\fi% End thickness
    \v@lX=\ptT@unit@\v@lX\v@lY=\ptT@unit@\v@lY%
    \edef\XR@d{\repdecn@mb\v@lX}\edef\YR@d{\repdecn@mb\v@lY}%
    \Q@s@tfillmode{yes}\Pss@tspecifSt{color=\fcbgc@lor}%
    \figdrawarcell\p@int;\XR@d,\YR@d(0,360,0)%
    \Q@s@tfillmode{no}\Psrest@reSt{color=\fcbgc@lor}\figdrawarcell\p@int;\XR@d,\YR@d(0,360,0)}}
\ctr@ld@f\def\fcn@deL{\delt@=\ptT@unit@\delt@\edef\t@ille{\repdecn@mb\delt@}%
    \@ecfor\p@int:=\list@num\do{\Figg@tXYa{\p@int}\d@butn@de%
    \ifdim\v@lX>\v@lY\itis@Ktrue\else\itis@Kfalse\fi%
    \advance\v@lXa-\v@lX\Figp@intreg-1:(\v@lXa,\v@lYa)%
    \advance\v@lXa\v@lX\advance\v@lYa-\v@lY\Figp@intreg-2:(\v@lXa,\v@lYa)%
    \advance\v@lXa\v@lX\advance\v@lYa\v@lY\Figp@intreg-3:(\v@lXa,\v@lYa)%
    \advance\v@lXa-\v@lX\advance\v@lYa\v@lY\Figp@intreg-4:(\v@lXa,\v@lYa)%
    \ifdim\thickn@ss\p@>\z@% Begin thickness
    \Figg@tXYa{\p@int}\Q@s@tfillmode{yes}\Pss@tspecifSt{color=\DDV@thickcolor}%
    \c@lpt@xt{-1}{-4}\c@lpt@xt@\v@lXa\v@lYa\v@lX\v@lY\c@rre\delt@%
    \Figp@intregDD-9:(\v@lZ,\v@lYa)\Figp@intregDD-11:(\v@lZa,\v@lYa)%
    \c@lpt@xt{-4}{-3}\c@lpt@xt@\v@lYa\v@lXa\v@lY\v@lX\delt@\c@rre%
    \Figp@intregDD-12:(\v@lXa,\v@lZ)\Figp@intregDD-10:(\v@lXa,\v@lZa)%
    \ifitis@K\figptstra-7=-9,-10,-11/\thickn@ss,-8/\figdrawline[-9,-11,-5,-6,-7]\else%
    \figptstra-7=-10,-11,-12/\thickn@ss,-8/\figdrawline[-10,-12,-5,-6,-7]\fi%
    \Psrest@reSt{color=\DDV@thickcolor}\fi% End thickness
    \Q@s@tfillmode{yes}\Pss@tspecifSt{color=\fcbgc@lor}\figdrawline[-1,-2,-3,-4]%
    \Q@s@tfillmode{no}\Psrest@reSt{color=\fcbgc@lor}\figdrawline[-1,-2,-3,-4,-1]}}
\ctr@ld@f\def\c@lpt@xt#1#2{\figvectN-7[#1,#2]\vecunit@{-7}{-7}\figpttra-5:=#1/\t@ille,-7/%
    \figvectP-7[#1,#2]\Figg@tXY{-7}\c@rre=\v@lX\delt@=\v@lY\Figg@tXY{-5}}
\ctr@ld@f\def\c@lpt@xt@#1#2#3#4#5#6{\v@lZ=#6\invers@{\v@lZ}{\v@lZ}\v@leur=\repdecn@mb{#5}\v@lZ%
    \v@lZ=#2\advance\v@lZ-#4\mili@u=\repdecn@mb{\v@leur}\v@lZ%
    \v@lZ=#3\advance\v@lZ\mili@u\v@lZa=-\v@lZ\advance\v@lZa\tw@#1}
\ctr@ld@f\def\fcn@deR{\@ecfor\p@int:=\list@num\do{\Figg@tXYa{\p@int}\d@butn@de%
    \advance\v@lXa-0.5\v@lX\advance\v@lYa-0.5\v@lY\Figp@intreg-1:(\v@lXa,\v@lYa)%
    \advance\v@lXa\v@lX\Figp@intreg-2:(\v@lXa,\v@lYa)%
    \advance\v@lYa\v@lY\Figp@intreg-3:(\v@lXa,\v@lYa)%
    \advance\v@lXa-\v@lX\Figp@intreg-4:(\v@lXa,\v@lYa)%
    \ifdim\thickn@ss\p@>\z@% Begin thickness
    \Q@s@tfillmode{yes}\Pss@tspecifSt{color=\DDV@thickcolor}%
    \Figv@ctCreg-5(-\delt@,-\delt@)\figpttra-9:=-1/1,-5/%
    \Figv@ctCreg-5(\delt@,-\delt@)\figpttra-10:=-2/1,-5/%
    \Figv@ctCreg-5(\delt@,\delt@)\figpttra-11:=-3/1,-5/%
    \figptstra-7=-9,-10,-11/\thickn@ss,-8/\figdrawline[-9,-11,-5,-6,-7]%
    \Psrest@reSt{color=\DDV@thickcolor}\fi% End thickness
    \Q@s@tfillmode{yes}\Pss@tspecifSt{color=\fcbgc@lor}\figdrawline[-1,-2,-3,-4]%
    \Q@s@tfillmode{no}\Psrest@reSt{color=\fcbgc@lor}\figdrawline[-1,-2,-3,-4,-1]}}
\ctr@ld@f\def\Pssetfl@wchart#1=#2|{\keln@mtr#1|%
    \def\n@mref{arr}\ifx\l@debut\n@mref\expandafter\keln@mtr\l@suite|%
     \def\n@mref{owp}\ifx\l@debut\n@mref\update@ttr\D@FTfcarrowposition\P@setfcarrowposition{#2}\else% arrowposition
     \def\n@mref{owr}\ifx\l@debut\n@mref\update@ttr\D@FTfcarrowrefpt\P@setfcarrowrefpt{#2}\else% arrowrefpt
     \W@rnmesAttr{figset flowchart}{#1}\fi\fi\else%
    \def\n@mref{bgc}\ifx\l@debut\n@mref\update@ttr\D@FTfcbgcolor\P@setfcbgcolor{#2}\else% background color
    \def\n@mref{lin}\ifx\l@debut\n@mref\update@ttr\D@FTfcline\P@setfcline{#2}\else% line
    \def\n@mref{pad}\ifx\l@debut\n@mref\update@ttr\D@FTfcxpadding\P@setfcxpadding{#2}%
                                       \update@ttr\D@FTfcypadding\P@setfcypadding{#2}\else% padding
    \def\n@mref{rad}\ifx\l@debut\n@mref\update@ttr\D@FTfcradius\P@setfcradius{#2}\else% connection radius
    \def\n@mref{sha}\ifx\l@debut\n@mref\update@ttr\D@FTfcshape\P@setfcshape{#2}\else% shape
    \def\n@mref{thi}\ifx\l@debut\n@mref\expandafter\keln@mtr\l@suite|%
     \def\n@mref{ckc}\ifx\l@debut\n@mref\update@ttr\D@FTref\P@setfcthickcolor{#2}\else% thickness color
     \def\n@mref{ckn}\ifx\l@debut\n@mref\update@ttr\D@FTfcthickness\P@setfcthickness{#2}\else% thickness
     \W@rnmesAttr{figset flowchart}{#1}\fi\fi\else%
    \def\n@mref{xpa}\ifx\l@debut\n@mref\update@ttr\D@FTfcxpadding\P@setfcxpadding{#2}\else% xpadding
    \def\n@mref{ypa}\ifx\l@debut\n@mref\update@ttr\D@FTfcypadding\P@setfcypadding{#2}\else% ypadding
    \W@rnmesAttr{figset flowchart}{#1}\fi\fi\fi\fi\fi\fi\fi\fi\fi}
\ctr@ln@m\@rrowp@s
\ctr@ld@f\def\P@setfcarrowposition#1{\edef\@rrowp@s{#1}}
\ctr@ln@m\@rrowr@fpt
\ctr@ld@f\def\P@setfcarrowrefpt#1{\setfcr@fpt#1|}
\ctr@ld@f\def\setfcr@fpt#1#2|{\if#1e\def\@rrowr@fpt{1}\else\def\@rrowr@fpt{0}\fi}
\ctr@ln@m\fcbgc@lor
\ctr@ld@f\def\P@setfcbgcolor#1{\edef\fcbgc@lor{#1}}
\ctr@ln@m\fclin@typ@
\ctr@ld@f\def\P@setfcline#1{\setfccurv@#1|}
\ctr@ld@f\def\setfccurv@#1#2|{\if#1c\def\fclin@typ@{0}\else\def\fclin@typ@{1}\fi}
\ctr@ln@m\fclin@r@d
\ctr@ld@f\def\P@setfcradius#1{\edef\fclin@r@d{#1}}
\ctr@ln@m\fcn@de \ctr@ln@m\fcsh@pe
\ctr@ln@m\h@rdfcXp@dd \ctr@ln@m\h@rdfcYp@dd
\ctr@ld@f\def\P@setfcshape#1{\setfcshap@#1|}
\ctr@ld@f\def\setfcshap@#1#2|{%
    \if#1e\let\fcn@de=\fcn@deE\def\h@rdfcXp@dd{4pt}\def\h@rdfcYp@dd{4pt}%
     \edef\fcsh@pe{ellipse}\else%
    \if#1l\let\fcn@de=\fcn@deL\def\h@rdfcXp@dd{4pt}\def\h@rdfcYp@dd{4pt}%
     \edef\fcsh@pe{lozenge}\else%
          \let\fcn@de=\fcn@deR\def\h@rdfcXp@dd{6pt}\def\h@rdfcYp@dd{6pt}%
     \edef\fcsh@pe{rectangle}\fi\fi}
\ctr@ln@m\DDV@thickcolor
\ctr@ld@f\def\P@setfcthickcolor#1{\edef\DDV@thickcolor{#1}}
\ctr@ln@m\thickn@ss
\ctr@ld@f\def\P@setfcthickness#1{\edef\thickn@ss{#1}}
\ctr@ln@m\Xp@dd
\ctr@ld@f\def\P@setfcxpadding#1{\edef\Xp@dd{#1}}
\ctr@ln@m\Yp@dd
\ctr@ld@f\def\P@setfcypadding#1{\edef\Yp@dd{#1}}
\ctr@ld@f\def\figdrawline[#1]{{\ifCUR@PS\ifGR@cri\PSc@mment{line Points=#1}%
    \let\figdrawlign@=\Pslign@P\Pslin@{#1}\PSc@mment{End line}\fi\fi}}
\ctr@ld@f\def\figdrawlineF#1{{\ifCUR@PS\ifGR@cri\PSc@mment{lineF Filename=#1}%
    \let\figdrawlign@=\Pslign@F\Pslin@{#1}\PSc@mment{End lineF}\fi\fi}}
\ctr@ld@f\def\figdrawlineC(#1){{\ifCUR@PS\ifGR@cri\PSc@mment{lineC}%
    \let\figdrawlign@=\Pslign@C\Pslin@{#1}\PSc@mment{End lineC}\fi\fi}}
\ctr@ld@f\def\Pslin@#1{\iffillm@de\figdrawlign@{#1}%
    \f@gfill%
    \else\figdrawlign@{#1}\ifx\derp@int\premp@int%
    \f@gclosestroke%
    \else\f@gstroke\fi\fi}
\ctr@ld@f\def\Pslign@P#1{\def\list@num{#1}\extrairelepremi@r\p@int\de\list@num%
    \edef\premp@int{\p@int}\f@gnewpath%
    \PSwrit@cmd{\p@int}{\c@mmoveto}{\fwf@g}%
    \@ecfor\p@int:=\list@num\do{\PSwrit@cmd{\p@int}{\c@mlineto}{\fwf@g}%
    \edef\derp@int{\p@int}}}
\ctr@ld@f\def\Pslign@F#1{\s@uvc@ntr@l\et@tPslign@F\setc@ntr@l{2}\openin\frf@g=#1\relax%
    \ifeof\frf@g\message{*** File #1 not found !}\end\else%
    \read\frf@g to\tr@c\edef\premp@int{\tr@c}\expandafter\extr@ctCF\tr@c:%
    \f@gnewpath\PSwrit@cmd{-1}{\c@mmoveto}{\fwf@g}%
    \loop\read\frf@g to\tr@c\ifeof\frf@g\mored@tafalse\else\mored@tatrue\fi%
    \ifmored@ta\expandafter\extr@ctCF\tr@c:\PSwrit@cmd{-1}{\c@mlineto}{\fwf@g}%
    \edef\derp@int{\tr@c}\repeat\fi\closein\frf@g\resetc@ntr@l\et@tPslign@F}
\ctr@ln@m\extr@ctCF
\ctr@ld@f\def\extr@ctCFDD#1 #2:{\v@lX=#1\unit@\v@lY=#2\unit@\Figp@intregDD-1:(\v@lX,\v@lY)}
\ctr@ld@f\def\extr@ctCFTD#1 #2 #3:{\v@lX=#1\unit@\v@lY=#2\unit@\v@lZ=#3\unit@%
    \Figp@intregTD-1:(\v@lX,\v@lY,\v@lZ)}
\ctr@ld@f\def\Pslign@C#1{\s@uvc@ntr@l\et@tPslign@C\setc@ntr@l{2}%
    \def\list@num{#1}\extrairelepremi@r\p@int\de\list@num%
    \edef\premp@int{\p@int}\f@gnewpath%
    \expandafter\Pslign@C@\p@int:\PSwrit@cmd{-1}{\c@mmoveto}{\fwf@g}%
    \@ecfor\p@int:=\list@num\do{\expandafter\Pslign@C@\p@int:%
    \PSwrit@cmd{-1}{\c@mlineto}{\fwf@g}\edef\derp@int{\p@int}}%
    \resetc@ntr@l\et@tPslign@C}
\ctr@ld@f\def\Pslign@C@#1 #2:{{\def\t@xt@{#1}\ifx\t@xt@\empty\Pslign@C@#2:% Discard leading spaces
    \else\extr@ctCF#1 #2:\fi}}
\ctr@ld@f\def\Pssetm@sh#1=#2|{\keln@mde#1|%
    \def\n@mref{co}\ifx\l@debut\n@mref\update@ttr\D@FTref\P@setmeshcolor{#2}\else% color
    \def\n@mref{da}\ifx\l@debut\n@mref\update@ttr\D@FTref\P@setmeshdash{#2}\else% dash
    \def\n@mref{di}\ifx\l@debut\n@mref\update@ttr\D@FTmeshdiag\Q@s@tmeshdiag{#2}\else% diag
    \def\n@mref{wi}\ifx\l@debut\n@mref\update@ttr\D@FTref\P@setmeshwidth{#2}\else% width
    \W@rnmesAttr{figset mesh}{#1}\fi\fi\fi\fi}
\ctr@ln@m\c@ntrolmesh
\ctr@ld@f\def\Q@s@tmeshdiag#1{\edef\c@ntrolmesh{#1}}
\ctr@ld@f\def\D@FTmeshdiag{0}    % Valeur par defaut
\ctr@ln@m\DDV@meshcolor
\ctr@ld@f\def\P@setmeshcolor#1{\edef\DDV@meshcolor{#1}}
\ctr@ln@m\DDV@meshdash
\ctr@ld@f\def\P@setmeshdash#1{\edef\DDV@meshdash{#1}}
\ctr@ln@m\DDV@meshwidth
\ctr@ld@f\def\P@setmeshwidth#1{\edef\DDV@meshwidth{#1}}
\ctr@ld@f\def\figdrawmesh#1,#2[#3,#4,#5,#6]{{\ifCUR@PS\ifGR@cri%
    \PSc@mment{mesh N1=#1, N2=#2, Quadrangle=[#3,#4,#5,#6]}\s@uvc@ntr@l\et@tpsmesh%
    \Pss@tspecifSt{color=\DDV@meshcolor,dash=\DDV@meshdash,width=\DDV@meshwidth}%
    \setc@ntr@l{2}%
    \ifnum#1>\@ne\Psmeshp@rt#1[#3,#4,#5,#6]\fi%
    \ifnum#2>\@ne\Psmeshp@rt#2[#4,#5,#6,#3]\fi%
    \ifnum\c@ntrolmesh>\z@\Psmeshdi@g#1,#2[#3,#4,#5,#6]\fi%
    \ifnum\c@ntrolmesh<\z@\Psmeshdi@g#2,#1[#4,#5,#6,#3]\fi%
    \Psrest@reSt{color=\DDV@meshcolor,dash=\DDV@meshdash,width=\DDV@meshwidth}%
    \figdrawline[#3,#4,#5,#6,#3]\PSc@mment{End mesh}\resetc@ntr@l\et@tpsmesh\fi\fi}}
\ctr@ld@f\def\Psmeshp@rt#1[#2,#3,#4,#5]{{\l@mbd@un=\@ne\l@mbd@de=#1\loop%
    \ifnum\l@mbd@un<#1\advance\l@mbd@de\m@ne\figptbary-1:[#2,#3;\l@mbd@de,\l@mbd@un]%
    \figptbary-2:[#5,#4;\l@mbd@de,\l@mbd@un]\figdrawline[-1,-2]\advance\l@mbd@un\@ne\repeat}}
\ctr@ld@f\def\Psmeshdi@g#1,#2[#3,#4,#5,#6]{\figptcopy-2:/#3/\figptcopy-3:/#6/%
    \l@mbd@un=\z@\l@mbd@de=#1\loop\ifnum\l@mbd@un<#1%
    \advance\l@mbd@un\@ne\advance\l@mbd@de\m@ne\figptcopy-1:/-2/\figptcopy-4:/-3/%
    \figptbary-2:[#3,#4;\l@mbd@de,\l@mbd@un]%
    \figptbary-3:[#6,#5;\l@mbd@de,\l@mbd@un]\Psmeshdi@gp@rt#2[-1,-2,-3,-4]\repeat}
\ctr@ld@f\def\Psmeshdi@gp@rt#1[#2,#3,#4,#5]{{\l@mbd@un=\z@\l@mbd@de=#1\loop%
    \ifnum\l@mbd@un<#1\figptbary-5:[#2,#5;\l@mbd@de,\l@mbd@un]%
    \advance\l@mbd@de\m@ne\advance\l@mbd@un\@ne%
    \figptbary-6:[#3,#4;\l@mbd@de,\l@mbd@un]\figdrawline[-5,-6]\repeat}}
\ctr@ln@m\figdrawnormal
\ctr@ld@f\def\Q@normalDD#1,#2[#3,#4]{{\ifCUR@PS\ifGR@cri%
    \PSc@mment{normal Length=#1, Lambda=#2 [Pt1,Pt2]=[#3,#4]}%
    \s@uvc@ntr@l\et@tpsnormal\resetc@ntr@l{2}\figptendnormal-6::#1,#2[#3,#4]%
    \figptcopyDD-5:/-1/\figdrawarrow[-5,-6]%
    \PSc@mment{End normal}\resetc@ntr@l\et@tpsnormal\fi\fi}}
\ctr@ld@f\def\figreset#1{\trtlis@rg{#1}{\Psreset@}}
\ctr@ld@f\def\Psreset@#1|{\def\t@xt@{#1}\ifx\t@xt@\empty\P@resetg@n% general settings
    \else\keln@mde#1|%
    \def\n@mref{al}\ifx\l@debut\n@mref%
        \figset altitude(blcolor=default,bldash=default,blwidth=default,%
        sqcolor=default,sqdash=default,sqwidth=default)\else% altitude
    \def\n@mref{ar}\ifx\l@debut\n@mref\figresetarrowhead\else% arrowhead
    \def\n@mref{cu}\ifx\l@debut\n@mref\figset curve(roundness=\D@FTroundness)\else% curve
    \def\n@mref{ge}\ifx\l@debut\n@mref\P@resetg@n\else% general settings
    \def\n@mref{fl}\ifx\l@debut\n@mref%
        \figset flowchart(arrowp=\D@FTfcarrowposition,arrowr=\D@FTfcarrowrefpt,%
	bgcolor=\D@FTfcbgcolor,line=\D@FTfcline,radius=\D@FTfcradius,%
	shape=\D@FTfcshape,thickcolor=default,thickness=\D@FTfcthickness,%
	xpadd=\D@FTfcxpadding,ypadd=\D@FTfcypadding)\else% flow chart
    \def\n@mref{me}\ifx\l@debut\n@mref\figset mesh(diag=\D@FTmeshdiag,%
        color=default,dash=default,width=default)\else% mesh
    \def\n@mref{tr}\ifx\l@debut\n@mref%
        \figset trimesh(color=default,dash=default,width=default)\else% trimesh
    \W@rnmeskwd{figreset}{#1}\fi\fi\fi\fi\fi\fi\fi\fi}
\ctr@ld@f\def\P@resetg@n{\figset (color=\D@FTcolor,dash=\D@FTdash,fill=\D@FTfill,%
    join=\D@FTjoin,width=\D@FTwidth)}
\ctr@ld@f\def\figset#1(#2){\def\t@xt@{#1}\ifx\t@xt@\empty\trtlis@rg{#2}{\Pssetg@n}% general settings
    \else\keln@mde#1|%
    \def\n@mref{al}\ifx\l@debut\n@mref\trtlis@rg{#2}{\Psset@lti}\else% altitude
    \def\n@mref{ar}\ifx\l@debut\n@mref\trtlis@rg{#2}{\Psset@rrowhe@d}\else% arrowhead
    \def\n@mref{cu}\ifx\l@debut\n@mref\trtlis@rg{#2}{\Pssetc@rve}\else% curve
    \def\n@mref{fl}\ifx\l@debut\n@mref\trtlis@rg{#2}{\Pssetfl@wchart}\else% flow chart
    \def\n@mref{ge}\ifx\l@debut\n@mref\trtlis@rg{#2}{\Pssetg@n}\else% general settings
    \def\n@mref{me}\ifx\l@debut\n@mref\trtlis@rg{#2}{\Pssetm@sh}\else% mesh
    \def\n@mref{pr}\ifx\l@debut\n@mref\ifCUR@PS\W@rnmesIgn{figset proj(...)}%
     \else\trtlis@rg{#2}{\Figsetpr@j}\fi\else% projection
    \def\n@mref{tr}\ifx\l@debut\n@mref\trtlis@rg{#2}{\Pssettrim@sh}\else% trimesh
    \def\n@mref{wr}\ifx\l@debut\n@mref\let\M@cro=\Figsetwr@te\trtlis@rgtok{#2,|}\else% write
    \W@rnmeskwd{figset}{#1}\fi\fi\fi\fi\fi\fi\fi\fi\fi\fi\ignorespaces}
\ctr@ld@f\def\figsetdefault#1(#2){\ifCUR@PS\W@rnmesIgn{figsetdefault}\else%
    \def\t@xt@{#1}\ifx\t@xt@\empty\trtlis@rg{#2}{\Pssd@g@n}\else\keln@mun#1|% general settings
    \def\n@mref{a}\ifx\l@debut\n@mref\trtlis@rg{#2}{\Pssd@@rrowhe@d}\else% arrowhead
    \def\n@mref{c}\ifx\l@debut\n@mref\trtlis@rg{#2}{\Pssd@c@rve}\else% curve
    \def\n@mref{g}\ifx\l@debut\n@mref\trtlis@rg{#2}{\Pssd@g@n}\else% general settings
    \def\n@mref{f}\ifx\l@debut\n@mref\trtlis@rg{#2}{\Pssd@fl@wchart}\else% flow chart
    \def\n@mref{m}\ifx\l@debut\n@mref\trtlis@rg{#2}{\Pssd@m@sh}\else% mesh
    \W@rnmeskwd{figsetdefault}{#1}\fi\fi\fi\fi\fi\fi\initpss@ttings\fi}
\ctr@ld@f\def\Pssd@g@n#1=#2|{\keln@mun#1|%
    \def\n@mref{c}\ifx\l@debut\n@mref\edef\D@FTcolor{#2}\else% color
    \def\n@mref{d}\ifx\l@debut\n@mref\edef\D@FTdash{#2}\else% line dash
    \def\n@mref{f}\ifx\l@debut\n@mref\edef\D@FTfill{#2}\else% fillmode
    \def\n@mref{j}\ifx\l@debut\n@mref\edef\D@FTjoin{#2}\else% line join
    \def\n@mref{u}\ifx\l@debut\n@mref\edef\D@FTupdate{#2}\Q@s@tupdate{#2}\else% update
    \def\n@mref{w}\ifx\l@debut\n@mref\edef\D@FTwidth{#2}\else% line width
    \W@rnmesAttr{figsetdefault}{#1}\fi\fi\fi\fi\fi\fi}
\ctr@ld@f\def\Pssd@@rrowhe@d#1=#2|{\keln@mun#1|%
    \def\n@mref{a}\ifx\l@debut\n@mref\edef\D@FTarrowheadangle{#2}\else% angle
    \def\n@mref{f}\ifx\l@debut\n@mref\edef\D@FTarrowheadfill{#2}\else% fillmode
    \def\n@mref{l}\ifx\l@debut\n@mref\y@tiunit{#2}\ifunitpr@sent%
     \edef\D@FTh@rdahlength{#2}\else\edef\D@FTh@rdahlength{#2pt}%
     \message{*** \BS@ figsetdefault (..., #1=#2, ...) : unit is missing, pt is assumed.}%
     \fi\else% length
    \def\n@mref{o}\ifx\l@debut\n@mref\edef\D@FTarrowheadout{#2}\else% out
    \def\n@mref{r}\ifx\l@debut\n@mref\edef\D@FTarrowheadratio{#2}\else% ratio
    \W@rnmesAttr{figsetdefault arrowhead}{#1}\fi\fi\fi\fi\fi}
\ctr@ld@f\def\Pssd@c@rve#1=#2|{\keln@mun#1|%
    \def\n@mref{r}\ifx\l@debut\n@mref\edef\D@FTroundness{#2}\else%
    \W@rnmesAttr{figsetdefault curve}{#1}\fi}
\ctr@ld@f\def\Pssd@fl@wchart#1=#2|{\keln@mtr#1|%
    \def\n@mref{arr}\ifx\l@debut\n@mref\expandafter\keln@mtr\l@suite|%
     \def\n@mref{owp}\ifx\l@debut\n@mref\edef\D@FTfcarrowposition{#2}\else% arrowposition
     \def\n@mref{owr}\ifx\l@debut\n@mref\edef\D@FTfcarrowrefpt{#2}\else% arrowrefpt
                     \W@rnmesAttr{figsetdefault flowchart}{#1}\fi\fi\else%
    \def\n@mref{bgc}\ifx\l@debut\n@mref\edef\D@FTfcbgcolor{#2}\else% background color
    \def\n@mref{lin}\ifx\l@debut\n@mref\edef\D@FTfcline{#2}\else% line
    \def\n@mref{pad}\ifx\l@debut\n@mref\edef\D@FTfcxpadding{#2}%
                    \edef\D@FTfcypadding{#2}\else% padding
    \def\n@mref{rad}\ifx\l@debut\n@mref\edef\D@FTfcradius{#2}\else% connection radius
    \def\n@mref{sha}\ifx\l@debut\n@mref\edef\D@FTfcshape{#2}\else% shape
    \def\n@mref{thi}\ifx\l@debut\n@mref\expandafter\keln@mtr\l@suite|%
     \def\n@mref{ckn}\ifx\l@debut\n@mref\edef\D@FTfcthickness{#2}\else% thickness
                     \W@rnmesAttr{figsetdefault flowchart}{#1}\fi\else%
    \def\n@mref{xpa}\ifx\l@debut\n@mref\edef\D@FTfcxpadding{#2}\else% xpadding
    \def\n@mref{ypa}\ifx\l@debut\n@mref\edef\D@FTfcypadding{#2}\else% ypadding
    \W@rnmesAttr{figsetdefault flowchart}{#1}\fi\fi\fi\fi\fi\fi\fi\fi\fi}
\ctr@ld@f\def\D@FTfcarrowposition{0.5}
\ctr@ld@f\def\D@FTfcarrowrefpt{start}
\ctr@ld@f\def\D@FTfcbgcolor{1}
\ctr@ld@f\def\D@FTfcline{polygon}
\ctr@ld@f\def\D@FTfcradius{0}
\ctr@ld@f\def\D@FTfcshape{rectangle}
\ctr@ld@f\def\D@FTfcthickness{0}
\ctr@ld@f\def\D@FTfcxpadding{0}
\ctr@ld@f\def\D@FTfcypadding{0}
\ctr@ld@f\def\Pssd@m@sh#1=#2|{\keln@mun#1|%
    \def\n@mref{d}\ifx\l@debut\n@mref\edef\D@FTmeshdiag{#2}\else%
    \W@rnmesAttr{figsetdefault mesh}{#1}\fi}
\ctr@ln@w{newif}\iffillm@de
\ctr@ld@f\def\Q@s@tfillmode#1{\expandafter\setfillm@de#1:}
\ctr@ld@f\def\setfillm@de#1#2:{\if#1n\fillm@defalse\else\fillm@detrue\fi}
\ctr@ld@f\def\D@FTfill{no}     % Valeur par defaut
\ctr@ln@w{newif}\ifGRupdatem@de
\ctr@ld@f\def\Q@s@tupdate#1{\ifCUR@PS\W@rnmesIgn{figset (update=...)}%
    \else\expandafter\setupd@te#1:\fi}
\ctr@ld@f\def\setupd@te#1#2:{\if#1n\GRupdatem@defalse\else\GRupdatem@detrue\fi}
\ctr@ld@f\def\D@FTupdate{no}     % Valeur par defaut
\ctr@ln@m\CUR@color \ctr@ln@m\CUR@colorc@md
\ctr@ld@f\def\s@uvcolor#1{\edef#1{\CUR@color}}
\ctr@ld@f\def\D@FTcolor{0}       % Valeur par defaut
\ctr@ld@f\def\Pssetc@lor#1{\ifGR@cri\result@tent=\@ne\expandafter\c@lnbV@l#1 :%
    \def\CUR@color{}\def\CUR@colorc@md{}%
    \ifcase\result@tent\or\Q@s@tgray{#1}\or\or\Q@s@trgb{#1}\or\Q@s@tcmyk{#1}\fi\fi}
\ctr@ln@m\CUR@colorc@mdStroke
\ctr@ld@f\def\Q@s@tcmyk#1{\ifGR@cri\def\CUR@color{#1}\def\CUR@colorc@md{\c@msetcmykcolor}%
    \def\CUR@colorc@mdStroke{\c@msetcmykcolorStroke}%
    \ifCUR@PS\PSc@mment{setcmyk Color=#1}\us@primarC@lor\fi\fi}
\ctr@ld@f\def\Q@s@trgb#1{\ifGR@cri\def\CUR@color{#1}\def\CUR@colorc@md{\c@msetrgbcolor}%
    \def\CUR@colorc@mdStroke{\c@msetrgbcolorStroke}%
    \ifCUR@PS\PSc@mment{setrgb Color=#1}\us@primarC@lor\fi\fi}
\ctr@ld@f\def\Q@s@tgray#1{\ifGR@cri\def\CUR@color{#1}\def\CUR@colorc@md{\c@msetgray}%
    \def\CUR@colorc@mdStroke{\c@msetgrayStroke}%
    \ifCUR@PS\PSc@mment{setgray Gray level=#1}\us@primarC@lor\fi\fi}
\ctr@ln@m\fillc@md
\ctr@ld@f\def\us@primarC@lor{\immediate\write\fwf@g{\d@fprimarC@lor}%
    \let\fillc@md=\prfillc@md}
\ctr@ld@f\def\prfillc@md{\d@fprimarC@lor\space\c@mfill}
\ctr@ld@f\def\c@lnbV@l#1 #2:{\def\t@xt@{#1}\relax\ifx\t@xt@\empty\c@lnbV@l#2:% Discard leading spaces
    \else\c@lnbV@l@#1 #2:\fi}
\ctr@ld@f\def\c@lnbV@l@#1 #2:{\def\t@xt@{#2}\ifx\t@xt@\empty%
    \def\t@xt@{#1}\ifx\t@xt@\empty\advance\result@tent\m@ne\fi% Discard trailing spaces
    \else\advance\result@tent\@ne\c@lnbV@l@#2:\fi}
\ctr@ld@f\def\Blackcmyk{0 0 0 1}
\ctr@ld@f\def\Whitecmyk{0 0 0 0}
\ctr@ld@f\def\Cyancmyk{1 0 0 0}
\ctr@ld@f\def\Magentacmyk{0 1 0 0}
\ctr@ld@f\def\Yellowcmyk{0 0 1 0}
\ctr@ld@f\def\Redcmyk{0 1 1 0}
\ctr@ld@f\def\Greencmyk{1 0 1 0}
\ctr@ld@f\def\Bluecmyk{1 1 0 0}
\ctr@ld@f\def\Graycmyk{0 0 0 0.50}
\ctr@ld@f\def\BrickRedcmyk{0 0.89 0.94 0.28} % PANTONE 1805
\ctr@ld@f\def\Browncmyk{0 0.81 1 0.60} % PANTONE 1615
\ctr@ld@f\def\ForestGreencmyk{0.91 0 0.88 0.12} % PANTONE 349
\ctr@ld@f\def\Goldenrodcmyk{ 0 0.10 0.84 0} % PANTONE 109
\ctr@ld@f\def\Marooncmyk{0 0.87 0.68 0.32} % PANTONE 201
\ctr@ld@f\def\Orangecmyk{0 0.61 0.87 0} % PANTONE ORANGE-021
\ctr@ld@f\def\Purplecmyk{0.45 0.86 0 0} % PANTONE PURPLE
\ctr@ld@f\def\RoyalBluecmyk{1. 0.50 0 0} % No PANTONE match
\ctr@ld@f\def\Violetcmyk{0.79 0.88 0 0} % PANTONE VIOLET
\ctr@ld@f\def\Blackrgb{0 0 0}
\ctr@ld@f\def\Whitergb{1 1 1}
\ctr@ld@f\def\Redrgb{1 0 0}
\ctr@ld@f\def\Greenrgb{0 1 0}
\ctr@ld@f\def\Bluergb{0 0 1}
\ctr@ld@f\def\Cyanrgb{0 1 1}
\ctr@ld@f\def\Magentargb{1 0 1}
\ctr@ld@f\def\Yellowrgb{1 1 0}
\ctr@ld@f\def\Grayrgb{0.5 0.5 0.5}
\ctr@ld@f\def\Chocolatergb{0.824 0.412 0.118}
\ctr@ld@f\def\DarkGoldenrodrgb{0.722 0.525 0.043}
\ctr@ld@f\def\DarkOrangergb{1 0.549 0}
\ctr@ld@f\def\Firebrickrgb{0.698 0.133 0.133}
\ctr@ld@f\def\ForestGreenrgb{0.133 0.545 0.133}
\ctr@ld@f\def\Goldrgb{1 0.843 0}
\ctr@ld@f\def\HotPinkrgb{1 0.412 0.706}
\ctr@ld@f\def\Maroonrgb{0.690 0.188 0.376}
\ctr@ld@f\def\Pinkrgb{1 0.753 0.796}
\ctr@ld@f\def\RoyalBluergb{0.255 0.412 0.882}
\ctr@ld@f\def\Pssetg@n#1=#2|{\keln@mun#1|%
    \def\n@mref{c}\ifx\l@debut\n@mref\update@ttr\D@FTcolor\Pssetc@lor{#2}\else% color
    \def\n@mref{d}\ifx\l@debut\n@mref\update@ttr\D@FTdash\Q@s@tdash{#2}\else% line dash
    \def\n@mref{f}\ifx\l@debut\n@mref\update@ttr\D@FTfill\Q@s@tfillmode{#2}\else% fillmode
    \def\n@mref{j}\ifx\l@debut\n@mref\update@ttr\D@FTjoin\Q@s@tjoin{#2}\else% line join
    \def\n@mref{u}\ifx\l@debut\n@mref\update@ttr\D@FTupdate\Q@s@tupdate{#2}\else% update
    \def\n@mref{w}\ifx\l@debut\n@mref\update@ttr\D@FTwidth\Q@s@twidth{#2}\else% line width
    \W@rnmesAttr{figset}{#1}\fi\fi\fi\fi\fi\fi}
\ctr@ln@m\CUR@dash
\ctr@ld@f\def\s@uvdash#1{\edef#1{\CUR@dash}}
\ctr@ld@f\def\D@FTdash{1}        % Valeur par defaut (numero sans espace)
\ctr@ld@f\def\Q@s@tdash#1{\ifGR@cri\edef\CUR@dash{#1}\ifCUR@PS\expandafter\Pssetd@sh#1 :\fi\fi}
\ctr@ld@f\def\Pssetd@shI#1{\PSc@mment{setdash Index=#1}\ifcase#1%
    \or\immediate\write\fwf@g{[] 0 \c@msetdash}%         Index=1
    \or\immediate\write\fwf@g{[6 2] 0 \c@msetdash}%      Index=2
    \or\immediate\write\fwf@g{[4 2] 0 \c@msetdash}%      Index=3
    \or\immediate\write\fwf@g{[2 2] 0 \c@msetdash}%      Index=4
    \or\immediate\write\fwf@g{[1 2] 0 \c@msetdash}%      Index=5
    \or\immediate\write\fwf@g{[2 4] 0 \c@msetdash}%      Index=6
    \or\immediate\write\fwf@g{[3 5] 0 \c@msetdash}%      Index=7
    \or\immediate\write\fwf@g{[3 3] 0 \c@msetdash}%      Index=8
    \or\immediate\write\fwf@g{[3 5 1 5] 0 \c@msetdash}%  Index=9
    \or\immediate\write\fwf@g{[6 4 2 4] 0 \c@msetdash}%  Index=10
    \fi}
\ctr@ld@f\def\Pssetd@sh#1 #2:{{\def\t@xt@{#1}\ifx\t@xt@\empty\Pssetd@sh#2:% Discard leading spaces
    \else\def\t@xt@{#2}\ifx\t@xt@\empty\Pssetd@shI{#1}\else\s@mme=\@ne\def\debutp@t{#1}%
    \an@lysd@sh#2:\ifodd\s@mme\edef\debutp@t{\debutp@t\space\finp@t}\def\finp@t{0}\fi%
    \PSc@mment{setdash Pattern=#1 #2}%
    \immediate\write\fwf@g{[\debutp@t] \finp@t\space\c@msetdash}\fi\fi}}
\ctr@ld@f\def\an@lysd@sh#1 #2:{\def\t@xt@{#2}\ifx\t@xt@\empty\def\finp@t{#1}\else%
    \edef\debutp@t{\debutp@t\space#1}\advance\s@mme\@ne\an@lysd@sh#2:\fi}
\ctr@ln@m\CUR@width
\ctr@ld@f\def\s@uvwidth#1{\edef#1{\CUR@width}}
\ctr@ld@f\def\D@FTwidth{0.4}     % Valeur par defaut
\ctr@ld@f\def\Q@s@twidth#1{\ifGR@cri\edef\CUR@width{#1}\ifCUR@PS%
    \PSc@mment{setwidth Width=#1}\immediate\write\fwf@g{#1 \c@msetlinewidth}\fi\fi}
\ctr@ln@m\CUR@join
\ctr@ld@f\def\s@uvjoin#1{\edef#1{\CUR@join}}
\ctr@ld@f\def\D@FTjoin{miter}   % Valeur par defaut
\ctr@ld@f\def\Q@s@tjoin#1{\ifGR@cri\edef\CUR@join{#1}\ifCUR@PS\expandafter\Pssetj@in#1:\fi\fi}
\ctr@ld@f\def\Pssetj@in#1#2:{\PSc@mment{setjoin join=#1}%
    \if#1r\def\t@xt@{1}\else\if#1b\def\t@xt@{2}\else\def\t@xt@{0}\fi\fi%
    \immediate\write\fwf@g{\t@xt@\space\c@msetlinejoin}}
\ctr@ld@f\def\Pss@tspecifSt#1{\trtlis@rg{#1}{\Pss@tspecifSt@}}
\ctr@ld@f\def\Pss@tspecifSt@#1=#2|{\keln@mun#1|%
    \def\n@mref{c}\ifx\l@debut\n@mref\def\n@mref{#2}\ifx\n@mref\D@FTref\else%
     \s@uvcolor{\typ@color}\Pssetc@lor{#2}\fi\else% color
    \def\n@mref{d}\ifx\l@debut\n@mref\def\n@mref{#2}\ifx\n@mref\D@FTref\else%
     \s@uvdash{\typ@dash}\Q@s@tdash{#2}\fi\else% line dash
    \def\n@mref{j}\ifx\l@debut\n@mref\def\n@mref{#2}\ifx\n@mref\D@FTref\else%
     \s@uvjoin{\typ@join}\Q@s@tjoin{#2}\fi\else% line join
    \def\n@mref{w}\ifx\l@debut\n@mref\def\n@mref{#2}\ifx\n@mref\D@FTref\else%
     \s@uvwidth{\typ@width}\Q@s@twidth{#2}\fi\else% line width
    \W@rnmeskwd{Pss@tspecifSt}{#1}\fi\fi\fi\fi}
\ctr@ld@f\def\Psrest@reSt#1{\trtlis@rg{#1}{\Psrest@reSt@}}
\ctr@ld@f\def\Psrest@reSt@#1=#2|{\keln@mun#1|%
    \def\n@mref{c}\ifx\l@debut\n@mref\def\n@mref{#2}\ifx\n@mref\D@FTref\else%
     \Pssetc@lor{\typ@color}\fi\else% color
    \def\n@mref{d}\ifx\l@debut\n@mref\def\n@mref{#2}\ifx\n@mref\D@FTref\else%
     \Q@s@tdash{\typ@dash}\fi\else% line dash
    \def\n@mref{j}\ifx\l@debut\n@mref\def\n@mref{#2}\ifx\n@mref\D@FTref\else%
     \Q@s@tjoin{\typ@join}\fi\else% line join
    \def\n@mref{w}\ifx\l@debut\n@mref\def\n@mref{#2}\ifx\n@mref\D@FTref\else%
     \Q@s@twidth{\typ@width}\fi\else% line width
    \W@rnmeskwd{Psrest@reSt}{#1}\fi\fi\fi\fi}
\ctr@ld@f\def\Pssettrim@sh#1=#2|{\keln@mde#1|%
    \def\n@mref{co}\ifx\l@debut\n@mref\update@ttr\D@FTref\P@settmeshcolor{#2}\else% color
    \def\n@mref{da}\ifx\l@debut\n@mref\update@ttr\D@FTref\P@settmeshdash{#2}\else% dash
    \def\n@mref{wi}\ifx\l@debut\n@mref\update@ttr\D@FTref\P@settmeshwidth{#2}\else% width
    \W@rnmesAttr{figset trimesh}{#1}\fi\fi\fi}
\ctr@ln@m\DDV@tmeshcolor
\ctr@ld@f\def\P@settmeshcolor#1{\edef\DDV@tmeshcolor{#1}}
\ctr@ln@m\DDV@tmeshdash
\ctr@ld@f\def\P@settmeshdash#1{\edef\DDV@tmeshdash{#1}}
\ctr@ln@m\DDV@tmeshwidth
\ctr@ld@f\def\P@settmeshwidth#1{\edef\DDV@tmeshwidth{#1}}
\ctr@ld@f\def\figdrawtrimesh#1[#2,#3,#4]{{\ifCUR@PS\ifGR@cri%
    \PSc@mment{trimesh Type=#1, Triangle=[#2,#3,#4]}%
    \s@uvc@ntr@l\et@tpstrimesh\ifnum#1>\@ne%
    \Pss@tspecifSt{color=\DDV@tmeshcolor,dash=\DDV@tmeshdash,width=\DDV@tmeshwidth}%
    \setc@ntr@l{2}%
    \Pstrimeshp@rt#1[#2,#3,#4]\Pstrimeshp@rt#1[#3,#4,#2]\Pstrimeshp@rt#1[#4,#2,#3]%
    \Psrest@reSt{color=\DDV@tmeshcolor,dash=\DDV@tmeshdash,width=\DDV@tmeshwidth}%
    \fi\figdrawline[#2,#3,#4,#2]%
    \PSc@mment{End trimesh}\resetc@ntr@l\et@tpstrimesh\fi\fi}}
\ctr@ld@f\def\Pstrimeshp@rt#1[#2,#3,#4]{{\l@mbd@un=\@ne\l@mbd@de=#1\loop\ifnum\l@mbd@de>\@ne%
    \advance\l@mbd@de\m@ne\figptbary-1:[#2,#3;\l@mbd@de,\l@mbd@un]%
    \figptbary-2:[#2,#4;\l@mbd@de,\l@mbd@un]\figdrawline[-1,-2]%
    \advance\l@mbd@un\@ne\repeat}}
\initpr@lim\initpss@ttings\initPDF@rDVI% Initialisation preliminaire
\ctr@ln@w{newbox}\figBoxA
\ctr@ln@w{newbox}\figBoxB
\ctr@ln@w{newbox}\figBoxC
\catcode`\@=12

%
\allowdisplaybreaks
\DisableLigatures{encoding = *, family = * }
%
\numberwithin{equation}{section}
%\pagestyle{myheadings}
%
\newtheorem{theorem}{Theorem}[section]
\newtheorem{proposition}{Proposition}[section]
\newtheorem{lemma}{Lemma}[section]
\newtheorem{definition}{Definition}[section]
\newtheorem{remark}{Remark}[section]
%
\numberwithin{equation}{section}
\numberwithin{theorem}{section}
\numberwithin{remark}{section}
\numberwithin{lemma}{section}
\numberwithin{proposition}{section}
\numberwithin{definition}{section}
%
\newcommand{\norm}[2]{{\left\|#1\right\|}_{#2}}
\newcommand{\fl}[2]{(-d_x^{\,2})^{#1}#2}
\newcommand{\rfl}[2]{A^{#1}_{\Omega}#2}
\newcommand{\hp}[1]{\hphantom{#1}}
\newcommand{\cns}{c_{N,s}}
\newcommand{\ccs}{c_{1,s}}
\newcommand{\ffl}[2]{(-d_x^{\,2})^{#1}#2}
\newcommand{\flh}[2]{\frac{1}{\Gamma(-s)}\int_0^{+\infty}\Big(e^{t\Delta}#2 - #2\Big)\frac{dt}{t^{1+#1}}}
\newcommand{\kernel}[1]{|x-y|^{#1}}
\newcommand{\dkj}{\delta_{kj}}
\newcommand{\intr}[1]{\underset{#1}{\int}}
\newcommand{\Do}[1]{D_{#1}}
\newcommand{\Hs}{H^s_0(\Omega)}
\newcommand{\ue}[1]{#1^{\,\varepsilon}}
\newcommand{\xHdot}[1]{\dot{H}^{#1}}
\newcommand{\ha}[2]{\mathbf{H}_{#1}^{#2}}
\newcommand{\lhi}{\mathcal{L}_i^h}
\newcommand{\NN}{\mathbb{N}}
\newcommand{\ZZ}{\mathbb{Z}}
\newcommand{\RR}{\mathbb{R}}
\newcommand{\CC}{\mathbb{C}}
\newcommand{\TT}{\mathbf{T}}
\newcommand\inter[1]{\llbracket #1\rrbracket}
\newcommand\mesh{\mathfrak{M}}
\newcommand{\rouge}[1]{\color{red}#1\color{black}}
\newcommand{\lop}[2]{\mathcal{L}_T(#1,#2)}
\newcommand{\loh}[2]{\mathcal{L}^h_T\left(#1,#2\right)}

%
\usepackage{tikz}
\usepackage{pgfplots}
\usepackage{pgfplotstable}
\usetikzlibrary{matrix,external,fit}
%
 \pgfplotsset{surface/.style={ %
               xmax=0.34,%
               axis z line=center,%
               axis x line=center,%
               axis y line=center,%
               %axis on top,%
               zmin=-1,%
               clip=false,%
               extra x ticks={1},%
               extra x tick label={$T=1$},%
               xtick={0},%
               ytick=\empty}}
%
%
\newcommand{\controldomain}[2]{ \addplot3 [surf,fill=violet,mesh/rows=2] coordinates {(0,#1,0) (1,#1,0) (0,#2,0) (1,#2,0)}; }
\newcommand{\controldomainT}[3]{ \addplot3 [surf,fill=violet,mesh/rows=2] coordinates {(0,#1,0) (#3,#1,0) (0,#2,0) (#3,#2,0)}; }

\newcommand{\couplingdomain}[2]{ \addplot3 [surf,fill=green,mesh/rows=2] coordinates {(0,#1,0) (1,#1,0) (0,#2,0) (1,#2,0)}; }
\newcommand{\couplingdomainT}[3]{ \addplot3 [surf,fill=green,mesh/rows=2] coordinates {(0,#1,0) (#3,#1,0) (0,#2,0) (#3,#2,0)}; }
%
 \pgfplotsset{erreurs/.style={scale=1,
     legend cell align=left,
     legend pos=outer north east,
     legend plot pos=right,
     legend style={cells={anchor=east},draw=none},
     xlabel=$h$,
    xmin=0.001,xmax=0.05}}
%
\tikzset{pente/.style={opacity=0.6}}
%
\pgfplotsset{cout/.style={black,mark=diamond*,mark size=2.5,mark options={fill=gray}}}
\pgfplotsset{cible/.style={black,mark=square*,mark size=2.5,mark options={fill=gray}}}
\pgfplotsset{cibleyT/.style={black,mark=*,mark size=2.5,mark options={fill=gray}}}
\pgfplotsset{CG/.style={black,mark=otimes*,mark size=2.5,mark options={fill=gray}}}
%\pgfplotsset{solex/.style={black,mark=diamond*,mark size=2.5,mark options={fill=gray}}}
\pgfplotsset{solex/.style={black,mark=*,mark size=2.5,mark options={fill=gray}}}
\pgfplotsset{energie/.style={black,mark=triangle*,mark size=2.5,mark options={fill=gray}}}
%
\newcounter{quotecount}
\newcommand{\MyQuote}[1]{\vspace{0.5cm}%\refstepcounter{quotecount}{1}%
     \parbox{12cm}{\em #1}\hspace*{1cm}($\mathcal{P}$)\\[0.5cm]}
 
 \newcommand{\NewQuote}[1]{\vspace{0.3cm}%\refstepcounter{quotecount}{1}%
   	\parbox{14cm}{\em #1}\\[0.3cm]}
%
%
% The mesh
\def\mesh{\mathfrak{M}}
\def\petitmesh{{\scriptscriptstyle \mathfrak{M}}}
% \def\petitmesh{{\tiny \mathfrak{M}}}
% \def\petitmesh{{\scriptscriptstyle  M}}

% Mesh size
\def\hM{h^\petitmesh}

% The space of discrete functions
\def\fM{{\R^\petitmesh}}  % On the primal mesh 
\def\fdM{{\R^{\overline{\petitmesh}}}} % On the dual mesh
\def\fMdM{{\R^{\petitmesh\cup\partial \petitmesh}}}  % On the primal mesh + boundary
% The  discrete Laplace
\def\AM{{\mathcal{A}^\petitmesh}}
% The parabolic operators
\def\PM{{\mathcal{P}^\petitmesh_\pm}}
\def\PMp{{\mathcal{P}^\petitmesh_+}}
\def\PMm{{\mathcal{P}^\petitmesh_-}}

\def\ah{a^\petitmesh}
\def\bh{b^\petitmesh}

\def\yh{y^\petitmesh}
\def\qh{q^\petitmesh}
\def\zh{z^\petitmesh}
\def\ph{p^\petitmesh}

\def\phz{p^{\petitmesh}_0}
\def\yhz{y^{\petitmesh}_0}
\def\whz{w^{\petitmesh}_0}

\def\LtwoDT{{L^2_{\DT}(0,T;\fM)}}

\def\vh{v^\petitmesh}
\def\xih{\xi^\petitmesh}

\def\yhdM{y^{\partial \petitmesh}}
\def\qhdM{q^{\partial \petitmesh}}
\def\phdM{p^{\partial \petitmesh}}
\def\zhdM{z^{\partial \petitmesh}}

\def\DT{{\delta t}}
\def\eps{\varepsilon}

\newcommand\dsp{\displaystyle}

\newcommand\inter[1]{\llbracket #1\rrbracket}

\newcommand\rouge[1]{{\color{red} #1}}

 \usepackage{tikz}
 \usepackage{pgfplots}
 \usepackage{subfig}
 \pgfplotsset{surface/.style={ %
               xmax=0.34,%
               axis z line=center,%
               axis x line=center,%
               axis y line=center,%
               %axis on top,%
               zmin=-1,%
               clip=false,%
               extra x ticks={1},%
               extra x tick label={$T=1$},%
               xtick={0},%
               ytick=\empty}}


\newcommand{\controldomain}[2]{ \addplot3 [surf,fill=violet,mesh/rows=2] coordinates {(0,#1,0) (1,#1,0) (0,#2,0) (1,#2,0)}; }
\newcommand{\controldomainT}[3]{ \addplot3 [surf,fill=violet,mesh/rows=2] coordinates {(0,#1,0) (#3,#1,0) (0,#2,0) (#3,#2,0)}; }

\newcommand{\couplingdomain}[2]{ \addplot3 [surf,fill=green,mesh/rows=2] coordinates {(0,#1,0) (1,#1,0) (0,#2,0) (1,#2,0)}; }
\newcommand{\couplingdomainT}[3]{ \addplot3 [surf,fill=green,mesh/rows=2] coordinates {(0,#1,0) (#3,#1,0) (0,#2,0) (#3,#2,0)}; }

 \pgfplotsset{erreurs/.style={scale=1,
     legend cell align=left,
     legend pos=outer north east,
     legend plot pos=right,
     legend style={cells={anchor=east},draw=none},
     xlabel=$h$,
    xmin=0.001,xmax=0.05}}

\tikzset{pente/.style={opacity=0.6}}

\pgfplotsset{cout/.style={black,mark=diamond*,mark size=2.5,mark options={fill=gray}}}
\pgfplotsset{cible/.style={black,mark=square*,mark size=2.5,mark options={fill=gray}}}
\pgfplotsset{cibleyT/.style={black,mark=*,mark size=2.5,mark options={fill=gray}}}
\pgfplotsset{CG/.style={black,mark=otimes*,mark size=2.5,mark options={fill=gray}}}
%\pgfplotsset{solex/.style={black,mark=diamond*,mark size=2.5,mark options={fill=gray}}}
\pgfplotsset{solex/.style={black,mark=*,mark size=2.5,mark options={fill=gray}}}
\pgfplotsset{energie/.style={black,mark=triangle*,mark size=2.5,mark options={fill=gray}}}


%%% Local Variables:
%%% mode: latex
%%% TeX-master: "insensitizing_semi"
%%% End:

%
%\title[Controllability fractional heat equation]{Controllability of a one-dimensional fractional heat equation: theoretical and numerical aspects}
\title{Controllability of a one-dimensional fractional heat equation: theoretical and numerical aspects}
%
\author{Umberto Biccari \thanks{DeustoTech, University of Deusto, 48007 Bilbao, Basque Country, Spain.}\,\,\thanks{Facultad Ingenier\'{\i}a, Universidad de Deusto, Avda Universidades 24, 48007 Bilbao, Basque Country, Spain. Emails: \texttt{umberto.biccari@deusto.es, u.biccari@gmail.com, victor.santamaria@deusto.es}}
\and 
V\'ictor Hern\'andez-Santamar\'ia$\phantom{i}^{\ast\,\dag}$% \thanks{DeustoTech, University of Deusto, 48007 Bilbao, Basque Country, Spain.}\,\,\thanks{Facultad Ingenier\'{\i}a, Universidad de Deusto, Avda Universidades 24, 48007 Bilbao, Basque Country, Spain. Email: \texttt{victor.santamaria@deusto.es}}
}
\date{}
%
%\documentclass[preprint,1p]{amsart}

\usepackage{amsfonts}
\usepackage{amsmath}
\usepackage{amsthm}
\usepackage[utf8]{inputenc}
\usepackage[english]{babel}
\usepackage{epstopdf}
\usepackage{bmpsize}
\usepackage{epsfig}
\usepackage{hyperref}
\usepackage[english]{varioref}
\usepackage{amssymb}
\usepackage{multicol}
\usepackage{dcolumn}
\usepackage{geometry}
\usepackage{fancyhdr}
\usepackage[mathcal]{eucal}
\usepackage{mathrsfs}
\usepackage{color, colortbl}
\usepackage{microtype}
%\usepackage{longtable}
\usepackage[toc,page]{appendix}
\usepackage{bm}
\usepackage{pifont}
%\usepackage{fleqn}
\usepackage{graphicx}
\usepackage{txfonts}
\usepackage{subfig}
%\usepackage{refcheck}
%\usepackage[pagewise]{lineno}\linenumbers
\usepackage[pagewise]{lineno}
\DeclareMathAlphabet{\mathbbm}{U}{bbm}{m}{n}% from bbm.sty

%\input fig4tex.tex

\allowdisplaybreaks
\DisableLigatures{encoding = *, family = * }

\numberwithin{equation}{section}
%\pagestyle{myheadings}

\newtheorem{theorem}{Theorem}[section]
\newtheorem{proposition}{Proposition}[section]
\newtheorem{lemma}{Lemma}[section]
\newtheorem{definition}{Definition}[section]
\newtheorem{remark}{Remark}[section]

\numberwithin{equation}{section}
\numberwithin{theorem}{section}
\numberwithin{remark}{section}
\numberwithin{lemma}{section}
\numberwithin{proposition}{section}
\numberwithin{definition}{section}

\newcommand{\norm}[2]{{\left\|#1\right\|}_{#2}}
\newcommand{\fl}[2]{(-d_x^2)^{#1}#2}
\newcommand{\rfl}[2]{A^{#1}_{\Omega}#2}
\newcommand{\hp}[1]{\hphantom{#1}}
\newcommand{\cns}{c_{N,s}}
\newcommand{\ccs}{c_{1,s}}
\newcommand{\ffl}[2]{(-d_x^{\,2})^{#1}#2}
\newcommand{\flh}[2]{\frac{1}{\Gamma(-s)}\int_0^{+\infty}\Big(e^{t\Delta}#2 - #2\Big)\frac{dt}{t^{1+#1}}}
\newcommand{\kernel}[1]{|x-y|^{#1}}
\newcommand{\dkj}{\delta_{kj}}
\newcommand{\intr}[1]{\underset{#1}{\int}}
\newcommand{\Do}[1]{D_{#1}}
\newcommand{\Hs}{H^s_0(\Omega)}
\newcommand{\ue}[1]{#1^{\,\varepsilon}}
\newcommand{\xHdot}[1]{\dot{H}^{#1}}
\newcommand{\ha}[2]{\mathbf{H}_{#1}^{#2}}
\newcommand{\lhi}{\mathcal{L}_i^h}
\newcommand{\NN}{\mathbb{N}}
\newcommand{\ZZ}{\mathbb{Z}}
\newcommand{\RR}{\mathbb{R}}
\newcommand{\CC}{\mathbb{C}}
\newcommand{\TT}{\mathbf{T}}
\newcommand\inter[1]{\llbracket #1\rrbracket}
\newcommand\mesh{\mathfrak{M}}

%\usepackage{tikz}
%\usepackage{pgfplots}
%\usepackage{pgfplotstable}
%\usetikzlibrary{matrix,external,fit}

\newcounter{quotecount}
\newcommand{\MyQuote}[1]{\vspace{0.5cm}%\refstepcounter{quotecount}{1}%
     \parbox{12cm}{\em #1}\hspace*{1cm}($\mathcal{P}$)\\[0.5cm]}

%
% The mesh
\def\mesh{\mathfrak{M}}
\def\petitmesh{{\scriptscriptstyle \mathfrak{M}}}
% \def\petitmesh{{\tiny \mathfrak{M}}}
% \def\petitmesh{{\scriptscriptstyle  M}}

% Mesh size
\def\hM{h^\petitmesh}

% The space of discrete functions
\def\fM{{\R^\petitmesh}}  % On the primal mesh 
\def\fdM{{\R^{\overline{\petitmesh}}}} % On the dual mesh
\def\fMdM{{\R^{\petitmesh\cup\partial \petitmesh}}}  % On the primal mesh + boundary
% The  discrete Laplace
\def\AM{{\mathcal{A}^\petitmesh}}
% The parabolic operators
\def\PM{{\mathcal{P}^\petitmesh_\pm}}
\def\PMp{{\mathcal{P}^\petitmesh_+}}
\def\PMm{{\mathcal{P}^\petitmesh_-}}

\def\ah{a^\petitmesh}
\def\bh{b^\petitmesh}

\def\yh{y^\petitmesh}
\def\qh{q^\petitmesh}
\def\zh{z^\petitmesh}
\def\ph{p^\petitmesh}

\def\phz{p^{\petitmesh}_0}
\def\yhz{y^{\petitmesh}_0}
\def\whz{w^{\petitmesh}_0}

\def\LtwoDT{{L^2_{\DT}(0,T;\fM)}}

\def\vh{v^\petitmesh}
\def\xih{\xi^\petitmesh}

\def\yhdM{y^{\partial \petitmesh}}
\def\qhdM{q^{\partial \petitmesh}}
\def\phdM{p^{\partial \petitmesh}}
\def\zhdM{z^{\partial \petitmesh}}

\def\DT{{\delta t}}
\def\eps{\varepsilon}

\newcommand\dsp{\displaystyle}

\newcommand\inter[1]{\llbracket #1\rrbracket}

\newcommand\rouge[1]{{\color{red} #1}}

 \usepackage{tikz}
 \usepackage{pgfplots}
 \usepackage{subfig}
 \pgfplotsset{surface/.style={ %
               xmax=0.34,%
               axis z line=center,%
               axis x line=center,%
               axis y line=center,%
               %axis on top,%
               zmin=-1,%
               clip=false,%
               extra x ticks={1},%
               extra x tick label={$T=1$},%
               xtick={0},%
               ytick=\empty}}


\newcommand{\controldomain}[2]{ \addplot3 [surf,fill=violet,mesh/rows=2] coordinates {(0,#1,0) (1,#1,0) (0,#2,0) (1,#2,0)}; }
\newcommand{\controldomainT}[3]{ \addplot3 [surf,fill=violet,mesh/rows=2] coordinates {(0,#1,0) (#3,#1,0) (0,#2,0) (#3,#2,0)}; }

\newcommand{\couplingdomain}[2]{ \addplot3 [surf,fill=green,mesh/rows=2] coordinates {(0,#1,0) (1,#1,0) (0,#2,0) (1,#2,0)}; }
\newcommand{\couplingdomainT}[3]{ \addplot3 [surf,fill=green,mesh/rows=2] coordinates {(0,#1,0) (#3,#1,0) (0,#2,0) (#3,#2,0)}; }

 \pgfplotsset{erreurs/.style={scale=1,
     legend cell align=left,
     legend pos=outer north east,
     legend plot pos=right,
     legend style={cells={anchor=east},draw=none},
     xlabel=$h$,
    xmin=0.001,xmax=0.05}}

\tikzset{pente/.style={opacity=0.6}}

\pgfplotsset{cout/.style={black,mark=diamond*,mark size=2.5,mark options={fill=gray}}}
\pgfplotsset{cible/.style={black,mark=square*,mark size=2.5,mark options={fill=gray}}}
\pgfplotsset{cibleyT/.style={black,mark=*,mark size=2.5,mark options={fill=gray}}}
\pgfplotsset{CG/.style={black,mark=otimes*,mark size=2.5,mark options={fill=gray}}}
%\pgfplotsset{solex/.style={black,mark=diamond*,mark size=2.5,mark options={fill=gray}}}
\pgfplotsset{solex/.style={black,mark=*,mark size=2.5,mark options={fill=gray}}}
\pgfplotsset{energie/.style={black,mark=triangle*,mark size=2.5,mark options={fill=gray}}}


%%% Local Variables:
%%% mode: latex
%%% TeX-master: "insensitizing_semi"
%%% End:


\title[FE approximation of the 1-d fractional Poisson equation]{A Finite Element approximation of the one-dimensional fractional Poisson equation with applications to numerical control}

\author{U.~Biccari}
\address{Umberto Biccari, DeustoTech, University of Deusto, 48007 Bilbao, Basque Country, Spain.}
\address{Umberto Biccari, Facultad Ingenier\'{\i}a, Universidad de Deusto, Avda Universidades 24, 48007 Bilbao, Basque Country, Spain.}
\email{umberto.biccari@deusto.es,u.biccari@gmail.com}
%
\author{V.~Hern\'andez-Santamar\'ia}
\address{Victor Hern\'andez-Santamar\'ia, DeustoTech, University of Deusto, 48007 Bilbao, Basque Country, Spain.}
\address{Victor Hern\'andez-Santamar\'ia, Facultad Ingenier\'{\i}a, Universidad de Deusto, Avda Universidades 24, 48007 Bilbao, Basque Country, Spain.}
\email{victor.santamaria@deusto.es}

\thanks{The work of Umberto Biccari was partially supported by the Advanced Grant DYCON (Dynamic Control) of the European Research Council Executive Agency, by the MTM2014-52347 Grant of the MINECO (Spain) and by the Air Force Office of Scientific Research under the Award No: FA9550-15-1-0027. The work of V\'ictor Hern\'andez-Santamar\'ia was partially supported by the Advanced Grant DYCON (Dynamic Control) of the European Research Council Executive Agency.}

\begin{document}

\bibliographystyle{acm}

\maketitle

\begin{abstract}
We analyze the controllability problem for a one-dimensional heat equation involving the fractional Laplacian $\fl{s}{}$ on the interval $(-1,1)$. Using classical results and techniques, we show that, acting from an open subset $\omega\subset(-1,1)$, the problem is null-controllable for $s>1/2$ and that for $s\leq 1/2$ we only have approximate controllability. Moreover, we deal with the numerical computation of the control employing the penalized Hilbert Uniqueness Method (HUM) and a finite element (FE) scheme for the approximation of the solution to the corresponding elliptic equation. We present several experiments confirming the expected controllability properties.
\end{abstract}

\section{Introduction and main results}\label{intro_sec}
Let $\omega\subset (-1,1)$ be an open and nonempty subset. In this work, we analyze controllability properties for the following nonlocal one-dimensional heat equation 
\begin{align}\label{heat_frac}
	\begin{cases}
		z_t + \fl{s}{z} = g\mathbf{1}_{\omega},\quad & (x,t)\in (-1,1)\times(0,T)
		\\
		z=0, & (x,t)\in[\,\RR\setminus (-1,1)\,]\times(0,T)
		\\
		z(x,0)=z_0(x), & x\in (-1,1),
	\end{cases}
\end{align} 
where $z_0\in L^2(-1,1)$ is a given initial datum. In more detail, we are interested in the resolution of the following control problem: given any $T>0$, find a control function $g\in L^2(\omega\times(0,T))$ such that the corresponding solution to \eqref{heat_frac} satisfies $z(x,T)=0$. 

In \eqref{heat_frac}, for all $s\in(0,1)$, $\fl{s}{}$ denotes the one-dimensional fractional Laplace operator, which is defined as the following singular integral
\begin{align}\label{fl_intro}
	\fl{s}{u}(x) = \ccs\,P.V.\,\int_{\RR}\frac{u(x)-u(y)}{|x-y|^{1+2s}}\,dy. 
\end{align}
Here, $\ccs$ is a normalization constant given by
\begin{align*}
	\ccs = \frac{s2^{2s}\Gamma\left(\frac{1+2s}{2}\right)}{\sqrt{\pi}\Gamma(1-s)},
\end{align*}
where $\Gamma$ is the usual Gamma function. Moreover, we have to mention that, for having a completely rigorous definition of the fractional Laplace operator, it is necessary to introduce also the class of functions $u$ for which computing $\fl{s}{u}$ makes sense. We postpone this discussion to the next section.

The analysis of non-local operators and non-local PDEs is a topic in continuous development.
A motivation for this growing interest relies in the large number of possible applications in the modeling of several complex phenomena for which a local approach turns up to be inappropriate or limiting.
Indeed, there is an ample spectrum of situations in which a non-local equation gives a
significantly better description than a PDE of the problem one wants to analyze.
Among others, we mention applications in turbulence (\cite{bakunin2008turbulence}), anomalous transport and diffusion (\cite{bologna2000anomalous,meerschaert2012fractional}), elasticity (\cite{dipierro2015dislocation}), image processing (\cite{gilboa2008nonlocal}), porous media flow (\cite{vazquez2012nonlinear}), wave propagation in heterogeneous high contrast media (\cite{zhu2014modeling}). Also, it is well known that the fractional Laplacian is the generator of s-stable processes, and it is often used in stochastic models with applications, for instance, in mathematical finance (\cite{levendorskii2004pricing,pham1997optimal}).

One of the main differences between these non-local models and classical Partial Differential Equations is that the fulfillment of a non-local equation at a point involves the values of the function far away from that point.

It is well-known that the classical local heat equation (as well as many more general variants) is null-controllable in any time $T>0$ (see, e.g., \cite{fattorini1971exact,fursikov1996controllability,lebeau1995controle}). Nevertheless, to the best of our knowledge, there are few results in the literature on the null-controllability of the fractional heat equation, and none of them is for a problem involving the fractional Laplacian in its integral form \eqref{fl_intro}. The existing results (\cite{micu2006controllability,miller2006controllability}), instead, deal with the so-called \textit{spectral} fractional Laplace operator, whose definition will be given later. 

In this paper, we deal with the controllability of \eqref{heat_frac}, both from the theoretical and the numerical point of view. Employing spectral analysis techniques based on the works \cite{kulczycki2010spectral,kwasnicki2012eigenvalues}, the first main result that we obtain is the following

\begin{theorem}\label{null_control_thm}
Given any $z_0\in L^2(-1,1)$ the parabolic problem \eqref{heat_frac} is null-controllable at time $T>0$ with a control function $g\in L^2(\omega\times(0,T))$ if and only if $s>1/2$.  
\end{theorem}

Furthermore, even if for $s\leq 1/2$ null controllability for \eqref{heat_frac} fails, we still have the following result of approximate controllability, as a consequence of unique continuation properties for the fractional Laplace operator (\cite{fall2014unique}). 

\begin{theorem}\label{approx_control_thm}
Let $s\in(0,1)$. Given any $z_0\in L^2(-1,1)$, there exists a control function $g\in L^2(\omega\times(0,T))$ such that the unique solution $z$ to the parabolic problem \eqref{heat_frac} is approximately controllable at time $T>0$.
\end{theorem}

Theorems \ref{null_control_thm} and \eqref{approx_control_thm} will then find a confirmation in the study of the corresponding numerical control problem. With this purpose, we will employ the penalized Hilbert Uniqueness Method, which relies on some classical works of Glowinski and Lions (\cite{glowinski1995exact,glowinski2008exact}). 

Notice that this method is very general, and it may by applied to a broad class of PDEs control problems (\cite{boyer2013penalised,boyer2017insensitizing,boyer2014approximate,khodja2017partial}). When using it for treating a nonlocal problem as \eqref{heat_frac}, new issues arise related to the discretization of the corresponding elliptic problem.    

In this framework, a preliminary step for the for the resolution of the numerical control problem will be a finite element (FE)  approximation of the solution to the following non-local Poisson equation
\begin{align}\label{PE}
	\begin{cases}
		\fl{s}{u} = f, & x\in(-L,L)
		\\
		u\equiv 0, & x\in\RR\setminus(-L,L).
	\end{cases}
\end{align}

In the recent past, the fractional Laplacian has been widely analyzed also from the point of view of numerical analysis. We refer, for instance, to the works \cite{acosta2017short,acosta2017fractional,borthagaray2017laplaciano}. There, the authors present a FE scheme for implementing the solution of \eqref{PE} in a bounded domain $\Omega\subset\RR^2$. In particular, they provide appropriate quadrature rules in order to solve numerically the variational formulation associated to the problem. Moreover, in \cite{acosta2017fractional,borthagaray2017laplaciano} it is also developed an accurate analysis of the efficiency of the FE method, employing several existing results. The techniques of the aforementioned works have then been applied in \cite{acosta2017finite}, combined with a convolution quadrature approach, for solving evolution equations involving the fractional Laplacian. For the sake of completeness, we also mention \cite{bonito2015numerical}, where it is presented a discretization of the spectral fractional Laplacian and its application to the evolutionary case \cite{bonito2017approximation}, and \cite{nochetto2015pde}, where the same problem is treated applying the well known extension of Caffarelli and Silvestre (\cite{caffarelli2007extension}).   

Our method deals with a FE approximation in one space dimension for the fractional Poisson equation. The main novelty of our work, with respect to \cite{acosta2017short,acosta2017fractional,borthagaray2017laplaciano}, relies on the fact that, since we are dealing with the one-dimensional case, we will not need any quadrature rule and each entry of the stiffness matrix can be computed explicitly. This has the great advantage of significantly reducing the computational cost of the algorithm and, therefore, our discretization method is suitable for being included in more general applications.  

\rouge{This paper is organized as follows. In Section \ref{theor_sec}, we present some existing theoretical results for the problems that we are going to analyze. In particular, we give a more accurate definition of the fractional Laplace operator and we introduce the variational formulation associated to \eqref{PE} (needed for the development of the FE scheme). Concerning the parabolic problem \eqref{heat_frac}, we present a couple of controllability results, which will help us in the verification of the accuracy of the numerical method. In Section \ref{fe_sec}, we describe our FE method for the elliptic equation \eqref{PE} and we present the algorithm for the penalized HUM, employed for the numerical control of \eqref{heat_frac}. In Section \ref{res_numerical} we present and comment the results of our numerical simulations. Finally, in Appendix \ref{appendix} we include the complete details for computing the stiffness matrix associated to our FE scheme.}


\section{Preliminary results}\label{theor_sec}

In this Section, we introduce some preliminary result that will be useful in the remaining of the paper.

\subsection{Elliptic problem}

We start by giving a more rigorous definition of the fractional Laplace operator, as we have anticipated in Section \ref{intro_sec}. Let
\begin{align*}
	\mathcal L^1_s(\RR) :=\left\{ u:\RR\longrightarrow\RR\,:\; u\textrm{ measurable },\;\int_{\RR}\frac{|u(x)|}{(1+|x|)^{1+2s}}\,dx<\infty\right\}.
\end{align*}
For any $u\in\mathcal L_s^1$ and $\varepsilon>0$ we set 
\begin{align*}
	(-d_x^2)^s_{\varepsilon} u(x) = \ccs\,\int_{|x,y|>\varepsilon}\frac{u(x)-u(y)}{|x-y|^{1+2s}}\,dy,\;\;\; x\in\RR.
\end{align*}
The fractional Laplacian is then defined by the following singular integral
\begin{align}\label{fl}
	\fl{s}{u}(x) = \ccs\,P.V.\,\int_{\RR}\frac{u(x)-u(y)}{|x-y|^{1+2s}}\,dy = \lim_{\varepsilon\to 0^+} (-d_x^2)^s_{\varepsilon} u(x), \;\;\; x\in\RR,
\end{align}
provided that the limit exists. 

We notice that if $0<s<1/2$ and $u$ is smooth, for example bounded and Lipschitz continuous on $\RR$, then the integral in \eqref{fl} is in fact not really singular near $x$ (see e.g. \cite[Remark 3.1]{dihitchhiker}). Moreover, $\mathcal L_s^1(\RR)$ is the right space for which $v:= (-d_x^2)^s_{\varepsilon} u$ exists for every $\varepsilon > 0$, $v$ being also continuous at the continuity points of $u$.

Let us now introduce the variational formulation associated to equation \eqref{PE}, which will be the starting point for the development of the FE approximation of the problem we are considering. That is, find $u\in H^s_0(-L,L)$ such that
\begin{align*}
	a(u,v) = \int_{-L}^L fv\,dx,	
\end{align*}
for all $v\in H_0^s(-L,L)$, where the bilinear form $a(\cdot,\cdot):H^s_0(-L,L)\times H^s_0(-L,L)\to \RR$ is given by
\begin{align*}
	a(u,v)=\frac{\ccs}{2} \int_{\RR}\int_{\RR}\frac{(u(x)-u(y))(v(x)-v(y))}{|x-y|^{1+2s}}\,dxdy.	
\end{align*}
Here, $H^s_0(-L,L)$ denotes the space 
\begin{align*}
	H^s_0(-L,L) :=\Big\{\, u\in H^s(\RR)\,:\,u=0 \textrm{ in } \RR\setminus(-L,L)\,\Big\}, 
\end{align*}
while with $H^s(\RR)$ we indicate the classical fractional Sobolev space of order $s$. We refer to \cite{dihitchhiker} for a complete description of these spaces.  

Since the bilinear form $a$ is continuous and coercive, Lax-Milgram Theorem immediately implies existence and uniqueness of solutions to the Dirichlet problem \eqref{PE}. In more detail, if $f\in H^{-s}(-L,L)$, then \eqref{PE} admits a unique weak solution $u\in H_0^s(-L,L)$ (see, e.g., \cite[Proposition 2.1]{biccari2017local}). Here $H^{-s}(-L,L)$ stands for the dual space of $H^s_0(-L,L)$. Furthermore, in the literature it is possible to find improved regularity results for the solution to \eqref{PE}, both in H\"older and Sobolev spaces. The interested reader may refer, for instance, to \cite{acosta2017fractional,biccari2017local,leonori2015basic,ros2014dirichlet,ros2014extremal}.

\subsection{Parabolic problem}

As we mentioned in Section \ref{intro_sec}, the main goal of the present paper is to obtain a FE discretization of the fractional Laplacian. An application for this approximation will then be the numerical resolution of the fractional heat equation \eqref{heat_frac} and the associated control problem. Before doing that, let us recall the following definitions of controllability.

\begin{definition}
System \eqref{heat_frac} is said to be \textit{null-controllable} at time $T$ if, for any $z_0\in L^2(-1,1)$, there exists $g\in L^2(\omega\times(0,T))$ such that the corresponding solution $z$ satisfies 
\begin{align*}
	z(x,T)=0.
\end{align*}

\end{definition}

\begin{definition}
System \eqref{heat_frac} is said to be \textit{approximately controllable} at time $T$ if, for any $z_0,z_T\in L^2(-1,1)$ and any $\delta>0$, there exists $g\in L^2(\omega\times(0,T))$ such that the corresponding solution $z$ satisfies \begin{align*}
	\norm{z(x,T)-z_T}{L^2(-1,1)}<\delta.
\end{align*}
\end{definition}

Therefore, given any initial datum $z_0\in L^2(-1,1)$ we are interested in computing numerically the control function $g$ that drives the solution $z$ to zero in time $T$. Before describing the methodology that we shall adopt, we recall the existing theoretical results on the controllability of the fractional heat equation \eqref{heat_frac}. This will give us a hint about what we should expect from our simulations, and will provide a validation of the accuracy of our numerical method.

First of all, it is worth to mention that the existence, uniqueness and regularity of the solutions to \eqref{heat_frac} has been studied by several authors. Among others, we mention the works \cite{biccari2017parabolic,fernandez2016boundary,leonori2015basic}. 

Concerning now the control problem, we have to mention that, to the best of our knowledge, there are few results in the literature on the null-controllability of the fractional heat equation, and none of them is for a problem involving the fractional Laplacian in its integral form \eqref{fl}. The existing results, instead, deal with the \textit{spectral} definition of the fractional Laplace operator, which is given as follows.

Let $\{\psi_k,\lambda_k\}_{k\in\NN}\subset H_0^1(-1,1)\times\RR^+$ be the set of normalized eigenfunctions and eigenvalues of the Laplace operator in $(-1,1)$ with homogeneous Dirichlet boundary conditions, so that $\{\psi_k\}_{k\in\NN}$ is an orthonormal basis of $L^2(-1,1)$ and         
\begin{align*}
	\begin{cases}
		-d_x^2\psi_k =\lambda_k\psi_k, & x\in (-1,1), 
		\\
		\psi_k(-1)=\psi_k(1)=0.
	\end{cases}
\end{align*}

Then, the \textit{spectral fractional Laplacian} $(-d_x^2)^s_S$ is defined by
\begin{align}\label{fl_spec}
	(-d_x^2)^s_S u(x) = \sum_{k\geq 1}\langle u,\psi_k\rangle \lambda_k^s\psi_k(x),
\end{align}
firstly for $u\in C_0^{\infty}(-1,1)$ and then for $u\in H_0^s(-1,1)$ employing a density argument.

It is important to notice that the spectral fractional Laplacian and the fractional Laplacian defined as in \eqref{fl} are two different operators. Indeed, definition \eqref{fl_spec} depends on the choice of the domain (in this case, $(-1,1)$), while the integral definition does not. For a complete discussion on the differences of these two operators, we refer to \cite{servadei2014spectrum}.

The fractional heat equation involving the operator $(-d_x^2)^s_S$ has been analyzed in \cite{micu2006controllability}, where the authors proved its null controllability, provided that $s>1/2$. For $s\leq 1/2$, instead, null controllability does not hold, not even for $T$ large. This negative result is based on the equivalence (consequence of M\"untz Theorem, see, e.g., \cite[Page 24]{schwartz1958etude}) between the controllability property (more specifically, the possibility of proving an observability inequality), and the following condition for the eigenvalues of the operator considered
%
\begin{align}\label{eigen_cond}
	\sum_{k\geq 1} \frac{1}{\lambda_k}<\infty,
\end{align} 
%
which is clearly not satisfied in the case of the spectral fractional Laplacian when $s\leq 1/2$, since in that case the eigenvalues are $\lambda_k = (k\pi)^{2s}$. Finally, in \cite{miller2006controllability}, the same result as in \cite{micu2006controllability} is obtained in a multi-dimensional setting, by means of a  \textit{spectral observability condition} for a negative self-adjoint operator, which allows to prove the null-controllability of the semi-group that it generates.

Even if we are not aware of any controllability result, neither positive nor negative, for the parabolic equation involving the integral fractional Laplacian, at least in the one space dimension, these properties are easily achievable. In more detail, we have the following.

\begin{proposition}
For all $z_0\in L^2(-1,1)$ the parabolic problem \eqref{heat_frac} is null-controllable with a control function $g\in L^2(\omega\times(0,T))$ if and only if $s>1/2$.  
\end{proposition}

\begin{proof}
First of all, multiplying \eqref{heat_frac} by $\varphi$ and integrating over $(-1,1)\times (0,T)$, it is straightforward to check that $z(x,T)=0$ if and only if 
\begin{align}\label{control_id}
	\int_0^T\int_{-1}^1 \varphi(x,t)g(x,t)\mathbf{1}_{\omega}(x)\,dxdt = -\int_{-1}^1 u_0(x)\varphi(x,0)\,dx,
\end{align}
for all $\varphi^T\in L^2(-1,1)$, where $\varphi(x,t)$ is the unique solution to the adjoint system
\begin{align}\label{heat_frac_adj}
	\begin{cases}
		-\varphi_t + \fl{s}{\varphi} = 0, & (x,t)\in (-1,1)\times(0,T)
		\\
		\varphi = 0, & (x,t)\in [\,\RR\setminus(-1,1)\,]\times(0,T)
		\\
		\varphi(x,T) = \varphi^T(x), & x\in (-1,1).
	\end{cases}
\end{align}

In turn, it is classical that \eqref{control_id} is equivalent to the existence of a constant $C>0$ such that the following observability inequality holds
\begin{align}\label{obs}
	\norm{\varphi(x,0)}{L^2(-1,1)}^2\leq C\int_0^T\left|\,\int_{-1}^1 \varphi(x,t)g(x,t)\mathbf{1}_{\omega}(x)\,dx\,\right|^2\,dt,
\end{align}

Notice that $\varphi$ can be written in terms of the basis of eigenfunctions $\{\varrho_k\}_{k\geq 1}$. Namely,
\begin{align}\label{adj_sol}
	\varphi(x,t) = \sum_{k\geq 1} \varphi_ke^{-\lambda_k(T-t)}\varrho_k(x), 
\end{align}
where $\varphi_k = \langle \varphi^T,\varrho_k\rangle$ and, for $k\geq 1$, $\varrho(x)$ are the solutions to the following eigenvalue problem 
\begin{align*}
	\begin{cases}
		\fl{s}{\varrho_k} = \lambda_k\varrho_k, & x\in (-1,1), \;\; k\in\NN
		\\
		\varrho_k = 0, & x\in \,\RR\setminus(-1,1).
	\end{cases}
\end{align*}

Now, plugging \eqref{adj_sol} into \eqref{obs}, using the orthonormality of the eigenfunctions $\varrho_k$ and employing the change of variables $T-t\mapsto t$, the observability inequality becomes 
\begin{align}\label{obs_spectr}
	\sum_{k\geq 1} |\varphi_k|^2e^{-2\lambda_k T} \leq C\int_0^T\left|\,\sum_{k\geq 1} \varphi_kg_k(t)e^{-\lambda_k t}\right|^2\,dt, 
\end{align}
where $g_k = \langle g\mathbf{1}_{\omega},\varrho_k\rangle$. 

By means of M\"untz Theorem, inequalities of the form \eqref{obs_spectr} are well known to be true if and only if \eqref{eigen_cond} holds. On the other hand, according to \cite{kulczycki2010spectral,kwasnicki2012eigenvalues} we have 
\begin{align*}
	\lambda_k = \left(\frac{k\pi}{2}-\frac{(1-s)\pi}{4}\right)^{2s}+O\left(\frac{1}{k}\right).
\end{align*}
 
Therefore, we easily see that the condition \eqref{eigen_cond} is satisfied if and only if $s>1/2$. If $s\leq 1/2$, instead, the series diverges, since it behaves as an harmonic series. In conclusion, the observability inequality \eqref{obs} is proved when $s>1/2$, but it is false when $s\leq 1/2$. This concludes the proof. 
\end{proof}

Even if for $s\leq 1/2$ null controllability for \eqref{heat_frac} fails, we still have the following result of approximate controllability. 

\begin{proposition}
Let $s\in(0,1)$. For all $z_0\in L^2(-1,1)$, there exists a control function $g\in L^2(\omega\times(0,T))$ such that the unique solution $z$ to the parabolic problem \eqref{heat_frac} is approximately controllable.
\end{proposition}

\begin{proof}

The result will be a consequence of the following unique continuation property for the solution to the adjoint equation \eqref{heat_frac_adj}

\MyQuote{
Given $s\in(0,1)$ and $\varphi^T_0\in L^2(-1,1)$, let $\varphi$ be the unique solution to the system \eqref{heat_frac_adj}. Let $\omega\subset (-1,1)$ be an arbitrary open set. If $\varphi = 0$ on $\omega\times(0,T)$, then $\varphi = 0$ on $(-1,1)\times(0,T)$.}

Therefore, we are reduced to the proof of the property ($\mathcal P$). To this end, let us recall that $\varphi$ can be expressed in the form \eqref{adj_sol} and let us assume that 
\begin{align}\label{uc}
	\varphi=0 \textrm{ in } \omega\times(0,T). 
\end{align}

Let $\{\psi_{k_j}\}_{1\leq k\leq m_k}$ be an orthonormal basis of $\ker(\lambda_k-\fl{s}{})$. Then, \eqref{adj_sol} can be rewritten as
\begin{align*}
	\varphi(x,t) = \sum_{k\geq 1} \left(\sum_{j=1}^{m_k} \varphi_{k_j}\psi_{k_j}(x)\right)e^{-\lambda_k(T-t)}, \;\;\; (x,t)\in (-1,1)\times(-\infty, T). 
\end{align*}

Let $z\in\CC$ with $\eta:=\Re(z)>0$ and let $N\in\NN$. Since the functions $\psi_{k_j}$, $1\leq j\leq m_k$, $1\leq k\leq N$ are orthonormal, we have that
\begin{align*}
	\norm{\sum_{k=1}^N \left(\sum_{j=1}^{m_k} \varphi_{k_j}\psi_{k_j}(x)\right)e^{z(t-T)}e^{-\lambda_k(T-t)}}{L^2(-1,1)}^2 & \leq \sum_{k=1}^N \left(\sum_{j=1}^{m_k} |\varphi_{k_j}|^2\right)e^{2\eta(t-T)}e^{-2\lambda_k(T-t)}
	\\
	& \leq \sum_{k\geq 1} \left(\sum_{j=1}^{m_k} |\varphi_{k_j}|^2\right)e^{2\eta(t-T)}e^{-2\lambda_k(T-t)} \leq Ce^{2\eta(t-T)}\norm{\varphi^T}{L^2(-1,1)}^2.
\end{align*}

Hence, letting 
\begin{align*}
w_N(x,t):= \sum_{k=1}^N \left(\sum_{j=1}^{m_k} \varphi_{k_j}\psi_{k_j}(x)\right)e^{z(t-T)}e^{-\lambda_k(T-t)},
\end{align*}
we have shown that $\norm{w_T(x,t)}{L^2(-1,1)}\leq Ce^{\eta(t-T)}\norm{\varphi^T}{L^2(-1,1)}$. Moreover, we have
\begin{align*}
	\int_{-\infty}^T e^{\eta(t-T)}\norm{\varphi^T}{L^2(-1,1)}\,dt = \frac{1}{\eta}\norm{\varphi^T}{L^2(-1,1)}\int_0^{+\infty} e^{-\tau}\,d\tau = \frac{1}{\eta}\norm{\varphi^T}{L^2(-1,1)}.
\end{align*}

Therefore, we can apply the Dominated Convergence Theorem, obtaining
\begin{align}\label{dct_identity}
	\lim_{N\to+\infty}\int_{-\infty}^T w_N(x,t)\,dt &= \int_{-\infty}^T\lim_{N\to+\infty} w_N(x,t)\,dt = \int_{-\infty}^T e^{z(t-T)} \sum_{k=1}^{+\infty} \left(\sum_{j=1}^{m_k} \varphi_{k_j}\psi_{k_j}(x)\right)e^{-\lambda_k(T-t)}\,dt \notag
	\\
	&=\sum_{k=1}^{+\infty}\sum_{j=1}^{m_k} \varphi_{k_j}\psi_{k_j}(x) \int_{-\infty}^T e^{z(t-T)}e^{-\lambda_k(T-t)}\,dt =\sum_{k=1}^{+\infty}\sum_{j=1}^{m_k} \varphi_{k_j}\psi_{k_j}(x) \int_0^{+\infty} e^{-(z+\lambda_k)\tau}\,d\tau \notag
	\\
	&=\sum_{k=1}^{+\infty}\sum_{j=1}^{m_k} \frac{\varphi_{k_j}}{z+\lambda_k}\psi_{k_j}(x), \;\;\; x\in (-1,1)\;\Re(z)>0.
\end{align} 

It follows from \eqref{uc} and \eqref{dct_identity} that 
\begin{align*}
	\sum_{k=1}^{+\infty}\sum_{j=1}^{m_k} \frac{\varphi_{k_j}}{z+\lambda_k}\psi_{k_j}(x)=0, \;\;\; x\in\omega,\;\Re(z)>0.
\end{align*}

This holds for every $z\in\CC\setminus\{-\lambda_k\}_{k\in\NN}$, using the analytic continuation in $z$. Hence, taking a suitable small circle around $-\lambda_{\ell}$ not including $\{-\lambda_k\}_{k\neq\ell}$ and integrating on that circle we get that
\begin{align*}
	w_\ell:=\sum_{j=1}^{m_{\ell}} \varphi_{\ell_j}\psi_{\ell_j}(x)=0, \;\;\; x\in\omega.
\end{align*}

According to \cite[Theorem 1.4]{fall2014unique}, $\fl{s}{}$ has the unique continuation property in the sense that if $\lambda_k$ is an eigenvalue of $\fl{s}{}$ on (-1,1) with Dirichlet boundary conditions, and $(\fl{s}{}-\lambda_k)\varrho_k = 0$ in $(-1,1)$ with $\varrho_k = 0$ in $\omega$, then $\varrho_k = 0$ in $(-1,1)$. 
This can applied to $w_{\ell}$, in order to conclude $w_{\ell} = 0$ in $(-1,1)$ for every $\ell$. Since $\{\psi_{\ell_j}\}_{1\leq j\leq m_{\ell}}$ are linearly independent in $L^2(-1,1)$, we get $\varphi_{\ell_j} = 0$, $1\leq j\leq m_k$, $\ell\in\NN$. It follows that $\varphi^T=0$ and hence, $\varphi=0$ in $(-1,1)\times(0,T)$, meaning that $\varphi$ enjoys the property $(\mathcal{P})$. As an immediate consequence, we have that our original equation \eqref{heat_frac} is approximately controllable. This last fact being classical (see, e.g., \cite[Theorem 2.5]{keyantuo2016interior}), we will leave the details to the reader. Our proof is then concluded. 
\end{proof}



\section{Proof of the controllability properties}

This section is devoted to study the control properties for the parabolic system \eqref{heat_frac}. We begin by proving the null controllability result.

\begin{proof}[Proof of Theorem \ref{null_control_thm}]
First of all, for all $\varphi^T\in L^2(-1,1)$, let $\varphi(x,t)$ be the unique solution to the adjoint system
\begin{align}\label{heat_frac_adj}
	\begin{cases}
		-\varphi_t + \fl{s}{\varphi} = 0, & (x,t)\in (-1,1)\times(0,T)
		\\
		\varphi = 0, & (x,t)\in [\,\RR\setminus(-1,1)\,]\times(0,T)
		\\
		\varphi(x,T) = \varphi^T(x), & x\in (-1,1).
	\end{cases}
\end{align}

Multiplying \eqref{heat_frac} by $\varphi$ and integrating over $(-1,1)\times (0,T)$, it is straightforward to check that $z(x,T)=0$ if and only if 
\begin{align}\label{control_id}
	\int_0^T\int_{-1}^1 \varphi(x,t)g(x,t)\mathbf{1}_{\omega}(x)\,dxdt = -\int_{-1}^1 u_0(x)\varphi(x,0)\,dx,
\end{align}

In turn, it is classical that \eqref{control_id} is equivalent to the existence of a constant $C>0$ such that the following observability inequality holds
\begin{align}\label{obs}
	\norm{\varphi(x,0)}{L^2(-1,1)}^2\leq C\int_0^T\left|\,\int_{-1}^1 \varphi(x,t)g(x,t)\mathbf{1}_{\omega}(x)\,dx\,\right|^2\,dt,
\end{align}

Notice that $\varphi$ can be expressed in the basis of the eigenfunctions of the fractional Laplacian on $(-1,1)$ with zero Dirichlet boundary conditions. Namely,
\begin{align}\label{adj_sol}
	\varphi(x,t) = \sum_{k\geq 1} \varphi_ke^{-\lambda_k(T-t)}\varrho_k(x), 
\end{align}
where $\varphi_k = \langle \varphi^T,\varrho_k\rangle$ and, for $k\geq 1$, $\varrho_k(x)$ are the solutions to the following eigenvalue problem 
\begin{align*}
	\begin{cases}
		\fl{s}{\varrho_k} = \lambda_k\varrho_k, & x\in (-1,1), \;\; k\in\NN
		\\
		\varrho_k = 0, & x\in \,\RR\setminus(-1,1).
	\end{cases}
\end{align*}
	
Now, plugging \eqref{adj_sol} into \eqref{obs}, using the orthonormality of the eigenfunctions $\varrho_k$ as a basis of $L^2(-1,1)$ and employing the change of variables $T-t\mapsto t$, the observability inequality becomes 
\begin{align}\label{obs_spectr}
	\sum_{k\geq 1} |\varphi_k|^2e^{-2\lambda_k T} \leq C\int_0^T\left|\,\sum_{k\geq 1} \varphi_kg_k(t)e^{-\lambda_k t}\right|^2\,dt, 
\end{align}
where $g_k = \langle g\mathbf{1}_{\omega},\varrho_k\rangle$. 
	
By means of the classical moment method (\cite{fattorini1971exact}), inequalities of the form \eqref{obs_spectr} are well known to be true if and only if \eqref{eigen_cond} holds and the eigenfunctions $\varrho_k$ satisfy the following lower bound:
\begin{align}\label{eigenf_bound}
	\norm{\varrho_k}{L^2(\omega)}\geq C>0, \,\,\forall \, k\geq 1,
\end{align}
where the constant $C$ is independent of $k$. The proof of \eqref{eigenf_bound} is an easy adaptation of the one of \cite[Lemma 2]{kwasnicki2012eigenvalues}. Moreover, according to \cite{kulczycki2010spectral,kwasnicki2012eigenvalues} we have 
\begin{align*}
	\lambda_k = \left(\frac{k\pi}{2}-\frac{(1-s)\pi}{4}\right)^{2s}+O\left(\frac{1}{k}\right).
\end{align*}
	
Therefore, we easily see that the condition \eqref{eigen_cond} is satisfied if and only if $s>1/2$. If $s\leq 1/2$, instead, the series diverges, since it behaves as an harmonic series. In conclusion, the observability inequality \eqref{obs} holds true when $s>1/2$, but it is false when $s\leq 1/2$. This concludes the proof. 
\end{proof}

Even if for $s\leq 1/2$ null controllability for \eqref{heat_frac} fails, Theorem \ref{approx_control_thm} ensures that, for all $s\in(0,1)$, we still have approximate controllability. This is consequence of a unique continuation property for the fractional Laplacian, which has been obtained in \cite{fall2014unique}.

\begin{proof}[Proof of Theorem \ref{approx_control_thm}]
It is classical that the result is true as soon as one has the following unique continuation property for the solution to the adjoint equation \eqref{heat_frac_adj} (see, e.g., \cite[Theorem 5.2]{micu2004introduction}).
	
\MyQuote{Given $s\in(0,1)$ and $\varphi^T_0\in L^2(-1,1)$, let $\varphi$ be the unique solution to the system \eqref{heat_frac_adj}. Let $\omega\subset (-1,1)$ be an arbitrary open set. If $\varphi = 0$ on $\omega\times(0,T)$, then $\varphi = 0$ on $(-1,1)\times(0,T)$.}
	
Therefore, we are reduced to the proof of the property ($\mathcal P$). To this end, let us recall that $\varphi$ can be expressed in the form \eqref{adj_sol} and let us assume that 
\begin{align}\label{uc}
	\varphi=0 \textrm{ in } \omega\times(0,T). 
\end{align}
Let $\{\psi_{k_j}\}_{1\leq k\leq m_k}$ be an orthonormal basis of $\ker(\lambda_k-\fl{s}{})$. Then, \eqref{adj_sol} can be rewritten as
\begin{align*}
	\varphi(x,t) = \sum_{k\geq 1} \left(\sum_{j=1}^{m_k} \varphi_{k_j}\psi_{k_j}(x)\right)e^{-\lambda_k(T-t)}, \;\;\; (x,t)\in (-1,1)\times(-\infty, T). 
\end{align*}

Let $z\in\CC$ with $\eta:=\Re(z)>0$ and let $N\in\NN$. Since the functions $\psi_{k_j}$, $1\leq j\leq m_k$, $1\leq k\leq N$ are orthonormal, we have that
\begin{align*}
	\norm{\sum_{k=1}^N \left(\sum_{j=1}^{m_k} \varphi_{k_j}\psi_{k_j}(x)\right)e^{z(t-T)}e^{-\lambda_k(T-t)}}{L^2(-1,1)}^2 & \leq \sum_{k=1}^N \left(\sum_{j=1}^{m_k} |\varphi_{k_j}|^2\right)e^{2\eta(t-T)}e^{-2\lambda_k(T-t)}
	\\
	& \leq \sum_{k\geq 1} \left(\sum_{j=1}^{m_k} |\varphi_{k_j}|^2\right)e^{2\eta(t-T)}e^{-2\lambda_k(T-t)} \leq Ce^{2\eta(t-T)}\norm{\varphi^T}{L^2(-1,1)}^2.
\end{align*}
Hence, letting 
\begin{align*}
	w_N(x,t):= \sum_{k=1}^N \left(\sum_{j=1}^{m_k} \varphi_{k_j}\psi_{k_j}(x)\right)e^{z(t-T)}e^{-\lambda_k(T-t)},
\end{align*}
we have shown that $\norm{w_T(x,t)}{L^2(-1,1)}\leq Ce^{\eta(t-T)}\norm{\varphi^T}{L^2(-1,1)}$. Moreover, we have
\begin{align*}
	\int_{-\infty}^T e^{\eta(t-T)}\norm{\varphi^T}{L^2(-1,1)}\,dt = \frac{1}{\eta}\norm{\varphi^T}{L^2(-1,1)}\int_0^{+\infty} e^{-\tau}\,d\tau = \frac{1}{\eta}\norm{\varphi^T}{L^2(-1,1)}.
\end{align*}
Therefore, we can apply the Dominated Convergence Theorem, obtaining
\begin{align}\label{dct_identity}
	\lim_{N\to+\infty}\int_{-\infty}^T w_N(x,t)\,dt &= \int_{-\infty}^T\lim_{N\to+\infty} w_N(x,t)\,dt = \int_{-\infty}^T e^{z(t-T)} \sum_{k=1}^{+\infty} \left(\sum_{j=1}^{m_k} \varphi_{k_j}\psi_{k_j}(x)\right)e^{-\lambda_k(T-t)}\,dt \notag
	\\
	&=\sum_{k=1}^{+\infty}\sum_{j=1}^{m_k} \varphi_{k_j}\psi_{k_j}(x) \int_{-\infty}^T e^{z(t-T)}e^{-\lambda_k(T-t)}\,dt =\sum_{k=1}^{+\infty}\sum_{j=1}^{m_k} \varphi_{k_j}\psi_{k_j}(x) \int_0^{+\infty} e^{-(z+\lambda_k)\tau}\,d\tau \notag
	\\
	&=\sum_{k=1}^{+\infty}\sum_{j=1}^{m_k} \frac{\varphi_{k_j}}{z+\lambda_k}\psi_{k_j}(x), \;\;\; x\in (-1,1)\;\Re(z)>0.
\end{align} 
It follows from \eqref{uc} and \eqref{dct_identity} that 
\begin{align*}
	\sum_{k=1}^{+\infty}\sum_{j=1}^{m_k} \frac{\varphi_{k_j}}{z+\lambda_k}\psi_{k_j}(x)=0, \;\;\; x\in\omega,\;\Re(z)>0.
\end{align*}

This holds for every $z\in\CC\setminus\{-\lambda_k\}_{k\in\NN}$, using the analytic continuation in $z$. Hence, taking a suitable small circle around $-\lambda_{\ell}$ not including $\{-\lambda_k\}_{k\neq\ell}$ and integrating on that circle we get that
\begin{align*}
	w_\ell:=\sum_{j=1}^{m_{\ell}} \varphi_{\ell_j}\psi_{\ell_j}(x)=0, \;\;\; x\in\omega.
\end{align*}
	
According to \cite[Theorem 1.4]{fall2014unique}, $\fl{s}{}$ has the unique continuation property in the sense that if $\lambda_k$ is an eigenvalue of $\fl{s}{}$ on (-1,1) with Dirichlet boundary conditions, and $(\fl{s}{}-\lambda_k)\varrho_k = 0$ in $(-1,1)$ with $\varrho_k = 0$ in $\omega$, then $\varrho_k = 0$ in $(-1,1)$. 
This can applied to $w_{\ell}$, in order to conclude $w_{\ell} = 0$ in $(-1,1)$ for every $\ell$. Since $\{\psi_{\ell_j}\}_{1\leq j\leq m_{\ell}}$ are linearly independent in $L^2(-1,1)$, we get $\varphi_{\ell_j} = 0$, $1\leq j\leq m_k$, $\ell\in\NN$. It follows that $\varphi^T=0$ and hence, $\varphi=0$ in $(-1,1)\times(0,T)$, meaning that $\varphi$ enjoys the property $(\mathcal{P})$. As an immediate consequence, we have that our original equation \eqref{heat_frac} is approximately controllable. Our proof is then concluded. 
\end{proof}

\begin{remark}
According to \cite{fall2014unique}, the elliptic unique continuation property for the fractional Laplacian holds in any space dimension. In view of that, Theorem \ref{approx_control_thm} may be extended also to the case $N>1$. On the other hand, the same does not applies to Theorem \eqref{null_control_thm}. Indeed, the proof of this result uses arguments that are designed specifically for one-dimensional problems (\cite{fattorini1971exact}). If one would analyze the null-controllability in a general multi-dimensional setting, other tools (such as a Carleman estimate) are needed. As far as we know, these techniques have not been fully developed yet for problems involving the fractional Laplacian on a domain. 
\end{remark}

\begin{remark}
For the sake of simplicity, in the results presented above we focused on the interval $(-1,1)$. Nevertheless, everything that we did in this section actually holds in the more general case $x\in(-L,L)$ and the extension is immediate.
\end{remark}
%Therefore, given any initial datum $z_0\in L^2(-1,1)$ we are interested in computing numerically the control function $g$ that drives the solution $z$ to zero in time $T$. Before describing the methodology that we shall adopt, we recall the existing theoretical results on the controllability of the fractional heat equation \eqref{heat_frac}. This will give us a hint about what we should expect from our simulations, and will provide a validation of the accuracy of our numerical method.




\section{Development of the numerical scheme}\label{fe_sec}

We devote this Section to the description of the numerical scheme that we are going to employ. Let us start with the elliptic case. 
%
\subsection{Finite element approximation of the elliptic problem}\label{fe_ell_sec}
%
In order to solve numerically \eqref{PE}, we will develop a finite element scheme on a uniform mesh. To this purpose, let us firstly recall the variational formulation associated to our problem: find $u\in H_0^s(-L,L)$ such that
\begin{align}\label{WF}
	\frac{\ccs}{2} \int_{\RR}\int_{\RR}\frac{(u(x)-u(y))(v(x)-v(y))}{|x-y|^{1+2s}}\,dxdy = \int_{-L}^L fv\,dx,	
\end{align}
for all $v\in H_0^s(-L,L)$. 

Let us introduce a partition of the interval $(-L,L)$ as follows:
\begin{align*}
	-L = x_0<x_1<\ldots <x_i<x_{i+1}<\ldots<x_{N+1}=L\,,
\end{align*}
with $x_{i+1}=x_i+h$, $i=0,\ldots N$. We call $\mathfrak{M}$ the mesh composed by the points $\{x_i\,:\, i=1,\ldots,N\}$, while the set of the boundary points is denoted $\partial\mathfrak{M}:=\{x_0,x_{N+1}\}$. Now, define $K_i:=[x_i,x_{i+1}]$ and consider the discrete space 
\begin{align*}
	V_h :=\Big\{v\in H_0^s(-L,L)\,\big|\, \left. v\,\right|_{K_i}\in \mathcal{P}^1\Big\},
\end{align*} 
where $\mathcal{P}^1$ is the space of the continuous and piece-wise linear functions. Hence, we approximate \eqref{WF} with the following discrete problem: find $u_h\in V_h$ such that
\begin{align*}
	\frac{\ccs}{2} \int_{\RR}\int_{\RR}\frac{(u_h(x)-u_h(y))(v_h(x)-v_h(y))}{|x-y|^{1+2s}}\,dxdy = \int_{-L}^L fv_h\,dx,	
\end{align*}
for all $v_h\in V_h$. If now we indicate with $\big\{\phi_i\big\}_{i=1}^N$ a basis of $V_h$, it will be sufficient that \eqref{WFD} is satisfied for all the functions of the basis, since any element of $V_h$ is a linear combination of them. Therefore the problem takes the following form
\begin{align}\label{WFD}
	\frac{\ccs}{2} \int_{\RR}\int_{\RR}\frac{(u_h(x)-u_h(y))(\phi_i(x)-\phi_i(y))}{|x-y|^{1+2s}}\,dxdy = \int_{-L}^L fv_h\,dx,\;\;\; i=1,\ldots,N.	
\end{align}
Clearly, since $u_h\in V_h$, we have 
\begin{align*}
	u_h(x) = \sum_{j=1}^N u_j\phi_j(x),
\end{align*} 
where the coefficients $u_j$ are, a priori, unknown. In this way, \eqref{WFD} is reduced to solve the linear system $\mathcal A_h u=F$, where the stiffness matrix $\mathcal A_h\in \RR^{N\times N}$ has components
\begin{align}\label{stiffness_nc}
	a_{i,j}=\frac{\ccs}{2} \int_{\RR}\int_{\RR}\frac{(\phi_i(x)-\phi_i(y))(\phi_j(x)-\phi_j(y))}{|x-y|^{1+2s}}\,dxdy,	
\end{align}
while the vector $F\in\RR^N$ is given by $F=(F_1,\ldots,F_N)$ with
\begin{align*}
	F_i = \langle f,\phi_i\rangle = \int_{-L}^L f\phi_i\,dx,\;\;\; i=1,\ldots,N.
\end{align*}

Moreover, the basis $\big\{\phi_i\big\}_{i=1}^N$ that we will employ is the classical one in which each $\phi_i$ is the tent function with $supp(\phi_i)=(x_{i-1},x_{i+1})$ and verifying $\phi_i(x_j)=\delta_{i,j}$. In particular, for $x\in\{x_{i-1},x_i,x_{i+1}\}$ the $i^{th}$ function of the basis is explicitly defined as (see Fig. \ref{basis}) 
\begin{align}\label{basis_fun}
	\phi_i(x)= 1-\frac{|x-x_i|}{h}.
\end{align} 

\begin{figure}[h]
\figinit{0.7pt}
% Axes
\figpt 1:(-100,0) \figpt 2:(200,0)
\figpt 11:(-90,-10) \figpt 12:(-90,140)
%%%%
\figpt 3:(-50,0) \figpt 4:(0,100) 
\figpt 5:(50,0) \figpt 6:(0,0)
%
% Points for writing
\figpt 7:(-50,-10) \figpt 8:(0,110) 
\figpt 9:(50,-10) \figpt 10:(0,-10)

\figpt 13:(-100,130) \figpt 14:(180,-10)
\figpt 15:(50,50)

% 2. Creation of the graphical file
\figdrawbegin{}
\figdrawarrow[1,2]
\figdrawline[3,4]
\figdrawline[4,5]
\figset(dash=4)
\figdrawline[4,6]
\figdrawarrow[11,12]

\figdrawend

% 3. Writing text on the figure
\figvisu{\figBoxA}{}{
\figwritec [7]{$(x_{i-1},0)$}
\figwritec [8]{$(x_i,1)$}
\figwritec [9]{$(x_{i+1},0)$}
\figwritec [10]{$(x_i,0)$}
\figwritec [13]{$y$}
\figwritec [14]{$x$}
\figwritec [15]{$\phi_i(x)$}
}
\centerline{\box\figBoxA}
\caption{Basis function $\phi_i(x)$ on its support $(x_{i-1},x_{i+1})$.}\label{basis}
\end{figure}

%\begin{figure}[h]
%	\centering
%	\includegraphics[scale=1]{figure8.eps}	
%	\caption{Basis function $\phi_i(x)$ on its support $(x_{i-1},x_{i+1})$.}\label{basis}
%\end{figure}

\noindent Let us now describe our algorithm. Before that, we shall make the following preliminary comments.
\begin{remark}\label{rem_prel}
The following fact are worth noticing.
\begin{enumerate}
	\item It is evident from the definition \eqref{stiffness_nc} that $\mathcal A_h$ is symmetric. Therefore, in our algorithm we will only need to compute the values $a_{i,j}$ with $j\geq i$.
	
	\item Due to the non-local nature of the problem, the matrix $\mathcal A_h$ is full. However, while computing its components, we will encounter many simplifications, due to the fact that $supp(\phi_i)\cap supp(\phi_j) =\emptyset$ for $j\geq i+2$.
	
	\item While computing the values $a_{i,j}$, we will only work on the mesh $\mathfrak{M}$, not considering the points of the set $\partial\mathfrak{M}$. In this way, we will ensure that the basis functions $\phi_i$ satisfy the zero Dirichlet boundary conditions. In other words, in our FE approximation we are considering only the functions from $\phi_1$ to $\phi_N$. Instead, if we considered the points $x_0$ and $x_{N+1}$, then we would need to introduce in our discretization also the basis functions $\phi_0$ and $\phi_{N+1}$, which take value one at the boundary, and this would not be consistent with the continuous problem. Fig. \ref{basis_int} provides a graphical explanation of this last discussion.      	
\end{enumerate}
\end{remark}

\begin{figure}[h]
\figinit{0.7pt}
% Axes
\figpt 1:(-110,0) \figpt 2:(420,0)
\figpt 11:(-100,-10) \figpt 12:(-100,140)
%%%%
\figpt 3:(-75,0) \figpt 4:(-25,100) 
\figpt 5:(25,0) \figpt 6:(-25,0)

\figpt 31:(-25,0) \figpt 41:(25,100) 
\figpt 51:(75,0) \figpt 61:(25,0)

\figpt 32:(25,0) \figpt 42:(75,100) 
\figpt 52:(125,0) \figpt 62:(75,0)

\figpt 33:(265,0) \figpt 43:(315,100) 
\figpt 53:(365,0) \figpt 63:(315,0)

\figpt 200:(-100,100)
%
% Points for writing
\figpt 7:(-75,-10) \figpt 8:(-25,110) 
\figpt 9:(25,-10) \figpt 10:(-25,-10)

\figpt 71:(-25,-10) \figpt 81:(25,110) 
\figpt 91:(75,-10) \figpt 101:(25,-10)

\figpt 72:(25,-10) \figpt 82:(75,110) 
\figpt 92:(125,-10) \figpt 102:(75,-10)

\figpt 73:(265,-10) \figpt 83:(315,110) 
\figpt 93:(365,-10) \figpt 103:(315,-10)

\figpt 13:(-110,130) \figpt 14:(405,-10)
\figpt 15:(50,50) \figpt 16:(250,50)

\figpt 201:(-110,100) \figpt 202:(200,50)

% 2. Creation of the graphical file
\figdrawbegin{}
\figdrawarrow[1,2]
\figdrawline[3,4]
\figdrawline[4,5]
\figset (color=\Redrgb)
\figdrawline[31,41]
\figdrawline[41,51]
\figset (color=default)
\figset (color=\ForestGreenrgb)
\figdrawline[32,42]
\figdrawline[42,52]
\figset (color=default)
\figset (color=\Bluergb)
\figdrawline[33,43]
\figdrawline[43,53]
\figset (color=default)
\figset(dash=4)
\figdrawline[4,6]
\figdrawline[200,4]
\figdrawline[41,61]
\figdrawline[42,62]
\figdrawline[43,63]
\figdrawarrow[11,12]

\figdrawend

% 3. Writing text on the figure
\figvisu{\figBoxA}{}{
\figwritec [7]{$x_0=-L$}
\figwritec [8]{$\phi_1$}
\figwritec [9]{$x_2$}
\figwritec [10]{$x_1$}
\figwritec [81]{$\phi_2$}
\figwritec [91]{$x_3$}
\figwritec [82]{$\phi_3$}
\figwritec [92]{$x_4$}
\figwritec [73]{$x_{N-1}$}
\figwritec [83]{$\phi_N$}
\figwritec [93]{$x_{N+1}=L$}
\figwritec [103]{$x_N$}
\figwritec [13]{$y$}
\figwritec [14]{$x$}
\figwritec [201]{$1$}
\figwritec [202]{$\ldots\ldots\ldots$}
\figwritec[3,53]{$\bullet$}
%\figwritec [7,93]
%\figwritec [15]{$\phi_i(x)$}
%\figwritec [16]{$\color{red}\phi_j(x)$}
}
\centerline{\box\figBoxA}
\caption{Basis functions $\phi_i(x)$ on the whole interval $(-L,L)$.}\label{basis_int}
\end{figure}

%\begin{figure}[h]
%	\centering
%	\includegraphics[scale=1]{figure9.eps}
%	\caption{Basis functions $\phi_i(x)$ on the whole interval $(-L,L)$.}\label{basis_int}
%\end{figure}

We now start building the stiffness matrix $\mathcal A_h$. This will be done it in three steps, since the values of the matrix can be computed differentiating among three well defined regions: the upper triangle, corresponding to $j\geq i+2$, the upper diagonal corresponding to $j=i+1$ and the diagonal, corresponding to $j=i$ (see Fig. \ref{matrix_fig}). In fact, as it will be clear during our computations, in each of these regions the intersections among the support of the basis functions are different, thus generating different values of the bilinear form. In what follows, we will briefly present which will be the contributions to the matrix in each of these three steps, including the complete computations as an appendix at the end of the paper. 

\begin{figure}[!h]

\centering
\begin{tikzpicture}
 \matrix[matrix of math nodes,left delimiter = (,right delimiter = ),row sep=10pt,column sep = 10pt] (m)
 {
 a_{1,1}  & a_{1,2} & a_{1,3} & \ldots\ldots & \ldots\ldots & a_{1,N}\\
   &a_{2,2} & a_{2,3} & a_{2,4} & \ldots\ldots &a_{2,N}\\
   & & \ddots & \ddots & \ddots &\vdots\\
   & & & \ddots & \ddots &a_{N-2,N}\\	   
   & & & & \hspace{-0.25cm}a_{N-1,N-1} &a_{N-1,N}\\
   & & & & & a_{N,N}\\
 };
\draw[color=red] (m-1-3.north west) -- (m-1-6.north east) -- +(0.14,0) -- (m-4-6.south east) -- (m-4-6.south west) -- (m-1-3.west) -- (m-1-3.north west);
\draw[color=blue] (m-1-2.north west) -- +(0.6,0) -- +(5.37,-3.96) -- (m-5-6.north east) -- (m-5-6.south east) -- (m-5-6.south west) -- (m-1-2.south) -- (m-1-2.south west) -- (m-1-2.north west); 
\draw[color=black] (m-1-1.north west) -- (m-1-1.north east) -- +(0.2,0) -- +(0.2,-0.75) -- +(0.9,-0.75) -- +(5.75,-4.8) -- +(6.55,-4.8) -- +(6.55,-5.3) -- +(4.65,-5.3) -- (m-1-1.south west) -- (m-1-1.north west); 
\end{tikzpicture}
\caption{Structure of the stiffness matrix $\mathcal{A}_h$.}\label{matrix_fig}
\end{figure}

%\begin{figure}[!h]
%
%\centering
%\includegraphics[scale=1]{figure1.eps}
%\caption{Structure of the stiffness matrix $\mathcal{A}_h$.}\label{matrix_fig}
%\end{figure}

\subsubsection*{Step 1: $j\geq i+2$}
As we mentioned in Remark \ref{rem_prel}, in this case we have $supp(\phi_i)\cap supp(\phi_j) =\emptyset$ (see also Fig. \ref{basis_upp_tri}). Hence, \eqref{stiffness_nc} is reduced to computing only the integral
\begin{align}\label{elem_noint}
	a_{i,j}=-2 \int_{x_{j-1}}^{x_{j+1}}\int_{x_{i-1}}^{x_{i+1}}\frac{\phi_i(x)\phi_j(y)}{|x-y|^{1+2s}}\,dxdy.
\end{align}

\begin{figure}[h]
\figinit{0.7pt}
% Axes
\figpt 1:(-100,0) \figpt 2:(320,0)
\figpt 11:(-90,-10) \figpt 12:(-90,140)
%%%%
\figpt 3:(-50,0) \figpt 4:(0,100) 
\figpt 5:(50,0) \figpt 6:(0,0)

\figpt 31:(150,0) \figpt 41:(200,100) 
\figpt 51:(250,0) \figpt 61:(200,0)
%
% Points for writing
\figpt 7:(-50,-10) \figpt 8:(0,110) 
\figpt 9:(50,-10) \figpt 10:(0,-10)

\figpt 71:(150,-10) \figpt 81:(200,110) 
\figpt 91:(250,-10) \figpt 101:(200,-10)

\figpt 13:(-100,130) \figpt 14:(300,-10)
\figpt 15:(50,50) \figpt 16:(250,50)

% 2. Creation of the graphical file
\figdrawbegin{}
\figdrawarrow[1,2]
\figdrawline[3,4]
\figdrawline[4,5]
\figset (color=\Redrgb)
\figdrawline[31,41]
\figdrawline[41,51]
\figset (color=default)
\figset(dash=4)
\figdrawline[4,6]
\figdrawline[41,61]
\figdrawarrow[11,12]

\figdrawend

% 3. Writing text on the figure
\figvisu{\figBoxA}{}{
\figwritec [7]{$(x_{i-1},0)$}
\figwritec [8]{$(x_i,1)$}
\figwritec [9]{$(x_{i+1},0)$}
\figwritec [10]{$(x_i,0)$}
\figwritec [71]{$(x_{j-1},0)$}
\figwritec [81]{$(x_j,1)$}
\figwritec [91]{$(x_{j+1},0)$}
\figwritec [101]{$(x_j,0)$}
\figwritec [13]{$y$}
\figwritec [14]{$x$}
\figwritec [15]{$\phi_i(x)$}
\figwritec [16]{$\color{red}\phi_j(x)$}
}
\centerline{\box\figBoxA}
\caption{Basis functions $\phi_i(x)$ and $\phi_j(x)$ for $j\geq i+1$. In this case, the supports are disjoint.}\label{basis_upp_tri}
\end{figure}

%\begin{figure}[h]
%	\centering
%	\includegraphics[scale=1]{figure10.eps}
%	\caption{Basis functions $\phi_i(x)$ and $\phi_j(x)$ for $j\geq i+1$. In this case, the supports are disjoint.}\label{basis_upp_tri}
%\end{figure}

Taking into account the definition of the basis function \eqref{basis_fun}, from \eqref{elem_noint} we obtain
\begin{align*}
	a_{i,j}=-2 \int_{x_{j-1}}^{x_{j+1}}\int_{x_{i-1}}^{x_{i+1}}\frac{\left(1-\frac{|x-x_i|}{h}\right)\left(1-\frac{|y-x_j|}{h}\right)}{|x-y|^{1+2s}}\,dxdy.
\end{align*}

Finally, this last integral can be computed explicitly employing the following  change of variables:
\begin{align}\label{cv}
	\frac{x-x_i}{h}=\hat{x},\;\;\; \frac{y-x_i}{h}=\hat{y}.
\end{align}

In this way, for the elements $a_{i,j}$, $j\geq i+2$, we get the following values: 
\begin{align*}
	a_{i,j} = \begin{cases}
			\displaystyle -h^{1-2s}\,\frac{4(k+1)^{3-2s} + 4(k-1)^{3-2s}-6k^{3-2s}-(k+2)^{3-2s}-(k-2)^{3-2s}}{2s(1-2s)(1-s)(3-2s)}, & k=j-i,\;\; \displaystyle s\neq\frac{1}{2} 
			\\
			\\
			-4(j-i+1)^2\log(j-i+1)-4(j-i-1)^2\log(j-i-1) &  \displaystyle s=\frac{1}{2},\;\; j> i+2
			\\
			\;\;\;+6(j-i)^2\log(j-i)+(j-i+2)^2\log(j-i+2)+(j-i-2)^2\log(j-i-2), 
			\\
			\\
			56\ln(2)-36\ln(3), & \displaystyle s=\frac{1}{2},\;\; j= i+2.
		\end{cases}	
\end{align*}

\subsubsection*{Step 2: $j= i+1$}
This is the most cumbersome case, since it is the one with the most interactions between the basis functions (see Fig. \ref{basis_upp_dia}). 

\begin{figure}[h]
\figinit{0.7pt}
% Axes
\figpt 1:(-100,0) \figpt 2:(200,0)
\figpt 11:(-90,-10) \figpt 12:(-90,140)
%%%%
\figpt 3:(-50,0) \figpt 4:(0,100) 
\figpt 5:(50,0) \figpt 6:(0,0)

\figpt 31:(0,0) \figpt 41:(50,100) 
\figpt 51:(100,0) \figpt 61:(50,0)
%
% Points for writing
\figpt 7:(-50,-10) \figpt 8:(0,110) 
\figpt 9:(50,-10) \figpt 10:(0,-10)

\figpt 71:(0,-25) \figpt 81:(50,110) 
\figpt 91:(100,-25) \figpt 101:(50,-25)

\figpt 13:(-100,130) \figpt 14:(180,-10)
\figpt 15:(-50,50) \figpt 16:(100,50)

% 2. Creation of the graphical file
\figdrawbegin{}
\figdrawarrow[1,2]
\figdrawline[3,4]
\figdrawline[4,5]
\figset (color=\Redrgb)
\figdrawline[31,41]
\figdrawline[41,51]
\figset (color=default)
\figset(dash=4)
\figdrawline[4,6]
\figdrawline[41,61]
\figdrawarrow[11,12]

\figdrawend

% 3. Writing text on the figure
\figvisu{\figBoxA}{}{
\figwritec [7]{$(x_{i-1},0)$}
\figwritec [8]{$(x_i,1)$}
\figwritec [9]{$(x_{i+1},0)$}
\figwritec [10]{$(x_i,0)$}
\figwritec [71]{$(x_{j-1},0)$}
\figwritec [81]{$(x_j,1)$}
\figwritec [91]{$(x_{j+1},0)$}
\figwritec [101]{$(x_j,0)$}
\figwritec [13]{$y$}
\figwritec [14]{$x$}
\figwritec [15]{$\phi_i(x)$}
\figwritec [16]{$\color{red}\phi_j(x)$}
}
\centerline{\box\figBoxA}
\caption{Basis functions $\phi_i(x)$ and $\phi_{i+1}(x)$. In this case, the intersection of the supports is the interval $[x_i,x_{i+1}]$.}\label{basis_upp_dia}
\end{figure}

%\begin{figure}[h]
%	\centering
%	\includegraphics[scale=1]{figure11.eps}
%	\caption{Basis functions $\phi_i(x)$ and $\phi_{i+1}(x)$. In this case, the intersection of the supports is the interval $[x_i,x_{i+1}]$.}\label{basis_upp_dia}
%\end{figure}

According to \eqref{stiffness_nc}, and using the symmetry of the integral with respect to the bisector $y=x$, we have 
	\begin{align*}
	a_{i,i+1}= & \int_{\RR}\int_{\RR}\frac{(\phi_i(x)-\phi_i(y))(\phi_{i+1}(x)-\phi_{i+1}(y))}{|x-y|^{1+2s}}\,dxdy
	\\
	= & \int_{x_{i+1}}^{+\infty}\int_{x_{i+1}}^{+\infty} \ldots\,dxdy + 2\int_{x_{i+1}}^{+\infty}\int_{x_i}^{x_{i+1}} \ldots\,dxdy + 2\int_{x_{i+1}}^{+\infty}\int_{-\infty}^{x_i} \ldots\,dxdy 
	\\
	& + \int_{x_i}^{x_{i+1}}\int_{x_i}^{x_{i+1}} \ldots\,dxdy + 2\int_{x_i}^{x_{i+1}}\int_{-\infty}^{x_i} \ldots\,dxdy + \int_{-\infty}^{x_i}\int_{-\infty}^{x_i} \ldots\,dxdy 
	\\
	:= & Q_1 + Q_2 + Q_3 + Q_4 + Q_5 + Q_6.
\end{align*}

These contributions will be calculated separately, employing changes of variables analogous to \eqref{cv}. After several computations, we obtain
\begin{align*}
	a_{i,i+1} = \begin{cases}
					\displaystyle h^{1-2s}\frac{3^{3-2s}-2^{5-2s}+7}{2s(1-2s)(1-s)(3-2s)}, & \displaystyle s\neq \frac{1}{2}
					\\
					9\ln 3-16\ln 2, & \displaystyle s=\frac{1}{2}.
				\end{cases}	
\end{align*}

\subsubsection*{Step 3: $j= i$}
As a last step, we fill the diagonal of the matrix $\mathcal A_h$, which collects the values corresponding to the case $\phi_i(x)=\phi_j(x)$ (see Fig. \ref{basis_dia}). We have
	\begin{align*}
	a_{i,i}= & \int_{\RR}\int_{\RR}\frac{(\phi_i(x)-\phi_i(y))^2}{|x-y|^{1+2s}}\,dxdy
	\\
	= & \int_{x_{i+1}}^{+\infty}\int_{x_{i+1}}^{+\infty} \ldots\,dxdy + 2\int_{x_{i+1}}^{+\infty}\int_{x_{i-1}}^{x_{i+1}} \ldots\,dxdy + \int_{x_{i+1}}^{+\infty}\int_{-\infty}^{x_{i-1}} \ldots\,dxdy 
	\\
	& + \int_{x_{i-1}}^{x_{i+1}}\int_{x_{i-1}}^{x_{i+1}} \ldots\,dxdy + 2\int_{-\infty}^{x_{i-1}}\int_{x_{i-1}}^{x_{i+1}} \ldots\,dxdy + + \int_{-\infty}^{x_{i-1}}\int_{x{i+1}}^{+\infty} \ldots\,dxdy 
	\\
	& +  \int_{-\infty}^{x_{i-1}}\int_{-\infty}^{x_{i-1}} \ldots\,dxdy := R_1 + R_2 + R_3 + R_4 + R_5 + R_6 + R_7.
\end{align*}

\begin{figure}[h]
\figinit{0.7pt}
% Axes
\figpt 1:(-100,0) \figpt 2:(200,0)
\figpt 11:(-90,-10) \figpt 12:(-90,140)
%%%%
\figpt 3:(-50,0) \figpt 4:(0,100) 
\figpt 5:(50,0) \figpt 6:(0,0)

\figpt 31:(-40,20) \figpt 41:(-30,40) 
\figpt 51:(-20,60) \figpt 61:(-10,80)

\figpt 32:(40,20) \figpt 42:(30,40) 
\figpt 52:(20,60) \figpt 62:(10,80)
%
% Points for writing
\figpt 7:(-50,-10) \figpt 8:(0,110) 
\figpt 9:(50,-10) \figpt 10:(0,-10)

\figpt 71:(-50,-40) \figpt 81:(50,110) 
\figpt 91:(50,-40) \figpt 101:(0,-40)

\figpt 72:(-50,-25) \figpt 82:(50,110) 
\figpt 92:(50,-25) \figpt 102:(0,-25)

\figpt 13:(-100,130) \figpt 14:(180,-10)
\figpt 15:(-50,50) \figpt 16:(80,50)

% 2. Creation of the graphical file
\figdrawbegin{}
\figdrawarrow[1,2]
\figset (color=\Redrgb)
\figdrawline[3,4]
\figdrawline[4,5]
\figset (width=default,color=default)
%\figdrawline[3,4]
%\figdrawline[4,5]
%\figset (color=default)
\figset(dash=4)
\figdrawline[4,6]
\figdrawline[41,61]
\figdrawarrow[11,12]

\figdrawend

% 3. Writing text on the figure
\figvisu{\figBoxA}{}{
\figwritec [7]{$(x_{i-1},0)$}
\figwritec [8]{$(x_i,1)=(x_j,1)$}
\figwritec [9]{$(x_{i+1},0)$}
\figwritec [10]{$(x_i,0)$}
\figwritec [71]{$(x_{j-1},0)$}
\figwritec [91]{$(x_{j+1},0)$}
\figwritec [101]{$(x_j,0)$}
\figwritec [72,92,102]{$\shortparallel$}
\figwritec [13]{$y$}
\figwritec [14]{$x$}
\figwritec [16]{$\phi_i(x)=\color{red}\phi_j(x)$}
\figset write(mark=$\figBullet$)
\figwritep[3,31,41,51,61,4,32,42,52,62,5]
}
\centerline{\box\figBoxA}
\caption{Basis functions $\phi_i(x)$ and $\phi_j(x)$. In this case, the two functions coincide.}\label{basis_dia}
\end{figure}

%\begin{figure}[h]
%	\centering
%	\includegraphics[scale=1]{figure12.eps}	
%	\caption{Basis functions $\phi_i(x)$ and $\phi_j(x)$. In this case, the two functions coincide.}\label{basis_dia}
%\end{figure}

Once again, the terms $R_i$, $i=1,\ldots,7$ will be computed separately, obtaining
\begin{align*}
	a_{i,i} = \begin{cases}
			\displaystyle h^{1-2s}\,\frac{2^{3-2s}-4}{s(1-2s)(1-s)(3-2s)}, & \displaystyle s\neq\frac{1}{2}
			\\
			\\
			8\ln 2, & \displaystyle s=\frac{1}{2}.			
			\end{cases}	
\end{align*}

\subsubsection*{Conclusion}
Summarizing, we have the following values for the elements of the stiffness matrix $\mathcal{A}_h$: for $s\neq 1/2$
\begin{align*}
	a_{i,j} =  -h^{1-2s} \begin{cases}
			\displaystyle \,\frac{4(k+1)^{3-2s} + 4(k-1)^{3-2s}-6k^{3-2s}-(k+2)^{3-2s}-(k-2)^{3-2s}}{2s(1-2s)(1-s)(3-2s)}, &  \displaystyle k=j-i,\,k\geq 2
			\\
			\\
			\displaystyle\frac{3^{3-2s}-2^{5-2s}+7}{2s(1-2s)(1-s)(3-2s)}, & \displaystyle j=i+1
			\\
			\\
			\displaystyle\frac{2^{3-2s}-4}{s(1-2s)(1-s)(3-2s)}, & \displaystyle j=i.
		\end{cases}	
\end{align*}

For $s=1/2$, instead, we have

\begin{align*}
	a_{i,j} = \begin{cases}
			-4(j-i+1)^2\log(j-i+1)-4(j-i-1)^2\log(j-i-1) 
			\\
			\;\;\;+6(j-i)^2\log(j-i)+(j-i+2)^2\log(j-i+2)+(j-i-2)^2\log(j-i-2), &  \displaystyle j> i+2
			\\
			\\
			56\ln(2)-36\ln(3), & \displaystyle j= i+2.
			\\
			\\
			\displaystyle 9\ln 3-16\ln 2, & \displaystyle j=i+1
			\\
			\\
			\displaystyle 8\ln 2, & \displaystyle j=i.
		\end{cases}	
\end{align*}

\begin{remark}
We point out the following facts:
\begin{enumerate}
	\item The matrix $\mathcal A_h$ has the structure of a $N$-diagonal matrix, meaning that value of its elements remain constant along its diagonals. This is in analogy with the tridiagonal matrix approximating the classical Laplace operator. Notice, however, that in our case we obtain a full matrix. This is consistent with the non-local nature of the operator that we are discretizing.
	
	\item The value of each element $a_{i,j}$ is given explicitly, and it only depends on $i$, $j$, $s$ and $h$. In other words, when approximating the left hand side of \eqref{WFD}, no numerical integration is needed. This significantly improve the efficiency of our method.
	
	\item For $s=1/2$, the elements $a_{i,j}$ do not depend on the value of $h$ which, in turn, is a function of $N$. This implies that, in this particular case, no matter how many points we consider in our mesh, the matrix $\mathcal A_h$ will always have the same entries. 
\end{enumerate}
\end{remark}

In Section \ref{res_numerical} below, we will present the numerical simulations associated to  the elliptic problem \eqref{PE}, discussing in detail the convergence properties of the algorithm. 

\subsection{Control problem for the fractional heat equation}%\label{fe_parabolic}

Let us now give a brief description of the so called penalised Hilbert Uniqueness Method (HUM in what follows) that we shall employ for computing the controls for equation \eqref{heat_frac}. Here, we will mostly refer to the work of Boyer \cite{boyer2013penalised}.

We start recalling the classical HUM, as it has been introduced in the pioneering works \cite{glowinski1995exact,glowinski2008exact}. Let $(E,\langle\cdot,\cdot\rangle)$ be a Hilbert space whose norm is denoted by $\norm{\cdot}{}$. Let $(A,\mathcal D(A))$ be an unbounded operator in $E$ such that $-A$ generates an analytic semi-group in $E$ that we indicate by $t\mapsto e^{-tA}$. Also, we denote $(A^*,\mathcal D(A^*))$ the adjoint of this operator and by $t\mapsto e^{-tA^*}$ the corresponding semi-group. 

Let $(U,[\cdot,\cdot])$ be another Hilbert space whose norm is denoted by $\inter{\cdot}$. Let  $B$ be an unbounded operator from $U$ to $\mathcal D(A^*)'$ and let $B^*:\mathcal D(A^*)\to U$ be its adjoint. Let $T>0$ be given and, for any $y_0\in E$ and $v\in L^2(0,T;U)$, let us consider the non-homogeneous evolution problem  
\begin{align}\label{abstract_pb}
	\begin{cases}
		y_t+Ay=Bv, & t\in[0,T]
		\\
		y(0)=y_0.
	\end{cases}
\end{align}

The well posedness of \eqref{abstract_pb} is guaranteed by \cite[Theorem 2.37]{coron2007control}. From now on, we will refer to the solution as $t\mapsto y_{v,y_0}(t)$.

Notice that we have $y_{0,y_0}(t)= e^{-tA}y_0$. Moreover, for simplicity, the solution at time $T$, which is of particular interest in what follows, will be denoted by 
\begin{align*}
	y_{v,y_0}(T)=\mathcal{L}_T(v|y_0).
\end{align*} 
The linear operator $\mathcal{L}_T(\cdot|\cdot)$ is then continuous from $L^2(0,T;U)\times E$ into $E$.

In the framework of both controllability notions that we introduced in Section \ref{theor_sec}, if one control exists it is certainly not unique. For instance, the classical HUM approach consists in finding the control with the minimal $L^2(0,T;U)$-norm. Nevertheless, even though in this way we can identify a precise control, its computation can be a difficult task due to the nature of the constraints involved (see, e.g., \cite{fernandez2000cost,micu2004introduction}). 

Because of what we just described, it is convenient to deal with a penalized version of the above mentioned optimization problems.

In the penalized version of the HUM, we look for a control that is solution to a different optimization problem. In particular, we for any $\varepsilon>0$, we shall find

\begin{align}\label{min_ve}
	v_\varepsilon=\min_{v\in L^2(0,T;U)} F_\varepsilon (v)
\end{align}
where
\begin{align*}
	F_\varepsilon(v):=\frac{1}{2}\int_0^T\inter{v(t)}^2\,dt+\frac{1}{2\varepsilon}\norm{\mathcal L(v|y_0)}{}^2, \quad \forall v\in L^2(0,T;U).
\end{align*}

Notice that, for any $\varepsilon > 0$, the functional $F_\varepsilon$ has a unique minimiser on $L^2(0,T;U)$  that we denote by $v_\varepsilon$. This is due to the fact that $F_\varepsilon$ is strictly convex, continuous and coercive. 

However, the space $L^2(0,T;U)$ in which one has to minimise $F_\varepsilon$ is a quite big one and it depends on the time $T$. This makes the minimization problem computationally expensive. On the other hand, this issue can be bypassed by considering a different optimization problem, defined on the smaller space $E$. Namely we have to find  

\begin{align}\label{min_je}
	q^T_\varepsilon=\min_{q\in E} J_\varepsilon (q^T)
\end{align}
where
\begin{align}\label{penalized_fun}
	J_\varepsilon(q^T):=\frac{1}{2}\int_0^T\inter{B^*e^{-(T-t)A^*}q^T}^2\,dt+\frac{\varepsilon}{2}\norm{q^T}{}^2 + \left\langle y_0,e^{-TA^*}q^T\right\rangle, \quad \forall q^T\in E.
\end{align}

Notice that \eqref{min_ve} and \eqref{min_je} are equivalent since, according to \cite[Proposition 1.5]{boyer2013penalised}, for any $\varepsilon > 0$, the minimisers $v_\varepsilon$ and $q_\varepsilon^T$ of the functionals $F_\varepsilon$ and $J_\varepsilon$, respectively, are related through the formula
\begin{align*}
	v_\varepsilon = B^*e^{-(T-t)A^*}q_\varepsilon^T, \textrm{ for a.e. } t\in(0,T).
\end{align*} 

Notice also that we can express the approximate and null controllability properties of the system, for a given initial datum $y_0$, in terms of the behaviour of the penalised HUM approach described above. In particular we have 

\begin{theorem}[Theorem 1.7 of \cite{boyer2013penalised}]\label{theorem_hum}
Problem \eqref{abstract_pb} is approximately controllable from the initial datum $y_0$ if and only if we have
\begin{align*}
	\mathcal{L}_T(v_\varepsilon|y_0) = y_{v_\varepsilon,y_0}(T)\rightarrow 0,\;\;\;\textrm{ as }\;\varepsilon\to 0.
\end{align*}
Problem \eqref{abstract_pb} is null-controllable from the initial datum $y_0$ if and only if we have
\begin{align*}
	M_{y_0}^2:=2\sup_{\varepsilon>0}\left( \inf_{L^2(0,T;U)}F_\varepsilon\right)<+\infty.
\end{align*}
In this case, we have 
\begin{align*}
	\inter{v_\varepsilon}_{L^2(0,T;U)}\leq M_{y_0},\nonumber
	\\
	\norm{\mathcal L_T(v_\varepsilon|y_0)}{}\leq M_{y_0}\sqrt{\varepsilon}.
\end{align*} 
\end{theorem} 

Since the fractional Laplacian $\fl{s}{}$ has the properties required for the operator $A$, the penalized HUM that we just described can be applied to the control problem \eqref{heat_frac}. To this purpose, let us now present its numerical implementation. 

Having obtained a FE approximation $\mathcal A_h$ of the operator $\fl{s}{}$, we can compute the fully-discrete version of \eqref{heat_frac}. For any given mesh $\mathfrak M$ and any integer $M>0$, we set $\delta t=T/M$ and we consider an implicit Euler method, with respect to the time variable. More precisely, we consider
%
\begin{align}\label{frac_heat_num}
	\begin{cases}
		\displaystyle\mathcal M_h \frac{z^{n+1}-z^n}{\delta t}+\mathcal A_h z^{n+1}=\mathbf{1}_\omega v_h^{n+1}, \quad \forall n\in \left\{1,\ldots,M-1\right\}
		\\
		z^0=z_0, 
	\end{cases}
\end{align}
%
where $z_0\in \mathbb R^\mesh$ and $\mathcal M_h$ is the classical mass matrix with entries $m_{i,j}=\langle \phi_i,\phi_j\rangle$.

Here, $v_{h,\delta t}=(v_h^n)_{1\leq n\leq M}$ is a fully-discrete control function whose cost, that is the discrete $L_{\delta t}^2(0,T;\mathbb R^\mesh)$-norm, is defined by
%
\begin{align*}
\|v_{\delta t}\|_{L_{\delta t}^2(0,T;\mathbb R^\mesh)}:=\left(\sum_{i=1}^M\delta t |v^n|^2_{L^2(\RR^\mesh)}\right)^{1/2},
\end{align*}
%
and where $| \cdot |_{L^2(\RR^\mesh)}$ stands for the norm associated to the $L^2$-inner product on $\mathbb{R}^\mathfrak M$
%
\begin{align*}
(u,v)_{L^2(\RR^\mesh)}=h \sum_{i=1}^N u_i v_i.
\end{align*}
%

With the above notation and according to the penalized HUM strategy, we introduce, for some penalization parameter $\varepsilon>0$, the following primal fully-discrete functional 
%
\begin{align*}
F_{\varepsilon,h,\delta t}(v_{\delta t})=\sum_{n=1}^{M}\delta t |v^n|^2_{L^2(\RR^\mesh)}+\frac{1}{2\varepsilon}|z^M|^2_{L^2(\RR^\mesh)}, \quad \forall \, v_{\delta t}\in L_{\delta t}^2(0,T;\mathbb R^\mesh),
\end{align*}
%
that we wish to minimize onto the whole fully-discrete control space $L_{\delta t}^2(0,T;\mathbb R^\mesh)$ and where $z^M$ is the final value of the controlled problem \eqref{frac_heat_num}. 

We can apply Fenchel-Rockafellar theory results to obtain the corresponding dual functional, which reads as follows
%
\begin{align}\label{dual_fully}
	J_{\varepsilon,h,\delta t}(\varphi^T)=\frac{1}{2}\sum_{n=1}^{M}\delta t|\mathbf 1_\omega\varphi|^2_{L^2(\RR^\mesh)}+\frac{\varepsilon}{2}|\varphi^T|^2_{L^2(\RR^\mesh)}+(\varphi^1,y_0)_{L^2(\RR^\mesh)}, \quad \forall \varphi^T\in L_{h,\delta t}^2(0,1)
\end{align}
%
where $\varphi=(\varphi^n)_{1\leq n\leq M+1}$ is solution to the adjoint system
%
\begin{align}\label{frac_adj_num}
	\begin{cases}
		\displaystyle\mathcal M_h \frac{\varphi^n-\varphi^{n+1}}{\delta t}+\mathcal A_h \varphi^n=0, & \forall n\in\inter{1,M}
		\\
		\varphi^{M+1}=\varphi^T. 
	\end{cases}
\end{align}

Notice that \eqref{dual_fully} is the fully-discrete approximation of \eqref{penalized_fun}. Moreover, it can be readily verified that this functional has a unique minimizer without any additional assumption on the problem. Therefore, by minimizing  \eqref{dual_fully}, and from duality theory, we obtain a control function 
\begin{align*}
	v_{\varepsilon,h,\delta t}=\left(\mathbf{1}_\omega\varphi_{\varepsilon,h,\delta t}^n\right)_{1\leq n\leq M},
\end{align*}
where $\varphi_\epsilon$ is the solution to \eqref{frac_adj_num} evaluated in the optimal datum $\varphi_\varepsilon^T$. 

Thus, the optimal penalized control always exists and is unique. Deducing controllability properties amounts to study the behavior of this control with respect to the penalization parameter $\varepsilon$, in connection with the discretization parameters.  

It is well known that, in general, we cannot expect for a given bounded family of initial data that the fully-discrete controls are uniformly bounded when the discretization parameters $h$, $\delta t$ and the penalization term $\varepsilon$ tend to zero independently. 

Instead, we expect to obtain uniform bounds by taking the penalization parameter $\varepsilon=\phi(h)$ that tends to zero in connection with the mesh size not too fast (see \cite{boyer2013penalised}) and a time step $\delta t$ verifying some weak condition of the kind $\delta t\leq \zeta(h)$ where $\zeta$ tends to zero logaritmically when $h\to 0$ (see \cite{boyer2011uniform}).

These facts will be confirmed by the numerical simulations that we are going to present in Section \ref{control_exp} below, by observing the behavior of the norm of the control, the optimal energy $\inf F_\varepsilon$, and the norm of the solution at time $T$. In this way, as predicted by Theorem \ref{theorem_hum}, we obtain a numerical evidence of the properties of null and approximate controllability for equation \eqref{heat_frac}, in accordance with the theoretical results in Section \ref{theor_sec}. 

\section{Numerical results} \label{res_numerical}
In this Section, we present the numerical simulations corresponding to the algorithm previously described, and we provide a complete discussion of the results obtained. 

First of all, we test numerically the accuracy of our method for the resolution of the elliptic equation \eqref{PE}, by applying it to the following problem 
\begin{align}\label{PE_real}
	\left\{\begin{array}{ll}
		\fl{s}{u} = 1, & x\in(-1,1)
		\\
		u\equiv 0, & x\in\RR\setminus(-1,1).
	\end{array}\right.
\end{align}

In this particular case, the unique solution to \eqref{PE_real} can be computed exactly and it is given in \cite{getoor1961first}. It reads as follows, 
\begin{align}\label{real_sol}
	u(x)=\frac{2^{-2s}\sqrt{\pi}}{\Gamma\left(\frac{1+2s}{2}\right)\Gamma(1+s)}\Big(1-x^2\Big)^s\cdot\mathbf{1}_{(-1,1)}.
\end{align}

In Fig.  \ref{smas12}, we show a comparison for different values of $s$ between the exact solution \eqref{real_sol} and the computed numerical approximation. Here we consider $N=50$. One can notice that when $s=0.1$ (and also for other small values of s), the computed solution is to a certain extent different from the exact solution. However, one should be careful with such result and a more precise analysis of the error should be carried. 
\begin{figure}[!h]
	\pgfplotstableread{fl_01_50_sym.txt}{\datpu}
	\pgfplotstableread{fl_04_50_sym.txt}{\ddatpu}
	\pgfplotstableread{fl_05_50_sym.txt}{\Ddatpu}
	\pgfplotstableread{fl_08_50_sym.txt}{\Dddatpu}

		\subfloat[$s=0.1$]{
		\begin{tikzpicture}[scale=0.8]
		\begin{axis}[xmin=-1.1, xmax=1.1,legend style={at={(0.77,0.25)}}]
		
			\addplot [color=blue, mark=none,  thick] table[x=0,y=1]{\datpu};	
			\addplot [color=red, mark=x, only marks, thick] table[x=2,y=3]{\datpu};		
			\addlegendentry{Numerical solution}
			\addlegendentry{Real solution}
		\end{axis}
	\end{tikzpicture}
	\label{s01a}
	}
		\hspace{1cm}\subfloat[$s=0.4$]{
		\begin{tikzpicture}[scale=0.8]
		\begin{axis}[xmin=-1.1, xmax=1.1,legend style={at={(0.77,0.25)}}]
			\addplot [color=blue, mark=none, thick] table[x=0,y=1]{\ddatpu};	
			\addplot [color=red, mark=x, only marks, thick] table[x=2,y=3]{\ddatpu};		
		\end{axis}
	\end{tikzpicture}
	}
\\
		\subfloat[$s=0.5$]{
		\begin{tikzpicture}[scale=0.8]
		\begin{axis}[xmin=-1.1, xmax=1.1,legend style={at={(0.77,0.25)}}]
			\addplot [color=blue, mark=none, thick] table[x=0,y=1]{\Ddatpu};	
			\addplot [color=red, mark=x, only marks, thick] table[x=2,y=3]{\Ddatpu};		
		\end{axis}
	\end{tikzpicture}
	}
		\hspace{1cm}\subfloat[$s=0.8$]{
		\begin{tikzpicture}[scale=0.8]
		\begin{axis}[xmin=-1.1, xmax=1.1,legend style={at={(0.77,0.25)}}]
			\addplot [color=blue, mark=none, thick] table[x=0,y=1]{\Dddatpu};	
			\addplot [color=red, mark=x, only marks, thick] table[x=2,y=3]{\Dddatpu};		
		\end{axis}
	\end{tikzpicture}
	}
	\caption{Plot for different values of $s$.}
	\label{smas12}
\end{figure}

In the same spirit as in \cite{acosta2017short}, the computation of the error in the space $H_0^s(-1,1)$ can be readily done by using the definition of the bilinear form, namely
\begin{align*}
	\norm{u-u_h}{H_0^s(-1,1)}^2 &=a(u-u_h,u-u_h) =a(u,u-u_h) =\int_{-1}^1f(x)\left(u(x)-u_h(x)\right)dx,
\end{align*}
where have used the orthogonality condition $a(v_h,u-u_h)=0$ $\forall v_h \in V_h$.

For this particular test, since $f\equiv 1$ in $(-1,1)$, the problem is therefore reduced to
\begin{align*}
	\norm{u-u_h}{H^s_0(-1,1)}=\left(\int_{-1}^1\left( u(x)-u_h(x) \right)\,dx\right)^{1/2}
\end{align*}
where the right-hand side can be easily computed, since we have the closed formula 
\begin{align*}
	\int_{-1}^1u\,dx= \frac{\pi}{2^{2s}\Gamma(s+\frac{1}{2})\Gamma(s+\frac{3}{2})}
\end{align*}
and the term corresponding to $\int_{-1}^1u_h$ can be carried out numerically. 

In Fig.  \ref{error}, we present the computational errors evaluated for different values of $s$ and $h$. 
\begin{figure}[!h]
 \centering
  \pgfplotstableread{result_convergence_01.org}{\datpu}
  \pgfplotstableread{result_convergence_03.org}{\datpt}
   \pgfplotstableread{result_convergence_05.org}{\datpcnum}
  \pgfplotstableread{result_convergence_07.org}{\datps}
  \pgfplotstableread{result_convergence_09.org}{\datpn}
   \pgfplotsset{
     legend cell align=left,
     legend pos=outer north east,
     legend plot pos=right,
     legend style={cells={anchor=east},draw=none},
     }
%
  %\subfloat[Convergence of the error]{\label{fig_case_O_subset_omega}
  \begin{tikzpicture}%[scale=1] 
  \label{conv}
  \begin{loglogaxis}[xlabel=$h$, ymax=2e-1]
  	%\addplot [color=blue, very thick] table [x=0, y=2] {\datpt};
	\addplot [color=black, mark=o, thick] table [x=0, y=1] {\datpu};
	\addlegendentry{$s=0.1$};
	\addplot [color=black, mark=diamond, mark size=3 pt, thick] table [x=0, y=1] {\datpt};
	\addlegendentry{${s}=0.3$};
	\addplot [color=black, mark=pentagon, thick] table [x=0, y=1] {\datpcnum};
	\addlegendentry{${s}=0.5$};
	\addplot [color=black, mark=square, thick] table [x=0, y=1] {\datps};
	\addlegendentry{$ s=0.7$};
	\addplot [color=black, mark=triangle, mark size=3 pt, thick] table [x=0, y=1] {\datpn};
	\addlegendentry{$ s=0.9$};
%
 \draw [pente]  (axis cs: 0.0008,4e-2) -- ++ (axis cs: 1, {10^(0.5)}) -- ++ (axis cs: 10, 1) -- cycle;
 \node at (axis cs:0.0008,10e-2) [right,pente] {\small slope $0.5$};
%
  \end{loglogaxis}
\end{tikzpicture}%
\caption{Convergence of the error.}
\label{error}
\end{figure}

The rates of convergence shown are of order (in $h$) of $1/2$. This is in accordance with the following result: 
\begin{theorem}[{\cite[Theorem 4.6]{acosta2017short}}]
For the solution $u$ of \eqref{WF} and its FE approximation $u_h$ given by \eqref{WFD}, if $h$ is sufficiently small, the following estimates hold

\begin{align*}
	&\norm{u-u_h}{H^s_0(-1,1)}\leq C h^{1/2}|\!\ln h|\,\norm{f}{C^{1/2-s}(-1,1)}, \quad \textnormal{if}\quad s<1/2, 
	\\
	&\norm{u-u_h}{H^s_0(-1,1)}\leq C h^{1/2} |\!\ln h|\, \norm{f}{L^\infty(-1,1)}, \quad \textnormal{if}\quad  s=1/2 
	\\
	&\norm{u-u_h}{H^s_0(-1,1)}\leq \tfrac{C}{2s-1} h^{1/2} \sqrt{|\!\ln h|}\, \norm{f}{C^\beta(-1,1)}, \quad \textnormal{if} \quad s>1/2,
\end{align*}
where $C$ is a positive constant not depending on $h$. 
\end{theorem}

Moreover, Fig.  \ref{error} shows that the convergence rate is maintained also for small values of $s$. This confirms that the behavior shown in Fig. \ref{smas12} (A) is not in contrast with the known theoretical results. Indeed, since it is well-known that the notion of trace is not defined for the spaces $H^s(-1,1)$ with $s\leq 1/2$ (see \cite{jllions1972non,tartar2007introduction}), it is somehow natural that we cannot expect a point-wise convergence in this case.  

\subsection{Control experiments}\label{control_exp}

To address the actual computation of fully-discrete controls for a given problem, we use the methodology described, for instance, in \cite{glowinski2008exact}. We apply an optimization algorithm to the dual functional \eqref{dual_fully}. Since these functionals are quadratic and coercive, the conjugate gradient is a natural and quite simple choice.

In the same spirit as \cite{boyer2011uniform}, the computation of the gradient at each iteration amounts to solve first the homogeneous equation \eqref{frac_adj_num}. Then, set $v^n=\mathbf{1}_\omega\varphi^n$ and finally solve \eqref{frac_heat_num}. In this way, the procedure to compute the control for a given problem basically requires to solve two parabolic equations: a homogeneous backward one associated with the final data $\varphi^T$, and a non-homogeneous forward problem with zero initial data. 

We present now some results obtained with the described methodology. In accordance with the discussion in Section \ref{fe_sec}, we use the finite-element approximation of $\fl{s}{}$ for the space discretization and the implicit Euler scheme in the time variable. We denote by $N$ the number of points in the mesh and by $M$ the number of time intervals. As discussed in \cite{boyer2011uniform}, the results in this kind of problems does not depend too much in the time step, as soon as it is chosen to ensure at least the same accuracy as the space discretization. The same remains true here, and therefore we always take $M=2000$ in order to concentrate the discussion on the dependency of the results with respect to the mesh size $h$ and the parameter $s$.

As we mentioned, we choose the penalization term $\varepsilon$ as a function of $h$. A reasonable practical rule (\cite{boyer2013penalised}) is to systematically choose $\phi(h)\sim h^{2p}$ where $p$ is the order of accuracy in space of the numerical method employed for the discretization of the spatial operator involved (in this case the fractional Laplacian \eqref{fl}).

\begin{theorem}[{\cite[proposition 3.3.2]{borthagaray2017laplaciano}}]
Let $s\in(0,1)$, $f\in L^2(-1,1)$ and $u$ be the solution to \eqref{PE}. Given a uniform mesh $\mathfrak{M}$ with mesh size $h$,
and the space $V_h$ defined as in \eqref{Vh}, let $u_h$ be the finite element solution to the corresponding discrete 
problem. Then, it holds that
\begin{align}\label{error_l2}
	\norm{u-u_h}{L^2(-1,1)}\leq C(s,\alpha)h^{2\alpha}\norm{f}{L^2(-1,1)},
\end{align}
where $\alpha:=\min\{s, 1/2 -\varepsilon\}$, for all $\varepsilon>0$.
\end{theorem}%



 We recall that for the elliptic problem that we are considering, this order of convergence is $1/2$. Thus, hereinafter we always assume $\varepsilon=\phi(h)=h$.

We begin by plotting on Fig.  \ref{sol_surf} the time evolution of the uncontrolled solution as well as the controlled solution. Here, we set $s=0.8$, $\omega=(-0.3,0.8)$ and $T=0.3$, and as an initial condition we take $z_0(x) = \sin(\pi x)$. The control domain is represented as highlighted zone on the plane $(t,x)$. As expected, we observe that the uncontrolled solution is damped with time, but does not reach zero at time $T$, while the controlled solution does. 

%\begin{figure}[ht]
%%%  Datos del experimento 
%%%  T=0.3, epsilon=0.01*h, s=0.8
% \centering
%% 
%\subfloat[The adjoint][Uncontrolled solution]{
%\begin{tikzpicture}
%\begin{axis}[surface,zmax=1.0]
%  
%\addplot3 [mesh/ordering=y varies, surf, shader=flat,draw=black,opacity=0.6] file {soly_s08_sc.dat};
%\node at (axis cs:0.34,0,0) [below] {$T=0.3$};
%
%\end{axis}
%\end{tikzpicture}}
%%
%\hspace{1 cm}
%  \subfloat[The state][Controlled solution ({\tikz \fill [green] (0,0) rectangle (0.2,0.2);}=control domain)]{
%\begin{tikzpicture}
%\begin{axis}[surface, zmax=1.0]
%
%  %\controldomainT{-0.3}{0.8}{0.3};
%  \couplingdomainT{-0.3}{0.8}{0.3};
%  
%\addplot3 [mesh/ordering=y varies, surf, shader=flat,draw=black,opacity=0.6] file {soly_s08_cc.dat};
%\node at (axis cs:0.34,0,0) [below] {$T=0.3$};
%
%\end{axis}
%\end{tikzpicture}}%
%\caption{Time evolution of system \eqref{frac_heat_num}.}\label{fig_heat_frac}
%\label{sol_surf}
%\end{figure}

%\begin{figure}[ht]
%	\centering
%	\includegraphics[scale=1]{figure5.eps}
%	\caption{Time evolution of system \eqref{frac_heat_num}.}\label{fig_heat_frac}
%\label{sol_surf}
%\end{figure}

In figure \ref{figure_case1}, we present the computed values of various quantities of interest when the mesh size goes to zero. More precisely, we observe that the control cost $\|v_{\delta t}\|_{L^2_{\delta t}(0,T;\mathbb R^\mesh)}$ and the optimal energy $\inf F_{\phi(h),h,\delta t}$ remain bounded as $h\to 0$. On the other hand, we see that 
\begin{align}\label{control_norm_behavior}
	|y^M|_{L^2(\RR^\mesh)}\,\sim\,C\sqrt{\phi(h)}=Ch^{1/2}. 
\end{align}

We know that, for $s=0.8$, system \eqref{heat_frac} is null controllable. This is now confirmed by \eqref{control_norm_behavior}, according to Theorem \ref{theorem_hum}.  In fact, the same experiment can be repeated for different values of $s>1/2$, obtaining the same conclusions. 
\begin{figure}[h]
  \centering
\begin{tikzpicture}
  \begin{loglogaxis}[erreurs, ymin=5e-5,ymax=1e1]
 
    \pgfplotstableread[ignore chars={|},skip first n=2]{heat_frac_s=08.org}\resultats
    %% Original file  resultats_09-06-2017_11h18.org

    \addplot[cout] table[x=dx,y=Nv] \resultats;
    \addlegendentry{Cost of the control};
    \addplot[cible] table[x=dx,y=NyT] \resultats;
    \addlegendentry{Size of $y^M$};
    \addplot[energie] table[x=dx,y=Inf_eps(F_eps)] \resultats;
    \addlegendentry{Optimal energy};
    \draw [pente]  (axis cs: 0.003,84e-4) -- ++ (axis cs: 1, 10^1) -- ++ (axis cs: 10, 1) -- cycle;
    \node at (axis cs:0.003,20e-3) [right,pente] {\small sl. $1$};
    
  \end{loglogaxis}
  
\end{tikzpicture}
\caption{Convergence properties of the method for controllability of the fractional heat equation for $s=0.8$.}\label{figure_case1}
\end{figure}

According to the discussion in Section \ref{theor_sec}, one can prove that null controllability does not hold for system \eqref{heat_frac} in the case $s\leq 1/2$. However approximate controllability can be proved by means of the unique continuation property of the operator $\fl{s}{}$. We would like to illustrate this property in Fig. \ref{figure_case3}.

We observe that the results are different from what we obtained in Fig.  \ref{figure_case1}. In fact, the cost of the control and the optimal energy increase in both cases, while the target $y^M$ tends to zero with a slower rate than $h^{1/2}$. This seems to confirm that a uniform observability estimate for \eqref{heat_frac} does not hold and that we can only expect to have approximate controllability (see Theorem \ref{theorem_hum}).
\begin{figure}
\centering
\subfloat[$s=0.4$]{
\begin{tikzpicture}[scale=0.9]
  \begin{loglogaxis}[erreurs, ymin=8e-2,ymax=2e1,title={$s=0.4$}]
 
    \pgfplotstableread[ignore chars={|},skip first n=2]{heat_frac_s=04.org}\resultats
    %% Original file  resultats_09-06-2017_11h47.org

    \addplot[cout] table[x=dx,y=Nv] \resultats;
    %\addlegendentry{Cost of the control};
    \addplot[cible] table[x=dx,y=NyT] \resultats;
    %\addlegendentry{Size of $y^M$};
    \addplot[energie] table[x=dx,y=Inf_eps(F_eps)] \resultats;
    %\addlegendentry{Optimal energy};
    \draw [pente]  (axis cs: 0.003,2.75e-1) -- ++ (axis cs: 1, 6^0.15) -- ++ (axis cs: 6, 1) -- cycle;
    \node at (axis cs:0.003,4.2e-1) [right,pente] {\small sl. $0.35$};
    
    \draw [pente]  (axis cs: 0.003,1.7e0) -- ++ (axis cs: 1, {6^(-0.7)}) -- ++ (axis cs: 6, 1) -- cycle;
    \draw [pente]  (axis cs: 0.003,0.99e0) -- ++ (axis cs: 1, {5.9^(-0.4)}) -- ++ (axis cs: 5.9, 1) -- cycle;
    \node at (axis cs:0.003,0.55e0) [right,pente] {\small sl. $-0.4/-0.7$};
    
  \end{loglogaxis}
  \end{tikzpicture}
  }
 \subfloat[$s=0.5$]{
  \begin{tikzpicture}[scale=0.9]
  \begin{loglogaxis}[erreurs, ymin=2e-2,ymax=20.4e1,title={$s=0.5$}]
    \pgfplotstableread[ignore chars={|},skip first n=2]{heat_frac_s=05.org}\resultats
    %% Original file  resultats_09-06-2017_11h07.org

    \addplot[cout] table[x=dx,y=Nv] \resultats;
    %\addlegendentry{Cost of the control};
    \addplot[cible] table[x=dx,y=NyT] \resultats;
    %\addlegendentry{Size of $y^M$};
    \addplot[energie] table[x=dx,y=Inf_eps(F_eps)] \resultats;
    %\addlegendentry{Optimal energy};
    \draw [pente]  (axis cs: 0.003,10e-2) -- ++ (axis cs: 1, 10^0.4) -- ++ (axis cs: 10, 1) -- cycle;
    \node at (axis cs:0.003,18e-2) [right,pente] {\small slope $0.4$};
    
    \draw [pente]  (axis cs: 0.003,2.65e1) -- ++ (axis cs: 1, {10^(-0.18)}) -- ++ (axis cs: 10, 1) -- cycle;
    \draw [pente]  (axis cs: 0.003,3.5e1) -- ++ (axis cs: 1, {10^(-0.3)}) -- ++ (axis cs: 10, 1) -- cycle;
    \node at (axis cs:0.003,1.35e1) [right,pente] {\small sl. $-0.18/-0.3$};
    
  \end{loglogaxis}
\end{tikzpicture}
}
\caption{Convergence properties of the method for $s<1/2$. Same legend as in Fig.  \ref{figure_case1}}\label{figure_case3}
\end{figure}


{\appendix

\section{Explicit computations of the elements of the matrix $\mathcal A_h$}\label{appendix}

We present here the explicit computations for each element $a_{i,j}$ of the stiffness matrix, completing the discussion that we started in Section \ref{fe_sec}.

\subsubsection*{Step 1: $j\geq i+2$}
We recall that, in this case, the value of $a_{i,j}$ is given by the integral
\begin{align}\label{elem_noint_app}
	a_{i,j}=-2 \int_{x_{j-1}}^{x_{j+1}}\int_{x_{i-1}}^{x_{i+1}}\frac{\phi_i(x)\phi_j(y)}{|x-y|^{1+2s}}\,dxdy.
\end{align}

In Fig. \ref{upp_tri}, we give a scheme of the region of interaction (marked in grey) between the basis functions in this case. 
\begin{figure}[h]
\figinit{0.7pt}
% Axes
\figpt 0:(-90,-10)
\figpt 1:(-100,0) \figpt 2:(220,0)
\figpt 3:(-80,-20) \figpt 4:(-80,290)

%%%%
\figpt 5:(-50,0) \figpt 6:(-50,230)
\figpt 7:(-25,0) \figpt 8:(-25,230)
\figpt 9:(0,0) \figpt 10:(0,230)
%
\figpt 11:(-80,30) \figpt 12:(150,30)
\figpt 13:(-80,55) \figpt 14:(150,55)
\figpt 15:(-80,80) \figpt 16:(150,80)
%
\figpt 17:(100,0) \figpt 18:(100,230)
\figpt 19:(125,0) \figpt 20:(125,230)
\figpt 21:(150,0) \figpt 22:(150,230)
%
\figpt 23:(-80,180) \figpt 24:(150,180)
\figpt 25:(-80,205) \figpt 26:(150,205)
\figpt 27:(-80,230) \figpt 28:(150,230)

% Grey squares 
\figpt 29:(-50,30) \figpt 30:(0,30)
\figpt 31:(-50,80) \figpt 32:(0,80)

\figpt 33:(-50,180) \figpt 34:(0,180)
\figpt 35:(-50,230) \figpt 36:(0,230)

\figpt 37:(100,180) \figpt 38:(150,180)
\figpt 39:(100,230) \figpt 40:(150,230)

\figpt 41:(100,30) \figpt 42:(150,30)
\figpt 43:(100,80) \figpt 44:(150,80)

% Dotted crosses
\figpt 45:(-50,55) \figpt 46:(0,55)
\figpt 47:(-25,30) \figpt 48:(-25,80)

\figpt 49:(-50,205) \figpt 50:(0,205)
\figpt 51:(-25,180) \figpt 52:(-25,230)

\figpt 53:(100,205) \figpt 54:(150,205)
\figpt 55:(125,180) \figpt 56:(125,230)

\figpt 57:(100,55) \figpt 58:(150,55)
\figpt 59:(125,30) \figpt 60:(125,80)

% Points for writing
\figpt 61:(-50,-10) \figpt 62:(-25,-10) \figpt 63:(0,-10) 
\figpt 64:(-92,30) \figpt 65:(-92,55) \figpt 66:(-92,80) 
\figpt 67:(100,-10) \figpt 68:(125,-10) \figpt 69:(150,-10) 
\figpt 70:(-92,180) \figpt 71:(-92,205) \figpt 72:(-92,230) 

\figpt 73:(-92,280) \figpt 74:(210,-10)

% Diagonal

\figpt 75:(-80,0) \figpt 76:(180,260)


% 2. Creation of the graphical file
\figdrawbegin{}
\figdrawarrow[1,2]
\figdrawarrow[3,4]
\figset(dash=4)
\figdrawline[5,6]
\figdrawline[7,8]
\figdrawline[9,10]
\figdrawline[11,12]
\figdrawline[13,14]
\figdrawline[15,16]
\figdrawline[17,18]
\figdrawline[19,20]
\figdrawline[21,22]
\figdrawline[23,24]
\figdrawline[25,26]
\figdrawline[27,28]
\figset(dash=default)

\figset(fillmode=yes, color=0.8)
%\figdrawline[29,30,32,31,29]
\figdrawline[33,34,36,35,33]
%\figdrawline[37,38,40,39,37]
\figdrawline[41,42,44,43,41]
\figset (fillmode=no, color=default)
%\figdrawline[29,30,32,31,29]
\figdrawline[33,34,36,35,33]
%\figdrawline[37,38,40,39,37]
\figdrawline[41,42,44,43,41]

\figset(dash=4)
\figdrawline[45,46]
\figdrawline[47,48]
\figdrawline[49,50]
\figdrawline[51,52]
\figdrawline[53,54]
\figdrawline[55,56]
\figdrawline[57,58]
\figdrawline[59,60]

\figset(dash=5)
\figdrawline[75,76]

\figdrawend

% 3. Writing text on the figure
\figvisu{\figBoxA}{}{
\figwritec [0]{$O$}
\figwritec [61,64]{$x_{i-1}$}
\figwritec [62,65]{$x_i$}
\figwritec [63,66]{$x_{i+1}$}
\figwritec [67,70]{$x_{j-1}$}
\figwritec [68,71]{$x_j$}
\figwritec [69,72]{$x_{j+1}$}
\figwritec [73]{$y$}
\figwritec [74]{$x$}
}
\centerline{\box\figBoxA}
\caption{Interactions between the basis function $\phi_i$ and $\phi_j$ when $j\geq i+2$.}\label{upp_tri}
\end{figure}

%\begin{figure}[h]
%	\centering
%	\includegraphics[scale=1]{figure13.eps}
%	\caption{Interactions between the basis function $\phi_i$ and $\phi_j$ when $j\geq i+2$.}\label{upp_tri}
%\end{figure}

Now, taking into account the definition of the basis function \eqref{basis_fun}, the integral \eqref{elem_noint_app} becomes
\begin{align*}
	a_{i,j}=-2 \int_{x_{j-1}}^{x_{j+1}}\int_{x_{i-1}}^{x_{i+1}}\frac{\left(1-\frac{|x-x_i|}{h}\right)\left(1-\frac{|y-x_j|}{h}\right)}{|x-y|^{1+2s}}\,dxdy.
\end{align*}

Let us introduce the following change of variables:
\begin{align*}
	\frac{x-x_i}{h}=\hat{x},\;\;\; \frac{y-x_i}{h}=\hat{y}.
\end{align*}

Then, rewriting (with some abuse of notations since there is no possibility of confusion) $\hat{x}=x$ and $\hat{y}=y$, we get 
\begin{align}\label{elem_noint_cv}
	a_{i,j}=-2h^{1-2s} \int_{-1}^1\int_{-1}^1\frac{(1-|x\,|\,)(1-|y\,|\,)}{|x-y+i-j\,|^{1+2s}}\,dxdy.
\end{align}

The integral \eqref{elem_noint_cv} can be computed explicitly in the following way. First of all, for simplifying the notation, let us define $k=j-i$. We have 
\begin{align*}
	a_{i,j} = &\, -2h^{1-2s} \int_{-1}^1\int_{-1}^1\frac{(1-|x\,|\,)(1-|y\,|\,)}{|x-y+i-j\,|^{1+2s}}\,dxdy =-2h^{1-2s} \int_{-1}^1\int_{-1}^1\frac{(1-|x\,|\,)(1-|y\,|\,)}{|x-y-k\,|^{1+2s}}\,dxdy
	\\
	= &\, -2h^{1-2s} \int_0^1\int_0^1\frac{(1-x)(1-y)}{(y-x+k)^{1+2s}}\,dxdy - 2h^{1-2s} \int_0^1\int_{-1}^0\frac{(1+x)(1-y)}{(y-x+k)^{1+2s}}\,dxdy 
	\\
	&- 2h^{1-2s} \int_{-1}^0\int_0^1\frac{(1-x)(1+y)}{(y-x+k)^{1+2s}}\,dxdy - 2h^{1-2s} \int_{-1}^0\int_{-1}^0\frac{(1+x)(1+y)}{(y-x+k)^{1+2s}}\,dxdy
	\\
	= &\, - 2h^{1-2s}(B_1 + B_2 + B_3 + B_4).
\end{align*}

These terms $B_i$, $i=1,2,3,4$, can be computed integrating by parts several times. In more detail, we have
\begin{align*}
	& B_1 = \frac{1}{4s(1-2s)}\left[2k^{1-2s}-\frac{(k+1)^{2-2s}-(k-1)^{2-2s}}{1-s}-\frac{2k^{3-2s}-(k+1)^{3-2s}-(k-1)^{3-2s}}{(1-s)(3-2s)}\right]
	\\
	& B_2 = \frac{1}{4s(1-2s)}\left[-2k^{1-2s}+\frac{2(k+1)^{2-2s}-2k^{2-2s}}{1-s}+\frac{2(k+1)^{3-2s}-k^{3-2s}-(k+2)^{3-2s}}{(1-s)(3-2s)}\right]
	\\
	& B_3 = \frac{1}{4s(1-2s)}\left[-2k^{1-2s}+\frac{2k^{2-2s}-2(k-1)^{2-2s}}{1-s}+\frac{2(k-1)^{3-2s}-k^{3-2s}-(k-2)^{3-2s}}{(1-s)(3-2s)}\right]
	\\
	& B_4 = \frac{1}{4s(1-2s)}\left[2k^{1-2s}-\frac{(k+1)^{2-2s}-(k-1)^{2-2s}}{1-s}-\frac{2k^{3-2s}-(k+1)^{3-2s}-(k-1)^{3-2s}}{(1-s)(3-2s)}\right].
\end{align*} 

Therefore, we obtain
\begin{align*}
	a_{i,j} = - h^{1-2s}\,\frac{4(k+1)^{3-2s} + 4(k-1)^{3-2s}-6k^{3-2s}-(k+2)^{3-2s}-(k-2)^{3-2s}}{2s(1-2s)(1-s)(3-2s)}.
\end{align*} 

We notice that, when $s=1/2$, both the numerator and the denominator of the expression above are zero. Hence, in this particular case, it would not be possible to introduce the value that we just encountered in our code. Nevertheless, this difficulty can be overcome noting that we can easily compute
\begin{align*}
	\lim_{s\to\frac{1}{2}} &- h^{1-2s}\,\frac{4(k+1)^{3-2s} + 4(k-1)^{3-2s}-6k^{3-2s}-(k+2)^{3-2s}-(k-2)^{3-2s}}{2s(1-2s)(1-s)(3-2s)}
	\\
	& = -4(k+1)^2\log(k+1)-4(k-1)^2\log(k-1)+6k^2\log(k)+(k+2)^2\log(k+2)+(k-2)^2\log(k-2),
\end{align*} 
if $k\neq 2$. When $k=2$, instead, since 
\begin{align*}
	\lim_{k\to 2} (k-2)^2\log(k-2) =0,
\end{align*}
the corresponding value $a_{i,j}=a_{i,i+2}$ if given by 
\begin{align*}
	a_{i,i+2} = 56\ln(2)-36\ln(3).
\end{align*}


\subsubsection*{Step 2: $j= i+1$}
This is the most cumbersome case, since it is the one with the most interactions between the basis functions (see Fig. \ref{basis_upp_dia}). According to \eqref{stiffness_nc}, and using the symmetry of the integral with respect to the bisector $y=x$, we have 
	\begin{align*}
	a_{i,i+1}= & \int_{\RR}\int_{\RR}\frac{(\phi_i(x)-\phi_i(y))(\phi_{i+1}(x)-\phi_{i+1}(y))}{|x-y|^{1+2s}}\,dxdy
	\\
	= & \int_{x_{i+1}}^{+\infty}\int_{x_{i+1}}^{+\infty} \ldots\,dxdy + 2\int_{x_{i+1}}^{+\infty}\int_{x_i}^{x_{i+1}} \ldots\,dxdy + 2\int_{x_{i+1}}^{+\infty}\int_{-\infty}^{x_i} \ldots\,dxdy 
	\\
	& + \int_{x_i}^{x_{i+1}}\int_{x_i}^{x_{i+1}} \ldots\,dxdy + 2\int_{x_i}^{x_{i+1}}\int_{-\infty}^{x_i} \ldots\,dxdy + \int_{-\infty}^{x_i}\int_{-\infty}^{x_i} \ldots\,dxdy 
	\\
	:= & Q_1 + Q_2 + Q_3 + Q_4 + Q_5 + Q_6.
\end{align*}

In Fig. \ref{upp_dia}, we give a scheme of the regions of interactions between the basis functions $\phi_i$ and $\phi_{i+1}$ enlightening the domain of integration of the $Q_i$. The regions in grey are the ones that produce a contribution to $a_{i,i+1}$, while on the regions in white the integrals will be zero.
\begin{figure}[h]
\figinit{0.7pt}
% Axes
\figpt 0:(-50,70)
\figpt 1:(-100,80) \figpt 2:(220,80)
\figpt 3:(-40,-20) \figpt 4:(-40,290)

%%%%
\figpt 5:(40,0) \figpt 6:(40,270)
\figpt 7:(65,0) \figpt 8:(65,270)
\figpt 9:(90,0) \figpt 10:(90,270)
%
\figpt 11:(-80,160) \figpt 12:(200,160)
\figpt 13:(-80,185) \figpt 14:(200,185)
\figpt 15:(-80,210) \figpt 16:(200,210)
%

% Grey part 
\figpt 29:(65,210) \figpt 30:(90,210)
\figpt 31:(65,235) \figpt 32:(40,235) 
\figpt 33:(40,210) \figpt 34:(90,185)
\figpt 35:(115,185) \figpt 36:(115,160)
\figpt 37:(90,160) \figpt 38:(65,185)
\figpt 39:(40,185) \figpt 40:(65,160) 
\figpt 41:(115,0) \figpt 42:(115,270) 
\figpt 43:(115,210) \figpt 44:(-80,235) 
\figpt 45:(90,235) \figpt 46:(200,235) 

% Diagonal 
\figpt 47:(-40,80) \figpt 48:(150,270)

% Points for writing
\figpt 49:(-50,164) \figpt 50:(-47,189) 
\figpt 51:(-50,215) \figpt 52:(-50,240) 

\figpt 53:(28,74) \figpt 54:(59,74) 
\figpt 55:(80,74) \figpt 56:(104,74) 

\figpt 73:(-52,280) \figpt 74:(210,70)

\figpt 75:(160,255) \figpt 76:(77.5,255)
\figpt 77:(160,197.5) \figpt 78:(52.5,222.5)
\figpt 79:(102.5,172.5) \figpt 80:(77.5,197.5)
\figpt 81:(77.5,28) \figpt 82:(-15,197.5)
\figpt 83:(-15,28) 


% 2. Creation of the graphical file
\figdrawbegin{}

\figset(fillmode=yes, color=0.9)
\figdrawline[29,30,10,8,29]
\figdrawline[29,31,32,33,29]
\figdrawline[30,34,14,16,30]
\figdrawline[35,36,37,34,35]
\figdrawline[34,38,7,9,34]
\figdrawline[38,29,15,13,38]
\figset(fillmode=yes, color=0.6)
\figdrawline[29,30,34,38,29]
\figset(fillmode=no, color=0)
\figdrawarrow[1,2]
\figdrawarrow[3,4]
\figdrawline[8,31]
\figdrawline[10,30]
\figdrawline[30,16]
\figdrawline[35,14]
\figdrawline[35,36]
\figdrawline[36,37]
\figdrawline[37,9]
\figdrawline[38,7]
\figdrawline[38,13]
\figdrawline[33,15]
\figdrawline[32,33]
\figdrawline[32,31]
\figset(dash=4)
\figdrawline[6,32]
\figdrawline[39,5]
\figdrawline[40,11]
\figdrawline[36,12]
\figdrawline[36,41]
\figdrawline[42,43]
\figdrawline[32,44]
\figdrawline[45,46]

\figdrawline[33,30]
\figdrawline[31,45]
\figdrawline[38,35]
\figdrawline[40,37]
\figdrawline[33,39]
\figdrawline[31,38]
\figdrawline[30,37]
\figdrawline[43,35]
\figset(dash=5)
\figdrawline[47,48]
\figdrawend

% 3. Writing text on the figure
\figvisu{\figBoxA}{}{
\figwritec [0]{$O$}
\figwritec [49,53]{$x_{i-1}$}
\figwritec [50,54]{$x_i$}
\figwritec [51,55]{$x_{i+1}$}
\figwritec [52,56]{$x_{i+2}$}
\figwritec [73]{$y$}
\figwritec [74]{$x$}
\figwritec [75]{$Q_1$}
\figwritec [76,77]{$Q_2$}
\figwritec [78,79]{$Q_3$}
\figwritec [80]{$Q_4$}
\figwritec [81,82]{$Q_5$}
\figwritec [83]{$Q_6$}
}
\centerline{\box\figBoxA}
\caption{Interactions between the basis function $\phi_i$ and $\phi_{i+1}$.}\label{upp_dia}
\end{figure}

%\begin{figure}[h]
%	\centering
%	\includegraphics[scale=1]{figure14.eps}
%	\caption{Interactions between the basis function $\phi_i$ and $\phi_{i+1}$.}\label{upp_dia}
%\end{figure}

Le us now compute the terms $Q_i$, $i=1,\ldots,6$, separately. 
\subsubsection*{Computation of $Q_1$}
Since $\phi_i = 0$ on the domain of integration we have
\begin{align*}
	Q_1 &= \int_{x_{i+1}}^{+\infty}\int_{x_{i+1}}^{+\infty} \frac{\phi_{i+1}(x)-\phi_{i+1}(y)}{|x-y|^{1+2s}}\,dxdy 
	\\
	&= \int_{x_{i+1}}^{+\infty}\int_{x_{i+1}}^{+\infty} \frac{\phi_{i+1}(x)}{|x-y|^{1+2s}}\,dxdy - \int_{x_{i+1}}^{+\infty}\int_{x_{i+1}}^{+\infty} \frac{\phi_{i+1}(y)}{|x-y|^{1+2s}}\,dxdy = 0.
\end{align*}

\subsubsection*{Computation of $Q_2$}
We have
\begin{align*}
	Q_2 &= 2\int_{x_{i+1}}^{+\infty}\int_{x_i}^{x_{i+1}} \frac{\phi_i(x)(\phi_{i+1}(x)-\phi_{i+1}(y))}{|x-y|^{1+2s}}\,dxdy. 
\end{align*}

Now, using Fubini's theorem we can exchange the order of the integrals, obtaining 
\begin{align*}
	Q_2 &= 2\int_{x_i}^{x_{i+1}}\phi_i(x)\phi_{i+1}(x)\left(\int_{x_{i+1}}^{+\infty} \frac{dy}{|x-y|^{1+2s}}\right)\,dx - 2\int_{x_{i+1}}^{x_{i+2}}\int_{x_i}^{x_{i+1}} \frac{\phi_i(x)\phi_{i+1}(y)}{|x-y|^{1+2s}}\,dxdy 
	\\
	&= \frac{1}{s}\int_{x_i}^{x_{i+1}}\frac{\phi_i(x)\phi_{i+1}(x)}{(x_{i+1}-x)^{2s}}\,dx - 2\int_{x_{i+1}}^{x_{i+2}}\int_{x_i}^{x_{i+1}} \frac{\phi_i(x)\phi_{i+1}(y)}{|x-y|^{1+2s}}\,dxdy
	\\
	&= \frac{1}{s}\int_{x_i}^{x_{i+1}}\frac{\left(1-\frac{|x-x_i|}{h}\right)\left(1-\frac{|x-x_{i+1}|}{h}\right)}{(x_{i+1}-x)^{2s}}\,dx - 2\int_{x_{i+1}}^{x_{i+2}}\int_{x_i}^{x_{i+1}} \frac{\left(1-\frac{|x-x_i|}{h}\right)\left(1-\frac{|y-x_{i+1}|}{h}\right)}{|x-y|^{1+2s}}\,dxdy:= Q_2^1 + Q_2^2.
\end{align*}

The two integrals above can be computed explicitly. Indeed, employing the change of variables
\begin{align*}
	\frac{x_{i+1}-x}{h}=\hat{x},
\end{align*}
and then renaming $\hat{x}=x$, $R_2^1$ becomes
\begin{align*}
	Q_2^1=\frac{h^{1-2s}}{s}\int_0^1 x^{1-2s}(1-x)\,dx = \frac{h^{1-2s}}{s(2-2s)(3-2s)}.
\end{align*}

For computing $Q_2^2$, instead, we introduce the change of variables
\begin{align}\label{cv3}
	\frac{x_i-x}{h}=\hat{x},\;\;\;\frac{y-x_{i+1}}{h}=\hat{y},
\end{align}
and we obtain
\begin{align*}
	Q_2^2 = -2h^{1-2s}\int_0^1\int_0^1\frac{(1-x)(1-y)}{(y-x+1)^{1+2s}}\,dxdy = h^{1-2s}\frac{2^{1-2s}+s-2}{s(1-s)(3-2s)}.
\end{align*}

Adding the two contributions, we get the following expression for the term $R_2$
\begin{align*}
	Q_2 = h^{1-2s}\frac{2^{2-2s}+2s-3}{s(2-2s)(3-2s)}.
\end{align*}

\subsubsection*{Computation of $Q_3$}
In this case, we simply take into account the intervals in which the basis functions are supported, so that we obtain
\begin{align*}
	Q_3 &= -2\int_{x_{i+1}}^{x_{i+2}}\int_{x_{i-1}}^{x_i} \frac{\phi_i(x)\phi_{i+1}(y)}{|x-y|^{1+2s}}\,dxdy = - 2\int_{x_{i+1}}^{x_{i+2}}\int_{x_{i-1}}^{x_i} \frac{\left(1-\frac{|x-x_i|}{h}\right)\left(1-\frac{|y-x_{i+1}|}{h}\right)}{|x-y|^{1+2s}}\,dxdy.
\end{align*}

This integral can be computed applying again \eqref{cv3}, and we get
\begin{align}\label{Q3}
	Q_3 = -2h^{1-2s}\int_0^1\int_{-1}^0 \frac{(1+x)(1-y)}{(y-x+1)^{1+2s}}\,dxdy = h^{1-2s}\frac{13-5\cdot 2^{3-2s}+3^{3-2s}+s(2^{4-2s}-14)+4s^2}{2s(1-2s)(1-s)(3-2s)},
\end{align}
if $s\neq 1/2$. If $s=1/2$, instead, we have 
\begin{align*}
	Q_3 &= -2\int_0^1\int_{-1}^0 \frac{(1+x)(1-y)}{(y-x+1)^2}\,dxdy = 1+9\ln 3-16\ln(2).
\end{align*}

Notice that this last value could have been computed directly from \eqref{Q3}, by taking the limit as $s\to 1/2$ in that expression, being this limit exactly $1+9\ln 3-16\ln(2)$.

\subsubsection*{Computation of $Q_4$}
In this case, we are in the intersection of the supports of $\phi_i$ and $\phi_{i+1}$. Therefore, we have
\begin{align*}
	Q_4 &= \int_{x_i}^{x_{i+1}}\int_{x_i}^{x_{i+1}} \frac{(\phi_i(x)-\phi_i(y))(\phi_{i+1}(x)-\phi_{i+1}(y))}{|x-y|^{1+2s}}\,dxdy. 
\end{align*}

Moreover, we notice that, this time, it is possible that $x=y$, meaning that $Q_4$ could be a singular integral. To deal with this difficulty, we will exploit the explicit definition of the basis function. We have (see also Fig. \ref{basis2})
\begin{align*}
	\phi_i(x) = 1-\frac{x-x_i}{h}, \;\;\; x\in (x_i,x_{i+1}),
	\\
	\phi_{i+1}(x) = \frac{x_{i+1}-x}{h}, \;\;\; x\in (x_i,x_{i+1}).
\end{align*}
\begin{figure}[h]
\figinit{0.8pt}
% Axes
\figpt 1:(-100,0) \figpt 2:(200,0)
\figpt 11:(-90,-10) \figpt 12:(-90,140)
%%%%
\figpt 3:(-50,0) \figpt 4:(25,100) 
\figpt 5:(-50,100) \figpt 6:(25,0)
%
% Points for writing
\figpt 7:(-50,-10) \figpt 8:(25,110) 
\figpt 9:(-50,110) \figpt 10:(25,-10)

\figpt 13:(-100,130) \figpt 14:(180,-10)
\figpt 15:(-20,82) \figpt 16:(-22,10)

% 2. Creation of the graphical file
\figdrawbegin{}
\figdrawarrow[1,2]
\figset (color=\Redrgb)
\figdrawline[3,4]
\figset (color=default)
\figdrawline[5,6]
\figset(dash=4)
\figdrawline[4,6]
\figdrawarrow[11,12]
\figdrawline[3,5]
\figdrawend

% 3. Writing text on the figure
\figvisu{\figBoxA}{}{
\figwritec [7]{$(x_i,0)$}
\figwritec [8]{$(x_{i+1},1)$}
\figwritec [9]{$(x_i,1)$}
\figwritec [10]{$(x_{i+1},0)$}
\figwritec [13]{$y$}
\figwritec [14]{$x$}
\figwritec [15]{$\phi_i(x)$}
\figwritec [16]{$\color{red}\phi_{i+1}(x)$}
}
\centerline{\box\figBoxA}
\caption{Functions $\phi_i(x)$ and $\phi_{i+1}(x)$ on the interval $(x_i,x_{i+1})$.}\label{basis2}
\end{figure}

%\begin{figure}[h]
%	\centering
%	\includegraphics[scale=1]{figure15.eps}
%	\caption{Functions $\phi_i(x)$ and $\phi_{i+1}(x)$ on the interval $(x_i,x_{i+1})$.}\label{basis2}
%\end{figure}
$\newline$
Therefore, 
\begin{align*}
	(\phi_i(x)-\phi_i(y))(\phi_{i+1}(x)-\phi_{i+1}(y)) = \left(\frac{y-x}{h}\right)\left(\frac{x-y}{h}\right) = -\frac{|x-y|^2}{h^2},
\end{align*}
and the integral becomes
\begin{align*}
	Q_4 &= -\int_{x_i}^{x_{i+1}}\int_{x_i}^{x_{i+1}} |x-y|^{1-2s}\,dxdy = -\frac{h^{1-2s}}{(1-s)(3-2s)}. 
\end{align*}

\subsubsection*{Computation of $Q_5$}
Here the procedure is analogous to the one for $Q_2$ before. Using again Fubini's theorem we have
\begin{align*}
	Q_5 &= 2\int_{x_i}^{x_{i+1}}\phi_i(y)\phi_{i+1}(y)\left(\int_{-\infty}^{x_i} \frac{dx}{|x-y|^{1+2s}}\right)dy - 2\int_{x_i}^{x_{i+1}}\int_{x_{i-1}}^{x_i} \frac{\phi_i(x)\phi_{i+1}(y)}{|x-y|^{1+2s}}\,dxdy 
	\\
	&= \frac{1}{s}\int_{x_i}^{x_{i+1}}\frac{\phi_i(y)\phi_{i+1}(y)}{(y-x_i)^{2s}}\,dy - 2\int_{x_i}^{x_{i+1}}\int_{x_{i-1}}^{x_i} \frac{\phi_i(x)\phi_{i+1}(y)}{|x-y|^{1+2s}}\,dxdy. 
\end{align*}
Applying again \eqref{cv3}, it is now easy to check that $Q_5=Q_2$.

\subsubsection*{Computation of $Q_6$}
In analogy with what we did for $Q_1$, we can show that also $Q_6=0$.

\subsubsection*{Conclusion}
The elements $a_{i,i+1}$ are now given by the sum $2Q_2+Q_3+Q_4$, according to the corresponding values that we computed. In particular, we have
\begin{align*}
	a_{i,i+1} = \begin{cases}
					\displaystyle h^{1-2s}\frac{3^{3-2s}-2^{5-2s}+7}{2s(1-2s)(1-s)(3-2s)}, & \displaystyle s\neq \frac{1}{2}
					\\
					9\ln 3-16\ln 2, & \displaystyle s=\frac{1}{2}.
				\end{cases}	
\end{align*}

\subsubsection*{Step 3: $j= i$}
As a last step, we fill the diagonal of the matrix $\mathcal A_h$. In this case we have
	\begin{align*}
	a_{i,i}= & \int_{\RR}\int_{\RR}\frac{(\phi_i(x)-\phi_i(y))^2}{|x-y|^{1+2s}}\,dxdy
	\\
	= & \int_{x_{i+1}}^{+\infty}\int_{x_{i+1}}^{+\infty} \ldots\,dxdy + 2\int_{x_{i+1}}^{+\infty}\int_{x_{i-1}}^{x_{i+1}} \ldots\,dxdy + \int_{x_{i+1}}^{+\infty}\int_{-\infty}^{x_{i-1}} \ldots\,dxdy 
	\\
	& + \int_{x_{i-1}}^{x_{i+1}}\int_{x_{i-1}}^{x_{i+1}} \ldots\,dxdy + 2\int_{-\infty}^{x_{i-1}}\int_{x_{i-1}}^{x_{i+1}} \ldots\,dxdy + + \int_{-\infty}^{x_{i-1}}\int_{x{i+1}}^{+\infty} \ldots\,dxdy 
	\\
	& +  \int_{-\infty}^{x_{i-1}}\int_{-\infty}^{x_{i-1}} \ldots\,dxdy := R_1 + R_2 + R_3 + R_4 + R_5 + R_6 + R_7.
\end{align*}

In Fig. \ref{upp_dia}, we give a scheme of the regions of interactions between the basis functions $\phi_i(x)$ and $\phi_i(y)$ enlightening the domain of integration of the $R_i$. The regions in grey are the ones that produce a contribution to $a_{i,i}$, while on the regions in white the integrals will be zero.
\begin{figure}
\figinit{0.7pt}
% Axes
\figpt 0:(-50,70)
\figpt 1:(-100,80) \figpt 2:(220,80)
\figpt 3:(-40,-20) \figpt 4:(-40,290)

%%%%
\figpt 5:(40,0) \figpt 6:(40,270)
\figpt 7:(65,0) \figpt 8:(65,270)
\figpt 9:(90,0) \figpt 10:(90,270)
%
\figpt 11:(-80,160) \figpt 12:(200,160)
\figpt 13:(-80,185) \figpt 14:(200,185)
\figpt 15:(-80,210) \figpt 16:(200,210)

\figpt 53:(65,250) \figpt 54:(65,250)
\figpt 55:(65,230) \figpt 56:(65,25)
\figpt 57:(65,41) \figpt 58:(200,210)

\figpt 59:(-25,185) \figpt 60:(140,185)
\figpt 61:(-5,185) \figpt 62:(160,185)
\figpt 63:(65,230) \figpt 64:(65,25)

%

% Grey part 
\figpt 29:(40,210) \figpt 30:(90,210)
\figpt 31:(40,160) \figpt 32:(90,160)

% Diagonal 
\figpt 33:(-40,80) \figpt 34:(150,270)

% Points for writing
\figpt 35:(-49,164) \figpt 36:(-47,189) \figpt 37:(-49,215) 
\figpt 38:(30,74) \figpt 39:(58,74) \figpt 40:(80,74) 

\figpt 41:(150,240) \figpt 42:(65,240) \figpt 43:(150,185) 
\figpt 44:(-15,240) \figpt 45:(52.5,172.5) \figpt 46:(77.5,172.5) 
\figpt 47:(52.5,195.5) \figpt 48:(77.5,195.5) \figpt 49:(-15,185) 
\figpt 50:(65,30) \figpt 51:(150,30) \figpt 52:(-15,30) 

\figpt 73:(-52,280) \figpt 74:(210,70)


% 2. Creation of the graphical file
\figdrawbegin{}

\figset(fillmode=yes, color=0.9)
\figdrawline[29,30,10,6,29]
\figdrawline[30,32,12,16,30]
\figdrawline[32,31,5,9,32]
\figdrawline[31,11,15,29,31]
\figset(fillmode=yes, color=0.6)
\figdrawline[29,30,32,31,29]
\figset(fillmode=no, color=0)
\figdrawarrow[1,2]
\figdrawarrow[3,4]
\figdrawline[5,6]
\figdrawline[9,10]
\figdrawline[11,12]
\figdrawline[15,16]
\figset(dash=4)
\figdrawline[7,56]
\figdrawline[57,55]
\figdrawline[53,8]
\figdrawline[13,59]
\figdrawline[60,61]
\figdrawline[62,14]
\figset(dash=5)
\figdrawline[33,34]
\figdrawend

% 3. Writing text on the figure
\figvisu{\figBoxA}{}{
\figwritec [0]{$O$}
\figwritec [35,38]{$x_{i-1}$}
\figwritec [36,39]{$x_i$}
\figwritec [37,40]{$x_{i+1}$}
\figwritec [73]{$y$}
\figwritec [41]{$R_1$}
\figwritec [42,43]{$R_2$}
\figwritec [44]{$R_3$}
\figwritec [45]{$R_4^1$}
\figwritec [46]{$R_4^2$}
\figwritec [47]{$R_4^3$}
\figwritec [48]{$R_4^4$}
\figwritec [49,50]{$R_5$}
\figwritec [51]{$R_6$}
\figwritec [52]{$R_7$}
}
\centerline{\box\figBoxA}
\caption{Interactions between the basis function $\phi_i(x)$ and $\phi_i(y)$.}\label{dia}
\end{figure}
%\begin{figure}
%	\centering
%	\includegraphics[scale=1]{figure16.eps}
%	\caption{Interactions between the basis function $\phi_i(x)$ and $\phi_i(y)$.}\label{dia}
%\end{figure}

Le us now compute the terms $R_i$, $i=1,\ldots,7$, separately. First of all, according to Fig. \ref{dia} we have that $R_1=R_3=R_6=R_7=0.$ This is due to the fact that the corresponding regions are all away from the support of the basis functions. 
\subsubsection*{Computation of $R_2$}
Since $\phi_i(y) = 0$ on the domain of integrations we have
\begin{align*}
	R_2 &= 2\int_{x_{i+1}}^{+\infty}\int_{x_{i-1}}^{x_{i+1}} \frac{\phi_i^2(x)}{|x-y|^{1+2s}}\,dxdy = 2\int_{x_{i-1}}^{x_{i+1}}\phi_i^2(x)\left(\int_{x_{i+1}}^{+\infty} \frac{dy}{|x-y|^{1+2s}}\right)\,dx = \frac{1}{s}\int_{x_{i-1}}^{x_{i+1}}\frac{\phi_i^2(x)}{(x_{i+1}-x)^{2s}}\,dxdy.
\end{align*}

This integral is computed employing \eqref{cv}, and we obtain
\begin{align*}
	R_2 = \frac{h^{1-2s}}{s}\int_{-1}^1 \frac{(1-|x|\,)^2}{(1-x)^{2s}}\,dx = h^{1-2s}\frac{4s-6+2^{3-2s}}{s(1-2s)(1-s)(3-2s)}, 
\end{align*}
if $s\neq 1/2$. If $s=1/2$, instead, we have
\begin{align*}
	R_2 = 2\int_{-1}^1 \frac{(1-|x|\,)^2}{1-x}\,dx = 2\ln 16-4.
\end{align*}

\subsubsection*{Computation of $R_4$}
In this case, we are in the intersection of the supports of $\phi_i(x)$ and $\phi_i(y)$. Therefore, we have
\begin{align*}
	R_4 &= \int_{x_{i-1}}^{x_{i+1}}\int_{x_{i-1}}^{x_{i+1}} \frac{(\phi_i(x)-\phi_i(y))^2}{|x-y|^{1+2s}}\,dxdy. 
\end{align*}

In order to compute this integral, we divide it in four components as follows:
\begin{align*}
	R_4 &= \int_{x_{i-1}}^{x_i}\int_{x_{i-1}}^{x_i} \ldots\,dxdy + \int_{x_{i-1}}^{x_i}\int_{x_i}^{x_{i+1}} \ldots\,dxdy +
	\int_{x_i}^{x_{i+1}}\int_{x_{i-1}}^{x_i} \ldots\,dxdy +  \int_{x_i}^{x_{i+1}}\int_{x_i}^{x_{i+1}} \ldots\,dxdy
	\\
	&= R_4^1 + R_4^2 + R_4^3 + R_4^4.
\end{align*}

Moreover, we notice that, due to symmetry reason, we have $R_4^2 = R_4^3$. Therefore, we can compute only one of this two terms and add its value twice when building the matrix $\mathcal A_h$. Also, notice that in these two region it cannot happen that $x=y$. On the other hand, $R_4^1$ and $R_4^4$ may be singular integrals, and we shall deal with them as we did before. 

\subsubsection*{Computation of $R_4^1$}
Using again the explicit expression of the basis functions we find 
\begin{align*}
	(\phi_i(x)-\phi_i(y))^2 = \frac{|x-y|^2}{h^2},
\end{align*}
and the integral becomes
\begin{align*}
	R_4^1 &= \int_{x_{i-1}}^{x_i}\int_{x_{i-1}}^{x_i} |x-y|^{1-2s}\,dxdy = \frac{h^{1-2s}}{(1-s)(3-2s)}. 
\end{align*}

\subsubsection*{Computation of $R_4^2$}
In this case, we simply have
\begin{align*}
	R_4^2 &= \int_{x_{i-1}}^{x_i}\int_{x_i}^{x_{i+1}} \frac{(\phi_i(x)-\phi_i(y))^2}{|x-y|^{1+2s}}\,dxdy.
\end{align*}
Employing \eqref{cv} we obtain
\begin{align*}
	R_4^2 = h^{1-2s}\int_{-1}^0\int_0^1 \frac{(x+y)^2}{(x-y)^{1+2s}}\,dxdy = h^{1-2s}\frac{2s^2-5s+4-2^{2-2s}}{s(1-2s)(1-s)(3-2s)}, 
\end{align*}
if $s\neq 1/2$. If $s=1/2$, instead, we get
\begin{align*}
	R_4^2 = \int_{-1}^0\int_0^1 \frac{(x+y)^2}{(x-y)^2}\,dxdy = 3-4\ln 2.
\end{align*}
\subsubsection*{Computation of $R_4^4$}
Also in this case we can use the explicit expression of the basis functions and the integral becomes
\begin{align*}
	R_4^4 &= \int_{x_i}^{x_{i+1}}\int_{x_i}^{x_{i+1}} |x-y|^{1-2s}\,dxdy = \frac{h^{1-2s}}{(1-s)(3-2s)}=R_4^1. 
\end{align*}
Adding the values that we just computed, we therefore obtain
\begin{align*}
	R_4 = 2(R_4^1+R_4^2) = \begin{cases}
					\displaystyle h^{1-2s}\frac{8-8s-2^{3-2s}}{2s(1-2s)(1-s)(3-2s)}, & \displaystyle s\neq \frac{1}{2}
					\\
					8\ln 3-8\ln 2, & \displaystyle s=\frac{1}{2}.
				\end{cases}	
\end{align*}

\subsubsection*{Computation of $R_5$}
Since, once again, $\phi_i(y) = 0$ on the domain of integration we have
\begin{align*}
	R_5 &= 2\int_{-\infty}^{x_{i-1}}\int_{x_{i-1}}^{x_{i+1}} \frac{\phi_i^2(x)}{|x-y|^{1+2s}}\,dxdy = 2\int_{x_{i-1}}^{x_{i+1}}\phi_i^2(x)\left(\int_{-\infty}^{x_{i-1}} \frac{dy}{|x-y|^{1+2s}}\right)\,dx = \frac{1}{s}\int_{x_{i-1}}^{x_{i+1}}\frac{\phi_i^2(x)}{(x-x_{i-1})^{2s}}\,dxdy.
\end{align*}
Employing one last time \eqref{cv}, we get
\begin{align*}
	R_5 = \frac{h^{1-2s}}{s}\int_{-1}^1 \frac{(1-|x|\,)^2}{(1+x)^{2s}}\,dx = h^{1-2s}\frac{4s-6+2^{3-2s}}{s(1-2s)(3-2s)(1-s)}=R_2,
\end{align*}
if $s\neq 1/2$. If $s=1/2$, instead, we have
\begin{align*}
	R_5 = 2\int_{-1}^1 \frac{(1-|x|\,)^2}{1+x}\,dx = 8\ln 2-4.
\end{align*}

\subsubsection*{Conclusion}
The elements $a_{i,i}$ are now given by the sum $2R_2+R_4$, according to the corresponding values that we computed. In particular, we have 
\begin{align*}
	a_{i,i} = \begin{cases}
			\displaystyle h^{1-2s}\,\frac{2^{3-2s}-4}{s(1-2s)(1-s)(3-2s)}, & \displaystyle s\neq\frac{1}{2}
			\\
			\\
			8\ln 2, & \displaystyle s=\frac{1}{2}.			
			\end{cases}	
\end{align*}

}

\section*{Funding}
The work of Umberto Biccari was partially supported by the European Research Council [694126 DYCON (Dynamic control and numerics of partial differential equations)], by the Ministerio de Econom\'ia, Industria y Competividad (MINECO, Spain) [MTM2014-52347] and by the Air Force Office of Scientific Research [FA9550-15-1-0027]. 

The work of V\'ictor Hern\'andez-Santamar\'ia was partially supported by the European Research Council [694126 DYCON (Dynamic control and numerics of partial differential equations)].
 

\section*{Acknowledgements}
The authors wish to acknowledge F. Boyer (Institut de Math\'ematiques de Toulouse) for fruitful discussion on the numerical implementation of the control problem. Moreover, a special thanks goes to J. Loh\'eac (Laboratoire de Sciences Num\'eriques de Nantes), for helping with some computations in Section \ref{fe_sec}.  

\bibliography{biblio}

\end{document}
