\documentclass[preprint,1p]{amsart}

\usepackage{amsfonts}
\usepackage{amsmath}
\usepackage{amsthm}
\usepackage[utf8]{inputenc}
\usepackage[english]{babel}
\usepackage{epstopdf}
\usepackage{bmpsize}
\usepackage{epsfig}
\usepackage{hyperref}
\usepackage[english]{varioref}
\usepackage{amssymb}
\usepackage{multicol}
\usepackage{dcolumn}
\usepackage{geometry}
\usepackage{fancyhdr}
\usepackage[mathcal]{eucal}
\usepackage{mathrsfs}
\usepackage{color, colortbl}
\usepackage{microtype}
\usepackage{longtable}
\usepackage[toc,page]{appendix}
\usepackage{bm}
\usepackage{pifont}
\usepackage{fleqn}
\usepackage{graphicx}
\usepackage{txfonts}
\usepackage{subfig}
%\usepackage{refcheck}
%\usepackage[pagewise]{lineno}\linenumbers
\usepackage[pagewise]{lineno}
\DeclareMathAlphabet{\mathbbm}{U}{bbm}{m}{n}% from bbm.sty
%
\input{fig4tex.tex}
%
\allowdisplaybreaks
\DisableLigatures{encoding = *, family = * }
%
\numberwithin{equation}{section}
%\pagestyle{myheadings}
%
\newtheorem{theorem}{Theorem}[section]
\newtheorem{proposition}{Proposition}[section]
\newtheorem{lemma}{Lemma}[section]
\newtheorem{definition}{Definition}[section]
\newtheorem{remark}{Remark}[section]
%
\numberwithin{equation}{section}
\numberwithin{theorem}{section}
\numberwithin{remark}{section}
\numberwithin{lemma}{section}
\numberwithin{proposition}{section}
\numberwithin{definition}{section}
%
\newcommand{\norm}[2]{{\left\|#1\right\|}_{#2}}
\newcommand{\fl}[2]{(-d_x^{\,2})^{#1}#2}
\newcommand{\rfl}[2]{A^{#1}_{\Omega}#2}
\newcommand{\hp}[1]{\hphantom{#1}}
\newcommand{\cns}{c_{N,s}}
\newcommand{\ccs}{c_{1,s}}
\newcommand{\ffl}[2]{(-d_x^{\,2})^{#1}#2}
\newcommand{\flh}[2]{\frac{1}{\Gamma(-s)}\int_0^{+\infty}\Big(e^{t\Delta}#2 - #2\Big)\frac{dt}{t^{1+#1}}}
\newcommand{\kernel}[1]{|x-y|^{#1}}
\newcommand{\dkj}{\delta_{kj}}
\newcommand{\intr}[1]{\underset{#1}{\int}}
\newcommand{\Do}[1]{D_{#1}}
\newcommand{\Hs}{H^s_0(\Omega)}
\newcommand{\ue}[1]{#1^{\,\varepsilon}}
\newcommand{\xHdot}[1]{\dot{H}^{#1}}
\newcommand{\ha}[2]{\mathbf{H}_{#1}^{#2}}
\newcommand{\lhi}{\mathcal{L}_i^h}
\newcommand{\NN}{\mathbb{N}}
\newcommand{\ZZ}{\mathbb{Z}}
\newcommand{\RR}{\mathbb{R}}
\newcommand{\CC}{\mathbb{C}}
\newcommand{\TT}{\mathbf{T}}
\newcommand\inter[1]{\llbracket #1\rrbracket}
\newcommand\mesh{\mathfrak{M}}
\newcommand{\rouge}[1]{\color{red}#1\color{black}}
\newcommand{\lop}[2]{\mathcal{L}_T(#1,#2)}
\newcommand{\loh}[2]{\mathcal{L}^h_T\left(#1,#2\right)}

%
\usepackage{tikz}
\usepackage{pgfplots}
\usepackage{pgfplotstable}
\usetikzlibrary{matrix,external,fit}
%
 \pgfplotsset{surface/.style={ %
               xmax=0.34,%
               axis z line=center,%
               axis x line=center,%
               axis y line=center,%
               %axis on top,%
               zmin=-1,%
               clip=false,%
               extra x ticks={1},%
               extra x tick label={$T=1$},%
               xtick={0},%
               ytick=\empty}}
%
%
\newcommand{\controldomain}[2]{ \addplot3 [surf,fill=violet,mesh/rows=2] coordinates {(0,#1,0) (1,#1,0) (0,#2,0) (1,#2,0)}; }
\newcommand{\controldomainT}[3]{ \addplot3 [surf,fill=violet,mesh/rows=2] coordinates {(0,#1,0) (#3,#1,0) (0,#2,0) (#3,#2,0)}; }

\newcommand{\couplingdomain}[2]{ \addplot3 [surf,fill=green,mesh/rows=2] coordinates {(0,#1,0) (1,#1,0) (0,#2,0) (1,#2,0)}; }
\newcommand{\couplingdomainT}[3]{ \addplot3 [surf,fill=green,mesh/rows=2] coordinates {(0,#1,0) (#3,#1,0) (0,#2,0) (#3,#2,0)}; }
%
 \pgfplotsset{erreurs/.style={scale=1,
     legend cell align=left,
     legend pos=outer north east,
     legend plot pos=right,
     legend style={cells={anchor=east},draw=none},
     xlabel=$h$,
    xmin=0.001,xmax=0.05}}
%
\tikzset{pente/.style={opacity=0.6}}
%
\pgfplotsset{cout/.style={black,mark=diamond*,mark size=2.5,mark options={fill=gray}}}
\pgfplotsset{cible/.style={black,mark=square*,mark size=2.5,mark options={fill=gray}}}
\pgfplotsset{cibleyT/.style={black,mark=*,mark size=2.5,mark options={fill=gray}}}
\pgfplotsset{CG/.style={black,mark=otimes*,mark size=2.5,mark options={fill=gray}}}
%\pgfplotsset{solex/.style={black,mark=diamond*,mark size=2.5,mark options={fill=gray}}}
\pgfplotsset{solex/.style={black,mark=*,mark size=2.5,mark options={fill=gray}}}
\pgfplotsset{energie/.style={black,mark=triangle*,mark size=2.5,mark options={fill=gray}}}
%
\newcounter{quotecount}
\newcommand{\MyQuote}[1]{\vspace{0.5cm}%\refstepcounter{quotecount}{1}%
     \parbox{12cm}{\em #1}\hspace*{1cm}($\mathcal{P}$)\\[0.5cm]}
 
 \newcommand{\NewQuote}[1]{\vspace{0.3cm}%\refstepcounter{quotecount}{1}%
   	\parbox{14cm}{\em #1}\\[0.3cm]}
%
%\input{macros_franck.tex}
%
\title[Controllability fractional heat equation]{Controllability of a one-dimensional fractional heat equation: theoretical and numerical aspects}
%
\author{U.~Biccari}
\address{Umberto Biccari, DeustoTech, University of Deusto, 48007 Bilbao, Basque Country, Spain.}
\address{Umberto Biccari, Facultad Ingenier\'{\i}a, Universidad de Deusto, Avda Universidades 24, 48007 Bilbao, Basque Country, Spain.}
\email{umberto.biccari@deusto.es,u.biccari@gmail.com}
%%
\author{V.~Hern\'andez-Santamar\'ia}
\address{Victor Hern\'andez-Santamar\'ia, DeustoTech, University of Deusto, 48007 Bilbao, Basque Country, Spain.}
\address{Victor Hern\'andez-Santamar\'ia, Facultad Ingenier\'{\i}a, Universidad de Deusto, Avda Universidades 24, 48007 Bilbao, Basque Country, Spain.}
\email{victor.santamaria@deusto.es}
%
\thanks{The work of Umberto Biccari was partially supported by the Advanced Grant DYCON (Dynamic Control) of the European Research Council Executive Agency, by the MTM2014-52347 Grant of the MINECO (Spain) and by the Air Force Office of Scientific Research under the Award No: FA9550-15-1-0027. The work of V\'ictor Hern\'andez-Santamar\'ia was partially supported by the Advanced Grant DYCON (Dynamic Control) of the European Research Council Executive Agency.}

%\input{packages.tex}

\begin{document}

\bibliographystyle{acm}

\maketitle

\begin{abstract}
We analyze the controllability problem for a one-dimensional heat equation involving the fractional Laplacian $\fl{s}{}$ on the interval $(-1,1)$. Using classical results and techniques, we show that, acting from an open subset $\omega\subset(-1,1)$, the problem is null-controllable for $s>1/2$ and that for $s\leq 1/2$ we only have approximate controllability. Moreover, we deal with the numerical computation of the control employing the penalized Hilbert Uniqueness Method (HUM) and a finite element (FE) scheme for the approximation of the solution to the corresponding elliptic equation. We present several experiments confirming the expected controllability properties.
\end{abstract}

\section{Introduction and main results}\label{intro_sec}
Let $\omega\subset (-1,1)$ be an open and nonempty subset. In this work, we analyze controllability properties for the following nonlocal one-dimensional heat equation 
\begin{align}\label{heat_frac}
	\begin{cases}
		z_t + \fl{s}{z} = g\mathbf{1}_{\omega},\quad & (x,t)\in (-1,1)\times(0,T)
		\\
		z=0, & (x,t)\in[\,\RR\setminus (-1,1)\,]\times(0,T)
		\\
		z(x,0)=z_0(x), & x\in (-1,1),
	\end{cases}
\end{align} 
where $z_0\in L^2(-1,1)$ is a given initial datum. In more detail, we are interested in the resolution of the following control problem: given any $T>0$, find a control function $g\in L^2(\omega\times(0,T))$ such that the corresponding solution to \eqref{heat_frac} satisfies $z(x,T)=0$. 

In \eqref{heat_frac}, for all $s\in(0,1)$, $\fl{s}{}$ denotes the one-dimensional fractional Laplace operator, which is defined as the following singular integral
\begin{align}\label{fl_intro}
	\fl{s}{u}(x) = \ccs\,P.V.\,\int_{\RR}\frac{u(x)-u(y)}{|x-y|^{1+2s}}\,dy. 
\end{align}
Here, $\ccs$ is a normalization constant given by
\begin{align*}
	\ccs = \frac{s2^{2s}\Gamma\left(\frac{1+2s}{2}\right)}{\sqrt{\pi}\Gamma(1-s)},
\end{align*}
where $\Gamma$ is the usual Gamma function. Moreover, we have to mention that, for having a completely rigorous definition of the fractional Laplace operator, it is necessary to introduce also the class of functions $u$ for which computing $\fl{s}{u}$ makes sense. We postpone this discussion to the next section.

The analysis of non-local operators and non-local PDEs is a topic in continuous development.
A motivation for this growing interest relies in the large number of possible applications in the modeling of several complex phenomena for which a local approach turns up to be inappropriate or limiting.
Indeed, there is an ample spectrum of situations in which a non-local equation gives a
significantly better description than a PDE of the problem one wants to analyze.
Among others, we mention applications in turbulence (\cite{bakunin2008turbulence}), anomalous transport and diffusion (\cite{bologna2000anomalous,meerschaert2012fractional}), elasticity (\cite{dipierro2015dislocation}), image processing (\cite{gilboa2008nonlocal}), porous media flow (\cite{vazquez2012nonlinear}), wave propagation in heterogeneous high contrast media (\cite{zhu2014modeling}). Also, it is well known that the fractional Laplacian is the generator of s-stable processes, and it is often used in stochastic models with applications, for instance, in mathematical finance (\cite{levendorskii2004pricing,pham1997optimal}).

One of the main differences between these non-local models and classical Partial Differential Equations is that the fulfillment of a non-local equation at a point involves the values of the function far away from that point.

It is well-known that the classical local heat equation (as well as many more general variants) is null-controllable in any time $T>0$ (see, e.g., \cite{fattorini1971exact,fursikov1996controllability,lebeau1995controle}). Nevertheless, to the best of our knowledge, there are few results in the literature on the null-controllability of the fractional heat equation, and none of them is for a problem involving the fractional Laplacian in its integral form \eqref{fl_intro}. The existing results (\cite{micu2006controllability,miller2006controllability}), instead, deal with the so-called \textit{spectral} fractional Laplace operator, whose definition will be given later. 

In this paper, we deal with the controllability of \eqref{heat_frac}, both from the theoretical and the numerical point of view. Employing spectral analysis techniques based on the works \cite{kulczycki2010spectral,kwasnicki2012eigenvalues}, the first main result that we obtain is the following

\begin{theorem}\label{null_control_thm}
Given any $z_0\in L^2(-1,1)$ the parabolic problem \eqref{heat_frac} is null-controllable at time $T>0$ with a control function $g\in L^2(\omega\times(0,T))$ if and only if $s>1/2$.  
\end{theorem}

Furthermore, even if for $s\leq 1/2$ null controllability for \eqref{heat_frac} fails, we still have the following result of approximate controllability, as a consequence of unique continuation properties for the fractional Laplace operator (\cite{fall2014unique}). 

\begin{theorem}\label{approx_control_thm}
Let $s\in(0,1)$. Given any $z_0\in L^2(-1,1)$, there exists a control function $g\in L^2(\omega\times(0,T))$ such that the unique solution $z$ to the parabolic problem \eqref{heat_frac} is approximately controllable at time $T>0$.
\end{theorem}

Theorems \ref{null_control_thm} and \eqref{approx_control_thm} will then find a confirmation in the study of the corresponding numerical control problem. With this purpose, we will employ the penalized Hilbert Uniqueness Method, which relies on some classical works of Glowinski and Lions (\cite{glowinski1995exact,glowinski2008exact}). 

Notice that this method is very general, and it may by applied to a broad class of PDEs control problems (\cite{boyer2013penalised,boyer2017insensitizing,boyer2014approximate,khodja2017partial}). When using it for treating a nonlocal problem as \eqref{heat_frac}, new issues arise related to the discretization of the corresponding elliptic problem.    

In this framework, a preliminary step for the for the resolution of the numerical control problem will be a finite element (FE)  approximation of the solution to the following non-local Poisson equation
\begin{align}\label{PE}
	\begin{cases}
		\fl{s}{u} = f, & x\in(-L,L)
		\\
		u\equiv 0, & x\in\RR\setminus(-L,L).
	\end{cases}
\end{align}

In the recent past, the fractional Laplacian has been widely analyzed also from the point of view of numerical analysis. We refer, for instance, to the works \cite{acosta2017short,acosta2017fractional,borthagaray2017laplaciano}. There, the authors present a FE scheme for implementing the solution of \eqref{PE} in a bounded domain $\Omega\subset\RR^2$. In particular, they provide appropriate quadrature rules in order to solve numerically the variational formulation associated to the problem. Moreover, in \cite{acosta2017fractional,borthagaray2017laplaciano} it is also developed an accurate analysis of the efficiency of the FE method, employing several existing results. The techniques of the aforementioned works have then been applied in \cite{acosta2017finite}, combined with a convolution quadrature approach, for solving evolution equations involving the fractional Laplacian. For the sake of completeness, we also mention \cite{bonito2015numerical}, where it is presented a discretization of the spectral fractional Laplacian and its application to the evolutionary case \cite{bonito2017approximation}, and \cite{nochetto2015pde}, where the same problem is treated applying the well known extension of Caffarelli and Silvestre (\cite{caffarelli2007extension}).   

Our method deals with a FE approximation in one space dimension for the fractional Poisson equation. The main novelty of our work, with respect to \cite{acosta2017short,acosta2017fractional,borthagaray2017laplaciano}, relies on the fact that, since we are dealing with the one-dimensional case, we will not need any quadrature rule and each entry of the stiffness matrix can be computed explicitly. This has the great advantage of significantly reducing the computational cost of the algorithm and, therefore, our discretization method is suitable for being included in more general applications.  

\rouge{This paper is organized as follows. In Section \ref{theor_sec}, we present some existing theoretical results for the problems that we are going to analyze. In particular, we give a more accurate definition of the fractional Laplace operator and we introduce the variational formulation associated to \eqref{PE} (needed for the development of the FE scheme). Concerning the parabolic problem \eqref{heat_frac}, we present a couple of controllability results, which will help us in the verification of the accuracy of the numerical method. In Section \ref{fe_sec}, we describe our FE method for the elliptic equation \eqref{PE} and we present the algorithm for the penalized HUM, employed for the numerical control of \eqref{heat_frac}. In Section \ref{res_numerical} we present and comment the results of our numerical simulations. Finally, in Appendix \ref{appendix} we include the complete details for computing the stiffness matrix associated to our FE scheme.}


\section{Preliminary results on the fractional Laplace operator}\label{theor_sec}

In this Section, we introduce some preliminary result that will be useful in the remainder of the paper.

We start by giving a more rigorous definition of the fractional Laplace operator, as we have anticipated in Section \ref{intro_sec}. Consider the space
\begin{align*}
	\mathcal L^1_s(\RR) :=\left\{ u:\RR\longrightarrow\RR\,:\; u\textrm{ measurable },\;\int_{\RR}\frac{|u(x)|}{(1+|x|)^{1+2s}}\,dx<\infty\right\}.
\end{align*}
For any $u\in\mathcal L_s^1$ and $\varepsilon>0$, we set 
\begin{align*}
	(-d_x^{\,2})^s_{\varepsilon}\, u(x) = \ccs\,\int_{|x-y|>\varepsilon}\frac{u(x)-u(y)}{|x-y|^{1+2s}}\,dy,\;\;\; x\in\RR.
\end{align*}
The fractional Laplacian is then defined by the following singular integral
\begin{align}\label{fl}
	\fl{s}{u}(x) = \ccs\,P.V.\,\int_{\RR}\frac{u(x)-u(y)}{|x-y|^{1+2s}}\,dy = \lim_{\varepsilon\to 0^+} (-d_x^2)^s_{\varepsilon} u(x), \;\;\; x\in\RR,
\end{align}
provided that the limit exists. 

We notice that if $0<s<1/2$ and $u$ is a smooth function, for example bounded and Lipschitz continuous on $\RR$, then the integral in \eqref{fl} is in fact not really singular near $x$ (see e.g. \cite[Remark 3.1]{dihitchhiker}). Moreover, $\mathcal L_s^1(\RR)$ is the right space for which $v:= (-d_x^{\,2})^s_{\varepsilon}\, u$ exists for every $\varepsilon > 0$, $v$ being also continuous at the continuity points of $u$.

It is by now well-known (see, e.g., \cite{dihitchhiker}) that the natural functional setting for problems involving the fractional Laplacian is the one of the fractional Sobolev spaces. Since these spaces are not as familiar as the classical integral order ones, for the sake of completeness, we recall here their definition. 

Given $s\in(0,1)$, the fractional Sobolev space $H^s(-1,1)$ is defined as
\begin{align*}
	H^s(-1,1):= \left\{u\in L^2(-1,1)\,:\, \frac{|u(x)-u(y)|}{|x-y|^{\frac 12+s}}\in L^2\Big((-1,1)\times(-1,1)\Big) \right\}.
\end{align*}

It is classical that this is a Hilbert space, endowed with the norm (derived from the scalar product)
\begin{align*}
	\norm{u}{H^s(-1,1)} := \left[\norm{u}{L^2(-1,1)}^2 + |u|_{H^s(-1,1)}^2\right]^{\frac 12},
\end{align*}
where the term 
\begin{align*}
	|u|_{H^s(-1,1)}:= \left(\int_{-1}^1\int_{-1}^1 \frac{|u(x)-u(y)|^2}{|x-y|^{1+2s}}\,dxdy\right)^{\frac 12}
\end{align*}
is the so-called Gagliardo seminorm of $u$. We set 
\begin{align*}
H_0^s(-1,1):= \overline{C_0^\infty(-1,1)}^{\,H^s(-1,1)}
\end{align*}
the closure of the continuous infinitely differentiable functions with compact support in $(-1,1)$ with respect to the $H^s(-1,1)$-norm. The following facts are well-known.
\begin{itemize}
	\item[$\bullet$] For $0<s\leq\frac 12$, the identity $H_0^s(-1,1) = H^s(-1,1)$ holds. This is because, in this case, the $C_0^\infty(-1,1)$ functions are dense in $H^s(-1,1)$ (see, e.g., \cite[Theorem 11.1]{jllions1972non}).
	
	\item[$\bullet$] For $\frac 12<s<1$, we have $H_0^s(-1,1)=\left\{ u\in H^s(\RR)\,:\,u=0\textrm{ in } \RR\setminus (-1,1)\right\}$ (\cite{fiscella2015density}).
\end{itemize}

Finally, in what follows we will indicate with $H^{-s}(-1,1)=\left(H^s(-1,1)\right)'$ the dual space of $H^s(-1,1)$ with respect to the pivot space $L^2(-1,1)$.

A more exhaustive description of fractional Sobolev spaces and of their properties can be found in several classical references (see, e.g., \cite{adams2003sobolev,dihitchhiker,jllions1972non}).

To discuss the parabolic equation \eqref{heat_frac}, we firstly need to introduce the variational formulation associated to equation \eqref{PE}, namely: find $u\in H^s_0(-1,1)$ such that
\begin{align*}
a(u,v) = \int_{-1}^1 fv\,dx,	
\end{align*}
for all $v\in H_0^s(-1,1)$, where the bilinear form $a(\cdot,\cdot):H^s_0(-1,1)\times H^s_0(-1,1)\to \RR$ is given by
\begin{align}\label{bilinear}
	a(u,v)=\frac{\ccs}{2} \int_{\RR}\int_{\RR}\frac{(u(x)-u(y))(v(x)-v(y))}{|x-y|^{1+2s}}\,dxdy.	
\end{align}

Since the bilinear form $a$ is continuous and coercive, Lax-Milgram Theorem immediately implies existence and uniqueness of solutions to the Dirichlet problem \eqref{PE}. In more detail, if $f\in H^{-s}(-1,1)$, then \eqref{PE} admits a unique weak solution $u\in H_0^s(-1,1)$ (see, e.g., \cite[Proposition 2.1]{biccari2017local}). Furthermore, in the literature it is possible to find improved regularity results for the solution to \eqref{PE}, both in H\"older and Sobolev spaces. The interested reader may refer, for instance, to \cite{acosta2017fractional,biccari2017local,leonori2015basic,ros2014dirichlet,ros2014extremal}.

Concerning now equation \eqref{heat_frac}, first of all we recall the following definition of a weak solution (see \cite[Definition 25]{leonori2015basic}).
\begin{definition} 
We say that $z\in L^2(0,T;H_0^s(-1,1))\cap C([0,T],L^2(-1,1))$ with $z_t\in L^2(0,T;H^{-s}(-1,1))$ is a weak solution for the parabolic problem \eqref{heat_frac} with $g\in L^2(0,T;H^{-s}(-1,1))$ and $z_0\in L^2(-1,1)$ if it satisfies
\begin{align*}
	\int_0^T\int_{-1}^1 z_tw\,dxdt + \int_0^T a(z,w)\,dt = \int_0^T \langle f,w\rangle_{-s,s}\,dt,
\end{align*}
for any $w\in L^2(0,T;H_0^s(-1,1))$, where $a(\cdot,\cdot)$ is the bilinear form defined in \eqref{bilinear}.
\end{definition}

Moreover, we have the following.

\begin{theorem}[{\cite[Theorem 26]{leonori2015basic}}]
Assume that $f\in L^2(0,T;H^{-s}(-1,1))$. Then for any $z_0\in L^2(-1,1)$, problem \eqref{heat_frac} has a unique weak solution.
\end{theorem}
Notice that taking as in our case $g\in L^2(\omega\times(0,T))$, the same result holds due to the continuous injection of $L^2$ into $H^{-s}$. Finally, we mention that well posedness results in H\"older spaces and improved regularity properties for \eqref{heat_frac} have been obtained in \cite{biccari2017parabolic,fernandez2016boundary}.  

In this paper we are mainly interested in the study of control properties for the parabolic system \eqref{heat_frac}. For the sake of completeness, we include below the definitions of null and approximate controllability.

\begin{definition}
System \eqref{heat_frac} is said to be \textit{null-controllable} at time $T>0$ if, for any $z_0\in L^2(-1,1)$, there exists $g\in L^2(\omega\times(0,T))$ such that the corresponding solution $z$ satisfies $z(x,T)=0$.
\end{definition}

\begin{definition}
	System \eqref{heat_frac} is said to be \textit{approximately controllable} at time $T>0$ if, for any $z_0,z_T\in L^2(-1,1)$ and any $\delta>0$, there exists $g\in L^2(\omega\times(0,T))$ such that the corresponding solution $z$ satisfies 
	\begin{align*}
		\norm{z(x,T)-z_T}{L^2(-1,1)}\leq\delta.
	\end{align*}
\end{definition}

We already mentioned that, to the best of our knowledge, there are no results in the literature concerning the controllability of the fractional heat equation involving the integral operator \eqref{fl}. The existing ones deal with the \textit{spectral} definition of the fractional Laplace operator, which is given as follows.

Let $\{\psi_k,\lambda_k\}_{k\in\NN}\subset H_0^1(-1,1)\times\RR^+$ be the set of normalized eigenfunctions and eigenvalues of the Laplace operator in $(-1,1)$ with homogeneous Dirichlet boundary conditions, so that $\{\psi_k\}_{k\in\NN}$ is an orthonormal basis of $L^2(-1,1)$ and         
\begin{align*}
	\begin{cases}
		-d_x^2\psi_k =\lambda_k\psi_k, & x\in (-1,1), 
		\\
		\psi_k(-1)=\psi_k(1)=0.
	\end{cases}
\end{align*}

Then, the \textit{spectral fractional Laplacian} $(-d_x^{\,2})^s_S$ is defined by
\begin{align}\label{fl_spec}
	(-d_x^{\,2})^s_S u(x) = \sum_{k\geq 1}\langle u,\psi_k\rangle \lambda_k^s\psi_k(x),
\end{align}
firstly for $u\in C_0^{\infty}(-1,1)$ and then for $u\in H_0^s(-1,1)$ employing a density argument.

It is important to notice that the spectral fractional Laplacian and the fractional Laplacian defined as in \eqref{fl} are two different operators. For instance, definition \eqref{fl_spec} depends on the choice of the domain, while the integral definition does not. For a complete discussion on the differences of these two operators, we refer to \cite{servadei2014spectrum}.

The control problem for the fractional heat equation involving the operator $(-d_x^{\,2})^s_S$ has been analyzed in \cite{micu2006controllability}, where the authors proved null controllability provided that $s>1/2$. For $s\leq 1/2$, instead, null controllability does not hold, not even for $T$ large. This negative result is based on the equivalence (consequence of M\"untz Theorem, see, e.g., \cite[Page 24]{schwartz1958etude}) between the controllability property (more specifically, the possibility of proving an observability inequality), and the following condition for the eigenvalues of the considered operator 
\begin{align}\label{eigen_cond}
	\sum_{k\geq 1} \frac{1}{\lambda_k}<\infty,
\end{align} 
which is clearly not satisfied for the spectral fractional Laplacian when $s\leq 1/2$, since in that case the eigenvalues are $\lambda_k = (k\pi)^{2s}$. Finally, in \cite{miller2006controllability}, the same result as in \cite{micu2006controllability} is obtained in a multi-dimensional setting, by means of a  \textit{spectral observability condition} for a negative self-adjoint operator, which allows to prove the null-controllability of the semi-group that it generates.

As we anticipated in Section \ref{intro_sec}, the same null controllability result holds for the parabolic equation \eqref{heat_frac}.
This will be obtained by means of classical tools (\cite{fattorini1971exact}) and by an explicit approximations of the eigenvalues and the eigenfunctions the fractional Laplacian with homogeneous Dirichlet boundary conditions. We stress that \eqref{heat_frac} is a different model with respect to the ones analyzed in \cite{micu2004introduction,miller2006controllability}, since the operators \eqref{fl} and \eqref{fl_spec} are not equivalent. 

\section{Proof of the controllability properties}

This section is devoted to study the control properties for the parabolic system \eqref{heat_frac}. We begin by proving the null controllability result.

\begin{proof}[Proof of Theorem \ref{null_control_thm}]
First of all, for all $\varphi^T\in L^2(-1,1)$, let $\varphi(x,t)$ be the unique solution to the adjoint system
\begin{align}\label{heat_frac_adj}
	\begin{cases}
		-\varphi_t + \fl{s}{\varphi} = 0, & (x,t)\in (-1,1)\times(0,T)
		\\
		\varphi = 0, & (x,t)\in [\,\RR\setminus(-1,1)\,]\times(0,T)
		\\
		\varphi(x,T) = \varphi^T(x), & x\in (-1,1).
	\end{cases}
\end{align}

Multiplying \eqref{heat_frac} by $\varphi$ and integrating over $(-1,1)\times (0,T)$, it is straightforward to check that $z(x,T)=0$ if and only if 
\begin{align}\label{control_id}
	\int_0^T\int_{-1}^1 \varphi(x,t)g(x,t)\mathbf{1}_{\omega}(x)\,dxdt = -\int_{-1}^1 u_0(x)\varphi(x,0)\,dx,
\end{align}

In turn, it is classical that \eqref{control_id} is equivalent to the existence of a constant $C>0$ such that the following observability inequality holds
\begin{align}\label{obs}
	\norm{\varphi(x,0)}{L^2(-1,1)}^2\leq C\int_0^T\left|\,\int_{-1}^1 \varphi(x,t)g(x,t)\mathbf{1}_{\omega}(x)\,dx\,\right|^2\,dt,
\end{align}

Notice that $\varphi$ can be expressed in the basis of the eigenfunctions of the fractional Laplacian on $(-1,1)$ with zero Dirichlet boundary conditions. Namely,
\begin{align}\label{adj_sol}
	\varphi(x,t) = \sum_{k\geq 1} \varphi_ke^{-\lambda_k(T-t)}\varrho_k(x), 
\end{align}
where $\varphi_k = \langle \varphi^T,\varrho_k\rangle$ and, for $k\geq 1$, $\varrho_k(x)$ are the solutions to the following eigenvalue problem 
\begin{align*}
	\begin{cases}
		\fl{s}{\varrho_k} = \lambda_k\varrho_k, & x\in (-1,1), \;\; k\in\NN
		\\
		\varrho_k = 0, & x\in \,\RR\setminus(-1,1).
	\end{cases}
\end{align*}
	
Now, plugging \eqref{adj_sol} into \eqref{obs}, using the orthonormality of the eigenfunctions $\varrho_k$ as a basis of $L^2(-1,1)$ and employing the change of variables $T-t\mapsto t$, the observability inequality becomes 
\begin{align}\label{obs_spectr}
	\sum_{k\geq 1} |\varphi_k|^2e^{-2\lambda_k T} \leq C\int_0^T\left|\,\sum_{k\geq 1} \varphi_kg_k(t)e^{-\lambda_k t}\right|^2\,dt, 
\end{align}
where $g_k = \langle g\mathbf{1}_{\omega},\varrho_k\rangle$. 
	
By means of the classical moment method (\cite{fattorini1971exact}), inequalities of the form \eqref{obs_spectr} are well known to be true if and only if \eqref{eigen_cond} holds and the eigenfunctions $\varrho_k$ satisfy the following lower bound:
\begin{align}\label{eigenf_bound}
	\norm{\varrho_k}{L^2(\omega)}\geq C>0, \,\,\forall \, k\geq 1,
\end{align}
where the constant $C$ is independent of $k$. The proof of \eqref{eigenf_bound} is an easy adaptation of the one of \cite[Lemma 2]{kwasnicki2012eigenvalues}. Moreover, according to \cite{kulczycki2010spectral,kwasnicki2012eigenvalues} we have 
\begin{align*}
	\lambda_k = \left(\frac{k\pi}{2}-\frac{(1-s)\pi}{4}\right)^{2s}+O\left(\frac{1}{k}\right).
\end{align*}
	
Therefore, we easily see that the condition \eqref{eigen_cond} is satisfied if and only if $s>1/2$. If $s\leq 1/2$, instead, the series diverges, since it behaves as an harmonic series. In conclusion, the observability inequality \eqref{obs} holds true when $s>1/2$, but it is false when $s\leq 1/2$. This concludes the proof. 
\end{proof}

Even if for $s\leq 1/2$ null controllability for \eqref{heat_frac} fails, Theorem \ref{approx_control_thm} ensures that, for all $s\in(0,1)$, we still have approximate controllability. This is consequence of a unique continuation property for the fractional Laplacian, which has been obtained in \cite{fall2014unique}.

\begin{proof}[Proof of Theorem \ref{approx_control_thm}]
It is classical that the result is true as soon as one has the following unique continuation property for the solution to the adjoint equation \eqref{heat_frac_adj} (see, e.g., \cite[Theorem 5.2]{micu2004introduction}).
	
\MyQuote{Given $s\in(0,1)$ and $\varphi^T_0\in L^2(-1,1)$, let $\varphi$ be the unique solution to the system \eqref{heat_frac_adj}. Let $\omega\subset (-1,1)$ be an arbitrary open set. If $\varphi = 0$ on $\omega\times(0,T)$, then $\varphi = 0$ on $(-1,1)\times(0,T)$.}
	
Therefore, we are reduced to the proof of the property ($\mathcal P$). To this end, let us recall that $\varphi$ can be expressed in the form \eqref{adj_sol} and let us assume that 
\begin{align}\label{uc}
	\varphi=0 \textrm{ in } \omega\times(0,T). 
\end{align}
Let $\{\psi_{k_j}\}_{1\leq k\leq m_k}$ be an orthonormal basis of $\ker(\lambda_k-\fl{s}{})$. Then, \eqref{adj_sol} can be rewritten as
\begin{align*}
	\varphi(x,t) = \sum_{k\geq 1} \left(\sum_{j=1}^{m_k} \varphi_{k_j}\psi_{k_j}(x)\right)e^{-\lambda_k(T-t)}, \;\;\; (x,t)\in (-1,1)\times(-\infty, T). 
\end{align*}

Let $z\in\CC$ with $\eta:=\Re(z)>0$ and let $N\in\NN$. Since the functions $\psi_{k_j}$, $1\leq j\leq m_k$, $1\leq k\leq N$ are orthonormal, we have that
\begin{align*}
	\norm{\sum_{k=1}^N \left(\sum_{j=1}^{m_k} \varphi_{k_j}\psi_{k_j}(x)\right)e^{z(t-T)}e^{-\lambda_k(T-t)}}{L^2(-1,1)}^2 & \leq \sum_{k=1}^N \left(\sum_{j=1}^{m_k} |\varphi_{k_j}|^2\right)e^{2\eta(t-T)}e^{-2\lambda_k(T-t)}
	\\
	& \leq \sum_{k\geq 1} \left(\sum_{j=1}^{m_k} |\varphi_{k_j}|^2\right)e^{2\eta(t-T)}e^{-2\lambda_k(T-t)} \leq Ce^{2\eta(t-T)}\norm{\varphi^T}{L^2(-1,1)}^2.
\end{align*}
Hence, letting 
\begin{align*}
	w_N(x,t):= \sum_{k=1}^N \left(\sum_{j=1}^{m_k} \varphi_{k_j}\psi_{k_j}(x)\right)e^{z(t-T)}e^{-\lambda_k(T-t)},
\end{align*}
we have shown that $\norm{w_T(x,t)}{L^2(-1,1)}\leq Ce^{\eta(t-T)}\norm{\varphi^T}{L^2(-1,1)}$. Moreover, we have
\begin{align*}
	\int_{-\infty}^T e^{\eta(t-T)}\norm{\varphi^T}{L^2(-1,1)}\,dt = \frac{1}{\eta}\norm{\varphi^T}{L^2(-1,1)}\int_0^{+\infty} e^{-\tau}\,d\tau = \frac{1}{\eta}\norm{\varphi^T}{L^2(-1,1)}.
\end{align*}
Therefore, we can apply the Dominated Convergence Theorem, obtaining
\begin{align}\label{dct_identity}
	\lim_{N\to+\infty}\int_{-\infty}^T w_N(x,t)\,dt &= \int_{-\infty}^T\lim_{N\to+\infty} w_N(x,t)\,dt = \int_{-\infty}^T e^{z(t-T)} \sum_{k=1}^{+\infty} \left(\sum_{j=1}^{m_k} \varphi_{k_j}\psi_{k_j}(x)\right)e^{-\lambda_k(T-t)}\,dt \notag
	\\
	&=\sum_{k=1}^{+\infty}\sum_{j=1}^{m_k} \varphi_{k_j}\psi_{k_j}(x) \int_{-\infty}^T e^{z(t-T)}e^{-\lambda_k(T-t)}\,dt =\sum_{k=1}^{+\infty}\sum_{j=1}^{m_k} \varphi_{k_j}\psi_{k_j}(x) \int_0^{+\infty} e^{-(z+\lambda_k)\tau}\,d\tau \notag
	\\
	&=\sum_{k=1}^{+\infty}\sum_{j=1}^{m_k} \frac{\varphi_{k_j}}{z+\lambda_k}\psi_{k_j}(x), \;\;\; x\in (-1,1)\;\Re(z)>0.
\end{align} 
It follows from \eqref{uc} and \eqref{dct_identity} that 
\begin{align*}
	\sum_{k=1}^{+\infty}\sum_{j=1}^{m_k} \frac{\varphi_{k_j}}{z+\lambda_k}\psi_{k_j}(x)=0, \;\;\; x\in\omega,\;\Re(z)>0.
\end{align*}

This holds for every $z\in\CC\setminus\{-\lambda_k\}_{k\in\NN}$, using the analytic continuation in $z$. Hence, taking a suitable small circle around $-\lambda_{\ell}$ not including $\{-\lambda_k\}_{k\neq\ell}$ and integrating on that circle we get that
\begin{align*}
	w_\ell:=\sum_{j=1}^{m_{\ell}} \varphi_{\ell_j}\psi_{\ell_j}(x)=0, \;\;\; x\in\omega.
\end{align*}
	
According to \cite[Theorem 1.4]{fall2014unique}, $\fl{s}{}$ has the unique continuation property in the sense that if $\lambda_k$ is an eigenvalue of $\fl{s}{}$ on (-1,1) with Dirichlet boundary conditions, and $(\fl{s}{}-\lambda_k)\varrho_k = 0$ in $(-1,1)$ with $\varrho_k = 0$ in $\omega$, then $\varrho_k = 0$ in $(-1,1)$. 
This can applied to $w_{\ell}$, in order to conclude $w_{\ell} = 0$ in $(-1,1)$ for every $\ell$. Since $\{\psi_{\ell_j}\}_{1\leq j\leq m_{\ell}}$ are linearly independent in $L^2(-1,1)$, we get $\varphi_{\ell_j} = 0$, $1\leq j\leq m_k$, $\ell\in\NN$. It follows that $\varphi^T=0$ and hence, $\varphi=0$ in $(-1,1)\times(0,T)$, meaning that $\varphi$ enjoys the property $(\mathcal{P})$. As an immediate consequence, we have that our original equation \eqref{heat_frac} is approximately controllable. Our proof is then concluded. 
\end{proof}

\begin{remark}
According to \cite{fall2014unique}, the elliptic unique continuation property for the fractional Laplacian holds in any space dimension. In view of that, Theorem \ref{approx_control_thm} may be extended also to the case $N>1$. On the other hand, the same does not applies to Theorem \eqref{null_control_thm}. Indeed, the proof of this result uses arguments that are designed specifically for one-dimensional problems (\cite{fattorini1971exact}). If one would analyze the null-controllability in a general multi-dimensional setting, other tools (such as Carleman estimates) are needed. As far as we know, these techniques have not been fully developed yet for problems involving the fractional Laplacian on a domain. 
\end{remark}

\begin{remark}
For the sake of simplicity, in the results presented above we focused on the interval $(-1,1)$. Nevertheless, everything that we did in this section actually holds in the more general case $x\in(-L,L)$ and the extension is immediate.
\end{remark}
%Therefore, given any initial datum $z_0\in L^2(-1,1)$ we are interested in computing numerically the control function $g$ that drives the solution $z$ to zero in time $T$. Before describing the methodology that we shall adopt, we recall the existing theoretical results on the controllability of the fractional heat equation \eqref{heat_frac}. This will give us a hint about what we should expect from our simulations, and will provide a validation of the accuracy of our numerical method.




\input{fe.tex}
\section{Numerical results} \label{res_numerical}
In this Section, we present the numerical simulations corresponding to the algorithm previously described, and we provide a complete discussion of the results obtained. 

First of all, we test numerically the accuracy of our method for the resolution of the elliptic equation \eqref{PE}, by applying it to the following problem 
\begin{align}\label{PE_real}
	\left\{\begin{array}{ll}
		\fl{s}{u} = 1, & x\in(-1,1)
		\\
		u\equiv 0, & x\in\RR\setminus(-1,1).
	\end{array}\right.
\end{align}

In this particular case, the unique solution to \eqref{PE_real} can be computed exactly and it is given in \cite{getoor1961first}. It reads as follows, 
\begin{align}\label{real_sol}
	u(x)=\frac{2^{-2s}\sqrt{\pi}}{\Gamma\left(\frac{1+2s}{2}\right)\Gamma(1+s)}\Big(1-x^2\Big)^s\cdot\mathbf{1}_{(-1,1)}.
\end{align}

In Fig.  \ref{smas12}, we show a comparison for different values of $s$ between the exact solution \eqref{real_sol} and the computed numerical approximation. Here we consider $N=50$. One can notice that when $s=0.1$ (and also for other small values of s), the computed solution is to a certain extent different from the exact solution. However, one should be careful with such result and a more precise analysis of the error should be carried. 
\begin{figure}[!h]
	\pgfplotstableread{fl_01_50_sym.txt}{\datpu}
	\pgfplotstableread{fl_04_50_sym.txt}{\ddatpu}
	\pgfplotstableread{fl_05_50_sym.txt}{\Ddatpu}
	\pgfplotstableread{fl_08_50_sym.txt}{\Dddatpu}

		\subfloat[$s=0.1$]{
		\begin{tikzpicture}[scale=0.8]
		\begin{axis}[xmin=-1.1, xmax=1.1,legend style={at={(0.77,0.25)}}]
		
			\addplot [color=blue, mark=none,  thick] table[x=0,y=1]{\datpu};	
			\addplot [color=red, mark=x, only marks, thick] table[x=2,y=3]{\datpu};		
			\addlegendentry{Numerical solution}
			\addlegendentry{Real solution}
		\end{axis}
	\end{tikzpicture}
	\label{s01a}
	}
		\hspace{1cm}\subfloat[$s=0.4$]{
		\begin{tikzpicture}[scale=0.8]
		\begin{axis}[xmin=-1.1, xmax=1.1,legend style={at={(0.77,0.25)}}]
			\addplot [color=blue, mark=none, thick] table[x=0,y=1]{\ddatpu};	
			\addplot [color=red, mark=x, only marks, thick] table[x=2,y=3]{\ddatpu};		
		\end{axis}
	\end{tikzpicture}
	}
\\
		\subfloat[$s=0.5$]{
		\begin{tikzpicture}[scale=0.8]
		\begin{axis}[xmin=-1.1, xmax=1.1,legend style={at={(0.77,0.25)}}]
			\addplot [color=blue, mark=none, thick] table[x=0,y=1]{\Ddatpu};	
			\addplot [color=red, mark=x, only marks, thick] table[x=2,y=3]{\Ddatpu};		
		\end{axis}
	\end{tikzpicture}
	}
		\hspace{1cm}\subfloat[$s=0.8$]{
		\begin{tikzpicture}[scale=0.8]
		\begin{axis}[xmin=-1.1, xmax=1.1,legend style={at={(0.77,0.25)}}]
			\addplot [color=blue, mark=none, thick] table[x=0,y=1]{\Dddatpu};	
			\addplot [color=red, mark=x, only marks, thick] table[x=2,y=3]{\Dddatpu};		
		\end{axis}
	\end{tikzpicture}
	}
	\caption{Plot for different values of $s$.}
	\label{smas12}
\end{figure}

In the same spirit as in \cite{acosta2017short}, the computation of the error in the space $H_0^s(-1,1)$ can be readily done by using the definition of the bilinear form, namely
\begin{align*}
	\norm{u-u_h}{H_0^s(-1,1)}^2 &=a(u-u_h,u-u_h) =a(u,u-u_h) =\int_{-1}^1f(x)\left(u(x)-u_h(x)\right)dx,
\end{align*}
where have used the orthogonality condition $a(v_h,u-u_h)=0$ $\forall v_h \in V_h$.

For this particular test, since $f\equiv 1$ in $(-1,1)$, the problem is therefore reduced to
\begin{align*}
	\norm{u-u_h}{H^s_0(-1,1)}=\left(\int_{-1}^1\left( u(x)-u_h(x) \right)\,dx\right)^{1/2}
\end{align*}
where the right-hand side can be easily computed, since we have the closed formula 
\begin{align*}
	\int_{-1}^1u\,dx= \frac{\pi}{2^{2s}\Gamma(s+\frac{1}{2})\Gamma(s+\frac{3}{2})}
\end{align*}
and the term corresponding to $\int_{-1}^1u_h$ can be carried out numerically. 

In Fig.  \ref{error}, we present the computational errors evaluated for different values of $s$ and $h$. 
\begin{figure}[!h]
 \centering
  \pgfplotstableread{result_convergence_01.org}{\datpu}
  \pgfplotstableread{result_convergence_03.org}{\datpt}
   \pgfplotstableread{result_convergence_05.org}{\datpcnum}
  \pgfplotstableread{result_convergence_07.org}{\datps}
  \pgfplotstableread{result_convergence_09.org}{\datpn}
   \pgfplotsset{
     legend cell align=left,
     legend pos=outer north east,
     legend plot pos=right,
     legend style={cells={anchor=east},draw=none},
     }
%
  %\subfloat[Convergence of the error]{\label{fig_case_O_subset_omega}
  \begin{tikzpicture}%[scale=1] 
  \label{conv}
  \begin{loglogaxis}[xlabel=$h$, ymax=2e-1]
  	%\addplot [color=blue, very thick] table [x=0, y=2] {\datpt};
	\addplot [color=black, mark=o, thick] table [x=0, y=1] {\datpu};
	\addlegendentry{$s=0.1$};
	\addplot [color=black, mark=diamond, mark size=3 pt, thick] table [x=0, y=1] {\datpt};
	\addlegendentry{${s}=0.3$};
	\addplot [color=black, mark=pentagon, thick] table [x=0, y=1] {\datpcnum};
	\addlegendentry{${s}=0.5$};
	\addplot [color=black, mark=square, thick] table [x=0, y=1] {\datps};
	\addlegendentry{$ s=0.7$};
	\addplot [color=black, mark=triangle, mark size=3 pt, thick] table [x=0, y=1] {\datpn};
	\addlegendentry{$ s=0.9$};
%
 \draw [pente]  (axis cs: 0.0008,4e-2) -- ++ (axis cs: 1, {10^(0.5)}) -- ++ (axis cs: 10, 1) -- cycle;
 \node at (axis cs:0.0008,10e-2) [right,pente] {\small slope $0.5$};
%
  \end{loglogaxis}
\end{tikzpicture}%
\caption{Convergence of the error.}
\label{error}
\end{figure}

The rates of convergence shown are of order (in $h$) of $1/2$. This is in accordance with the following result: 
\begin{theorem}[{\cite[Theorem 4.6]{acosta2017short}}]
For the solution $u$ of \eqref{WF} and its FE approximation $u_h$ given by \eqref{WFD}, if $h$ is sufficiently small, the following estimates hold

\begin{align*}
	&\norm{u-u_h}{H^s_0(-1,1)}\leq C h^{1/2}|\!\ln h|\,\norm{f}{C^{1/2-s}(-1,1)}, \quad \textnormal{if}\quad s<1/2, 
	\\
	&\norm{u-u_h}{H^s_0(-1,1)}\leq C h^{1/2} |\!\ln h|\, \norm{f}{L^\infty(-1,1)}, \quad \textnormal{if}\quad  s=1/2 
	\\
	&\norm{u-u_h}{H^s_0(-1,1)}\leq \tfrac{C}{2s-1} h^{1/2} \sqrt{|\!\ln h|}\, \norm{f}{C^\beta(-1,1)}, \quad \textnormal{if} \quad s>1/2,
\end{align*}
where $C$ is a positive constant not depending on $h$. 
\end{theorem}

Moreover, Fig.  \ref{error} shows that the convergence rate is maintained also for small values of $s$. This confirms that the behavior shown in Fig. \ref{smas12} (A) is not in contrast with the known theoretical results. Indeed, since it is well-known that the notion of trace is not defined for the spaces $H^s(-1,1)$ with $s\leq 1/2$ (see \cite{jllions1972non,tartar2007introduction}), it is somehow natural that we cannot expect a point-wise convergence in this case.  

\subsection{Control experiments}\label{control_exp}

To address the actual computation of fully-discrete controls for a given problem, we use the methodology described, for instance, in \cite{glowinski2008exact}. We apply an optimization algorithm to the dual functional \eqref{dual_fully}. Since these functionals are quadratic and coercive, the conjugate gradient is a natural and quite simple choice.

In the same spirit as \cite{boyer2011uniform}, the computation of the gradient at each iteration amounts to solve first the homogeneous equation \eqref{frac_adj_num}. Then, set $v^n=\mathbf{1}_\omega\varphi^n$ and finally solve \eqref{frac_heat_num}. In this way, the procedure to compute the control for a given problem basically requires to solve two parabolic equations: a homogeneous backward one associated with the final data $\varphi^T$, and a non-homogeneous forward problem with zero initial data. 

We present now some results obtained with the described methodology. In accordance with the discussion in Section \ref{fe_sec}, we use the finite-element approximation of $\fl{s}{}$ for the space discretization and the implicit Euler scheme in the time variable. We denote by $N$ the number of points in the mesh and by $M$ the number of time intervals. As discussed in \cite{boyer2011uniform}, the results in this kind of problems does not depend too much in the time step, as soon as it is chosen to ensure at least the same accuracy as the space discretization. The same remains true here, and therefore we always take $M=2000$ in order to concentrate the discussion on the dependency of the results with respect to the mesh size $h$ and the parameter $s$.

As we mentioned, we choose the penalization term $\varepsilon$ as a function of $h$. A reasonable practical rule (\cite{boyer2013penalised}) is to systematically choose $\phi(h)\sim h^{2p}$ where $p$ is the order of accuracy in space of the numerical method employed for the discretization of the spatial operator involved (in this case the fractional Laplacian \eqref{fl}).

\begin{theorem}[{\cite[proposition 3.3.2]{borthagaray2017laplaciano}}]
Let $s\in(0,1)$, $f\in L^2(-1,1)$ and $u$ be the solution to \eqref{PE}. Given a uniform mesh $\mathfrak{M}$ with mesh size $h$,
and the space $V_h$ defined as in \eqref{Vh}, let $u_h$ be the finite element solution to the corresponding discrete 
problem. Then, it holds that
\begin{align}\label{error_l2}
	\norm{u-u_h}{L^2(-1,1)}\leq C(s,\alpha)h^{2\alpha}\norm{f}{L^2(-1,1)},
\end{align}
where $\alpha:=\min\{s, 1/2 -\varepsilon\}$, for all $\varepsilon>0$.
\end{theorem}%



 We recall that for the elliptic problem that we are considering, this order of convergence is $1/2$. Thus, hereinafter we always assume $\varepsilon=\phi(h)=h$.

We begin by plotting on Fig.  \ref{sol_surf} the time evolution of the uncontrolled solution as well as the controlled solution. Here, we set $s=0.8$, $\omega=(-0.3,0.8)$ and $T=0.3$, and as an initial condition we take $z_0(x) = \sin(\pi x)$. The control domain is represented as highlighted zone on the plane $(t,x)$. As expected, we observe that the uncontrolled solution is damped with time, but does not reach zero at time $T$, while the controlled solution does. 

%\begin{figure}[ht]
%%%  Datos del experimento 
%%%  T=0.3, epsilon=0.01*h, s=0.8
% \centering
%% 
%\subfloat[The adjoint][Uncontrolled solution]{
%\begin{tikzpicture}
%\begin{axis}[surface,zmax=1.0]
%  
%\addplot3 [mesh/ordering=y varies, surf, shader=flat,draw=black,opacity=0.6] file {soly_s08_sc.dat};
%\node at (axis cs:0.34,0,0) [below] {$T=0.3$};
%
%\end{axis}
%\end{tikzpicture}}
%%
%\hspace{1 cm}
%  \subfloat[The state][Controlled solution ({\tikz \fill [green] (0,0) rectangle (0.2,0.2);}=control domain)]{
%\begin{tikzpicture}
%\begin{axis}[surface, zmax=1.0]
%
%  %\controldomainT{-0.3}{0.8}{0.3};
%  \couplingdomainT{-0.3}{0.8}{0.3};
%  
%\addplot3 [mesh/ordering=y varies, surf, shader=flat,draw=black,opacity=0.6] file {soly_s08_cc.dat};
%\node at (axis cs:0.34,0,0) [below] {$T=0.3$};
%
%\end{axis}
%\end{tikzpicture}}%
%\caption{Time evolution of system \eqref{frac_heat_num}.}\label{fig_heat_frac}
%\label{sol_surf}
%\end{figure}

%\begin{figure}[ht]
%	\centering
%	\includegraphics[scale=1]{figure5.eps}
%	\caption{Time evolution of system \eqref{frac_heat_num}.}\label{fig_heat_frac}
%\label{sol_surf}
%\end{figure}

In figure \ref{figure_case1}, we present the computed values of various quantities of interest when the mesh size goes to zero. More precisely, we observe that the control cost $\|v_{\delta t}\|_{L^2_{\delta t}(0,T;\mathbb R^\mesh)}$ and the optimal energy $\inf F_{\phi(h),h,\delta t}$ remain bounded as $h\to 0$. On the other hand, we see that 
\begin{align}\label{control_norm_behavior}
	|y^M|_{L^2(\RR^\mesh)}\,\sim\,C\sqrt{\phi(h)}=Ch^{1/2}. 
\end{align}

We know that, for $s=0.8$, system \eqref{heat_frac} is null controllable. This is now confirmed by \eqref{control_norm_behavior}, according to Theorem \ref{theorem_hum}.  In fact, the same experiment can be repeated for different values of $s>1/2$, obtaining the same conclusions. 
\begin{figure}[h]
  \centering
\begin{tikzpicture}
  \begin{loglogaxis}[erreurs, ymin=5e-5,ymax=1e1]
 
    \pgfplotstableread[ignore chars={|},skip first n=2]{heat_frac_s=08.org}\resultats
    %% Original file  resultats_09-06-2017_11h18.org

    \addplot[cout] table[x=dx,y=Nv] \resultats;
    \addlegendentry{Cost of the control};
    \addplot[cible] table[x=dx,y=NyT] \resultats;
    \addlegendentry{Size of $y^M$};
    \addplot[energie] table[x=dx,y=Inf_eps(F_eps)] \resultats;
    \addlegendentry{Optimal energy};
    \draw [pente]  (axis cs: 0.003,84e-4) -- ++ (axis cs: 1, 10^1) -- ++ (axis cs: 10, 1) -- cycle;
    \node at (axis cs:0.003,20e-3) [right,pente] {\small sl. $1$};
    
  \end{loglogaxis}
  
\end{tikzpicture}
\caption{Convergence properties of the method for controllability of the fractional heat equation for $s=0.8$.}\label{figure_case1}
\end{figure}

According to the discussion in Section \ref{theor_sec}, one can prove that null controllability does not hold for system \eqref{heat_frac} in the case $s\leq 1/2$. However approximate controllability can be proved by means of the unique continuation property of the operator $\fl{s}{}$. We would like to illustrate this property in Fig. \ref{figure_case3}.

We observe that the results are different from what we obtained in Fig.  \ref{figure_case1}. In fact, the cost of the control and the optimal energy increase in both cases, while the target $y^M$ tends to zero with a slower rate than $h^{1/2}$. This seems to confirm that a uniform observability estimate for \eqref{heat_frac} does not hold and that we can only expect to have approximate controllability (see Theorem \ref{theorem_hum}).
\begin{figure}
\centering
\subfloat[$s=0.4$]{
\begin{tikzpicture}[scale=0.9]
  \begin{loglogaxis}[erreurs, ymin=8e-2,ymax=2e1,title={$s=0.4$}]
 
    \pgfplotstableread[ignore chars={|},skip first n=2]{heat_frac_s=04.org}\resultats
    %% Original file  resultats_09-06-2017_11h47.org

    \addplot[cout] table[x=dx,y=Nv] \resultats;
    %\addlegendentry{Cost of the control};
    \addplot[cible] table[x=dx,y=NyT] \resultats;
    %\addlegendentry{Size of $y^M$};
    \addplot[energie] table[x=dx,y=Inf_eps(F_eps)] \resultats;
    %\addlegendentry{Optimal energy};
    \draw [pente]  (axis cs: 0.003,2.75e-1) -- ++ (axis cs: 1, 6^0.15) -- ++ (axis cs: 6, 1) -- cycle;
    \node at (axis cs:0.003,4.2e-1) [right,pente] {\small sl. $0.35$};
    
    \draw [pente]  (axis cs: 0.003,1.7e0) -- ++ (axis cs: 1, {6^(-0.7)}) -- ++ (axis cs: 6, 1) -- cycle;
    \draw [pente]  (axis cs: 0.003,0.99e0) -- ++ (axis cs: 1, {5.9^(-0.4)}) -- ++ (axis cs: 5.9, 1) -- cycle;
    \node at (axis cs:0.003,0.55e0) [right,pente] {\small sl. $-0.4/-0.7$};
    
  \end{loglogaxis}
  \end{tikzpicture}
  }
 \subfloat[$s=0.5$]{
  \begin{tikzpicture}[scale=0.9]
  \begin{loglogaxis}[erreurs, ymin=2e-2,ymax=20.4e1,title={$s=0.5$}]
    \pgfplotstableread[ignore chars={|},skip first n=2]{heat_frac_s=05.org}\resultats
    %% Original file  resultats_09-06-2017_11h07.org

    \addplot[cout] table[x=dx,y=Nv] \resultats;
    %\addlegendentry{Cost of the control};
    \addplot[cible] table[x=dx,y=NyT] \resultats;
    %\addlegendentry{Size of $y^M$};
    \addplot[energie] table[x=dx,y=Inf_eps(F_eps)] \resultats;
    %\addlegendentry{Optimal energy};
    \draw [pente]  (axis cs: 0.003,10e-2) -- ++ (axis cs: 1, 10^0.4) -- ++ (axis cs: 10, 1) -- cycle;
    \node at (axis cs:0.003,18e-2) [right,pente] {\small slope $0.4$};
    
    \draw [pente]  (axis cs: 0.003,2.65e1) -- ++ (axis cs: 1, {10^(-0.18)}) -- ++ (axis cs: 10, 1) -- cycle;
    \draw [pente]  (axis cs: 0.003,3.5e1) -- ++ (axis cs: 1, {10^(-0.3)}) -- ++ (axis cs: 10, 1) -- cycle;
    \node at (axis cs:0.003,1.35e1) [right,pente] {\small sl. $-0.18/-0.3$};
    
  \end{loglogaxis}
\end{tikzpicture}
}
\caption{Convergence properties of the method for $s<1/2$. Same legend as in Fig.  \ref{figure_case1}}\label{figure_case3}
\end{figure}


\input{appendix.tex}

\section*{Acknowledgements}
The authors wish to acknowledge F. Boyer (Institut de Math\'ematiques de Toulouse) for fruitful discussion on the numerical implementation of the control problem. Moreover, a special thanks goes to J. Loh\'eac (Laboratoire de Sciences Num\'eriques de Nantes), for helping with some computations in Section \ref{fe_sec}.  

\bibliography{biblio}

\end{document}
