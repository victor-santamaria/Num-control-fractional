\section{Introduction and main results}\label{intro_sec}
Let $\omega\subset (-1,1)$ be an open and nonempty subset. In this work, we analyze controllability properties for the following nonlocal one-dimensional heat equation 
\begin{align}\label{heat_frac}
	\begin{cases}
		z_t + \fl{s}{z} = g\mathbf{1}_{\omega},\quad & (x,t)\in (-1,1)\times(0,T)
		\\
		z=0, & (x,t)\in[\,\RR\setminus (-1,1)\,]\times(0,T)
		\\
		z(x,0)=z_0(x), & x\in (-1,1),
	\end{cases}
\end{align} 
where $z_0\in L^2(-1,1)$ is a given initial datum. In more detail, we are interested in the resolution of the following control problem: given any $T>0$, find a control function $g\in L^2(\omega\times(0,T))$ such that the corresponding solution to \eqref{heat_frac} satisfies $z(x,T)=0$. 

In \eqref{heat_frac}, for all $s\in(0,1)$, $\fl{s}{}$ denotes the one-dimensional fractional Laplace operator, which is defined as the following singular integral
\begin{align}\label{fl_intro}
	\fl{s}{u}(x) = \ccs\,P.V.\,\int_{\RR}\frac{u(x)-u(y)}{|x-y|^{1+2s}}\,dy. 
\end{align}
Here, $\ccs$ is a normalization constant given by
\begin{align*}
	\ccs = \frac{s2^{2s}\Gamma\left(\frac{1+2s}{2}\right)}{\sqrt{\pi}\Gamma(1-s)},
\end{align*}
where $\Gamma$ is the usual Gamma function. Moreover, we have to mention that, for having a completely rigorous definition of the fractional Laplace operator, it is necessary to introduce also the class of functions $u$ for which computing $\fl{s}{u}$ makes sense. We postpone this discussion to the next section.

The analysis of non-local operators and non-local PDEs is a topic in continuous development.
A motivation for this growing interest relies in the large number of possible applications in the modeling of several complex phenomena for which a local approach turns up to be inappropriate or limiting.
Indeed, there is an ample spectrum of situations in which a non-local equation gives a
significantly better description than a PDE of the problem one wants to analyze.
Among others, we mention applications in turbulence (\cite{bakunin2008turbulence}), anomalous transport and diffusion (\cite{bologna2000anomalous,meerschaert2012fractional}), elasticity (\cite{dipierro2015dislocation}), image processing (\cite{gilboa2008nonlocal}), porous media flow (\cite{vazquez2012nonlinear}), wave propagation in heterogeneous high contrast media (\cite{zhu2014modeling}). Also, it is well known that the fractional Laplacian is the generator of s-stable processes, and it is often used in stochastic models with applications, for instance, in mathematical finance (\cite{levendorskii2004pricing,pham1997optimal}).

One of the main differences between these non-local models and classical Partial Differential Equations is that the fulfillment of a non-local equation at a point involves the values of the function far away from that point.

It is well-known that the classical local heat equation (as well as many more general variants) is null-controllable in any time $T>0$ (see, e.g., \cite{fattorini1971exact,fursikov1996controllability,lebeau1995controle}). Nevertheless, to the best of our knowledge, there are few results in the literature on the null-controllability of the fractional heat equation, and none of them is for a problem involving the fractional Laplacian in its integral form \eqref{fl_intro}. The existing results (\cite{micu2006controllability,miller2006controllability}), instead, deal with the so-called \textit{spectral} fractional Laplace operator, whose definition will be given later. 

In this paper, we deal with the controllability of \eqref{heat_frac}, both from the theoretical and the numerical point of view. Employing spectral analysis techniques based on the works \cite{kulczycki2010spectral,kwasnicki2012eigenvalues}, the first main result that we obtain is the following

\begin{theorem}\label{null_control_thm}
Given any $z_0\in L^2(-1,1)$ the parabolic problem \eqref{heat_frac} is null-controllable at time $T>0$ with a control function $g\in L^2(\omega\times(0,T))$ if and only if $s>1/2$.  
\end{theorem}

Furthermore, even if for $s\leq 1/2$ null controllability for \eqref{heat_frac} fails, we still have the following result of approximate controllability, as a consequence of unique continuation properties for the fractional Laplace operator (\cite{fall2014unique}). 

\begin{theorem}\label{approx_control_thm}
Let $s\in(0,1)$. Given any $z_0\in L^2(-1,1)$, there exists a control function $g\in L^2(\omega\times(0,T))$ such that the unique solution $z$ to the parabolic problem \eqref{heat_frac} is approximately controllable at time $T>0$.
\end{theorem}

Theorems \ref{null_control_thm} and \eqref{approx_control_thm} will then find a confirmation in the study of the corresponding numerical control problem. With this purpose, we will employ the penalized Hilbert Uniqueness Method, which relies on some classical works of Glowinski and Lions (\cite{glowinski1995exact,glowinski2008exact}). 

Notice that this method is very general, and it may by applied to a broad class of PDEs control problems (\cite{boyer2013penalised,boyer2017insensitizing,boyer2014approximate,khodja2017partial}). When using it for treating a nonlocal problem as \eqref{heat_frac}, new issues arise related to the discretization of the corresponding elliptic problem.    

In this framework, a preliminary step for the for the resolution of the numerical control problem will be a finite element (FE)  approximation of the solution to the following non-local Poisson equation
\begin{align}\label{PE}
	\begin{cases}
		\fl{s}{u} = f, & x\in(-L,L)
		\\
		u\equiv 0, & x\in\RR\setminus(-L,L).
	\end{cases}
\end{align}

In the recent past, the fractional Laplacian has been widely analyzed also from the point of view of numerical analysis. We refer, for instance, to the works \cite{acosta2017short,acosta2017fractional,borthagaray2017laplaciano}. There, the authors present a FE scheme for implementing the solution of \eqref{PE} in a bounded domain $\Omega\subset\RR^2$. In particular, they provide appropriate quadrature rules in order to solve numerically the variational formulation associated to the problem. Moreover, in \cite{acosta2017fractional,borthagaray2017laplaciano} it is also developed an accurate analysis of the efficiency of the FE method, employing several existing results. The techniques of the aforementioned works have then been applied in \cite{acosta2017finite}, combined with a convolution quadrature approach, for solving evolution equations involving the fractional Laplacian. For the sake of completeness, we also mention \cite{bonito2015numerical}, where it is presented a discretization of the spectral fractional Laplacian and its application to the evolutionary case \cite{bonito2017approximation}, and \cite{nochetto2015pde}, where the same problem is treated applying the well known extension of Caffarelli and Silvestre (\cite{caffarelli2007extension}).   

In the present paper, we propose a FE approximation for the fractional Poisson equation \eqref{PE} which does not requires any quadrature rule. Indeed, exploiting the one-dimensional nature of the problem, each entry of the stiffness matrix can be computed explicitly in terms only of its position, of the parameter $s$ and of the mesh size. This, in particular, allows a quick and simple implementation of the control problem.

\rouge{This paper is organized as follows. In Section \ref{theor_sec}, we present some existing theoretical results for the problems that we are going to analyze. In particular, we give a more accurate definition of the fractional Laplace operator and we introduce the variational formulation associated to \eqref{PE} (needed for the development of the FE scheme). Concerning the parabolic problem \eqref{heat_frac}, we present a couple of controllability results, which will help us in the verification of the accuracy of the numerical method. In Section \ref{fe_sec}, we describe our FE method for the elliptic equation \eqref{PE} and we present the algorithm for the penalized HUM, employed for the numerical control of \eqref{heat_frac}. In Section \ref{res_numerical} we present and comment the results of our numerical simulations. Finally, in Appendix \ref{appendix} we include the complete details for computing the stiffness matrix associated to our FE scheme.}

