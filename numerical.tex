\section{Numerical results} \label{res_numerical}
In this Section, we present the numerical simulations corresponding to the algorithm previously described, and we provide a complete discussion of the results obtained. 

First of all, we test numerically the accuracy of our method for the resolution of the elliptic equation \eqref{PE}, by applying it to the following problem 
\begin{align}\label{PE_real}
	\left\{\begin{array}{ll}
		\fl{s}{u} = 1, & x\in(-1,1)
		\\
		u\equiv 0, & x\in\RR\setminus(-1,1).
	\end{array}\right.
\end{align}

In this particular case, the unique solution to \eqref{PE_real} can be computed exactly and it is given in \cite{getoor1961first}. It reads as follows, 
\begin{align}\label{real_sol}
	u(x)=\frac{2^{-2s}\sqrt{\pi}}{\Gamma\left(\frac{1+2s}{2}\right)\Gamma(1+s)}\Big(1-x^2\Big)^s\cdot\mathbf{1}_{(-1,1)}.
\end{align}

In Fig.  \ref{smas12}, we show a comparison for different values of $s$ between the exact solution \eqref{real_sol} and the computed numerical approximation. Here we consider $N=50$. One can notice that when $s=0.1$ (and also for other small values of s), the computed solution is to a certain extent different from the exact solution. However, one should be careful with such result and a more precise analysis of the error should be carried. 
\begin{figure}[!h]
	\pgfplotstableread{fl_01_50_sym.txt}{\datpu}
	\pgfplotstableread{fl_04_50_sym.txt}{\ddatpu}
	\pgfplotstableread{fl_05_50_sym.txt}{\Ddatpu}
	\pgfplotstableread{fl_08_50_sym.txt}{\Dddatpu}

		\subfloat[$s=0.1$]{
		\begin{tikzpicture}[scale=0.8]
		\begin{axis}[xmin=-1.1, xmax=1.1,legend style={at={(0.77,0.25)}}]
		
			\addplot [color=blue, mark=none,  thick] table[x=0,y=1]{\datpu};	
			\addplot [color=red, mark=x, only marks, thick] table[x=2,y=3]{\datpu};		
			\addlegendentry{Numerical solution}
			\addlegendentry{Real solution}
		\end{axis}
	\end{tikzpicture}
	\label{s01a}
	}
		\hspace{1cm}\subfloat[$s=0.4$]{
		\begin{tikzpicture}[scale=0.8]
		\begin{axis}[xmin=-1.1, xmax=1.1,legend style={at={(0.77,0.25)}}]
			\addplot [color=blue, mark=none, thick] table[x=0,y=1]{\ddatpu};	
			\addplot [color=red, mark=x, only marks, thick] table[x=2,y=3]{\ddatpu};		
		\end{axis}
	\end{tikzpicture}
	}
\\
		\subfloat[$s=0.5$]{
		\begin{tikzpicture}[scale=0.8]
		\begin{axis}[xmin=-1.1, xmax=1.1,legend style={at={(0.77,0.25)}}]
			\addplot [color=blue, mark=none, thick] table[x=0,y=1]{\Ddatpu};	
			\addplot [color=red, mark=x, only marks, thick] table[x=2,y=3]{\Ddatpu};		
		\end{axis}
	\end{tikzpicture}
	}
		\hspace{1cm}\subfloat[$s=0.8$]{
		\begin{tikzpicture}[scale=0.8]
		\begin{axis}[xmin=-1.1, xmax=1.1,legend style={at={(0.77,0.25)}}]
			\addplot [color=blue, mark=none, thick] table[x=0,y=1]{\Dddatpu};	
			\addplot [color=red, mark=x, only marks, thick] table[x=2,y=3]{\Dddatpu};		
		\end{axis}
	\end{tikzpicture}
	}
	\caption{Plot for different values of $s$.}
	\label{smas12}
\end{figure}

In the same spirit as in \cite{acosta2017short}, the computation of the error in the space $H_0^s(-1,1)$ can be readily done by using the definition of the bilinear form, namely
\begin{align*}
	\norm{u-u_h}{H_0^s(-1,1)}^2 &=a(u-u_h,u-u_h) =a(u,u-u_h) =\int_{-1}^1f(x)\left(u(x)-u_h(x)\right)dx,
\end{align*}
where have used the orthogonality condition $a(v_h,u-u_h)=0$ $\forall v_h \in V_h$.

For this particular test, since $f\equiv 1$ in $(-1,1)$, the problem is therefore reduced to
\begin{align*}
	\norm{u-u_h}{H^s_0(-1,1)}=\left(\int_{-1}^1\left( u(x)-u_h(x) \right)\,dx\right)^{1/2}
\end{align*}
where the right-hand side can be easily computed, since we have the closed formula 
\begin{align*}
	\int_{-1}^1u\,dx= \frac{\pi}{2^{2s}\Gamma(s+\frac{1}{2})\Gamma(s+\frac{3}{2})}
\end{align*}
and the term corresponding to $\int_{-1}^1u_h$ can be carried out numerically. 

In Fig.  \ref{error}, we present the computational errors evaluated for different values of $s$ and $h$. 
\begin{figure}[!h]
 \centering
  \pgfplotstableread{result_convergence_01.org}{\datpu}
  \pgfplotstableread{result_convergence_03.org}{\datpt}
   \pgfplotstableread{result_convergence_05.org}{\datpcnum}
  \pgfplotstableread{result_convergence_07.org}{\datps}
  \pgfplotstableread{result_convergence_09.org}{\datpn}
   \pgfplotsset{
     legend cell align=left,
     legend pos=outer north east,
     legend plot pos=right,
     legend style={cells={anchor=east},draw=none},
     }
%
  %\subfloat[Convergence of the error]{\label{fig_case_O_subset_omega}
  \begin{tikzpicture}%[scale=1] 
  \label{conv}
  \begin{loglogaxis}[xlabel=$h$, ymax=2e-1]
  	%\addplot [color=blue, very thick] table [x=0, y=2] {\datpt};
	\addplot [color=black, mark=o, thick] table [x=0, y=1] {\datpu};
	\addlegendentry{$s=0.1$};
	\addplot [color=black, mark=diamond, mark size=3 pt, thick] table [x=0, y=1] {\datpt};
	\addlegendentry{${s}=0.3$};
	\addplot [color=black, mark=pentagon, thick] table [x=0, y=1] {\datpcnum};
	\addlegendentry{${s}=0.5$};
	\addplot [color=black, mark=square, thick] table [x=0, y=1] {\datps};
	\addlegendentry{$ s=0.7$};
	\addplot [color=black, mark=triangle, mark size=3 pt, thick] table [x=0, y=1] {\datpn};
	\addlegendentry{$ s=0.9$};
%
 \draw [pente]  (axis cs: 0.0008,4e-2) -- ++ (axis cs: 1, {10^(0.5)}) -- ++ (axis cs: 10, 1) -- cycle;
 \node at (axis cs:0.0008,10e-2) [right,pente] {\small slope $0.5$};
%
  \end{loglogaxis}
\end{tikzpicture}%
\caption{Convergence of the error.}
\label{error}
\end{figure}

The rates of convergence shown are of order (in $h$) of $1/2$. This is in accordance with the following result: 
\begin{theorem}[{\cite[Theorem 4.6]{acosta2017short}}]
For the solution $u$ of \eqref{WF} and its FE approximation $u_h$ given by \eqref{WFD}, if $h$ is sufficiently small, the following estimates hold

\begin{align*}
	&\norm{u-u_h}{H^s_0(-1,1)}\leq C h^{1/2}|\!\ln h|\,\norm{f}{C^{1/2-s}(-1,1)}, \quad \textnormal{if}\quad s<1/2, 
	\\
	&\norm{u-u_h}{H^s_0(-1,1)}\leq C h^{1/2} |\!\ln h|\, \norm{f}{L^\infty(-1,1)}, \quad \textnormal{if}\quad  s=1/2 
	\\
	&\norm{u-u_h}{H^s_0(-1,1)}\leq \tfrac{C}{2s-1} h^{1/2} \sqrt{|\!\ln h|}\, \norm{f}{C^\beta(-1,1)}, \quad \textnormal{if} \quad s>1/2,
\end{align*}
where $C$ is a positive constant not depending on $h$. 
\end{theorem}

Moreover, Fig.  \ref{error} shows that the convergence rate is maintained also for small values of $s$. This confirms that the behavior shown in Fig. \ref{smas12} (A) is not in contrast with the known theoretical results. Indeed, since it is well-known that the notion of trace is not defined for the spaces $H^s(-1,1)$ with $s\leq 1/2$ (see \cite{jllions1972non,tartar2007introduction}), it is somehow natural that we cannot expect a point-wise convergence in this case.  

\subsection{Control experiments}\label{control_exp}

To address the actual computation of fully-discrete controls for a given problem, we use the methodology described, for instance, in \cite{glowinski2008exact}. We apply an optimization algorithm to the dual functional \eqref{dual_fully}. Since these functionals are quadratic and coercive, the conjugate gradient is a natural and quite simple choice.

In the same spirit as \cite{boyer2011uniform}, the computation of the gradient at each iteration amounts to solve first the homogeneous equation \eqref{frac_adj_num}. Then, set $v^n=\mathbf{1}_\omega\varphi^n$ and finally solve \eqref{frac_heat_num}. In this way, the procedure to compute the control for a given problem basically requires to solve two parabolic equations: a homogeneous backward one associated with the final data $\varphi^T$, and a non-homogeneous forward problem with zero initial data. 

We present now some results obtained with the described methodology. In accordance with the discussion in Section \ref{fe_sec}, we use the finite-element approximation of $\fl{s}{}$ for the space discretization and the implicit Euler scheme in the time variable. We denote by $N$ the number of points in the mesh and by $M$ the number of time intervals. As discussed in \cite{boyer2011uniform}, the results in this kind of problems does not depend too much in the time step, as soon as it is chosen to ensure at least the same accuracy as the space discretization. The same remains true here, and therefore we always take $M=2000$ in order to concentrate the discussion on the dependency of the results with respect to the mesh size $h$ and the parameter $s$.

As we mentioned, we choose the penalization term $\varepsilon$ as a function of $h$. A reasonable practical rule (\cite{boyer2013penalised}) is to systematically choose $\phi(h)\sim h^{2p}$ where $p$ is the order of accuracy in space of the numerical method employed for the discretization of the spatial operator involved (in this case the fractional Laplacian \eqref{fl}).

\begin{theorem}[{\cite[proposition 3.3.2]{borthagaray2017laplaciano}}]
Let $s\in(0,1)$, $f\in L^2(-1,1)$ and $u$ be the solution to \eqref{DP}. Given a uniform mesh $\mathfrak{M}$ with mesh size $h$,
and the space $V_h$ defined as in \eqref{Vh}, let $u_h$ be the finite element solution to the corresponding discrete 
problem. Then, it holds that
\begin{align}\label{error_l2}
	\norm{u-u_h}{L^2(-1,1)}\leq C(s,\alpha)h^{2\alpha}\norm{f}{L^2(-1,1)},
\end{align}
where $\alpha:=\min\{s, 1/2 -\varepsilon\}$, for all $\varepsilon>0$.
\end{theorem}%



 We recall that for the elliptic problem that we are considering, this order of convergence is $1/2$. Thus, hereinafter we always assume $\varepsilon=\phi(h)=h$.

We begin by plotting on Fig.  \ref{sol_surf} the time evolution of the uncontrolled solution as well as the controlled solution. Here, we set $s=0.8$, $\omega=(-0.3,0.8)$ and $T=0.3$, and as an initial condition we take $z_0(x) = \sin(\pi x)$. The control domain is represented as highlighted zone on the plane $(t,x)$. As expected, we observe that the uncontrolled solution is damped with time, but does not reach zero at time $T$, while the controlled solution does. 

%\begin{figure}[ht]
%%%  Datos del experimento 
%%%  T=0.3, epsilon=0.01*h, s=0.8
% \centering
%% 
%\subfloat[The adjoint][Uncontrolled solution]{
%\begin{tikzpicture}
%\begin{axis}[surface,zmax=1.0]
%  
%\addplot3 [mesh/ordering=y varies, surf, shader=flat,draw=black,opacity=0.6] file {soly_s08_sc.dat};
%\node at (axis cs:0.34,0,0) [below] {$T=0.3$};
%
%\end{axis}
%\end{tikzpicture}}
%%
%\hspace{1 cm}
%  \subfloat[The state][Controlled solution ({\tikz \fill [green] (0,0) rectangle (0.2,0.2);}=control domain)]{
%\begin{tikzpicture}
%\begin{axis}[surface, zmax=1.0]
%
%  %\controldomainT{-0.3}{0.8}{0.3};
%  \couplingdomainT{-0.3}{0.8}{0.3};
%  
%\addplot3 [mesh/ordering=y varies, surf, shader=flat,draw=black,opacity=0.6] file {soly_s08_cc.dat};
%\node at (axis cs:0.34,0,0) [below] {$T=0.3$};
%
%\end{axis}
%\end{tikzpicture}}%
%\caption{Time evolution of system \eqref{frac_heat_num}.}\label{fig_heat_frac}
%\label{sol_surf}
%\end{figure}

%\begin{figure}[ht]
%	\centering
%	\includegraphics[scale=1]{figure5.eps}
%	\caption{Time evolution of system \eqref{frac_heat_num}.}\label{fig_heat_frac}
%\label{sol_surf}
%\end{figure}

In figure \ref{figure_case1}, we present the computed values of various quantities of interest when the mesh size goes to zero. More precisely, we observe that the control cost $\|v_{\delta t}\|_{L^2_{\delta t}(0,T;\mathbb R^\mesh)}$ and the optimal energy $\inf F_{\phi(h),h,\delta t}$ remain bounded as $h\to 0$. On the other hand, we see that 
\begin{align}\label{control_norm_behavior}
	|y^M|_{L^2(\RR^\mesh)}\,\sim\,C\sqrt{\phi(h)}=Ch^{1/2}. 
\end{align}

We know that, for $s=0.8$, system \eqref{heat_frac} is null controllable. This is now confirmed by \eqref{control_norm_behavior}, according to Theorem \ref{theorem_hum}.  In fact, the same experiment can be repeated for different values of $s>1/2$, obtaining the same conclusions. 
\begin{figure}[h]
  \centering
\begin{tikzpicture}
  \begin{loglogaxis}[erreurs, ymin=5e-4,ymax=0.2e1]
 
    \pgfplotstableread[ignore chars={|},skip first n=2]{heat_frac_s=08.org}\resultats
    %% Original file  resultats_09-06-2017_11h18.org

    \addplot[cout] table[x=dx,y=Nv] \resultats;
    \addlegendentry{Cost of the control};
    \addplot[cible] table[x=dx,y=NyT] \resultats;
    \addlegendentry{Size of $y^M$};
    \addplot[energie] table[x=dx,y=Inf_eps(F_eps)] \resultats;
    \addlegendentry{Optimal energy};
    \draw [pente]  (axis cs: 0.003,84e-4) -- ++ (axis cs: 1, 10^0.5) -- ++ (axis cs: 10, 1) -- cycle;
    \node at (axis cs:0.003,20e-3) [right,pente] {\small slope $0.5$};
    
  \end{loglogaxis}
  
\end{tikzpicture}
\caption{Convergence properties of the method for controllability of the fractional heat equation for $s=0.8$.}\label{figure_case1}
\end{figure}

According to the discussion in Section \ref{theor_sec}, one can prove that null controllability does not hold for system \eqref{heat_frac} in the case $s\leq 1/2$. However approximate controllability can be proved by means of the unique continuation property of the operator $\fl{s}{}$. We would like to illustrate this property in Fig. \ref{figure_case3}.

We observe that the results are different from what we obtained in Fig.  \ref{figure_case1}. In fact, the cost of the control and the optimal energy increase in both cases, while the target $y^M$ tends to zero with a slower rate than $h^{1/2}$. This seems to confirm that a uniform observability estimate for \eqref{heat_frac} does not hold and that we can only expect to have approximate controllability (see Theorem \ref{theorem_hum}).
\begin{figure}
\centering
\subfloat[$s=0.4$]{
\begin{tikzpicture}[scale=0.9]
  \begin{loglogaxis}[erreurs, ymin=4e-2,ymax=40.4e1,title={$s=0.4$}]
 
    \pgfplotstableread[ignore chars={|},skip first n=2]{heat_frac_s=04.org}\resultats
    %% Original file  resultats_09-06-2017_11h47.org

    \addplot[cout] table[x=dx,y=Nv] \resultats;
    %\addlegendentry{Cost of the control};
    \addplot[cible] table[x=dx,y=NyT] \resultats;
    %\addlegendentry{Size of $y^M$};
    \addplot[energie] table[x=dx,y=Inf_eps(F_eps)] \resultats;
    %\addlegendentry{Optimal energy};
    \draw [pente]  (axis cs: 0.003,1.2e-1) -- ++ (axis cs: 1, 10^0.35) -- ++ (axis cs: 10, 1) -- cycle;
    \node at (axis cs:0.003,20e-2) [right,pente] {\small slope $0.35$};
    
    \draw [pente]  (axis cs: 0.003,80.4e0) -- ++ (axis cs: 1, {10^(-0.4)}) -- ++ (axis cs: 10, 1) -- cycle;
    \draw [pente]  (axis cs: 0.003,56.7e0) -- ++ (axis cs: 1, {10^(-0.25)}) -- ++ (axis cs: 10, 1) -- cycle;
    \node at (axis cs:0.003,23e0) [right,pente] {\small sl. $-0.25/-0.4$};
    
  \end{loglogaxis}
  \end{tikzpicture}
  }
 \subfloat[$s=0.5$]{
  \begin{tikzpicture}[scale=0.9]
  \begin{loglogaxis}[erreurs, ymin=2e-2,ymax=20.4e1,title={$s=0.5$}]
    \pgfplotstableread[ignore chars={|},skip first n=2]{heat_frac_s=05.org}\resultats
    %% Original file  resultats_09-06-2017_11h07.org

    \addplot[cout] table[x=dx,y=Nv] \resultats;
    %\addlegendentry{Cost of the control};
    \addplot[cible] table[x=dx,y=NyT] \resultats;
    %\addlegendentry{Size of $y^M$};
    \addplot[energie] table[x=dx,y=Inf_eps(F_eps)] \resultats;
    %\addlegendentry{Optimal energy};
    \draw [pente]  (axis cs: 0.003,10e-2) -- ++ (axis cs: 1, 10^0.4) -- ++ (axis cs: 10, 1) -- cycle;
    \node at (axis cs:0.003,18e-2) [right,pente] {\small slope $0.4$};
    
    \draw [pente]  (axis cs: 0.003,2.65e1) -- ++ (axis cs: 1, {10^(-0.18)}) -- ++ (axis cs: 10, 1) -- cycle;
    \draw [pente]  (axis cs: 0.003,3.5e1) -- ++ (axis cs: 1, {10^(-0.3)}) -- ++ (axis cs: 10, 1) -- cycle;
    \node at (axis cs:0.003,1.35e1) [right,pente] {\small sl. $-0.18/-0.3$};
    
  \end{loglogaxis}
\end{tikzpicture}
}
\caption{Convergence properties of the method for $s<1/2$. Same legend as in Fig.  \ref{figure_case1}}\label{figure_case3}
\end{figure}

