\section{Preliminary results on the fractional Laplace operator}\label{theor_sec}

In this Section, we introduce some preliminary result that will be useful in the remainder of the paper.

%\subsection{Elliptic problem}

We start by giving a more rigorous definition of the fractional Laplace operator, as we have anticipated in Section \ref{intro_sec}. Let
\begin{align*}
	\mathcal L^1_s(\RR) :=\left\{ u:\RR\longrightarrow\RR\,:\; u\textrm{ measurable },\;\int_{\RR}\frac{|u(x)|}{(1+|x|)^{1+2s}}\,dx<\infty\right\}.
\end{align*}
For any $u\in\mathcal L_s^1$ and $\varepsilon>0$ we set 
\begin{align*}
	(-d_x^{\,2})^s_{\varepsilon}\, u(x) = \ccs\,\int_{|x-y|>\varepsilon}\frac{u(x)-u(y)}{|x-y|^{1+2s}}\,dy,\;\;\; x\in\RR.
\end{align*}
The fractional Laplacian is then defined by the following singular integral
\begin{align}\label{fl}
	\fl{s}{u}(x) = \ccs\,P.V.\,\int_{\RR}\frac{u(x)-u(y)}{|x-y|^{1+2s}}\,dy = \lim_{\varepsilon\to 0^+} (-d_x^2)^s_{\varepsilon} u(x), \;\;\; x\in\RR,
\end{align}
provided that the limit exists. 

We notice that if $0<s<1/2$ and $u$ is a smooth function, for example bounded and Lipschitz continuous on $\RR$, then the integral in \eqref{fl} is in fact not really singular near $x$ (see e.g. \cite[Remark 3.1]{dihitchhiker}). Moreover, $\mathcal L_s^1(\RR)$ is the right space for which $v:= (-d_x^{\,2})^s_{\varepsilon}\, u$ exists for every $\varepsilon > 0$, $v$ being also continuous at the continuity points of $u$.

It is by now well-known (see, e.g., \cite{dihitchhiker}) that the natural functional setting for problems involving the Fractional Laplacian is the one of the fractional Sobolev spaces. Since these spaces are not so familiar as the classical integral order ones, for the sake of completeness, we recall here their definition. 

Given $s\in(0,1)$, the fractional Sobolev space $H^s(-L,L)$ is defined as
\begin{align*}
	H^s(-L,L):= \left\{u\in L^2(-L,L)\,:\, \frac{|u(x)-u(y)|}{|x-y|^{\frac 12+s}}\in L^2\Big((-L,L)^2\Big)\right\}.
\end{align*}

It is classical that this is a Hilbert space, endowed with the norm (derived from the scalar product)
\begin{align*}
	\norm{u}{H^s(-L,L)} := \left[\norm{u}{L^2(-L,L)}^2 + |u|_{H^s(-L,L)}^2\right]^{\frac 12},
\end{align*}
where the term 
\begin{align*}
|u|_{H^s(-L,L)}:= \left(\int_{-L}^L\int_{-L}^L \frac{|u(x)-u(y)|^2}{|x-y|^{1+2s}}\,dxdy\right)^{\frac 12}
\end{align*}
is the so-called Gagliardo seminorm of $u$. We set 
\begin{align*}
H_0^s(-L,L):= \overline{C_0^\infty(-L,L)}^{\,H^s(-L,L)}
\end{align*}
the closure of the continuous infinitely differentiable functions compactly supported in $(-L,L)$ with respect to the $H^s(-L,L)$-norm. The following facts are well-known.
\begin{itemize}
	\item[$\bullet$] For $0<s\leq\frac 12$, the identity $H_0^s(-L,L) = H^s(-L,L)$ holds. This is because, in this case, the $C_0^\infty(-L,L)$ functions are dense in $H^s(-L,L)$ (see, e.g., \cite[Theorem 11.1]{jllions1972non}).
	
	\item[$\bullet$] For $\frac 12<s<1$, we have $H_0^s(-L,L)=\left\{ u\in H^s(\RR)\,:\,u=0\textrm{ in } \RR\setminus (-L,L)\right\}$ (\cite{fiscella2015density}).
\end{itemize}

Finally, in what follows we will indicate with $H^{-s}(-L,L)=\left(H^s(-L,L)\right)'$ the dual space of $H^s(-L,L)$ with respect tot the pivot space $L^2(-L,L)$.

A more exhaustive description of fractional Sobolev spaces and of their properties can be found in several classical references (see, e.g., \cite{adams2003sobolev,dihitchhiker,jllions1972non}).

Let us now introduce the variational formulation associated to equation \eqref{PE}. That is, find $u\in H^s_0(-L,L)$ such that
\begin{align*}
a(u,v) = \int_{-L}^L fv\,dx,	
\end{align*}
for all $v\in H_0^s(-L,L)$, where the bilinear form $a(\cdot,\cdot):H^s_0(-L,L)\times H^s_0(-L,L)\to \RR$ is given by
\begin{align*}
a(u,v)=\frac{\ccs}{2} \int_{\RR}\int_{\RR}\frac{(u(x)-u(y))(v(x)-v(y))}{|x-y|^{1+2s}}\,dxdy.	
\end{align*}

Since the bilinear form $a$ is continuous and coercive, Lax-Milgram Theorem immediately implies existence and uniqueness of solutions to the Dirichlet problem \eqref{PE}. In more detail, if $f\in H^{-s}(-L,L)$, then \eqref{PE} admits a unique weak solution $u\in H_0^s(-L,L)$ (see, e.g., \cite[Proposition 2.1]{biccari2017local}). Furthermore, in the literature it is possible to find improved regularity results for the solution to \eqref{PE}, both in H\"older and Sobolev spaces. The interested reader may refer, for instance, to \cite{acosta2017fractional,biccari2017local,leonori2015basic,ros2014dirichlet,ros2014extremal}.

Let us now discuss the parabolic equation \eqref{heat_frac}. First of all, we mention that the issues of existence, uniqueness and regularity of the solutions have been studied by several authors. Among others, we mention the works \cite{biccari2017parabolic,fernandez2016boundary,leonori2015basic}. In particular, in \cite[Theorem 26]{leonori2015basic} it is showed that, assuming $z_0\in L^2(\Omega)$ and $g\in L^2(0,T;H^{-s}(\Omega))$, then equation \eqref{heat_frac} admits a unique weak solution $z\in L^2(0,T;H_0^s(\Omega))\cap C([0,T];L^2(\Omega))$ with $z_t\in L^2(0,T;H^{-s}(\Omega))$. Notice that, taking as in our case $g\in L^2(\omega\times(0,T))$, the same result holds due to the continuous injection of $L^2$ into $H^{-s}$.

In this paper we are mainly interested in the study of control properties for the parabolic system \eqref{heat_frac}. For the sake of completeness, we include below the definitions of null and approximate controllability.

\begin{definition}
	System \eqref{heat_frac} is said to be \textit{null-controllable} at time $T$ if, for any $z_0\in L^2(-1,1)$, there exists $g\in L^2(\omega\times(0,T))$ such that the corresponding solution $z$ satisfies 
	\begin{align*}
		z(x,T)=0.
	\end{align*}
\end{definition}

\begin{definition}
	System \eqref{heat_frac} is said to be \textit{approximately controllable} at time $T$ if, for any $z_0,z_T\in L^2(-1,1)$ and any $\delta>0$, there exists $g\in L^2(\omega\times(0,T))$ such that the corresponding solution $z$ satisfies \begin{align*}
		\norm{z(x,T)-z_T}{L^2(-1,1)}<\delta.
	\end{align*}
\end{definition}

We already mentioned that, to the best of our knowledge, there are no results in the literature concerning the controllability of the fractional heat equation involving the integral operator \eqref{fl}. The existing ones deal with the \textit{spectral} definition of the fractional Laplace operator, which is given as follows.

Let $\{\psi_k,\lambda_k\}_{k\in\NN}\subset H_0^1(-1,1)\times\RR^+$ be the set of normalized eigenfunctions and eigenvalues of the Laplace operator in $(-1,1)$ with homogeneous Dirichlet boundary conditions, so that $\{\psi_k\}_{k\in\NN}$ is an orthonormal basis of $L^2(-1,1)$ and         
\begin{align*}
	\begin{cases}
		-d_x^2\psi_k =\lambda_k\psi_k, & x\in (-1,1), 
		\\
		\psi_k(-1)=\psi_k(1)=0.
	\end{cases}
\end{align*}

Then, the \textit{spectral fractional Laplacian} $(-d_x^{\,2})^s_S$ is defined by
\begin{align}\label{fl_spec}
	(-d_x^{\,2})^s_S u(x) = \sum_{k\geq 1}\langle u,\psi_k\rangle \lambda_k^s\psi_k(x),
\end{align}
firstly for $u\in C_0^{\infty}(-1,1)$ and then for $u\in H_0^s(-1,1)$ employing a density argument.

It is important to notice that the spectral fractional Laplacian and the fractional Laplacian defined as in \eqref{fl} are two different operators. For instance, definition \eqref{fl_spec} depends on the choice of the domain (in this case, $(-1,1)$), while the integral definition does not. For a complete discussion on the differences of these two operators, we refer to \cite{servadei2014spectrum}.

The control problem for the fractional heat equation involving the operator $(-d_x^{\,2})^s_S$ has been analyzed in \cite{micu2006controllability}, where the authors proved null controllability provided that $s>1/2$. For $s\leq 1/2$, instead, null controllability does not hold, not even for $T$ large. This negative result is based on the equivalence (consequence of M\"untz Theorem, see, e.g., \cite[Page 24]{schwartz1958etude}) between the controllability property (more specifically, the possibility of proving an observability inequality), and the following condition for the eigenvalues of the operator considered
\begin{align}\label{eigen_cond}
	\sum_{k\geq 1} \frac{1}{\lambda_k}<\infty,
\end{align} 
which is clearly not satisfied for the spectral fractional Laplacian when $s\leq 1/2$, since in that case the eigenvalues are $\lambda_k = (k\pi)^{2s}$. Finally, in \cite{miller2006controllability}, the same result as in \cite{micu2006controllability} is obtained in a multi-dimensional setting, by means of a  \textit{spectral observability condition} for a negative self-adjoint operator, which allows to prove the null-controllability of the semi-group that it generates.

As we anticipated in Section \ref{intro_sec}, the same null controllability result holds also for the parabolic equation \eqref{heat_frac}. This is obtained by means of classical tools (\cite{fattorini1971exact}), as well as explicit approximations of the eigenvalues and the eigenfunctions the fractional Laplacian with homogeneous Dirichlet boundary conditions. 

